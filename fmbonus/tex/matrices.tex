\section{Matrices}

This section unfortunately does not make any attempt to define anything, as I
can only think of one worthwhile question. When you have had the necessary
definitions and tools defined for you, you may attempt it.
\begin{enumerate}
 \item
  Let \(\vec f_n\) be a sequence of vectors defined as
  \begin{equation*}
   \vec f_n \defeq
   \begin{cases*}
    \begin{pmatrix} 0 \\ 1 \end{pmatrix} &
     if \(n = 0\) \\
    \begin{pmatrix} 0 & 1 \\ 1 & 1 \end{pmatrix}\vec f_{n - 1} &
     for \(n \in \Naturals\)
   \end{cases*}
  \end{equation*}
  Show that
  \(\vec f_{n + 2} = \vec f_{n + 1} + \vec f_n\) for all
  \(n \in \Naturals\), and that \((\vec f_1)_1 = (\vec f_2)_1 = 1\). By
  writing \(\begin{psmallmatrix} 0 & 1 \\ 1 & 1 \end{psmallmatrix}\) as
  \(\mat U \mat D \mat U^{-1}\) where \(\mat D\) is a diagonal matrix,
  find a closed form for the \(n\)th term of the Fibonacci sequence.
  % a nice way to do this is to write
  %             [ 0 1 ]
  % f_{n + 2} = [     ] f_{n + 1}
  %             [ 1 1 ]
  %
  %             [[ 1 0 ]   [ -1 1 ]]
  %           = [[     ] + [      ]] f_{n + 1}
  %             [[ 0 1 ]   [  1 0 ]]
  %
  %                         [ -1 1 ] [ 0 1 ]
  %           = f_{n + 1} + [      ] [     ] f_n
  %                         [  1 0 ] [ 1 1 ]
  %
  %           = f_{n + 1} + f_n
  %
  % which is basically Cayley-Hamiltonning the fact that
  % φ is an eigenvalue of this matrix (so M - I = M^-1)
\end{enumerate}
