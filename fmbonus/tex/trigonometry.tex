\section{Trigonometry}

\begin{enumerate}
 \item
  Prove the following identities (you should argue from definitions):
  \begin{itemize}
   \item
    \(\sin(\pi - \theta) \equiv \sin \theta\)
   \item
    \(\cos(\pi - \theta) \equiv - \cos \theta\)
   \item
    \(\sin(\theta + 2\pi) \equiv \sin \theta\)
   \item
    \(\cos(\theta + 2\pi) \equiv \cos \theta\)
   \item
    \(\sin(\frac \pi 2 - \theta) \equiv \cos \theta\)

    (Here you should start with the case \(\theta \in \intcc{0, \frac \pi 2}\)
    and then extend the result to \(\Reals\).)
   \item
    \(\tan(-\theta) \equiv -\tan \theta\)
   \item
    \(\tan(\theta + \pi) \equiv \tan \theta\)
   \item
    \(\tan(\frac \pi 2 - \theta) \equiv \dfrac 1{\tan \theta}\)
  \end{itemize}
  Consider the set of identities from this question, and the identities
  describing the parity of \(\sin\) and \(\cos\).

  Find a locally minimal subset of these identities that generates all the
  others (that is, from which you can prove all the others, but if you remove
  any element of the subset you can no longer prove all the others).
 \item
  Using the fact that the image of \(\arcsin\) is
  \(\intcc{-\frac \pi 2, \frac \pi 2}\), prove that
  \begin{equation*}
   \arcsin(\sin x) = x \Forall x \in \intcc{-\tfrac \pi 2, \tfrac \pi 2}
  \end{equation*}
 \item
  Prove Pythagoras' Theorem.

  Hence deduce the Pythagorean identity for \(\sin \theta\) and \(\cos \theta\)
  when \(\theta \in \intcc{0, \frac \pi 2}\):
  \begin{equation*}
   \sin^2 \theta + \cos^2 \theta \equiv 1
  \end{equation*}
  It is possible to show that this holds for all \(\theta\) by doing some boring
  stuff where you consider \(\abs{\sin \theta}\) and \(\abs{\cos \theta}\). I
  won't bore you with it here, you may just assume it.
 \item
  Prove that
  \begin{align*}
   \tan^2 \theta + 1 &\equiv \frac 1{\cos^2 \theta} \\
   \frac 1{\tan^2 \theta} + 1 &\equiv \frac 1{\sin^2 \theta}
  \end{align*}
 \item
  Draw a triangle \(\tri{ABC}\) with \(\lseg{AC}\) horizontal, and
  \(\angle{BAC}\) and \(\angle{BCA}\) acute. Let \(\angle{BAC} = \theta\).

  Label the sides in the conventional way:
  \(\lseg{AB} = c\), \(\lseg{BC} = a\), \(\lseg{CA} = b\).

  Drop an altitude\footnote{
   This is a word for the perpendicular to a side of a triangle the passes
   through the opposite vertex.
  } from \(B\) to the point \(D\) on \(\lseg{AC}\).

  Derive the \emph{cosine rule} for angles less than \(\degrees{90}\) by
  finding the length \(\lseg{BC}\).

  You may now assume that this also holds for \(\theta\) obtuse.
 \item
  In the same triangle, derive the \emph{sine area rule} for angles less than
  \(\degrees{90}\) by finding the area \(\area{\tri{ABC}}\).

  Noting that \(\tri{BCA}\) and \(\tri{CAB}\) are also triangles, immediately
  state two similar sine area rules.

  Hence deduce the \emph{sine rule}.

  You may now assume that both also hold for obtuse angles.
 \item
  Now draw a new triangle, \(\tri{ABC}\) where \(\lseg{BC}\) is vertical, with
  both \(\angle{ABC}\) and \(\angle{ACB}\) acute, and drop an altitude from
  \(A\) to the point \(D\) on \(\lseg{BC}\). Let \(\angle{BAD} = \alpha\),
  \(\angle{CAD} = \beta\).

  Also, let the triangle be scaled so that \(\lseg{AD} = 1\).

  By calculating the length \(\lseg{BC}\) in two different ways, determine the
  \emph{compound angle formula for cosine}\footnote{
   \label{foot_compound_angles}%
   for angles \(< \degrees{90}\)
  }:
  \begin{equation*}
   \cos(\alpha + \beta) \equiv \cos \alpha \cos \beta - \sin \alpha \sin \beta
  \end{equation*}
 \item
  In the same triangle, determine the
  \emph{compound angle formula for
  sine}\footnote{
   See \ref{foot_compound_angles}
  }, by calculating the area of the triangle in two different ways.
 \item
  Suppose that the compound angle formula for cosine holds for all
  \(\alpha, \beta\).

  Give a different proof of the compound angle formula for sine just using the
  parities of trig functions and the identity
  \begin{equation*}
   \sin x \equiv \cos(\tfrac \pi 2 - x)
  \end{equation*}
  You may now assume that both compound angle formulae hold for all
  \(\alpha, \beta\)\footnote{
   It's quite straightforward to show this by just bashing all the different
   extension cases, but not every interesting.}.

  The latter two identities describe the \emph{parity} of \(\cos\) and \(\sin\).
  Particularly, \(\cos\) is an \emph{even} function, and \(\sin\) is an
  \emph{odd} function.
 \item
  Prove the compound angle formula for tangent:
  \begin{equation*}
   \tan(\alpha + \beta) \equiv
   \frac{\tan \alpha + \tan \beta}{1 - \tan \alpha \tan \beta}
  \end{equation*}
 \item
  Use parities to state compound angle formulae for \(\sin(\alpha - \beta)\),
  \(\cos(\alpha - \beta)\), and \(\tan(\alpha - \beta)\)
 \item
  Write \(\sin 2x\) in terms of \(\sin x\) and \(\cos x\).
 \item
  Also write \(\cos 2x\) in terms of \(\sin x\) and \(\cos x\).

  Then use an identity to write \(\cos 2x\) in terms of just \(\sin x\), and
  also write \(\cos 2x\) in terms of just \(\cos x\).
 \item
  Hence write \(\sin^2 x\) as \(A \cos(ax + b) + B \sin(cx + d) + C\).

  Do the same for \(\cos^2 x\).
 \item
  By considering \(\sin(\alpha + \beta)\) and \(\sin(\alpha - \beta)\), write
  \(\sin \alpha \cos \beta\) as a linear combination\footnote{
   A linear combination of objects is a sum of each of the objects, multiplied
   by some scalar (a real number in this case).
  } of trigonometric functions of linear combinations of \(\alpha\) and
  \(\beta\). This is one of the \emph{product-sum} identities.
 \item
  Do the same for \(\sin \alpha \sin \beta\) and
  \(\cos \alpha \cos \beta\).
 \item
  Consider \(\sin \alpha + \sin \beta\). By finding \(\alpha'\) and \(\beta'\)
  such that \(\alpha' + \beta' = \alpha\) and \(\alpha' - \beta' = \beta\),
  write \(\sin \alpha + \sin \beta\) as a product of trigonometric functions of
  linear combinations of \(\alpha\) and \(\beta\), multiplied by some constant.

  Without doing any more algebra, determine a similar formula for
  \(\sin \alpha - \sin \beta\).

  Also find formulae for \(\cos \alpha + \cos \beta\) and
  \(\cos \alpha - \cos \beta\).
 \item
  Draw triangles to determine the values of
  \(\sin \frac \pi 6\), \(\sin \frac \pi 4\), \(\sin \frac \pi 3\),
  \(\cos \frac \pi 6\), \(\cos \frac \pi 4\), \(\cos \frac \pi 3\),
  \(\tan \frac \pi 6\), \(\tan \frac \pi 4\), \(\tan \frac \pi 3\).
 \item
  Find the values of \(\sin \frac{5\pi}{12}\) and \(\cos \frac{5\pi}{12}\) by
  writing \(5/12\) in a different way.

  Hence deduce the values of \(\sin \frac \pi{12}\) and \(\cos \frac \pi{12}\).
 \item
  If \(\theta = \frac \pi 8\), determine the value of
  \(\cos \theta\) by using an identity for \(\cos 2\theta\).

  Calculate the values of \(\tan \frac \pi 8\) and \(\tan \frac{3\pi}8\).

  Verify your answer by partitioning the regular octagon with unit sides into
  rectangles, squares, and right isosceles triangles, and finding the distances
  from the centre to a vertex and the midpoint of an edge.
 \item
  Show that if \(\theta = \frac \pi 5\), then \(\sin 2\theta = \sin 3\theta\).
  Expand both sides and solve for \(\cos \theta\) to determine the value of
  \(\cos \frac \pi 5\).

  Deduce the value of \(\sin \frac \pi 5\).

  Also show that if \(x\) is the length of any diagonal of a regular pentagon
  with unit sides, then \(x^2 = x + 1\).
 \item \label{q_trig_arcsin}
  Show that the function
  \begin{align*}
   f : \intcc{-1, 1} &\to \intcc{0, \pi} \\
   x &\mapsto \tfrac \pi 2 - \arcsin x
  \end{align*}
  satisfies the definition of \(\arccos x\).
 \item
  Do a rough sketch of \(\cos x + \sin x\).

  By guessing that \(\cos x + \sin x\) can be written in the form
  \(R \sin(x + \alpha)\) where \(R\), \(\alpha\) are constants, find
  suitable constants \(R \in \Reals\) and \(\alpha \in \intco{0, 2\pi}\). Hence
  justify the shape of your sketch, and find the minimum and maximum values.

  Also sketch \(\sqrt{12} \sin 2x + 2 - 4\sin^2 x\), stating its minimum and
  maximum values.
 \item
  If \(r_1 \sin x + r_2 \cos x \equiv R \sin(x + \alpha)\), where
  \(R \ge 0\) and \(\alpha \in \intco{0, 2 \pi}\), show that
  \(R^2 = r_1^2 + r_2^2\) and \(\tan \alpha = r_2 / r_2\).

  Can you deduce the values of both \(R\) and \(\alpha\) given this information?
 \item
  Prove the following identities:
  \begin{alignat*}2
   (\mathrm{i})&&\quad\frac 1{\sin^2 x} + \frac 1{\cos^2 x}
   &\equiv \frac 1{\cos^2 x \sin^2 x} \\
   (\mathrm{ii})&&\quad\cos x + \sin x \tan x
   &\equiv \frac 1{\cos x} \\
   (\mathrm{iii})&&\quad\frac 1{\tan x} + \tan x
   &\equiv \frac 2{\sin 2x} \\
   (\mathrm{iv})&&\quad\sin(x - y) \sin(x + y)
   &\equiv (\sin x - \sin y)(\sin x + \sin y)
  \end{alignat*}
 \item
  Use trigonometric identities to sketch the following:
  \begin{itemize}
   \item
    \(\displaystyle y = \frac{\cos 2x}{\sin(x + \frac \pi 2)}\).
   \item
    \(\sin(x + y) = \sin(x - y)\)
   \item
    \(\sin x = \sin y\)

    This last sketch is actually quite interesting: this is a general solution
    that comes up a lot. For instance, when you're looking at
    \(\sin x = \frac 12\), this is really just \(\sin x = \sin \frac \pi 6\),
    which is a special case of an equation of this form.

    Can you therefore characterise all solutions to \(\sin 13x = \sin 5x\)?
  \end{itemize}
 \item
  By considering
  \begin{itemize}
   \item
    The isosceles triangle with unit legs adjacent to an angle of \(\theta\)
   \item
    The sector of a unit circle subtended by an angle of \(\theta\)
   \item
    The right triangle with unit side adjacent to an angle of \(\theta\)
    (\emph{not} having unit hypotenuse)
  \end{itemize}
  (in that order) and drawing a suitable diagram, show that
  \(\sin \theta < \theta < \tan \theta\) for
  \(\theta \in \intoo{0, \frac \pi 2}\).
 \item
  Hence show that
  \begin{equation*}
   \cos x < \frac{\sin x}x < 1
  \end{equation*}
  for \(x \in \intoo{-\frac \pi 2, \frac \pi 2} \setminus \set{0}\).

  An \emph{extremely} useful consequence of this uses ``the squeeze theorem''.
  Since \(\frac{\sin x} x\) is bounded above and below by functions that tend to
  \(1\) as \(x \to 0\) (from either side of \(0\)),
  \begin{equation*}
   \lim_{x \to 0} \frac{\sin x}x = 1
  \end{equation*}
  We have now somewhat formally shown that ``\(\sin x\) looks roughly like \(x\)
  for small \(x\)''. This has many applications in engineering and physics.
 \item \label{q_trig_weierstrass}
  Show that if \(t = \tan \frac 12 x\), then
  \begin{align*}
   \tan x &\equiv \frac{2t}{1 - t^2} \\
   \sin x &\equiv \frac{2t}{1 + t^2} \\
   \cos x &\equiv \frac{1 - t^2}{1 + t^2}
  \end{align*}
  What is the value of \(\tan \frac 12 x\) if \(\sin x = \frac 5{13}\) and
  \(\cos x = \frac{12}{13}\)?

  Are there infinitely many values of \(\theta \in \intco{0, 2\pi}\) for which
  \(\cos \theta\) and \(\sin \theta\) are both rational?
\end{enumerate}
