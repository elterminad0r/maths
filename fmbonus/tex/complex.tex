\section*{Complex Numbers}

\begin{enumerate}
 \item
  Given two complex numbers
  \begin{align*}
   z_1 &= r_1(\cos \theta_1 + i \sin \theta_1) \\
   z_2 &= r_2(\cos \theta_2 + i \sin \theta_2)
  \end{align*}
  where \(r_1, r_2, \theta_1, \theta_2 \in \Reals\), draw a diagram in the
  complex plane\footnote{
   The complex plane is a two-dimensional plane with ``real part'' running along
   the horizontal axis, and ``imaginary part'' running along the vertical axis,
   used to represent complex numbers.
  } to interpret the values of
  \(r_1\), \(r_2\), \(\theta_1\), and \(\theta_2\).

  Calculate \(z_1 z_2\) and write it in the form
  \(R_1 \cos \phi_1 + R_2 \sin \phi_2\). Hence draw \(z_1 z_2\) on your diagram.

  Could \(z_1\) be any complex number? Write down a pair of values
  \((r_1, \theta_1)\) so that \(z_1 = -1 - \sqrt 3i\).

  Write down another such pair of values for which \(r_1\) is the same but
  \(\theta_1\) is different. Also write down another pair for which
  \(\theta_1\) is the same but \(r_1\) is different.
 \item
  Carefully write down the first eight terms of \(e^{ix}\). By comparing with
  the Maclaurin series of \(\sin x\) and \(\cos x\), write down a sensible
  formula for \(e^{ix}\), of the form \(\alpha \cos x + \beta \sin x\) where
  \(\alpha, \beta \in \Complex\).
 \item
  You may now assume that this formula is correct. You may also assume that
  \((e^z)^w = e^{zw}\) still holds \(\Forall z, w \in \Complex\).

  Write down the values of \(e^{i\pi}\), \(e^{i\pi / 2}\),
  \(e^{i\pi / 4}\), \(e^{i\pi / 3}\), and \(e^{2 i \pi}\).
 \item
  Use the formula to express
  \((\cos \theta + i \sin \theta)^n\) in the form \(\cos \phi + i \sin \phi\).

  This is called ``de Moivre's Theorem''.

  Give a more rigorous proof of de Moivre's Theorem by induction.
 \item
  By considering the real part of
  \(e^{5ix}\), deduce a formula for \(\cos 5x\) in terms of \(\cos x\).

  You may wish to write \(c = \cos x\) for convenience.
 \item
  Write down \(\alpha_1, \alpha_2, \beta_1, \beta_2 \in \Complex\) such that
  \begin{align*}
   e^{ix} &= \alpha_1 \cos x + \beta_1 \sin x \\
   e^{-ix} &= \alpha_2 \cos x + \beta_2 \sin x
  \end{align*}
  Solve this system of equations to find nice formulae for
  \(\sin x\) and \(\cos x\) in terms of \(e^{\pm ix}\).
 \item
  Hence find a solution \(z \in \Complex\) to \(\sin z = 2\).

  Is this the only solution?
\end{enumerate}
