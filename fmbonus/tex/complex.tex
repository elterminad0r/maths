\section{Complex Numbers}

\begin{enumerate}
 \item
  Show that addition and multiplication in the complex numbers are commutative
  and associative.
 \item
  Show that multiplication distributes over addition in the complex numbers.

  (This means that
  \begin{equation*}
   a(b + c) = ab + ac \Forall a, b, c \in \Complex)
  \end{equation*}
 \item
  Show that if \(x\) is real, then \(x(z_1, z_2) \equiv (x z_1, x z_2)\).
 \item
  Show that if \(x, y \in \Reals\), then \((x, y) \equiv x + i y\).

  This gives rise to the conventional notation \(x + iy\) for an arbitrary
  complex number. That is, if you are given \(z \in \Complex\), it fine to say
  ``let \(x, y \in \Reals\) be such that \(z = x + iy\).''
 \item
  Show that \(i^2 = -1\).
 \item
  Show that if \(z, w \in \Complex\), then \(\abs{zw} = \abs z \abs w\).
 \item
  If \(x, y \in \Reals\) and \(z = x + iy\), find a \(w \in \Complex\) such that
  \(zw = \abs{z}^2\). \(w\) is usually written \(\conj z\).

  If \(z \ne 0\), find a complex number \(z^{-1} \in \Complex\) such that
  \(zz^{-1} = 1\). Write it in terms of \(z\) and \(\conj z\).
 \item \label{q_compl_args}
  Given two complex numbers
  \begin{align*}
   z &= r_1(\cos \theta_1 + i \sin \theta_1) \\
   w &= r_2(\cos \theta_2 + i \sin \theta_2)
  \end{align*}
  where \(r_1, r_2, \theta_1, \theta_2 \in \Reals\),
  calculate the values of \(\abs{z}\) and \(\abs{w}\).

  Draw a diagram in the complex plane\footnote{
   The complex plane is a two-dimensional plane with ``real part'' running along
   the horizontal axis, and ``imaginary part'' running along the vertical axis,
   used to represent complex numbers.
  } to interpret the values of
  \(r_1\), \(r_2\), \(\theta_1\), and \(\theta_2\).

  Calculate \(zw\) and write it in the form
  \(R_1 \cos \phi_1 + R_2 \sin \phi_2\). Hence calculate \(\abs{zw}\) and draw
  \(zw\) on your diagram.

  Could \(z\) be any complex number? Write down a pair of values
  \((r_1, \theta_1)\) so that \(z = -1 - \sqrt 3i\).

  Write down another such pair of values for which \(r_1\) is the same but
  \(\theta_1\) is different. Also write down another pair for which
  \(\theta_1\) is the same but \(r_1\) is different.
 \item
  Carefully write down the first eight terms of the Maclaurin series of
  \(e^{i\theta}\) (by expanding \(e^x\) where \(x = i\theta\)). By comparing
  with the Maclaurin series of \(\sin \theta\) and \(\cos \theta\), write down a
  sensible formula for \(e^{i\theta}\), of the form
  \(\alpha \cos \theta + \beta \sin \theta\) where
  \(\alpha, \beta \in \Complex\).
 \item
  You may now assume that this formula is correct\footnote{
   Really, this is just how we define \(e^z\) for \(z \in \Complex\), so its
   correctness is literally trivial.
  }.

  Use question \ref{q_compl_args} to show that for \(\theta, \phi \in \Reals\),
  \(e^{i\theta} e^{i\phi} \equiv e^{i(\theta + \phi)}\).
 \item
  For \(z = x + iy \in \Complex\), we now further define
  \(e^z \defeq e^x e^{iy}\).

  Hence prove by induction that for \(z \in \Complex\), \(n \in \Naturals\),
  \((e^z)^n \equiv e^{nz}\).

  Write down the values of \(e^{i\pi}\), \(e^{i\pi / 2}\),
  \(e^{i\pi / 4}\), \(e^{i\pi / 3}\), and \(e^{2 i \pi}\).
 \item
  Use the formula to express
  \((\cos \theta + i \sin \theta)^n\) in the form \(\cos \phi + i \sin \phi\).

  This is called ``de Moivre's Theorem''.
 \item
  By considering the real part of \(e^{5ix}\), show that
  \begin{equation*}
   \cos 5x \equiv 16\cos^5 x - 20\cos^3 x + 5\cos x
  \end{equation*}
  You may wish to write \(c = \cos x\) and \(s = \sin x\) for convenience.
  % cos 5x = Re((c + is)^5)
  %        = c^5 - c^3 s^2 + c s^4
  %        = c^5 - 10 c^3 (1 - c^2) + 5 c (1 - c^2)^2
  %        = c^5 - 10c^3 + 10c^5 + 5c - 10c^3 + 5c^5
  %        = 16c^5 - 20c^3 + 5c
 \item
  Show that if \(n \in \Naturals\),
  \begin{align*}
   \cos 2nx &\equiv
    \sum_{r = 0}^{n - 1}
     (-1)^r \binom{2n}{2r} \cos^{2(n - r)} x \sin^{2r} x \\
   \sin 2nx &\equiv
    \sum_{r = 0}^{n - 1}
     (-1)^r \binom{2n}{2r + 1} \cos^{2(n - r) - 1} x \sin^{2r + 1} x
  \end{align*}
  and deduce that
  \begin{equation*}
   \tan 2nx \equiv
   \frac{\sum_{r = 0}^{n - 1} (-1)^r \binom{2n}{2r + 1} \tan^{2r + 1} x}
        {\sum_{r = 0}^{n - 1} (-1)^r \binom{2n}{2r} \tan^{2r} x}
  \end{equation*}
  % cos 2nx + i·sin 2nx = (c + is)^2n
  %                    = Σ (2n)Cr c^(2n - r) (is)^r
  %                    = Σ (2n)Cr c^(2n - r) i^r s^r
  %                    = Σ (2n)C(2r) c^(2n - 2r) i^2r s^2r +
  %                      Σ (2n)C(2r + 1) c^(2n - 2r - 1) i^(2r + 1) s^(2r + 1)
  %                    = Σ (2n)C(2r) c^(2n - 2r) (-1)^2r s^2r +
  %                     iΣ (2n)C(2r + 1) c^(2n - 2r - 1) (-1)^r s^(2r + 1)
 \item
  Write down \(\alpha_1, \alpha_2, \beta_1, \beta_2 \in \Complex\) such that
  \begin{align*}
   e^{ix} &= \alpha_1 \cos x + \beta_1 \sin x \\
   e^{-ix} &= \alpha_2 \cos x + \beta_2 \sin x
  \end{align*}
  Solve this system of equations to find nice formulae for
  \(\sin x\) and \(\cos x\) in terms of \(e^{\pm ix}\).
 \item
  Hence find a solution \(z \in \Complex\) to \(\sin z = 2\).

  Is this the only solution?
  %      (e^ix - e^-ix)/2i = 2
  % ⇔          e^2ix - 1 = 4ie^ix
  % ⇔ e^2ix - 4ie^ix - 1 = 0
  % ⇔               e^ix = (4i ± √(-12)) / 2
  %                      = i(2 ± √3)
  %                      = e^(iπ/2 + log(2 ± √3))
  % ⇔                 iz = iπ(1/2 + 2n) + log(2 ± √3)
  % ⇔                  z = π(1/2 + 2n) - i·log(2 ± √3)
 \item
  We write \(\Integers[i]\) to denote the Gaussian integers, which is the set of
  complex numbers with integer real and imaginary parts - ie
  \(\set{a + bi : a, b \in \Integers}\).

  Let \(n \in \Integers[i]\). Show that \(\abs{n}^2 \in \Integers\).

  Explain why \(n^2\) corresponds to a Pythagorean triple.
\end{enumerate}

\subsection{An aside on complex exponentiation}
%TODO
