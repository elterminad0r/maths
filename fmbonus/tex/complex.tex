\section{Complex Numbers}

\begin{enumerate}
 \item
  Show that addition and multiplication in the complex numbers are commutative
  and associative.
 \item
  Show that multiplication distributes over addition in the complex numbers.

  (This means that
  \begin{equation*}
   a(b + c) = ab + ac \Forall a, b, c \in \Complex)
  \end{equation*}
 \item
  Show that if \(x\) is real, then \(x(z_1, z_2) \equiv (x z_1, x z_2)\).
 \item
  Show that if \(x, y \in \Reals\), then \((x, y) \equiv x + i y\).

  This gives rise to the conventional notation \(x + iy\) for an arbitrary
  complex number. That is, if you are given \(z \in \Complex\), it fine to say
  ``let \(x, y \in \Reals\) be such that \(z = x + iy\).''
 \item
  Show that \(i^2 = -1\).
 \item
  Show that if \(z, w \in \Complex\), then \(\abs{zw} = \abs z \abs w\).
 \item
  If \(x, y \in \Reals\) and \(z = x + iy\), find a \(w \in \Complex\) such that
  \(zw = \abs{z}^2\). \(w\) is usually written \(\conj z\).

  If \(z \ne 0\), find a complex number \(z^{-1} \in \Complex\) such that
  \(zz^{-1} = 1\). Write it in terms of \(z\) and \(\conj z\).
 \item
  Given two complex numbers
  \begin{align*}
   z &= r_1(\cos \theta_1 + i \sin \theta_1) \\
   w &= r_2(\cos \theta_2 + i \sin \theta_2)
  \end{align*}
  where \(r_1, r_2, \theta_1, \theta_2 \in \Reals\),
  calculate the values of \(\abs{z}\) and \(\abs{w}\).

  Draw a diagram in the complex plane\footnote{
   The complex plane is a two-dimensional plane with ``real part'' running along
   the horizontal axis, and ``imaginary part'' running along the vertical axis,
   used to represent complex numbers.
  } to interpret the values of
  \(r_1\), \(r_2\), \(\theta_1\), and \(\theta_2\).

  Calculate \(zw\) and write it in the form
  \(R_1 \cos \phi_1 + R_2 \sin \phi_2\). Hence calculate \(\abs{zw}\) and draw
  \(zw\) on your diagram.

  Could \(z\) be any complex number? Write down a pair of values
  \((r_1, \theta_1)\) so that \(z = -1 - \sqrt 3i\).

  Write down another such pair of values for which \(r_1\) is the same but
  \(\theta_1\) is different. Also write down another pair for which
  \(\theta_1\) is the same but \(r_1\) is different.
 \item
  Carefully write down the first eight terms of \(e^{ix}\). By comparing with
  the Maclaurin series of \(\sin x\) and \(\cos x\), write down a sensible
  formula for \(e^{ix}\), of the form \(\alpha \cos x + \beta \sin x\) where
  \(\alpha, \beta \in \Complex\).
 \item
  You may now assume that this formula is correct. You may also assume that
  \((e^z)^w = e^{zw}\) still holds \(\Forall z, w \in \Complex\).

  Write down the values of \(e^{i\pi}\), \(e^{i\pi / 2}\),
  \(e^{i\pi / 4}\), \(e^{i\pi / 3}\), and \(e^{2 i \pi}\).
 \item
  Use the formula to express
  \((\cos \theta + i \sin \theta)^n\) in the form \(\cos \phi + i \sin \phi\).

  This is called ``de Moivre's Theorem''.

  Give a more rigorous proof of de Moivre's Theorem by induction.
 \item
  By considering the real part of
  \(e^{5ix}\), deduce a formula for \(\cos 5x\) in terms of \(\cos x\).

  You may wish to write \(c = \cos x\) for convenience.
 \item
  Write down \(\alpha_1, \alpha_2, \beta_1, \beta_2 \in \Complex\) such that
  \begin{align*}
   e^{ix} &= \alpha_1 \cos x + \beta_1 \sin x \\
   e^{-ix} &= \alpha_2 \cos x + \beta_2 \sin x
  \end{align*}
  Solve this system of equations to find nice formulae for
  \(\sin x\) and \(\cos x\) in terms of \(e^{\pm ix}\).
 \item
  Hence find a solution \(z \in \Complex\) to \(\sin z = 2\).

  Is this the only solution?
 \item
  We write \(\Integers[i]\) to denote the Gaussian integers, which is the set of
  complex numbers with integer real and imaginary parts - ie
  \(\set{a + bi : a, b \in \Integers}\).

  Let \(n \in \Integers[i]\). Show that \(\abs{n}^2 \in \Integers\).

  Explain why \(n^2\) corresponds to a Pythagorean triple.
\end{enumerate}
