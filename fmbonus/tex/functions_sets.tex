\section{Functions and Sets}

\begin{enumerate}
 \item
  Write down the set that is a subset of every set.
 \item
  Use quantifiers to write down all the properties that a subset of
  \(A \times B\) must satisfy for it to correspond to a function from
  \(A \to B\).

  Write down the conditions for such a subset to correspond to an injective and
  a surjective function, respectively.
 \item
  State a necessary and sufficient condition on \(A\) and \(B\) for
  \(A \times B\) to correspond to a function from \(A \to B\).
 \item
  Show that if \(A\) and \(B\) are sets, and \(A \subseteq B\) and
  \(B \subseteq A\), then \(A = B\).
 \item
  % "getallen"
  Let \(\Gamma\) be a set with the following properties:
  \begin{itemize}
   \item
    % "zero"
    There is an element \(\zeta \in \Gamma\).
   \item
    % "successor"
    There is a function \(\Sigma: \Gamma \to \Gamma \setminus \set{\zeta}\) that
    is a bijection, and satisfies the following:
   \item
    % "induction"
    If \(\BigIota\) is a set such that \(\zeta \in \BigIota\), and
    \(\Forall \gamma \in \Gamma :
      \gamma \in \BigIota
      \implies \Sigma(\gamma) \in \BigIota\),
    then \(\Gamma \subseteq \BigIota\).
  \end{itemize}
  Now we also define an \emph{operation}\footnote{
   An operation on a set \(S\) is nothing but a function from \(S^2 \to S\).
  } on \(\Gamma\) as follows:
  % "addition"
  \begin{align*}
   \alpha: \Gamma^2 &\to \Gamma \\
   (\gamma_1, \gamma_2) &\mapsto
   \begin{cases*}
    \gamma_1 & if \(\gamma_2 = \zeta\) \\
    \Sigma(\alpha(\gamma_1, \gamma')) & if \(\gamma_2 = \Sigma(\gamma')\)
   \end{cases*}
  \end{align*}
  Here is a proof that \(\alpha\) is well-defined for all
  \(\gamma_1, \gamma_2 \in \Gamma\):
  \begin{tcolorbox}
   Let \(\gamma_1\) be an arbitrary element of \(\Gamma\). Consider the
   following subset of \(\Gamma\):
   \begin{equation*}
    \BigIota
    = \set{\gamma \in \Gamma :
    \text{\(\alpha(\gamma_1, \gamma)\) is well-defined}}
   \end{equation*}
   Note firstly that \(\alpha(\gamma_1, \zeta)\) is well-defined, as it is an
   explicitly mentioned case that is equal to \(\gamma_1\). So
   \(\zeta \in \BigIota\).

   Note also that if some \(\gamma_2 \in \BigIota\), then
   \(\alpha(\gamma_1, \Sigma(\gamma_2)) = \Sigma(\alpha(\gamma_1, \gamma_2))\)
   by definition, and is therefore well-defined, so then also
   \(\Sigma(\gamma_2) \in \BigIota\). But then by the third property of
   \(\Gamma\), \(\Gamma \subseteq \BigIota\). Therefore, \(\Gamma = \BigIota\)
   and hence \(\alpha(\gamma_1, \gamma_2)\) is well-defined for all
   \(\gamma_1, \gamma_2 \in \Gamma\).
  \end{tcolorbox}
  Prove that \(\Forall \gamma \in \Gamma : \alpha(\zeta, \gamma) = \gamma\).

  Do this by considering the set of all elements \(\gamma \in \Gamma\) such that
  \(\alpha(\zeta, \gamma) = \gamma\).
 \item
  Prove that
  \(\forall \gamma_1, \gamma_2 \in \Gamma :
    \alpha(\Sigma(\gamma_1), \gamma_2) = \Sigma(\alpha(\gamma_1, \gamma_2))\).

  Do this by letting \(\gamma_1\) be arbitrary, and considering the set of all
  elements \(\gamma \in \Gamma\) such that
  \(\alpha(\Sigma(\gamma_1), \gamma) = \Sigma(\alpha(\gamma_1, \gamma))\).
 \item
  Prove that
  \(\Forall \gamma_1, \gamma_2, \gamma_3 \in \Gamma :
    \alpha(\alpha(\gamma_1, \gamma_2), \gamma_3) =
    \alpha(\gamma_1, \alpha(\gamma_2, \gamma_3))\).

  Do this by letting \(\gamma_1, \gamma_2\) be arbitrary and considering the set
  of all elements \(\gamma \in \Gamma\) such that
  \(\alpha(\alpha(\gamma_1, \gamma_2), \gamma) =
    \alpha(\gamma_1, \alpha(\gamma_2, \gamma))\).
 \item
  Hence show that
  \(\Forall \gamma_1, \gamma_2, \gamma_3, \gamma_4 \in \Gamma :
    \alpha(\gamma_1, \alpha(\gamma_2, \alpha(\gamma_3, \gamma_4))) =
    \alpha(\alpha(\alpha(\gamma_1, \gamma_2), \gamma_3), \gamma_4)\).

  (Do this without considering any more sets)
 \item
  Prove that
  \(\Forall \gamma_1, \gamma_2 \in \Gamma :
    \alpha(\gamma_1, \gamma_2) = \alpha(\gamma_2, \gamma_1)\).

  Do this by letting \(\gamma_1\) be arbitrary and considering the set of all
  elements \(\gamma \in \Gamma\) such that
  \(\alpha(\gamma_1, \gamma) = \alpha(\gamma, \gamma_1)\). Remember that you
  have already proved a number of useful results about \(\alpha\).
 \item
  Write down an expression for
  \(\alpha(\Sigma(\Sigma(\zeta)), \Sigma(\Sigma(\Sigma(\Sigma(\zeta)))))\) that
  does not involve \(\alpha\).

  Can you summarise everything that you have shown about \(\alpha\)? What do
  you think \(\alpha, \zeta, \Gamma, \Sigma\) represent? What does the property
  involving \(\BigIota\) represent?

  It is possible to walk through a similar process to define even more useful
  operations on and properties of \(\Gamma\). If you're interested, there is
  more to be found at this website:\\
  \url{uggc://jjjs.vzcrevny.np.hx/~ohmmneq/kran/angheny_ahzore_tnzr/} (ROT13
  employed to avoid spoilers in the name of the URL).
 \item
  We say two sets have the same size if there exists a bijection between them.

  Show from the definition that the sets \(\set{1, 2, 3, 4, 5, 6}\)
  and \(\set{0.1, -\pi, e, i, 3.14158, 0}\) have the same size.
 \item
  Prove from the definition that \(\Integers\) and \(\Naturals\) have the same
  size.
 \item
  Prove from the definition that
  \(\Forall a, b \in \Reals : a < b \implies
    \text{``\(\intcc{a, b}\) has the same size as \(\intcc{0, 1}\)''}\).
 \item
  Why is the following proof incorrect?
  \begin{tcolorbox}
   There is a function \(\sin : \Reals \to \intcc{-1, 1}\). Therefore
   \(\Reals\) and \(\intcc{-1, 1}\) have the same size.
  \end{tcolorbox}
  Can you give a correct proof from the definition that \(\Reals\) and
  \(\intcc{-1, 1}\) have the same size?
 \item
  Prove from the definition that \(\intcc{0, 1}\) and \(\intco{0, 1}\) have the
  same size.
 \item
  The \emph{Cantor-Schr\"oder-Bernstein Theorem} states that if \(A\) and \(B\)
  are sets, and there exists an injection from \(A\) to \(B\), and there exists
  an injection from \(B\) to \(A\), then there exists a bijection between
  \(A\) and \(B\)\footnote{
   The author will be happy to provide a proof of this fact.
  }. Use this to find a shorter proof that \(\intcc{0, 1}\) and \(\intco{0, 1}\)
  have the same size.
 \item
  Also use this to prove that \(\Rationals\) and \(\Naturals\) have the same
  size.
 \item
  Let \(\abs S\) denote ``the number of elements of the set \(S\)''.

  Let \(S_1, S_2, \dotsc S_n\) be finite sets. Write down formulae for:
  \begin{itemize}
   \item
    \(\abs{S_1 \cap S_2}\)
   \item
    \(\abs{S_1 \cap S_2 \cap S_3}\)
   \item
    \(\abs{S_1 \cap S_2 \cap S_3 \cap S_4}\)
  \end{itemize}
  Guess a formula for \(\abs{S_1 \cap S_2 \cap \dotsb \cap S_n}\) and prove its
  correctness by induction.
\end{enumerate}
