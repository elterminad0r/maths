\section{Functions and Sets}

\begin{enumerate}
 \item
  Write down the set that is a subset of every set.
 \item
  Use quantifiers to write down all the properties that a subset
  \(R \subseteq A \times B\) must satisfy for it to correspond to a function
  from \(A \to B\).
  % ∀ (a₁, b₁) ∈ R : ∀ (a₂, b₂) ∈ R : a₁ ≠ a₂
  % ∀ a ∈ A : ∃ (a', b') ∈ R : a = a'

  Write down the conditions for such a subset to correspond to an injective and
  a surjective function, respectively.
  % ∀ (a₁, b₁) ∈ R : ∀ (a₂, b₂) ∈ R : a₁ ≠ a₂ ⇒ b₁ ≠ b₂
  % ∀ b ∈ B : ∃ (a', b') ∈ R : b = b'
 \item
  State a necessary and sufficient condition on \(A\) and \(B\) for
  \(A \times B\) to correspond to a function from \(A \to B\).
  % A is the empty set, or |B| = 1
 \item
  Show that if \(A\) and \(B\) are sets, and \(A \subseteq B\) and
  \(B \subseteq A\), then \(A = B\).
 \item
  Show that if \(A\), \(B\), and \(C\) are sets, and
  \(f : A \to B\) and \(g : B \to C\) are bijections, then there is a bijection
  from \(A \to C\).
 \item
  If \(f: A \to B\) and \(g: B \to A\) are functions such that
  \begin{equation*}
   g(f(a)) = a \Forall a \in A
  \end{equation*}
  show that \(f\) is injective and \(g\) is surjective.

  Give examples to show that \(f\) does not have to be surjective, and \(g\)
  does not have to be injective.

  If \(f\) \emph{is} surjective, show that
  \begin{equation*}
   f(g(b)) = b \Forall b \in B
  \end{equation*}
  and deduce that \(g\) must be injective.
 \item
  Hence find the number of fractions of the form \(1 / n\) where
  \(n \in {1, \dotsc, 1000}\) that have a terminating radical expansion in base
  30.
 \item
  An operation on a set \(S\) is a function from \(S^2 \to S\).

  An operation \(f\) on a set \(S\) is called \emph{commutative} iff
  \begin{equation*}
   f(a, b) = f(b, a) \Forall a, b \in S
  \end{equation*}
  and \emph{associative} iff
  \begin{equation*}
   f(f(a, b), c) = f(a, f(b, c)) \Forall a, b, c \in S
  \end{equation*}
  Write down three associative commutative operations on \(\Integers\).
  % a * b = a + b
  % a * b = a · b
  % a * b = 0

 \item \label{q_functions_op_biject}
  Let \(S, S'\) be sets, \(f\) be an operation on \(S'\) and
  \(g\) be a bijection from \(S \to S'\), and \(h\) be an operation on \(S\)
  defined as \(h(a, b) \defeq g^{-1}(f(g(a), g(b)))\).

  Show that \(f\) is associative if and only if \(h\) is associative.
  % ⇒ implication given by
  % h(a, h(b, c)) = g^-1(f(g(a), g(g^-1(f(g(b), g(c))))))
  %               = g^-1(f(g(a), f(g(b), g(c))))
  %               = g^-1(f(f(g(a), g(b)), g(c)))
  %               = g^-1(f(g(g^-1(f(g(a), g(b)))), g(c)))
  %               = h(h(a, b), c)
  % and ⇐ implication given by noting that
  % f(a, b) = g(h(g^-1(a), g^-1(b))),
  % so you can recycle the argument.

  Show that \(f\) is commutative if and only if \(h\) is commutative.
  % similar process
 \item
  Can you think of a commutative non-associative operation on \(\Integers\)?
  % eg a * b := 2(a + b)

  How about a non-commutative associative operation on \(\Integers\)?
  % eg a * b := a
  %
  % Also, on ℕ, concatenation is such an operation. This operation can then be
  % applied to ℤ via question \ref{q_functions_op_biject}. This is already less
  % trivial, and somewhat suggests the following:
  %
  % Consider the question "construct an operation on ℤ that makes ℤ into a
  % non-abelian group". This is in some sense more interesting, because it means
  % that the operation will be less "trivial" than the easy answer above. It is
  % certainly a stronger condition, but also induces a certain way of thinking
  % that makes this question suddenly much more straightforward, if you know a
  % little group theory.
  %
  % GL₂(ℚ) is a countably infinite, non-abelian group, since
  % [1 1][1 0]   [2 1]
  % [   ][   ] = [   ]
  % [0 1][1 1]   [1 1]
  % but
  % [1 0][1 1]   [1 1]
  % [   ][   ] = [   ]
  % [1 1][0 1]   [1 2]
  % Since both GL₂(ℚ) and ℤ are countably infinite, there exists a bijection
  % between them. Then simply define
  % a * b = AB
  % where A, B are the matrices associated with a, b resp, and we mean to take
  % the number associated with the matrix AB.
 \item
  Find an operation \(f\) on \(\Naturals_0\), satisfying the following axioms:
  % one way is to say "think of the even numbers as the nonnegative integers,
  % and the odd numbers as the negative integers. Then just use addition as
  % defined on the integers."
  %
  % Alternatively, "bitwise XOR" also works.
  \begin{enumerate}
   \item
    \(f\) is an associative operation on \(\Naturals_0\)
   \item \label{set_gp_axiom}
    \(\Exists e \in \Naturals_0 : \Forall a \in \Naturals_0 : f(a, e) = a\)
   \item
    \(\Forall n \in \Naturals_0 : \Exists m \in \Naturals_0 : f(n, m) = e\),
    where \(e\) is an element of \(\Naturals_0\) satisfying Axiom
    \ref{set_gp_axiom}.
  \end{enumerate}
  The satisfaction of these axioms means that \((\Naturals_0, f)\) is a
  \emph{group}.

  Find an operation that makes the set of naturals congruent to
  \(1\) modulo \(3\)\footnote{
   written \(3\Naturals_0 + 1 \defeq \set{3n + 1 : n \in \Naturals_0}\)
  } a group.
  % biject them with N_0
 \item
  We say two sets have the same size if there exists a bijection between them.

  Show from the definition that the sets \(\set{1, 2, 3, 4, 5, 6}\)
  and \(\set{0.1, -\pi, \frac 12 (1 + \sqrt 5), i, 3.14158, 0}\) have the same
  size.
  % define case-by-case
 \item
  Prove from the definition that \(\Integers\) and \(\Naturals\) have the same
  size.
  % some bijection like even-odd positive-negative as above.
 \item
  Prove from the definition that
  \(\Forall a, b \in \Reals : a < b \implies
    \text{``\(\intcc{a, b}\) has the same size as \(\intcc{0, 1}\)''}\).
  % linear map bijecting
 \item
  Why is the following proof incorrect?
  \begin{tcolorbox}
   There is a function \(\sin : \Reals \to \intcc{-1, 1}\). Therefore
   \(\Reals\) and \(\intcc{-1, 1}\) have the same size.
  \end{tcolorbox}
  Can you give a correct proof from the definition that \(\Reals\) and
  \(\intcc{-1, 1}\) have the same size?
  % x / (|x| + 1) is a nice function for this.
 \item
  Prove from the definition that \(\intcc{0, 1}\) and \(\intco{0, 1}\) have the
  same size.
  % map 2^n to 2^(n - 1) where n ∈ ℤ, otherwise identity map
 \item
  The \emph{Cantor-Schr\"oder-Bernstein Theorem} states that if \(A\) and \(B\)
  are sets, and there exists an injection from \(A\) to \(B\), and there exists
  an injection from \(B\) to \(A\), then there exists a bijection between
  \(A\) and \(B\)\footnote{
   The author will be happy to provide a proof of this fact.
  }. Use this to find a simpler proof that \(\intcc{0, 1}\) and \(\intco{0, 1}\)
  have the same size.
  % x maps to x / 2 is an injection both ways
 \item
  Also use this to prove that \(\Rationals\) and \(\Naturals\) have the same
  size.
  % p / q (in lowest terms) maps to 2^p 3^q, and inclusion map
 \item
  Can you show that for all \(a, b \in \Reals\) with \(a < b\),
  \(\intcc{a, b} \cap \Rationals\) has the same size as \(\Naturals\)?
 \item
  % "getallen"
  Let \(\Gamma\) be a set with the following properties:
  \begin{itemize}
   \item
    % "zero"
    There is an element \(\zeta \in \Gamma\).
   \item
    % "successor"
    There is a function \(\Sigma: \Gamma \to \Gamma \setminus \set{\zeta}\) that
    is a bijection, and satisfies the following:
   \item
    % "induction"
    If \(\BigIota\) is a set such that \(\zeta \in \BigIota\), and
    \(\Forall \gamma \in \Gamma :
      \gamma \in \BigIota
      \implies \Sigma(\gamma) \in \BigIota\),
    then \(\Gamma \subseteq \BigIota\).
  \end{itemize}
  Now we also define an operation  on \(\Gamma\) as follows:
  % "addition"
  \begin{align*}
   \alpha: \Gamma^2 &\to \Gamma \\
   (\gamma_1, \gamma_2) &\mapsto
   \begin{cases*}
    \gamma_1 & if \(\gamma_2 = \zeta\) \\
    \Sigma(\alpha(\gamma_1, \gamma')) & if \(\gamma_2 = \Sigma(\gamma')\)
   \end{cases*}
  \end{align*}
  Here is a proof that \(\alpha\) is well-defined for all
  \(\gamma_1, \gamma_2 \in \Gamma\):
  \begin{tcolorbox}
   Let \(\gamma_1\) be an arbitrary element of \(\Gamma\). Consider the
   following subset of \(\Gamma\):
   \begin{equation*}
    \BigIota
    = \set{\gamma \in \Gamma :
    \text{\(\alpha(\gamma_1, \gamma)\) is well-defined}}
   \end{equation*}
   Note firstly that \(\alpha(\gamma_1, \zeta)\) is well-defined, as it is an
   explicitly mentioned case that is equal to \(\gamma_1\). So
   \(\zeta \in \BigIota\).

   Note also that if some \(\gamma_2 \in \BigIota\), then
   \(\alpha(\gamma_1, \Sigma(\gamma_2)) = \Sigma(\alpha(\gamma_1, \gamma_2))\)
   by definition, and is therefore well-defined, so then also
   \(\Sigma(\gamma_2) \in \BigIota\). But then by the third property of
   \(\Gamma\), \(\Gamma \subseteq \BigIota\). Therefore, \(\Gamma = \BigIota\)
   and hence \(\alpha(\gamma_1, \gamma_2)\) is well-defined for all
   \(\gamma_1, \gamma_2 \in \Gamma\).
  \end{tcolorbox}
  Prove that \(\Forall \gamma \in \Gamma : \alpha(\zeta, \gamma) = \gamma\).

  Do this by considering the set of all elements \(\gamma \in \Gamma\) such that
  \(\alpha(\zeta, \gamma) = \gamma\).
  % α(ζ, ζ) = ζ
  % α(ζ, Σ(γ)) = Σ(α(ζ, γ))
  %            = Σ(γ)
 \item
  Prove that
  \(\Forall \gamma_1, \gamma_2 \in \Gamma :
    \alpha(\Sigma(\gamma_1), \gamma_2) = \Sigma(\alpha(\gamma_1, \gamma_2))\).

  Do this by letting \(\gamma_1\) be arbitrary, and considering the set of all
  elements \(\gamma \in \Gamma\) such that
  \(\alpha(\Sigma(\gamma_1), \gamma) = \Sigma(\alpha(\gamma_1, \gamma))\).
  % α(Σ(γ₁), ζ) = Σ(γ₁) = Σ(α(γ₁, ζ))
  % α(Σ(γ₁), Σ(γ₂)) = Σ(α(Σ(γ₁), γ₂))
  %                 = Σ(Σ(α(γ₁, γ₂)))
  %                 = Σ(α(γ₁, Σ(γ₂))
 \item
  Prove that
  \(\Forall \gamma_1, \gamma_2, \gamma_3 \in \Gamma :
    \alpha(\alpha(\gamma_1, \gamma_2), \gamma_3) =
    \alpha(\gamma_1, \alpha(\gamma_2, \gamma_3))\).

  Do this by letting \(\gamma_1, \gamma_2\) be arbitrary and considering the set
  of all elements \(\gamma \in \Gamma\) such that
  \(\alpha(\alpha(\gamma_1, \gamma_2), \gamma)
  = \alpha(\gamma_1, \alpha(\gamma_2, \gamma))\).
  % α(α(γ₁, γ₂), ζ) = α(γ₁, γ₂)
  %                 = α(ζ, α(γ₁, γ₂))
  % α(α(γ₁, γ₂), Σ(γ₃)) = Σ(α(α(γ₁, γ₂), γ₃))
  %                     = Σ(α(γ₁, α(γ₂, γ₃)))
  %                     = α(γ₁, Σ(α(γ₂, γ₃)))
  %                     = α(γ₁, α(γ₂, Σ(γ₃)))
 \item
  Hence show that
  \(\Forall \gamma_1, \gamma_2, \gamma_3, \gamma_4 \in \Gamma :
    \alpha(\gamma_1, \alpha(\gamma_2, \alpha(\gamma_3, \gamma_4))) =
    \alpha(\alpha(\alpha(\gamma_1, \gamma_2), \gamma_3), \gamma_4)\).

  (Do this without considering any more sets)
  % α(γ₁, α(γ₂, α(γ₃, γ₄))) = α(α(γ₁, γ₂), α(γ₃, γ₄))
  %                         = α(α(α(γ₁, γ₂), γ₃), γ₄)
 \item
  Prove that
  \(\Forall \gamma_1, \gamma_2 \in \Gamma :
    \alpha(\gamma_1, \gamma_2) = \alpha(\gamma_2, \gamma_1)\).

  Do this by letting \(\gamma_1\) be arbitrary and considering the set of all
  elements \(\gamma \in \Gamma\) such that
  \(\alpha(\gamma_1, \gamma) = \alpha(\gamma, \gamma_1)\). Remember that you
  have already proved a number of useful results about \(\alpha\).
  % α(γ₁, ζ) = γ₁
  %          = α(ζ, γ₁)
  % α(γ₁, Σ(γ₂)) = Σ(α(γ₁, γ₂))
  %              = Σ(α(γ₂, γ₁))
  %              = α(Σ(γ)₂, γ₁)
 \item
  Write down an expression for
  \(\alpha(\Sigma(\Sigma(\zeta)), \Sigma(\Sigma(\Sigma(\Sigma(\zeta)))))\) that
  does not involve \(\alpha\).

  Can you summarise everything that you have shown about \(\alpha\)? What do
  you think \(\alpha, \zeta, \Gamma, \Sigma\) represent? What does the property
  involving \(\BigIota\) represent?

  It is possible to walk through a similar process to define even more useful
  operations on and properties of \(\Gamma\). If you're interested, there is
  more to be found at this website:\\
  \url{uggc://jjjs.vzcrevny.np.hx/~ohmmneq/kran/angheny_ahzore_tnzr/} (ROT13
  employed to avoid spoilers in the name of the URL).
\end{enumerate}
