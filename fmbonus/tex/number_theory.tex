\section{Numbers}

\begin{enumerate}
 \item
  If \(n \in \Naturals\), \(x^n\) is defined inductively as follows:
  \(x^1 \defeq x\) and \(x^{n + 1} \defeq x \cdot x^n\).

  Prove by induction on \(b\) that for \(a, b \in \Naturals\),
  \(x^{a + b} \equiv x^a x^b\).
  % x^(a+1) = x^a x^1 ✓
  % x^(a + b + 1) = x x^(a + b)
  %               = x x^a x^b
  %               = x^a x^(b + 1)

  You may assume commutativity and associativity of multiplication.
 \item
  Prove by induction on \(n\) that \((xy)^n \equiv x^n y^n\).
  % (xy)^1 = xy
  % (xy)^(n + 1) = xy (xy)^n
  %              = xy x^n y^n
  %              = x x^n y y^n
  %              = x^(n + 1) y^(n + 1)
 \item
  Show that if \(a, b \in \Naturals\), then
  \((x^a)^n \equiv x^{an}\) by induction on \(n\).
  % (x^a)^1 = x^(a · 1) ✓
  % (x^a)^(n + 1) = x^a x^(an)
  %               = x^(an + a)
  %               = x^(a(n + 1))

  Deduce that \((x^a)^n \equiv (x^n)^a\).
 \item
  Furthermore, we define \(x^0\) to be \(1\) and
  \(x^{-n}\) to be the multiplicative inverse of \(x^n\). It can be shown that
  the above identities still hold, by working through some cases, but that's a
  waste of everyone's time. Let's assume it.

  Now for \(x \in \Reals^+\)\footnote{
   Note that everything up til now only needs \(x\) and \(y\) to have some
   commutative associative multiplication defined on them.
  } and \(p/q \in \Rationals\), we define \(x^{p/q}\), as ``the positive real
  number \(t\) such that \(t^q = x^p\)''.

  It can be shown that there is always exactly one such number, but that
  requires defining the reals first, which is a \(\Reals\text{oyal}\) pain.

  We're just not defining it on the negative numbers since you can only define
  it some of the time, and then you have to start worrying about whether or not
  \(p/q\) is in lowest terms, and it's basically just annoying, and it's hardly
  even useful.

  Prove that for \(x \in \Reals^+\) and \(a/b, c/d \in \Rationals\),
  \(x^{a/b} x^{c/d} \equiv x^{(ad + bc)/bd}\).
  % (x^(a/b) x^(c/d))^(bd) = x^ad x^bc ✓

  Also show that \((x^{a/b})^{c/d} \equiv x^{ac/bd}\).
  % ((x^(a/b))^(c/d))^bd = (x^(a/b))^(bc)
  %                      = x^(ac)

  Also show that \((xy)^{a/b} \equiv x^{a/b} y^{a/b}\).
  % (x^(a/b) y^(a/b))^b = x^a y^a
  %                     = (xy)^a

  (You need not use induction for any of these.)
 \item
  State the Binomial Theorem
  (the expansion of \((a + b)^n\) for any \(n \in \Naturals\)), and define the
  general binomial coefficient.

  Prove that
  \begin{equation*}
   \binom nr + \binom n{r + 1} \equiv \binom{n + 1}{r + 1}
  \end{equation*}
  Hence prove the Binomial Theorem by induction.
 \item
  Prove that \(\Forall x, y \in \Reals : \abs{x + y} \le \abs x + \abs y\), by
  exhausting the different possible signs and magnitudes of \(x\) and \(y\).

  This is called the \emph{triangle inequality}.
 \item
  Let \(y = z - x\) in the triangle inequality to show that
  \(\abs{z - x} \ge \abs z - \abs x\). Also show that
  \(\abs{z - x} \ge \abs x - \abs z\), and hence deduce that
  \(\abs{z - x} \ge \abs[\big]{\abs z - \abs x}\).
 \item
  Let \(a, d, r \in \Reals\) be constants and let \(n \in \Naturals\).
  Guess and prove by induction formulas for
  \begin{align*}
   &\sum_{i = 1}^n \bracks{a + (i - 1)d} \\
   \text{and}\quad&\sum_{i = 1}^n ar^{i - 1}
  \end{align*}
  State for which (if any) values of \(a, d, r\) these formulas fail, and find
  formulas that work in these cases.

  These are known as \emph{arithmetic progressions} and
  \emph{geometric progressions}, respectively.
 \item
  Consider the sum
  \begin{equation*}
   \alpha^n + \alpha^{n - 1}\beta + \alpha^{n - 2}\beta^2 + \dotsb + \beta^n
  \end{equation*}
  Write this as a geometric progression, explicitly stating your values of
  \(a\), \(r\), and \(n\). Hence find and simplify a closed form expression for
  the sum.

  Hence state a formula for the factorisation of the difference of \(n\)th
  powers.
 \item
  For \(n\) odd, by writing \(\beta^n = -(-\beta)^n\), state a formula for the
  factorisation of the sum of powers.
 \item
  Factorise \(\alpha^{12} - \beta^{12}\) into 6 brackets, each with only integer
  coefficients.
 \item
  Find all solutions in \(\Reals^+\) to
  \(\sqrt[\sqrt x]x = {\sqrt x}^{\sqrt x}\).

  Strictly speaking, this question is being a bit silly because I haven't
  defined what most of this means yet. Once you have done Calculus you should be
  able to come up with a more systematic approach, but for now just think of
  this question as a fun little curiosity.
 \item
  Briefly explain why
  \begin{alignat*}3
   \Forall n \in \Naturals :
   \Forall a, b \in \Integers &:{}
   &ab&\equiv ba &&\pmod n \\
   \Forall n \in \Naturals :
   \Forall a, b \in \Integers &:{}
   &a + b&\equiv b + a &&\pmod n
  \end{alignat*}
 \item
  Prove using the definition of \(\equiv\) that
  \begin{equation*}
   \Forall n \in \Naturals :
   \Forall a, b, c \in \Integers :
   b \equiv c \pmod n \implies
   a + b \equiv a + c \pmod n
  \end{equation*}
 \item
  Prove using the definition of \(\equiv\) that
  \begin{equation*}
   \Forall n \in \Naturals :
   \Forall a, b, c \in \Integers :
   b \equiv c \pmod n \implies
   ab \equiv ac \pmod n
  \end{equation*}
 \item
  Prove by induction on \(n\) that
  \begin{equation*}
    \Forall m \in \Naturals :
    \Forall a, b \in \Integers :
    \Forall n \in \Naturals :
    a \equiv b \pmod m \implies
    a^n \equiv b^n \pmod m
  \end{equation*}
 \item
  Hence show that \(7^{4n} \equiv 1 \pmod{10} \Forall n \in \Naturals\), and
  \(7^{2n} \equiv 1 \pmod 4 \Forall n \in \Naturals\). Use
  this to find the last digit of
  \begin{equation*}
   {}^4 7 \defeq 7^{7^{7^7}} = 7^{(7^{(7^7)})}
  \end{equation*}
 \item
  Prove the ``Freshman's Dream'' theorem:

  Let \(p\) be a prime, and let \(a, b \in \Integers\). Then
  \begin{equation*}
   (a + b)^p \equiv a^p + b^p \pmod p
  \end{equation*}
 \item
  Hence prove \emph{Fermat's Little Theorem}:

  Let \(p\) be a prime, and \(a \in \Naturals\). Then
  \begin{equation*}
   a^p \equiv a \pmod p
  \end{equation*}
  by induction on \(a\).
 \item
  Prove that, for all \(n \in \Naturals\), for all \(a, b, c \in \Integers\),
  \begin{itemize}
   \item
    \(a \equiv a \pmod n\)
   \item
    \(a \equiv b \pmod n \implies b \equiv a \pmod n\)
   \item
    \(\text{``\(a \equiv b \pmod n\) and \(b \equiv c \pmod n\)''} \implies
      a \equiv c \pmod n\)
  \end{itemize}
  These three properties are \emph{reflexivity}, \emph{symmetry}, and
  \emph{transitivity}, respectively. Together, they mean that \(\equiv\) is an
  \emph{equivalence relation}.
 \item
  Let \(u_i, v_i \in \Reals\) be two finite sequences of real numbers, for
  \(i \in \set{1, 2, \dotsc, n}\) (ie both have length \(n\)).

  By considering the number of possible roots of the quadratic
  \begin{equation*}
   p(x) = \sum_{i = 1}^n (u_i x - v_i)^2
  \end{equation*}
  show that
  \begin{equation*}
   \parens[\Big]{\sum_{i = 1}^n u_i v_i}^2
   \le \parens[\Big]{\sum_{i = 1}^n u_i^2}
       \parens[\Big]{\sum_{i = 1}^n v_i^2}
  \end{equation*}
  and state the condition for equality to hold.

  This is called the \emph{Cauchy-Schwarz inequality}.
 \item
  Let \(p_i\), \(i = 1, \dotsc, 6\) be the probability that a biased die rolls
  the value \(i\). Use the Cauchy-Schwarz inequality to show that the
  probability of rolling the same value twice in two rolls cannot be made lower
  than \(\frac 16\). Hence show that there is only one possible set of
  probabilities for which this chance is exactly \(\frac 16\).
 \item
  For nonnegative real numbers \(x_i \ge 0\), \(i = 1, \dotsc, n\), define
  \begin{align*}
   M_1'(x_1, \dotsc, x_n) &\defeq
   \frac 1n \sum_{k = 0}^n x_k \\
   M_0'(x_1, \dotsc, x_n) &\defeq
   \parens[\Big]{
    \prod_{k = 0}^n x_k
   }^{\frac 1n}
  \end{align*}
  Prove that for all \(x_1, x_2 \in \Reals_0^+\),
  \begin{equation*}
   M_1'(x_1, x_2) \ge M_0'(x_1, x_2)
  \end{equation*}
  and state the condition for equality to hold.
 \item
  Show that if \(x_i, y_i \ge 0\), \(i = 1, \dotsc, n\), then
  \begin{align*}
   M_1'(M_1'(x_1, \dotsc, x_n), M_1'(y_1, \dotsc, x_n)) &=
   M_1'(x_1, \dotsc, x_n, y_1, \dotsc, y_n) \\
   \text{and}\quad
   M_0'(M_0'(x_1, \dotsc, x_n), M_0'(y_1, \dotsc, x_n)) &=
   M_0'(x_1, \dotsc, x_n, y_1, \dotsc, y_n)
  \end{align*}
  and hence deduce that if for some \(n \in \Naturals\),
  \begin{equation*}
   M_1'(x_1, \dotsc, x_n) \ge M_0'(x_1, \dotsc, x_n)
    \Forall x_1, \dotsc, x_n \in \Reals_0^+
  \end{equation*}
  then
  \begin{equation*}
   M_1'(x_1, \dotsc, x_{2n}) \ge M_0'(x_1, \dotsc, x_{2n})
    \Forall x_1, \dotsc, x_{2n} \in \Reals_0^+
  \end{equation*}
 \item
  Show that if \(x_i \ge 0\), \(i = 1, \dotsc, n\), then
  \begin{align*}
   M_1'(M_1'(x_1, \dotsc, x_n), x_1, \dotsc, x_n) &=
   M_1'(x_1, \dotsc, x_n) \\
   \text{and}\quad
   M_0'(M_0'(x_1, \dotsc, x_n), x_1, \dotsc, x_n) &=
   M_0'(x_1, \dotsc, x_n)
  \end{align*}
  and hence deduce that if for some \(n \in \Naturals_{\ge 2}\),
  \begin{equation*}
   M_1'(x_1, \dotsc, x_n) \ge M_0'(x_1, \dotsc, x_n)
    \Forall x_1, \dotsc, x_n \in \Reals_0^+
  \end{equation*}
  then
  \begin{equation*}
   M_1'(x_1, \dotsc, x_{n - 1}) \ge M_0'(x_1, \dotsc, x_{n - 1})
    \Forall x_1, \dotsc, x_{n - 1} \in \Reals_0^+
  \end{equation*}
 \item
  Prove by induction that
  \begin{equation*}
   M_1'(x_1, \dotsc, x_n) \ge M_0'(x_1, \dotsc, x_n)
    \Forall x_1, \dotsc, x_n \in \Reals_0^+
  \end{equation*}
  for all \(n\) of the form \(2^m\) for some \(m \in \Naturals\).

  Carefully argue what the condition for equality will be for each \(n\).

  Explain why we can now deduce that
  \begin{equation*}
   M_1'(x_1, \dotsc, x_n) \ge M_0'(x_1, \dotsc, x_n)
    \Forall x_1, \dotsc, x_n \in \Reals_0^+
  \end{equation*}
  for all \(n \in \Naturals\). This inequality is often referred to as the
  \emph{arithmetic mean-geometric mean} inequality, or ``AM-GM''.
 \item
  Use AM-GM to find the minimum value of
  \begin{equation*}
   \frac xy + \frac{2y}z + \frac{4z}x
  \end{equation*}
  where \(x, y, z \in \Reals^+\), stating values of \(x, y, z\) that attain this
  minimum.

  Also find the maximum value of
  \begin{equation*}
   (1 + x - y)(2 + y - z)(3 + z - x)
  \end{equation*}
  where each bracket is positive, stating values of \(x, y, z\) that attain this
  maximum.

  If we only require that \(x, y, z \in \Reals^+\), show that there is no
  maximum.
  % let z = 1, x = 5.
  % Then have (6 - y)(2 + y - 1)(3 + 1 - 5)
  %         = (y - 6)(y + 1)
  % which grows arbitrarily large as y → ∞
\end{enumerate}
