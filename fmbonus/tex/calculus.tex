\section{Calculus}

Throughout, you may assume the following properties of a limit:

\begin{tcolorbox}
 For nice functions \(r, s : \Reals \to \Reals\),
 \begin{itemize}
  \item
   If \(\lim_{t \to a} r(t) = b\) and \(\lim_{t \to a} s(t) = c\), then
   \(\lim_{t \to a} \bracks{r(t) + s(t)} = b + c\).
  \item
   If \(\lim_{t \to a} r(t) = b\) and \(\lim_{t \to a} s(t) = c\), then
   \(\lim_{t \to a} \bracks{r(t) s(t)} = bc\).
  \item
   If \(\lim_{t \to a} s(t) = b\) and \(\lim_{t \to b} r(t) = c\), then
   \(\lim_{t \to a} r(s(t)) = c\).
 \end{itemize}
 Here when we say \(\lim_{\ldots} \dotsb = \ell\) we mean that
 the limit exists, is finite, and is equal to \(\ell\).
\end{tcolorbox}

\begin{enumerate}
 \item
  Prove from first principles that, if
  \(a\) is a constant and \(f, g\) are differentiable functions,
  \begin{alignat*}2
   && \dv{x}(af(x)) &= af'(x) \\
   &\text{and}\quad&\dv{x}(f(x) + g(x)) &= f'(x) + g'(x)
  \end{alignat*}
  This means that differentiation is a \emph{linear operator}.
 \item
  Use the Binomial Theorem and the first-principles definition of the derivative
  to prove the \emph{power rule}
  \begin{equation*}
   \dv{x}(x^n) \equiv nx^{n - 1}
  \end{equation*}
  for all \(n \in \Naturals\).
 \item
  For dependent variables \(u = f(x)\), \(v = g(x)\) (where \(f\) and \(g\) are
  differentiable functions), this is the
  \emph{product rule} for differentiation:
  \begin{equation*}
   \dv{x}(uv) \equiv u'v + uv'
  \end{equation*}
  Prove the product rule from first principles, by choosing a suitable place to
  add and subtract a term \(f(x + h)g(x)\).
  %   lim [f(x + h)g(x + h) - f(x + h)g(x) + f(x + h)g(x) - f(x)g(x)] / h
  % = lim [f(x + h)(g(x + h) - g(x)) + g(x)(f(x + h) - f(x)] / h
  % = f(x) g'(x) + g(x) f'(x)
 \item
  Give an alternative proof of the power rule using induction and the product
  rule.
 \item
  For dependent variables \(u = f(x)\), \(v = g(x)\) (where \(f\) and \(g\) are
  infinitely differentiable functions), calculate and simplify expressions for
  \(\dv[2]{x}(uv)\), \(\dv[3]{x}(uv)\), \(\dv[4]{x}(uv)\), \(\dv[5]{x}(uv)\).
  Hence guess a formula for \(\dv[n]{x}(uv)\), for all \(n \in \Naturals\).
  Prove that it is correct by induction.

  (This is called the Leibniz rule, which generalises the product rule).
 \item
  Let \(f, g\) be differentiable functions. Let \(y = f(x) / g(x)\). Multiply
  both sides by \(g(x)\), differentiate both sides, and rearrange for
  \(\dv<y>{x}\) to obtain the \emph{quotient rule}:
  \begin{equation*}
   \dv{x}\parens[\Big]{\frac uv}
    \equiv \frac{vu' - uv'}{v^2}
  \end{equation*}
  This formula is probably worth memorising, but that will come through school.
  %          yv = u
  % ⇒ y'v + yv' = u'
  % ⇒        y' = (u' - yv') / v
  %             = (u' - uv'/v) / v
  %             = (u'v - uv') / v²
 \item
  Let \(n \in \Naturals\). Let \(y = x^{-n}\) (which is defined as \(1 / x^n\)).
  By multiplying both sides by \(x^n\) and then differentiating both sides, show
  that the power rule also holds for all \(n \in \Integers\).

  Use the quotient rule to demonstrate the same result instantly.
 \item
  If \(f\) is a differentiable function and \(k\) is a non-zero constant,
  write down the first-principles definition of \(\dv{x}(f(kx))\). By
  multiplying top and bottom by \(k\) and making an appropriate
  substitution\footnote{
   If \(t\) and \(t'\) are related variables so that
   \(t' \to 0 \implies t \to 0\), then
   \(\lim_{t \to 0} s(t) = \lim_{t' \to 0} s(t)\).
  }, show that it is equal to \(kf'(kx)\).

  Show that this also holds if \(k = 0\).
  %   lim_h [f(kx + kh) - f(kx)] / h
  % = lim_t [f(kx + t) - f(kx)] / (t / k)            | where t = kh
  % = kf'(kx)
 \item
  Use a similar proof to show that if \(f\) is a differentiable function and
  \(m\) and \(c\) are constants, then \(\dv{x}(f(mx + c)) \equiv mf'(mx + c)\).
 \item
  If \(g\) is a differentiable function, find the equation of the tangent to the
  graph \(y = g(x)\) at the point where \(x = x_0\), in the form
  \(y = mx + c\). By assuming that as \(x \to x_0\), \(g(x) \approx mx + c\),
  deduce the \emph{chain rule}:
  \begin{equation*}
   \dv{x}(f(g(x))) \equiv g'(x) f'(g(x))
  \end{equation*}
  Some people write it in the following form:
  \begin{equation*}
   \dv<y>{x} \equiv \dv<y>{u} \cdot \dv<u>{x}
  \end{equation*}
  where we let \(u = g(x)\) and \(y = f(u)\), to see that
  \(\dv<y>{x}
    = \dv{x}(u) \cdot f'(u)
    = \dv<u>{x} \cdot \dv{u}(y)\).

  This one is perhaps easier to remember, because it looks like you're just
  ``cancelling the \(\diff u\,\)s''. Of course, this is not formally what is
  happening as these are not fractions.
 \item
  Use the chain rule and the power rule to give yet another proof of the product
  rule, this time by writing \(y = uv = k[(u + v)^2 - (u - v)^2]\) for some
  constant \(k\).
 \item
  Let \(y = x^{p/q}\) where \(p, q \in \Integers, q \ne 0\). By raising both
  sides to the power \(q\) and then differentiating both sides (using the chain
  rule), show that the power rule also holds for all \(p / q \in \Rationals\)
  (when \(x > 0\)).
  %              y^q = x^p
  % ⇒ q y^(q - 1) y' = p x^(p - 1)
  %             ⇒ y' = (p / q) x^(p - 1)  y^(1 - q)
  %                  = (p / q) x^(p - 1)  x^((p / q)(1 - q))
  %                  = (p / q) x^((p / q)(1 - q) + p - 1)
  %                  = (p / q) x^((p(1 - q) + q(p - 1)) / q)
  %                  = (p / q) x^((p - q) / q)
  %                  = (p / q) x^((p / q) - 1)
 \item
  Let \(y = f^{-1}(x)\). Show that
  \begin{equation*}
   \dv<y>{x} = \frac 1{f'(y)}
  \end{equation*}
  % f(y) = x
  % f'(y) y' = 1
  % y' = 1 / f'(y)
  %    = 1 / (dx/dy)

  This fact is sometimes written as
  \begin{equation*}
   \dv<y>{x} = 1 \Big/ \dv<x>{y}
  \end{equation*}
  which is yet another ``fraction-like'' property of derivatives.
 \item
  If \(y = \sqrt x\)\footnote{
   \(\sqrt x\) is of course defined as \(x^{1/2}\).
  }, show that \(y \dv<y>{x}\) is constant.

  If \(y = \sqrt{x^3 + 1}\), calculate \(y \dv<y>{x}\) using the chain rule.

  In general, determine a formula for \(\sqrt{f(x)}\dv{x}(\sqrt{f(x)})\).

  Evaluate \(\dv{x}(y^2)\) using the chain rule, and compare.
 \item
  Let \(f\) a function that is differentiable at \(x_0\), and \(y = f(x)\) be a
  graph of that function.

  Let \(h\) be a constant. Find a formula for the gradient of the line through
  the points where \(x = x_0 - h\) and \(x = x_0 + h\).
 \item
  Use the following facts to find the derivative of \(\sin x\):
  \begin{itemize}
   \item
    \(\sin x\) is differentiable.
   \item
    If \(f\) is a function that is differentiable at \(x_0\), then, on the graph
    \(y = f(x)\), the gradient of the line through the points where
    \(x = x_0 - h\) and \(x = x_0 + h\) approaches the gradient of the tangent
    at \(x_0\) as \(h \to 0\).
   \item
    The compound angle formula for \(\sin\).
   \item
    \(\displaystyle
     \lim_{t \to 0} \frac{\sin t} t = 1
     \)
  \end{itemize}
 \item
  Hence use a trig identity and the chain rule to deduce the derivative of
  \(\cos x\), without evaluating any limits.
 \item
  Find the derivative of \(\tan x\). Write it in terms of \(\tan x\).
 \item \label{q_calc_arcsin}
  If \(y = \arcsin x\), then use implicit differentiation\footnote{
   This is the name of the trick where you apply a function to both sides, so
   you can use the chain rule.
  } to find
  \(\dv<y>{x}\) in terms of \(y\).

  Since \(-\frac \pi 2 \le y \le \frac \pi 2\), \(\cos y \ge 0\). Hence use the
  pythagorean identity to write the derivative in terms of \(\sin y\), and
  therefore write it in terms of \(x\).
  %    sin y = x
  % y' cos y = 1
  %       y' = 1 / cos y
  %          = 1 / sqrt(1 - x²)
 \item
  Use question \ref{q_trig_arcsin} from Trigonometry to immediately deduce the
  derivative of \(\arccos x\).
 \item
  Find by similar methods to question \ref{q_calc_arcsin} the derivative of
  \(\arctan x\).
 \item
  Show that if \(f\) and \(g\) are continuous functions and
  \(a, b, \alpha, \beta \in \Reals\) are constant, then
  \begin{equation*}
   \integ[a]<b>{(\alpha f(x) + \beta g(x))}{x} \equiv
   \alpha \integ[a]<b>{f(x)}{x} +
   \beta  \integ[a]<b>{g(x)}{x}
  \end{equation*}
  by using the Fundamental Theorem of Calculus and considering \(F, G\) such
  that \(F'(x) = f(x)\) and \(G'(x) = g(x)\), and letting
  \(L(x) \defeq \alpha F(x) + \beta G(x)\).
 \item
  Also use the Fundamental Theorem of Calculus to show that if \(f\) is
  continuous and \(a, b, c \in \Reals\) are constant, then
  \begin{equation*}
   \integ[a]<b>{f(x)}{x} + \integ[b]<c>{f(x)}{x} \equiv
   \integ[a]<c>{f(x)}{x}
  \end{equation*}
  This is not how these facts are normally proved (in fact, you would probably
  need them to prove the Theorem itself). If you think about areas, they should
  sound quite reasonable.
 \item \label{q_calc_int_pwr}
  If \(\alpha \in \Rationals\), use the power rule to work out
  \begin{equation*}
   \integ{x^\alpha}{x}
  \end{equation*}
  State the value of \(\alpha\) for which your answer breaks down.
 \item
  If \(f, g\) are continuous functions, show that
  \begin{equation*}
   \integ[a]<b>{g'(x) f(g(x))}{x} \equiv
   \integ[g(a)]<g(b)>{f(x)}{x}
  \end{equation*}
  The fraction-cancelling formulation of this fact is
  \begin{equation*}
   \integ[a]<b>{f(u) \dv<u>{x}}{x} \equiv
   \integ[g(a)]<g(b)>{f(u)}{u}
  \end{equation*}
  if \(u = g(x)\).

  This technique is called \emph{integration by substitution}.
 \item
  Hence evaluate
  \begin{itemize}
   \item \(\displaystyle
    \integ[a]<b>{f(kx)}{x}
    \) in terms of an antiderivative of \(f\),

    where \(k\) is constant. Convince yourself by means of a sketch that your
    answer is correct.
   \item \(\displaystyle
    \integ[0]<\sqrt \pi>{2x \sin(x^2)}{x}
    \)
   \item \(\displaystyle
    \integ{\frac x{\sqrt{x^2 + 1}}}{x}
    \)
  \end{itemize}
 \item
  Find the mistake in the reasoning:
  \begin{tcolorbox}
   Let
   \begin{equation*}
    I \defeq \integ[0]<\pi>{\sin x}{x}
   \end{equation*}
   This can be evaluated directly as \(\eval{0}{\pi}{-\cos x} = 2\).

   Also, the function \(\sin x\) can be written as \(g'(x) f(g(x))\) where
   \begin{align*}
    g(x) &\defeq \sin x \\
    f(x) &\defeq \frac x{\cos(\arcsin x)}
   \end{align*}
   so therefore
   \begin{equation*}
    I = \integ[\sin 0]<\sin \pi>{f(x)}{x} = 0
   \end{equation*}
   and we have \(0 = 2\).
  \end{tcolorbox}
 \item
  Abel and Bernstein are practising integration by substitution. They get the
  question:
  \begin{tcolorbox}
   Find
   \begin{equation*}
    \integ{\frac{2\tan x}{\cos^2 x}}{x}
   \end{equation*}
  \end{tcolorbox}
  Here is A's solution:
  \begin{tcolorbox}
   The integrand can be written in the form \(g'(x) f(g(x))\) where
   \begin{alignat*}2
    &&g(x) &\defeq \frac 1{\cos x} \\
    \implies{}&& g'(x) &=
     \frac{\sin x}{\cos^2 x} =
     \frac{\tan x}{\cos x} \\
    &&f(x) &\defeq 2x
   \end{alignat*}
   so the antiderivative is
   \begin{align*}
    \integ[x_0]<x>{\frac{2\tan t}{\cos^2 t}}{t}
     &= \integ[x_0']<1/\cos x>{2t}{t} \\
     &= \eval{x_0'}{1/\cos x}{t^2} \\
     &= \frac 1{\cos^2 x} + C
   \end{align*}
  \end{tcolorbox}
  Here is B's solution:
  \begin{tcolorbox}
   The integrand can be written in the form \(g'(x) f(g(x))\) where
   \begin{alignat*}2
    &&g(x) &\defeq \tan x \\
    \implies{}&& g'(x) &= \frac 1{\cos^2 x} \\
    &&f(x) &\defeq 2x
   \end{alignat*}
   so the antiderivative is
   \begin{align*}
    \integ[x_0]<x>{\frac{2\tan t}{\cos^2 t}}{t}
     &= \integ[x_0']<\tan x>{2t}{t} \\
     &= \eval{x_0'}{\tan x}{t^2} \\
     &= \tan^2 x + C
   \end{align*}
  \end{tcolorbox}
  Who is correct?
 \item
  If \(u\) and \(v\) are variables dependent on \(x\), use the product rule to
  show that
  \begin{equation*}
   \integ[a]<b>{uv'}{x} = \eval{a}{b}{uv} - \integ[a]<b>{u'v}{x}
  \end{equation*}
 \item
  Hence show that
  \begin{equation*}
   \lim_{t \to \infty}
    \bracks[\Big]{
     \integ[0]<t>{x^{n + 1} e^{-x}}{x}
    } =
   \lim_{t \to \infty}
    \bracks[\Big]{
     (n + 1)\integ[0]<t>{x^n e^{-x}}{x}
    }
  \end{equation*}
  and hence prove that for \(n \in \Naturals_0\),
  \begin{equation*}
   n! \equiv \lim_{t \to \infty}
    \bracks[\Big]{
     \integ[0]<t>{x^n e^{-x}}{x}
    }
  \end{equation*}
  This technique is called \emph{integration by parts}.
 \item
  This question is motivated by question \ref{q_calc_int_pwr}.

  Define
  \begin{align*}
   \ell : \Reals^+ &\to \Reals\\
   x &\mapsto \integ[1]<x>{\frac 1t}{t}
  \end{align*}
  By means of a substitution, show that for \(a, b \in \Reals^+\) constant,
  \begin{equation*}
   \ell(a) \equiv \integ[b]<ab>{\frac 1t}{t}
  \end{equation*}
  and deduce that \(\ell(a) + \ell(b) \equiv \ell(ab)\).
 \item
  By means of a different substitution, show that for \(a \in \Reals^+\) and
  \(n \in \Rationals\), \(n \cdot \ell(a) \equiv \ell(a^n)\).
 \item
  By rewriting \(-\ell(b)\), find a formula for
  \(\ell(a) - \ell(b)\) involving only one \(\ell\).
 \item
  Explain briefly why \(a < b \implies \ell(a) < \ell(b)\)\footnote{
   This means that \(\ell\) is \emph{monotonic increasing}.
  }.

  Draw a sketch to show that \(\frac 12 \le \ell(2) \le 1\). Hence show that
  \(\frac n2 \le \ell(2^n) \le n\) for \(n > 0\), and deduce that as
  \(x \to \infty\), \(\ell(x) \to \infty\).

  Also show that as\footnote{
   This notation means ``approaches from the right/from above''. This is
   necessary since we haven't defined \(\ell\) for \(x \le 0\).
  } \(x \to 0^+\), \(\ell(x) \to -\infty\).
 \item
  By means of a sketch, show that for \(n \in \Naturals\),
  \begin{equation*}
   \ell(n) \le \sum_{k = 1}^n \frac 1k \le 1 + \ell(n)
  \end{equation*}
  and deduce that the harmonic series (the above sum) does not converge.
 \item
  Since \(\ell\) is monotonic increasing, and it goes from \(-\infty\) and
  \(\infty\), it is a bijection\footnote{
   This fact is fairly easy to appreciate, but not so easy to prove. We will
   just assume it here.
  } between \(\Reals^+\) and \(\Reals\). Therefore
  it has an inverse, so let's define \(\xi: \Reals \to \Reals^+\) to be such
  that
  \begin{equation*}
   \ell(\xi(x)) = x \Forall x \in \Reals
  \end{equation*}
  For \(a, b \in \Reals\), show that \(\xi(a + b) \equiv \xi(a) \xi(b)\).

  Here you can use the identity defining \(\xi\) to write \(a\) and \(b\) in a
  different form.

  Also show that for \(n \in \Rationals\), \((\xi(a))^n \equiv \xi(an)\).
 \item
  Show that if \(t = \xi(\frac pq \cdot \ell(x))\) then \(t^q = x^p\).

  This means that \(\xi(n \cdot \ell(x))\) coincides with \(x^n\) for all
  rational \(n\) when \(x\) is positive. Since \(\xi\) and \(\ell\) are both
  quite nice, ``smooth'' functions, it is nice to \emph{define}
  \begin{equation*}
   x^t \defeq \xi(t \cdot \ell(x))
  \end{equation*}
  for \(t \in \Reals\) and \(x\) positive. This agrees with the old definitions,
  and ensures various nice continuity properties.

  If we \emph{define}\footnote{
   One should stress again that this is not the only definition of \(e\)
   (although any correct alternative definition should imply that this one is
   also true)
  } \(e \defeq \xi(1)\), then
  \(e^x \equiv \xi(x \cdot \ell(\xi(1))) \equiv \xi(x)\). So in fact \(\xi\) has
  been the famous ``exponential function'' all this time! This is one of the
  reasons that the number \(e\) is significant. From now on I will write \(e^x\)
  or \(\exp x\) instead of \(\xi(x)\), and \(\ln x\) or
  \(\log x\)\footnote{
   standing for ``natural logarithm'' (logarithmus naturalis) and ``logarithm'',
   respectively.
  }
  instead of \(\ell(x)\).
 \item
  If \(f\) is a positive, differentiable function, determine the derivative of
  \(\ln(f(x))\).
 \item
  Find the derivative of \(e^x\) (aka \(\xi(x)\)), using the definition of
  \(\ln x\) (aka \(\ell(x)\)).

  This property of \(e^x\) is an alternative way to define it. If it's defined
  this way, you can quickly deduce that its inverse function differentiates to
  \(1/x\) and work from there to deduce all the properties we've found.
 \item
  Hence find three different functions that are solutions to the
  \emph{differential equation}
  \begin{equation*}
   f'(x) = f(x)
  \end{equation*}
  Find three functions that are solutions to
  \begin{equation*}
   f'(x) = 2f(x)
  \end{equation*}
  Can you also find two solutions \(f_1, f_2\) to the equation
  \begin{equation*}
   f''(x) + 5f'(x) + 6f(x) = 0
  \end{equation*}
  and a constant \(x_0\) such that
  \(f_1(x_0) f_2'(x_0) - f_1'(x_0) f_2(x_0) \ne 0\)?
 \item
  Let \(a \in \Reals^+\) be constant. Find the derivative of \(a^x\) from the
  definition.
 \item
  Let \(t \in \Reals\) be constant. Show that the power rule for \(x^t\) holds
  for all \(t \in \Reals\) when \(x > 0\).

  This is the best we can do for now, since it's not so nice to define
  irrational powers on negative numbers (eg what should \((-1)^\pi\)
  be\footnote{
   In fact, if you were to define it it would almost certainly have to be a
   complex number.
  }?)

  An alternative route to get this result is to just define
  \(x^t\) for \(t \in \Reals\) as
  \begin{equation*}
   \lim_{k \to \infty} x^{(a_k)}
  \end{equation*}
  where \(a_k\) is some rational sequence converging to \(t\). This whole
  approach depends a little more subtly on how you've actually define the real
  numbers. It is basically an appeal to the continuity of the function
  \(x^t\).
 \item
  Show that \(x^a x^b \equiv x^{a + b}\) holds for \(x \in \Reals^+\),
  \(a, b \in \Reals\).
  % exp(a ln x) exp(b ln x) = exp((a + b) ln x)

  Also show that \((x^a)^b \equiv x^{ab}\).
  % exp(b ln exp(a ln x)) = exp(ba ln x)

  Also show that \((xy)^a \equiv x^a y^a\) if \(y \in \Reals^+\).
  % exp(a ln xy) = exp(a ln x + a ln y)
  %              = exp(a ln x) exp(a ln y)
 \item
  By writing \(\ln(uv) = \ln u + \ln v\), deduce a weak version of the product
  rule from the chain rule.

  This is a weak version because it is only valid for \(u, v > 0\). Can you use
  it, or use a similar proof, to deduce a weak product rule for
  \(u > 0\) and \(v < 0\)?
 \item
  Suppose that the function \(e^x\) can be written in the form
  \begin{align*}
   e^x
    &= a_0 + a_1 x + a_2 x^2 + \dotsb \\
    &= \sum_{k = 0}^\infty a_k x^k
  \end{align*}
  where \(a_k\) are constants to be determined. This is called a
  \emph{power series}. This particular type of power series is called a
  \emph{Taylor series}, and in fact this is a particular type of Taylor series
  called a \emph{Maclaurin series}.

  By letting \(x = 0\), deduce what \(a_0\) must be.

  By differentiating both sides, deduce what \(a_1\) must be.

  Find a \(k\)th term for \(a_k\).

  Assuming this series converges, show that
  \begin{equation*}
   e =
   \sum_{k = 0}^\infty \frac 1{k!} =
   1 + \frac 1{1!} + \frac 1{2!} + \frac 1{3!} + \dotsb
  \end{equation*}
  This series is in fact a popular alternative definition of \(e^x\) (and
  therefore also \(e\)). There is also the weird product formula
  \begin{equation*}
   e^x =
    \lim_{n \to \infty}
     \parens[\Big]{
      1 + \frac xn
     }^n
  \end{equation*}
  which the author thinks is weird\footnote{
   read: doesn't really understand
  }.
 \item
  Also determine Maclaurin series for \(\sin x\) and \(\cos x\).
 \item
  Determine the Maclaurin series of \((1 + x)^{-1}\).
 \item
  By writing \(t = x^2\), show that the Maclaurin series of \((1 + x^2)^{-1}\)
  is
  \begin{equation*}
   \sum_{k = 0}^\infty (-1)^k x^{2k} = 1 - x^2 + x^4 - x^6 + x^8 + \dotsb
  \end{equation*}
  By assuming that this series converges when \(x \in \intcc{0, 1}\) and
  calculating
  \begin{equation*}
   \integ[0]<1>{\frac 1{1 + x^2}}{x}
  \end{equation*}
  in two different ways, show that
  \begin{equation*}
   \frac \pi 4 = 1 - \frac 13 + \frac 15 - \frac 17 + \frac 19 + \dotsb
  \end{equation*}
  You may assume that to integrate a series you can integrate each term
  separately.
 \item
  Suppose that \(f\) is an infinitely differentiable function. Write down its
  Maclaurin series.
 \item
  Suppose that \(f\) is an infinitely differentiable function, and that it can
  be written in the form
  \begin{align*}
   f(x) &= a_0 + a_1(x - x_0) + a_2(x - x_0)^2 + \dotsb \\
        &= \sum_{k = 0}^\infty a_k(x - x_0)^k
  \end{align*}
  where \(x_0\) is some constant, and \(a_k\) are constants to be determined.
  Determine the constants.

  This is the full \emph{Taylor series} of \(f\).

  Beware that not every function is equal to all of its Taylor series at all
  points.
 \item
  Find \(\integ{\sin^2 x}{x}\).
 \item
  Find \(\integ{\sin 7x \cos 13 x}{x}\).
 \item
  Use the substitution \(g(x) = \cos x\) to determine \(\integ{\tan x}{x}\).
 \item
  Show that if \(t = \tan \frac 12 x\), then
  \begin{equation*}
   \dv<t>{x} = \frac 2{1 + t^2}
  \end{equation*}
  Use this in addition to an identity from question \ref{q_trig_weierstrass}
  in Trigonometry to determine
  \begin{equation*}
   \integ{\frac 1{\sin x}}{x}
  \end{equation*}
  This technique is called the \emph{Weierstrass substitution}. It is generally
  used to transform any integral involving trigonometric functions into an
  integral of a rational function (which unfortunately doesn't always make it
  look much easier).
\end{enumerate}
