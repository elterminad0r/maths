\section{Logic}

\begin{enumerate}
 \item
  Write down the negation of the statement
  \begin{equation*}
   \Forall x \in \Reals: f(x) < 0
  \end{equation*}
  You should only use positive quantifiers: ``there exists'' and ``for all''.
  You may have to change the innermost statement. You may wish to refer back to
  the compositeness example.
 \item
  Write down the negation of the statement
  \begin{equation*}
   \Exists x \in \Naturals : x_n \equiv 0 \pmod 2
  \end{equation*}
 \item
  Write down the negation of the statement
  \begin{equation*}
   \Forall n \in \Integers :
   \Exists x \in \Reals :
   f(n) < f(x)
  \end{equation*}
  Can you think of a function \(f : \Reals \to \Reals\) for which this is not
  true?
  % f(x) = cos 2πx
  % f(x) = 1
 \item
  Write down the negation of the statement
  \begin{equation*}
   \Forall \epsilon \in \set{t \in \Reals: t > 0} :
   \Exists N \in \Naturals :
   \Forall m \in \set{a \in \Naturals : a \ge N}:
   \abs{x_m - \ell} < \epsilon
  \end{equation*}
  Let the sequence \(x_n \defeq 9\sum_{k = 1}^n(\frac 1{10})^k\). Can you think
  of an \(\ell\) that makes this statement true?
  % 0.999... = 1
 \item
  Can you negate the completely general form of a quantified statement? This is
  \begin{equation*}
   \mathop{Q_1} x_1 \in S_1 :
   \mathop{Q_2} x_2 \in S_2 :
   \mathop{Q_3} x_3 \in S_3 :
   \dotsb
   \mathop{Q_n} x_n \in S_n :
   P(x_1, x_2, x_3, \dotsc, x_n)
  \end{equation*}
  where each \(Q_i\) is either \(\forall\) or \(\exists\), each \(S_i\) is a
  set, and \(P\) is some statement about \(x_1, x_2, \dotsc, x_n\).

  You may use the statement \(P'\), which is defined as
  ``\(P'(x_1, x_2, \dotsc, x_n)\) is true iff \(P(x_1, x_2, \dotsc, x_n)\) is
  false.''. You may also wish to define some new quantifiers \(Q_i'\) in terms
  of the old quantifiers.
 \item
  Here is a proof by induction that if \(n \in \Naturals\), then any finite
  subset \(S \subset \Naturals\) of \(n\) natural numbers consists of either all
  odd or all even numbers:
  \begin{tcolorbox}
   In the base case, \(n = 1\), there is a single number. This number is either
   odd or even, so there is nothing to check.

   Now suppose we have a set \(S \subset \Naturals\) of size \(n + 1\):
   \(S = \set{s_1, s_2, \dotsc, s_{n - 1}, s_n}\). Consider also the two sets
   \(A = \set{s_1, s_2, \dotsc, s_{n - 1}} \subset S\) and
   \(B = \set{s_2, \dotsc, s_{n - 1}, s_n} \subset S\).

   \(A\) and \(B\) are sets of size \(n\), so by inductive hypothesis, the
   elements of \(A\) are either all odd or all even, and similarly for \(B\).

   But then all elements of \(A\) and \(B\) have the same parity as \(s_2\), and
   hence all elements of \(S\) have the same parity as \(s_2\). So all elements
   of \(S\) have the same parity, and by induction we are done.
  \end{tcolorbox}
  Where is the mistake?
\end{enumerate}
