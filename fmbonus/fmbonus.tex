% This is a LaTeX document, compiled with pdflatex.
%
% Sometimes the comments in this document contain hints or partial answers for
% questions. If you wish not to be spoiled, do not read the source in its
% entirety.

\documentclass[fleqn,a4paper,11pt]{article}
\date{}
\author{}
\title{Additional maths questions (draft ed.)}

\usepackage{mymaths}
\usepackage{mysty}

\begin{document}
\maketitle
\tableofcontents

\section{Introduction}

This is a collection of bits of mathematics relating to the Further Maths
A-level and beyond. My alternative click-bait idea for a title was ``What they
don't teach you in Further Maths A-level''. This document is often concerned
much more with proof than you would normally see at A-level, but I think this is
a very important part of mathematics (and to me, it makes maths much easier to
really agree with).

In some cases, this document may well take things way too fast. You can always
wait until you've met more things at A-level before tackling eg some of the
tricker integration questions.

Also there are certainly questions here that you won't be able to answer, at
least not straight away (I hope). I'm not very good at gauging difficulty of
questions that I've written, so I won't always know which ones these are. If a
question seems just completely impossible, you can often just leave it.
Sometimes this will prohibit you from doing some other later questions, and if
so you will probably notice this and you can leave those too. It is worthwhile
to just let such questions simmer and see if you can come to an answer by
yourself eventually.

Unless a question is modulated by a phrase like ``briefly explain'', the
expectation is to provide sound mathematical arguments as answers. Of course
it's up to you how you write down your answers, but you should at least be
convinced that you could make your answers rigorous if you wanted to. Initial
scribblings never have to fulfil any arbitrary standard, but it can be useful to
then ``neaten up'' your answer once you have one.

\section{Notation and Definitions}

\subsection{Disclaimer}

None of this should be taken as gospel. Although I hope that any definitions I
make here (and later) are fairly sound, they are only made within the scope of
this document. Certainly there are many alternative ways to define things, but I
try to define them in a way that makes things not too complicated (sometimes).

\subsection{Logical and meta- symbols}
For expressions \(A\) and \(B\),
\begin{itemize}
 \item
  ``\(A \equiv B\)'' means ``\(A\) is identical to \(B\)''. Implicitly this
  means they have the same value for all values that any free variables can take
  (sometimes this means you have to make an inference about the domain of any
  free variables based on context).

  For instance, for \(x, y \in \Reals\):
  \begin{itemize}[label=\raisebox{0.2ex}{\tiny\textbullet}]
   \item \(\sin x \equiv \cos(\frac \pi 2 - x)\)
   \item \(x + x \equiv 2x\)
   \item \(x + y \equiv y + x\)
   \item \(x + 1 \nequiv x\)
   \item \(\abs x \nequiv x\)
  \end{itemize}
  although for \(x \in \Naturals\), \(\abs x \equiv x\).

  A statement involving a \(\equiv\) is often called an \emph{identity}.

  To prove an identity holds, you have to be very careful. You can't just ``do
  the same thing to both sides'' to obtain a tautology. In order for this
  strategy to work, each step must be reversible. It is often much safer to just
  start with the left hand side, and gradually keep rewriting it until it is
  equal to the right hand side (or vice versa).

  If you thought you could prove it by doing the same thing to both sides, you
  can just do that thing to the left hand side and then do it backwards to get
  to the right hand side, if it is invertible.
 \item
  ``\(A \defeq B\)'' means ``\(A\) is defined to be equal to \(B\)''. For
  instance (in most cases), \(x^2 \defeq x \cdot x\).
\end{itemize}
For statements \(A\) and \(B\),
\begin{itemize}
 \item
  ``\(A \implies B\)'' means ``\(A\) implies \(B\)'', or
  ``\(A\) is a sufficient condition for \(B\)''.

  This \emph{means} that ``whenever \(A\) is true, \(B\) must be true''.

  Counter-intuitively, this means that if \(A\) is always false,
  ``\(A \implies B\)'' is true (whatever \(B\) is). In logic, this is the
  principle of explosion - you can deduce anything from a contradiction.
 \item
  ``\(A \impliedby B\)'' means ``\(A\) is implied by \(B\)'', or
  ``\(A\) is a necessary condition for \(B\)''.

  It is equivalent to ``\(B \implies A\)``
 \item
  ``\(A \iff B\)'' means ``\(A\) and \(B\) are equivalent'', or
  ``\(A\) if and only if \(B\)'', or ``\(A\) iff \(B\)'', or
  ``\(A\) is a sufficient and necessary condition for \(B\)''.

  It is equivalent to ``\(A \implies B\) and \(A \impliedby B\)''. To
  \emph{prove} ``\(A \iff B\)'', you need to either prove both
  \(A \implies B\) and \(B \implies A\), or very carefully use a sequence of
  ``iff'' transformations to turn \(A\) into \(B\).
\end{itemize}

\subsubsection{Quantifiers}

The symbol \(\forall\) means ``for all''. This is called the \emph{universal
quantifier}. You can form statements with it by saying ``for a variable taking
all values in some set, some statement about that variable is true''. For
instance, to state that ``all real numbers have a square greater than or equal
to zero'' might be written
\begin{equation*}
 \Forall x \in \Reals : x^2 \ge 0
\end{equation*}
If the set you're talking about is the empty set, then the statement is
automatically true. This called ``vacuous truth'', and it really does make sense
(to disprove a universally quantified statement, you would have to provide a
counterexample from the set - but the whole point of the empty set is that
there's nothing in it to provide!) It is related to the principle of explosion.
More than likely you have used a vacuous truth jokingly to construct some
bizarre statement before: ``all integers that square to \(2\) have a decimal
representation consisting only of \(3\)s.''\footnote{
 Well, maybe not quite like that.
}

Similarly, the symbol \(\exists\) means ``there exists''. This is called the
\emph{existential quantifier}. You can form statements with it by saying ``there
exists a value in some set so that assigning that value to a variable makes some
statement about that variable true''. For instance, the fact that there exists a
natural number $n$ for which \(\frac 1{100!}n! > 100^n\) might be written
\begin{equation*}
 \Exists n \in \Naturals : \tfrac 1{100!}n! > 100^n
\end{equation*}
In much the same way that a universally quantified statement over the empty set
is always true, an existentially quantified statement over the empty set is
always false.

These ``quantifiers'' can be combined to make {\large HUGE} scary statements.
This is mostly discouraged when you're not doing formal logic but trying to
communicate maths to other people. An example of how they might be combined is
to say ``each integer has an additive inverse in the integers'':
\begin{equation*}
 \Forall n \in \Integers : \Exists m \in \Integers : m + n = 0
\end{equation*}
or if \(n \in \Naturals\) and \(n \ne 1\), ``\(n\) is prime'' might be written
as follows:
\begin{equation*}
 \Forall m \in \set{a \in \Naturals : 2 \le a < n} :
 \Forall k \in \Naturals :
 km \ne n
\end{equation*}
and ``\(n\) is composite'' might be written as follows:
\begin{equation*}
 \Exists m \in \set{a \in \Naturals : 2 \le a < n} :
 \Exists k \in \Naturals :
 km = n
\end{equation*}
(Can you spot the differences?)

By the way, many people have slightly different habits about putting in colons
or commas or whatnot between quantifiers. Generally a quantifier applies to
everything after it unless you see some other parenthesisation.

Sometimes people don't explicitly give the domain of a quantifying variable. For
example, in number theory, you might write something like
``\(\Forall n, x_n < x_{n + 1}\)'' (the sequence \(x_n\) is strictly increasing)
and by a mix of convention and context, it would be understood that this means
``\(\Forall n \in \Naturals\)''.

\subsection{Sets}

A set is a ``collection of objects'', in some sense, where we don't care about
which order they appear in, just about whether or not they appear. If \(S\) is a
set, and \(x\) is some object, we write \(x \in S\) to mean ``\(x\) is an
element of \(S\)'' and \(x \notin S\) to mean ``\(x\) is not an element of
\(S\)''.

``\(\set{s_1, s_2, s_3, \dotsc, s_n}\)'' means ``the set with elements
\(s_1, s_2, s_3\) up to \(s_n\)''.

``\(\set{s_1, s_2, s_3, \dotsc}\)'' means ``the set with elements \(s_i\) for
\(i\) a natural number''.

We say two sets are equal if they have precisely the same elements. Formally,
\(A = B\) iff \(\Forall a \in A: a \in B\) and \(\Forall b \in B: b \in A\).

I use the following conventions to talk about sets:
\begin{center}
 \begin{tabular}{cc}
  \toprule
  \bfseries Notation & \bfseries Set \\
  \midrule
  \(\emptyset\) & \(\set{}\) \\
  \(\Naturals\)
  & \(\set{1, 2, 3, \dotsc}\) \\
  \(\Naturals_0\)
  & \(\set{0, 1, 2, \dotsc}\) \\
  \(\intcc{a, b}\) & \(\set{x \in \Reals : a \le x \le b}\) \\
  \(\intco{a, b}\) & \(\set{x \in \Reals : a \le x < b}\) \\
  \(\intoc{a, b}\) & \(\set{x \in \Reals : a < x \le b}\) \\
  \(\intoo{a, b}\) & \(\set{x \in \Reals : a < x < b}\) \\
  \(\Reals^+\) & \(\intoo{0, \infty}\) \\
  \(\Reals_0^+\) & \(\intco{0, \infty}\) \\
  \(\Naturals_{\ge m}\) & \(\set{n \in \Naturals : n \ge m}\) \\
  \bottomrule
 \end{tabular}
\end{center}
A set is a \emph{subset} of another set if each of its elements is contained in
that set. Formally, \(A\) is a subset of \(B\) iff \(\Forall a \in A : a \in
B\). Furthermore, a set is a \emph{proper subset} of another set if it is a
subset and it is not equal to that set.

I will use the notation ``\(A \subseteq B\)'' to denote ``\(A\) is a subset
of \(B\)'', and ``\(A \subset B\)'' to denote ``\(A\) is a proper subset of
\(B\)''. Some people use the latter to mean the former, as the notion of proper
subset is not very often used.

The notion of ``set builder notation'' is also defined. For some well-defined
property \(P\) of elements of some set \(S\), we write
``\(\set{x \in S: P(x)}\)'' to mean ``the subset of elements of \(S\) that
satisfy \(P\)''.

The \emph{difference} of two sets \(A\) and \(B\) is the set of elements in
\(A\) that are not in \(B\). In symbols,
\(A \setminus B \defeq \set{a \in A : a \notin B}\).

The \emph{union} of two sets \(A\) and \(B\) is the set of all elements that are
in either \(A\) or \(B\), and is written \(A \cup B\).

The \emph{intersection} of two sets \(A\) and \(B\) is the set of all elements
that are in both \(A\) and \(B\), and is written \(A \cap B\).

Lastly, the ``Cartesian product'' of two sets \(A\) and \(B\) is the set of all
ordered pairs of elements of \(A\) then \(B\). It is written
\(A \times B \defeq \set{(a, b) : a \in A, b \in B}\). If \(S\) is a set, then
\(S^2\) is often used as a shorthand for \(S \times S\). Similarly, we can write
\(S^n\) to mean ``the set of ordered n-tuples of elements of \(S\)''.

\subsection{Functions}

A function is a way to assign a \emph{single} value in some set called a
codomain to each element in some set called a domain. The domain and codomain of
a function can be (are) very important! You might say that the functions \(f(x)
= x\) and \(f(x) = x^2\) are different, but if their domain was the set
\(\set{0, 1}\), then they would be the same function!

The notation ``\(f: A \to B\)'' means ``\(f\) is a function from the set \(A\)
to the set \(B\)'' (then \(A\) is the domain of \(f\), \(B\) is the codomain).

The expression ``\(x \mapsto x^2\)'' means ``the function taking \(x\) to
\(x^2\)'' (of course this also works for any other well-defined expression
involving \(x\)). Sometimes these are combined:
\begin{align*}
 f: \Reals &\to \Reals \\
    x &\mapsto x^2
\end{align*}
defines a function \(f\) from \(\Reals\) to \(\Reals\), that takes \(x\) to
\(x^2\). In more familiar applicative notation, \(f(x) = x^2\).

Note that a function need not hit every point in its codomain, we just need each
output of the function to lie in the codomain.

The subset of the codomain that \(f\) does hit is called the \emph{image} of
\(f\). Sometimes this is called the range.

\begin{itemize}
 \item
  A function that \emph{does} happen to hit each point of its codomain is called
  a \emph{surjection}, or a \emph{surjective function}. Formally, \(f: A \to B\)
  is surjective iff \(\Forall b \in B : \Exists a \in A : f(a) = b\).

  A surjective function is sometimes called ``onto''.

  Any function can be made into a surjection by restricting its codomain to be
  its image.
 \item
  A function for which each output is different is called an \emph{injection} or
  an \emph{injective function}. Formally, \(f: A \to B\) is injective iff
  \(\Forall a_1 \in A : \Forall a_2 \in A : a_1 \ne a_2
    \implies f(a_1) \ne f(a_2)\).

  An injective function is sometimes called ``into''. The concept of
  injectiveness is related to being ``one-to-one''. Certainly a non-injective
  function must be ``many-to-one''.
 \item
  A function that is both injective and surjective is called a \emph{bijection}
  or a \emph{bijective function}. A bijective function associates each element
  of its domain with each element of its codomain without repetition. This means
  that a function \(f: A \to B\) is bijective iff there exists a two-sided
  inverse: \(\Exists g: B \to A\) such that \(\Forall b \in B : fg(b) = b\) and
  \(\Forall a \in A : gf(a) = a\).
\end{itemize}

A formal way to think about a function \(f: A \to B\) is as a subset of
\(A \times B\) with certain properties. This does not often come up when you're
actually working with functions.

A useful piece of notation when defining functions is the ``by cases'' notation.
When we write
\begin{equation*}
 x \mapsto
 \begin{cases*}
  f_1(x) & if \(P_1(x)\) \\
  f_2(x) & if \(P_2(x)\) \\
  \dotsb
 \end{cases*}
\end{equation*}
where \(f_1(x), f_2(x), \dotsc\) are expressions involving \(x\) and
\(P_1(x), P_2(x), \dotsc\) are statements about \(x\), this means
``\(x\) maps to the value \(f_1(x)\) if \(P_1(x)\) is true, otherwise it maps to
\(f_2(x)\) if \(P_2(x)\) is true, \dots''
This is similar to a program in Python that looks something like this:
\begin{verbatim}
def f(x):
    if P_1(x):
        return f_1(x)
    elif P_2(x):
        return f_2(x)
    ...
\end{verbatim}
It is important to make sure that there is a value to take for each value that
\(x\) could take. Sometimes this is done by just writing ``otherwise'' for the
last condition. It is good manners to make sure that the conditions do not
overlap.

A nice example of how to use it is to define the absolute value function on
\(\Reals\):
\begin{equation*}
 \abs x =
 \begin{cases*}
  \phantom{-}x & if \(x \ge 0\) \\
  -x & if \(x < 0\)
 \end{cases*}
\end{equation*}

\subsection{Numbers}

\subsubsection{Modular arithmetic}

``\(a \equiv b \pmod n\)'' means ``\(a\) is congruent to \(b\) modulo \(n\)''.
This is defined to mean \\
``\(\Exists k \in \Integers : a = b + kn\)''.

Loosely, this means that \(a\) and \(b\) leave the same positive remainder when
divided by \(n\). If you know about ``mod'' from programming, this is perhaps a
more obvious sounding definition, but it is much harder to work with.

Note that this is not the same \(\equiv\) as in identities.

\subsubsection{Induction}

If you can prove that a statement is true for some natural number \(n\), and
that if the statement is true for some number \(m\)\footnote{
 This assumption is sometimes called the inductive hypothesis
}, it must also be true for
\(m + 1\), then you have proved that the statement is true for all \(m \ge n\),
by induction.

Induction should not be thought of as some arcane black magic method of proof.
Induction is really just a way to formalise things that are obviously true.
Whenever you write ``\ldots'', probably what you're really doing is proof by
induction. Induction is your friend.

\subsection{Trigonometry}

Angles are always in radians. One radian is the angle such that the arc
subtended by that angle in the unit circle\footnote{
 The unit circle is the circle with radius \(1\).
} has length \(1\).

For \(\theta \in \intcc{0, \frac \pi 2}\), \(\sin \theta\) is defined as the
length of the opposite side of the right triangle with unit hypotenuse and angle
\(\theta\). \(\cos \theta\) is defined as the length of the adjacent side in the
same triangle. (In the pathological cases \(\theta = 0\) and
\(\theta = \frac \pi 2\), the natural limiting values are taken.)

Then for \(\theta \in \Reals\), the definition is extended such that:
\begin{itemize}
 \item
  \(\sin(-\theta) \equiv -\sin \theta\) and
  \(\cos(-\theta) \equiv \cos \theta\).

  These two identities describe the \emph{parity} of the trigonometric
  functions. Particularly, this means that \(\sin\) is an \emph{odd} function
  and \(\cos\) is an \emph{even} function.
 \item
  \(\sin(\theta + \pi) \equiv -\sin \theta\) and
  \(\cos(\theta + \pi) \equiv -\cos \theta\).
\end{itemize}
You may wish to take some time to convince yourself that these completely and
unambiguously define the trigonometric functions as you know them (in the
familiar wavy shapes).

These definitions are useful because they make the functions be generally
``nice'', and smooth, in the way you expect. For instance,
\((\cos t, \sin t)\) for \(t \in \intco{0, 2\pi}\) now gives a smooth
parametrisation\footnote{
 This means the set of all points ``traced out'' by the co-ordinate
 \((\cos t, \sin t)\) as \(t\) varies over
 \(\intco{0, 2\pi}\).
} of the unit circle.

Lastly,
\begin{equation*}
 \tan \theta \defeq \frac{\sin \theta}{\cos \theta}
\end{equation*}

When I write ``arc-'' before a trigonometric function, that denotes the inverse
function. Precisely,
\begin{itemize}
 \item
  \(\arcsin\) is the function on \(\intcc{-1, 1}\) with image
  \(\intcc{-\frac \pi 2, \frac \pi 2}\) such that
  \begin{equation*}
   \sin(\arcsin x) = x \Forall x \in \intcc{1, 1}
  \end{equation*}
 \item
  \(\arccos\) is the function on \(\intcc{-1, 1}\) with image
  \(\intcc{0, \pi}\) such that
  \begin{equation*}
   \cos(\arccos x) = x \Forall x \in \intcc{1, 1}
  \end{equation*}
 \item
  \(\arctan\) is the function on \(\Reals\) with image
  \(\intoo{-\frac \pi 2, \frac \pi 2}\) such that
  \begin{equation*}
   \tan(\arctan x) = x \Forall x \in \Reals
  \end{equation*}
\end{itemize}
Their images are chosen in this way as this is the first sensible ``slice'' of
the regular trig function that hits all of \(\intcc{-1, 1}\).

\subsection{Calculus}

If \(y = f(x)\), then \(y\) is called a \emph{dependent variable}, which
\emph{depends} on the \emph{independent variable} \(x\).

If \(y = f(x)\), then the following are all equivalent expressions for ``the
derivative of \(y\) with respect to \(x\)'':
\begin{equation*}
 y'
 \equiv f'(x)
 \equiv \dv<y>{x}
 \equiv \dv{x}(f(x))
\end{equation*}
which is defined as follows:
\begin{equation*}
 f'(x_0) \defeq
  \lim_{h \to 0}
   \bracks[\bigg]{
    \frac{f(x_0 + h) - f(x_0)} h
   }
\end{equation*}
This is an equivalent definition:
\begin{equation*}
 f'(x_0) \defeq
  \lim_{x \to x_0}
  \bracks[\bigg]{
   \frac{f(x) - f(x_0)}{x - x_0}
  }
\end{equation*}
induced by the substitution \(x = x_0 + h\).

Furthermore the ``\(n\)th derivative of \(y\) with respect to \(x\)'' may be
written as:
\begin{equation*}
 f^{(n)}(x)
 \equiv \dv[n]<y>{x}
 \equiv \dv[n]{x}(f(x))
\end{equation*}
It can be defined recursively: \(f^{(1)}(x) \defeq f'(x)\) and
\(f^{(n + 1)}(x) \defeq \dv{x}(f^{(n)}(x)) \Forall n \in \Naturals\).

\subsection{Integrals}

The discussion of integrals will be, on the whole, somewhat less formal than
that of derivatives. This is because to really prove things about integrals
requires a fair bit of \emph{really} messy limit work, which isn't particularly
illuminating if you're doing calculus for the first time.

The \emph{definite integral} of a continuous\footnote{
 ``Continuous'', for now, means that the graph of \(y = f(x)\) can be drawn
 without taking your pen(cil) off the page.
} function \(f\) from \(a\) to \(b\)
is written
\begin{equation*}
 \integ[a]<b>{f(x)}{x}
\end{equation*}
If \(a < b\), it means ``the signed area between the graph \(y = f(x)\) and the
\(x\)-axis between \(x = a\) and \(x = b\)''.

``Signed area'' means that the area can be negative. Particularly, if \(f(x)\)
is negative for some interval, then the contribution of the area from that
interval is negative. For example,
\begin{equation*}
 \integ[0]<1>{(x - 1)}{x} = -\tfrac 12
\end{equation*}
You may wish to draw a diagram to convince yourself that this is true. This also
means that for instance,
\begin{equation*}
 \integ[0]<2>{(x - 1)}{x} = 0
\end{equation*}
so you really do have to be careful not to just talk about ``area''.

You may have noticed the parentheses here. Some people would omit these and just
write
\begin{equation*}
 \integ[0]<2>{x - 1}{x}
\end{equation*}
Because an integral is always opened by an integral sign and closed by a
differential, this isn't ambiguous per se, but the author disagrees with this
notation. This is because this notation comes from the concept of a Riemann
integral, which is a limit of the sum of little ``strips'' that make up the
area. Here ``\(\diff x\)'' is the (infinitesimal) width of the strips, and
``\(f(x)\)'' is the height of the strips as \(x\) varies. These are multiplied
together to give the area of each strip, and then summed by the sort of weird
S-shape, \(\int\). However, this isn't really conveyed by the unparenthesised
notation, as it just looks like only the \(1\) term is being multiplied by
\(\diff x\). Hence if you're integrating an integrand involving a sum, I prefer
to add parentheses, for explicitness' sake.

Moving on. If \(a > b\), then it is natural to define
\begin{equation*}
 \integ[a]<b>{f(x)}{x} \defeq -\integ[b]<a>{f(x)}{x}
\end{equation*}
and it is also natural to say that if \(a = b\), then
\begin{equation*}
 \integ[a]<b>{f(x)}{x} = 0
\end{equation*}

Note that the \(x\) inside a definite integral is a ``dummy'' variable. Outside
of the integral, the \(x\) doesn't mean anything, so we can change the \(x\) to
a \(t\), and the value of the definite integral remains the same. Make sure
never to use a variable as a dummy variable that already exists elsewhere.

\subsubsection{Fundamental Theorem of Calculus}

The Fundamental Theorem of Calculus tells you the following:
\begin{tcolorbox}
 If \(f\) is a continuous function, then there exist (infinitely many)
 antiderivatives of \(f\), each differing by a constant. Furthermore, if \(F\)
 is an antiderivative of \(f\) (that is, \(F'(x) \equiv f(x)\)), then
 \begin{equation*}
  \integ[a]<b>{f(x)}{x} \equiv F(b) - F(a)
 \end{equation*}
 Furthermore, if
 \begin{equation*}
  G(x) \defeq \integ[x_0]<x>{f(t)}{t}
 \end{equation*}
 for some \(x_0\), then \(G'(x) \equiv f(x)\)
\end{tcolorbox}
Sometimes people write \(\eval{a}{b}{F(x)}\) or \(\evalline{a}{b}{F(x)}\) to
mean \(F(b) - F(a)\).

Really, all this means is that ``when you add up all the small changes in
\(F\), you get the overall change'', which should sound like a reasonable
statement. The consequence of this theorem is basically that differentiation and
integration are inverse processes, in some sense.

Because of this theorem, people write the \emph{indefinite integral}
\begin{equation*}
 \integ{f(x)}{x}
\end{equation*}
to mean ``an antiderivative of \(f(x)\) with respect to \(x\)''. This is of
course only uniquely determined up to a constant, so this is where ``\({}+ C\)''
comes from.

\subsection{Complex Numbers}

\(\Complex\) is the set of complex numbers. \(\Complex\) is defined as
\(\Reals^2\) equipped with the operations of addition and multiplication as
follows:
\begin{align*}
 (z_1, z_2) + (w_1, w_2) &\defeq (z_1 + w_1, z_2 + w_2) \\
 (z_1, z_2) \cdot (w_1, w_2) &\defeq (z_1 w_1 - z_2 w_2, z_1 w_2 + z_2 w_1)
\end{align*}
We refer to the element \((0, 1)\) as \(i\). Also, if \(x\) is a real number,
then \(x\) is associated with the complex number \((x, 0) \in \Complex\).

If
\((z_1, z_2) \in \Complex\), then the \emph{real part} is defined as
\begin{equation*}
 \Re((z_1, z_2)) \defeq z_1
\end{equation*}
and the \emph{imaginary part} is defined as
\begin{equation*}
 \Im((z_1, z_2)) \defeq z_2
\end{equation*}
Also we write the \emph{modulus} as
\begin{equation*}
 \abs[\big]{(z_1, z_2)} \defeq \sqrt{z_1^2 + z_2^2}
\end{equation*}
If you have met complex numbers before, this definition may seem ``backwards''.
However, if you argue from this direction, it's much more obvious to show that
this number system is worthwhile at all.

By the way, when you construct \(\Complex\) this way, it is \emph{not}
set-theoretically accurate to say \(\Reals \subset \Complex\). It \emph{is} true
that \(\Reals\) is isomorphic to the subset
\(\Reals' = \set{(x, 0) : x \in \Reals} \subset \Complex\). This is not a very
consequential nitpick, but one that some people are very fond of making. (It also
often applies to \(\Naturals \subset \Integers\) and
\(\Integers \subset \Rationals\).)

\section{Logic}

\begin{enumerate}
 \item
  Write down the negation of the statement
  \begin{equation*}
   \Forall x \in \Reals: f(x) < 0
  \end{equation*}
  You should only use positive quantifiers: ``there exists'' and ``for all''.
  You may have to change the innermost statement. You may wish to refer back to
  the compositeness example.
 \item
  Write down the negation of the statement
  \begin{equation*}
   \Exists x \in \Naturals : x_n \equiv 0 \pmod 2
  \end{equation*}
 \item
  Write down the negation of the statement
  \begin{equation*}
   \Forall n \in \Integers :
   \Exists x \in \Reals :
   f(n) < f(x)
  \end{equation*}
  Can you think of a function \(f : \Reals \to \Reals\) for which this is not
  true?
 \item
  Write down the negation of the statement
  \begin{equation*}
   \Forall \epsilon \in \set{t \in \Reals: t > 0} :
   \Exists N \in \Naturals :
   \Forall m \in \set{a \in \Naturals : a \ge N}:
   \abs{x_m - \ell} < \epsilon
  \end{equation*}
  Let the sequence \(x_n \defeq \sum_{k = 0}^n(\frac 9{10})^k\). Can you think
  of an \(\ell\) that makes this statement true?
 \item
  Can you negate the completely general form of a quantified statement? This is
  \begin{equation*}
   \mathop{Q_1} x_1 \in S_1 :
   \mathop{Q_2} x_2 \in S_2 :
   \mathop{Q_3} x_3 \in S_3 :
   \dotsb
   \mathop{Q_n} x_n \in S_n :
   P(x_1, x_2, x_3, \dotsc, x_n)
  \end{equation*}
  where each \(Q_i\) is either \(\forall\) or \(\exists\), each \(S_i\) is a
  set, and \(P\) is some statement about \(x_1, x_2, \dotsc, x_n\).

  You may use the statement \(P'\), which is defined as
  ``\(P'(x_1, x_2, \dotsc, x_n)\) is true iff \(P(x_1, x_2, \dotsc, x_n)\) is
  false.''. You may also wish to define some new quantifiers \(Q_i'\) in terms
  of the old quantifiers.
\end{enumerate}

\section{Functions and Sets}

\begin{enumerate}
 \item
  Write down the set that is a subset of every set.
 \item
  Use quantifiers to write down all the properties that a subset of
  \(A \times B\) must satisfy for it to correspond to a function from
  \(A \to B\).

  Write down the conditions for such a subset to correspond to an injective and
  a surjective function, respectively.
 \item
  State a necessary and sufficient condition on \(A\) and \(B\) for
  \(A \times B\) to correspond to a function from \(A \to B\).
 \item
  Show that if \(A\) and \(B\) are sets, and \(A \subseteq B\) and
  \(B \subseteq A\), then \(A = B\).
 \item
  Let \(\abs S\) denote ``the number of elements of the set \(S\)''.

  Let \(S_1, S_2, \dotsc S_n\) be finite sets. Write down formulae for:
  \begin{itemize}
   \item
    \(\abs{S_1 \cap S_2}\)
   \item
    \(\abs{S_1 \cap S_2 \cap S_3}\)
   \item
    \(\abs{S_1 \cap S_2 \cap S_3 \cap S_4}\)
  \end{itemize}
  Guess a formula for \(\abs{S_1 \cap S_2 \cap \dotsb \cap S_n}\) and prove its
  correctness by induction.
 \item
  An operation on a set \(S\) is a function from \(S^2 \to S\).

  An operation \(f\) on a set \(S\) is called \emph{commutative} iff
  \begin{equation*}
   f(a, b) = f(b, a) \Forall a, b \in S
  \end{equation*}
  and \emph{associative} iff
  \begin{equation*}
   f(f(a, b), c) = f(a, f(b, c)) \Forall a, b, c \in S
  \end{equation*}
  Write down three associative commutative operations on \(\Integers\).
  Can you think of a commutative non-associative operation on \(\Integers\)?
  How about a non-associative commutative operation on \(\Integers\)?
 \item
  Find an operation \(f\) on \(\Naturals_0\), satisfying the following axioms:
  \begin{enumerate}
   \item
    \(f\) is an associative operation on \(\Naturals_0\)
   \item \label{set_gp_axiom}
    \(\Exists e \in \Naturals_0 : \Forall a \in \Naturals_0 : f(a, e) = a\)
   \item
    \(\Forall n \in \Naturals_0 : \Exists m \in \Naturals_0 : f(n, m) = e\),
    where \(e\) is an element of \(\Naturals_0\) satisfying Axiom
    \ref{set_gp_axiom}.
  \end{enumerate}
  The satisfaction of these axioms means that \((\Naturals_0, f)\) is a
  \emph{group}.

  Find an operation that makes the set of naturals congruent to
  \(1\) modulo \(3\)\footnote{
   written \(3\Naturals_0 + 1 \defeq \set{3n + 1 : n \in \Naturals}\)
  } a group.
 \item
  Let \(S, S'\) be sets, \(f\) be an operation on \(S'\) and
  \(g\) be a bijection from \(S \to S'\), and \(h\) be an operation on \(S\)
  defined as \(h(a, b) \defeq g^{-1}(f(g(a), g(b)))\).

  Show that \(f\) is associative if and only if \(h\) is associative.

  Show that \(f\) is commutative if and only if \(h\) is commutative.
 \item
  We say two sets have the same size if there exists a bijection between them.

  Show from the definition that the sets \(\set{1, 2, 3, 4, 5, 6}\)
  and \(\set{0.1, -\pi, e, i, 3.14158, 0}\) have the same size.
 \item
  Prove from the definition that \(\Integers\) and \(\Naturals\) have the same
  size.
 \item
  Prove from the definition that
  \(\Forall a, b \in \Reals : a < b \implies
    \text{``\(\intcc{a, b}\) has the same size as \(\intcc{0, 1}\)''}\).
 \item
  Why is the following proof incorrect?
  \begin{tcolorbox}
   There is a function \(\sin : \Reals \to \intcc{-1, 1}\). Therefore
   \(\Reals\) and \(\intcc{-1, 1}\) have the same size.
  \end{tcolorbox}
  Can you give a correct proof from the definition that \(\Reals\) and
  \(\intcc{-1, 1}\) have the same size?
 \item
  Prove from the definition that \(\intcc{0, 1}\) and \(\intco{0, 1}\) have the
  same size.
 \item
  The \emph{Cantor-Schr\"oder-Bernstein Theorem} states that if \(A\) and \(B\)
  are sets, and there exists an injection from \(A\) to \(B\), and there exists
  an injection from \(B\) to \(A\), then there exists a bijection between
  \(A\) and \(B\)\footnote{
   The author will be happy to provide a proof of this fact.
  }. Use this to find a shorter proof that \(\intcc{0, 1}\) and \(\intco{0, 1}\)
  have the same size.
 \item
  Also use this to prove that \(\Rationals\) and \(\Naturals\) have the same
  size.
 \item
  % "getallen"
  Let \(\Gamma\) be a set with the following properties:
  \begin{itemize}
   \item
    % "zero"
    There is an element \(\zeta \in \Gamma\).
   \item
    % "successor"
    There is a function \(\Sigma: \Gamma \to \Gamma \setminus \set{\zeta}\) that
    is a bijection, and satisfies the following:
   \item
    % "induction"
    If \(\BigIota\) is a set such that \(\zeta \in \BigIota\), and
    \(\Forall \gamma \in \Gamma :
      \gamma \in \BigIota
      \implies \Sigma(\gamma) \in \BigIota\),
    then \(\Gamma \subseteq \BigIota\).
  \end{itemize}
  Now we also define an operation  on \(\Gamma\) as follows:
  % "addition"
  \begin{align*}
   \alpha: \Gamma^2 &\to \Gamma \\
   (\gamma_1, \gamma_2) &\mapsto
   \begin{cases*}
    \gamma_1 & if \(\gamma_2 = \zeta\) \\
    \Sigma(\alpha(\gamma_1, \gamma')) & if \(\gamma_2 = \Sigma(\gamma')\)
   \end{cases*}
  \end{align*}
  Here is a proof that \(\alpha\) is well-defined for all
  \(\gamma_1, \gamma_2 \in \Gamma\):
  \begin{tcolorbox}
   Let \(\gamma_1\) be an arbitrary element of \(\Gamma\). Consider the
   following subset of \(\Gamma\):
   \begin{equation*}
    \BigIota
    = \set{\gamma \in \Gamma :
    \text{\(\alpha(\gamma_1, \gamma)\) is well-defined}}
   \end{equation*}
   Note firstly that \(\alpha(\gamma_1, \zeta)\) is well-defined, as it is an
   explicitly mentioned case that is equal to \(\gamma_1\). So
   \(\zeta \in \BigIota\).

   Note also that if some \(\gamma_2 \in \BigIota\), then
   \(\alpha(\gamma_1, \Sigma(\gamma_2)) = \Sigma(\alpha(\gamma_1, \gamma_2))\)
   by definition, and is therefore well-defined, so then also
   \(\Sigma(\gamma_2) \in \BigIota\). But then by the third property of
   \(\Gamma\), \(\Gamma \subseteq \BigIota\). Therefore, \(\Gamma = \BigIota\)
   and hence \(\alpha(\gamma_1, \gamma_2)\) is well-defined for all
   \(\gamma_1, \gamma_2 \in \Gamma\).
  \end{tcolorbox}
  Prove that \(\Forall \gamma \in \Gamma : \alpha(\zeta, \gamma) = \gamma\).

  Do this by considering the set of all elements \(\gamma \in \Gamma\) such that
  \(\alpha(\zeta, \gamma) = \gamma\).
 \item
  Prove that
  \(\forall \gamma_1, \gamma_2 \in \Gamma :
    \alpha(\Sigma(\gamma_1), \gamma_2) = \Sigma(\alpha(\gamma_1, \gamma_2))\).

  Do this by letting \(\gamma_1\) be arbitrary, and considering the set of all
  elements \(\gamma \in \Gamma\) such that
  \(\alpha(\Sigma(\gamma_1), \gamma) = \Sigma(\alpha(\gamma_1, \gamma))\).
 \item
  Prove that
  \(\Forall \gamma_1, \gamma_2, \gamma_3 \in \Gamma :
    \alpha(\alpha(\gamma_1, \gamma_2), \gamma_3) =
    \alpha(\gamma_1, \alpha(\gamma_2, \gamma_3))\).

  Do this by letting \(\gamma_1, \gamma_2\) be arbitrary and considering the set
  of all elements \(\gamma \in \Gamma\) such that
  \(\alpha(\alpha(\gamma_1, \gamma_2), \gamma) =
    \alpha(\gamma_1, \alpha(\gamma_2, \gamma))\).
 \item
  Hence show that
  \(\Forall \gamma_1, \gamma_2, \gamma_3, \gamma_4 \in \Gamma :
    \alpha(\gamma_1, \alpha(\gamma_2, \alpha(\gamma_3, \gamma_4))) =
    \alpha(\alpha(\alpha(\gamma_1, \gamma_2), \gamma_3), \gamma_4)\).

  (Do this without considering any more sets)
 \item
  Prove that
  \(\Forall \gamma_1, \gamma_2 \in \Gamma :
    \alpha(\gamma_1, \gamma_2) = \alpha(\gamma_2, \gamma_1)\).

  Do this by letting \(\gamma_1\) be arbitrary and considering the set of all
  elements \(\gamma \in \Gamma\) such that
  \(\alpha(\gamma_1, \gamma) = \alpha(\gamma, \gamma_1)\). Remember that you
  have already proved a number of useful results about \(\alpha\).
 \item
  Write down an expression for
  \(\alpha(\Sigma(\Sigma(\zeta)), \Sigma(\Sigma(\Sigma(\Sigma(\zeta)))))\) that
  does not involve \(\alpha\).

  Can you summarise everything that you have shown about \(\alpha\)? What do
  you think \(\alpha, \zeta, \Gamma, \Sigma\) represent? What does the property
  involving \(\BigIota\) represent?

  It is possible to walk through a similar process to define even more useful
  operations on and properties of \(\Gamma\). If you're interested, there is
  more to be found at this website:\\
  \url{uggc://jjjs.vzcrevny.np.hx/~ohmmneq/kran/angheny_ahzore_tnzr/} (ROT13
  employed to avoid spoilers in the name of the URL).
\end{enumerate}

\section{Numbers}

\begin{enumerate}
 \item
  If \(n \in \Naturals\), \(x^n\) is defined inductively as follows:
  \(x^1 \defeq x\) and \(x^{n + 1} \defeq x \cdot x^n\).

  Prove by induction on \(b\) that for \(a, b \in \Naturals\),
  \(x^{a + b} \equiv x^a x^b\).
  % x^(a+1) = x^a x^1 ✓
  % x^(a + b + 1) = x x^(a + b)
  %               = x x^a x^b
  %               = x^a x^(b + 1)

  You may assume commutativity and associativity of multiplication.
 \item
  Prove by induction on \(n\) that \((xy)^n \equiv x^n y^n\).
  % (xy)^1 = xy
  % (xy)^(n + 1) = xy (xy)^n
  %              = xy x^n y^n
  %              = x x^n y y^n
  %              = x^(n + 1) y^(n + 1)
 \item
  Show that if \(a, b \in \Naturals\), then
  \((x^a)^n \equiv x^{an}\) by induction on \(n\).
  % (x^a)^1 = x^(a · 1) ✓
  % (x^a)^(n + 1) = x^a x^(an)
  %               = x^(an + a)
  %               = x^(a(n + 1))

  Deduce that \((x^a)^n \equiv (x^n)^a\).
 \item
  We now wish to define \(x^n\) in general, for \(n \in \Integers\) (and this
  time for \(x \in \Reals^+\)\footnote{
   Note that everything up until now only needs \(x\) and \(y\) to have some
   commutative associative multiplication defined on them.
  }).

  If the identity \(x^{a + b} \equiv x^a x^b\) is to hold, show that
  we have no choice but to define \(x^0 \defeq 1\), and that therefore
  \(x^{-a} \equiv \frac 1{x^a}\).

  It can be shown that the other identities from above also hold under these
  definitions.
 \item
  Now for \(x \in \Reals^+\) and \(p/q \in \Rationals\), we define \(x^{p/q}\),
  as ``the positive real number \(t\) such that \(t^q = x^p\)''.

  It can be shown that there is always exactly one such number, but that
  requires defining the reals first, which is a \(\Reals\text{oyal}\) pain.

  Show that \(x^{a / 1} \equiv x^a\).

  Show that \(x^{a / b} \equiv x^{na/nb}\). (not just by simplifying the
  fraction - what we want to show is that it doesn't matter how you write the
  fraction. This is incidentally why we're just not defining it on the negative
  numbers since you can only define it some of the time, and then you have to
  start worrying about whether or not \(p/q\) is in lowest terms, and it's
  basically just annoying, and it's hardly even useful.)

  These two facts, in addition to the definitional fact that \((x^{a/b})^b
  \equiv x^a\), are good indications that this is the right definition. This
  definition will be further justified in the following questions.
 \item
  Prove that for \(x \in \Reals^+\) and \(a/b, c/d \in \Rationals\),
  \(x^{a/b} x^{c/d} \equiv x^{(ad + bc)/bd}\).
  % (x^(a/b) x^(c/d))^(bd) = x^ad x^bc ✓

  Also show that \((x^{a/b})^{c/d} \equiv x^{ac/bd}\).
  % ((x^(a/b))^(c/d))^bd = (x^(a/b))^(bc)
  %                      = x^(ac)

  Also show that \((xy)^{a/b} \equiv x^{a/b} y^{a/b}\).
  % (x^(a/b) y^(a/b))^b = x^a y^a
  %                     = (xy)^a

  (You need not use induction for any of these.)
 \item
  State the Binomial Theorem
  (the expansion of \((a + b)^n\) for any \(n \in \Naturals\)), and define the
  general binomial coefficient.

  Prove that
  \begin{equation*}
   \binom nr + \binom n{r + 1} \equiv \binom{n + 1}{r + 1}
  \end{equation*}
  Hence prove the Binomial Theorem by induction.
 \item
  Prove that \(\Forall x, y \in \Reals : \abs{x + y} \le \abs x + \abs y\), by
  exhausting the different possible signs and magnitudes of \(x\) and \(y\).

  This is called the \emph{triangle inequality}.
 \item
  Let \(y = z - x\) in the triangle inequality to show that
  \(\abs{z - x} \ge \abs z - \abs x\). Also show that
  \(\abs{z - x} \ge \abs x - \abs z\), and hence deduce that
  \(\abs{z - x} \ge \abs[\big]{\abs z - \abs x}\).
 \item
  Let \(a, d, r \in \Reals\) be constants and let \(n \in \Naturals\).
  Guess and prove by induction formulas for
  \begin{align*}
   &\sum_{i = 1}^n \bracks{a + (i - 1)d} \\
   \text{and}\quad&\sum_{i = 1}^n ar^{i - 1}
  \end{align*}
  State for which (if any) values of \(a, d, r\) these formulas fail, and find
  formulas that work in these cases.

  These are known as \emph{arithmetic progressions} and
  \emph{geometric progressions}, respectively.
 \item
  Consider the sum
  \begin{equation*}
   \alpha^n + \alpha^{n - 1}\beta + \alpha^{n - 2}\beta^2 + \dotsb + \beta^n
  \end{equation*}
  Write this as a geometric progression, explicitly stating your values of
  \(a\), \(r\), and \(n\). Hence find and simplify a closed form expression for
  the sum.

  Hence state a formula for the factorisation of the difference of \(n\)th
  powers.
 \item
  For \(n\) odd, by writing \(\beta^n = -(-\beta)^n\), state a formula for the
  factorisation of the sum of powers.
 \item
  Factorise \(\alpha^{12} - \beta^{12}\) into 6 brackets, each with only integer
  coefficients.
 \item
  Find all solutions in \(\Reals^+\) to
  \(\sqrt[\sqrt x]x = {\sqrt x}^{\sqrt x}\).

  Strictly speaking, this question is being a bit silly because I haven't
  defined what most of this means yet. Once you have done Calculus you should be
  able to come up with a more systematic approach, but for now just think of
  this question as a fun little curiosity.
 \item
  Briefly explain why
  \begin{alignat*}3
   \Forall n \in \Naturals :
   \Forall a, b \in \Integers &:{}
   &ab&\equiv ba &&\pmod n \\
   \Forall n \in \Naturals :
   \Forall a, b \in \Integers &:{}
   &a + b&\equiv b + a &&\pmod n
  \end{alignat*}
 \item
  Prove using the definition of \(\equiv\) that
  \begin{equation*}
   \Forall n \in \Naturals :
   \Forall a, b, c \in \Integers :
   b \equiv c \pmod n \implies
   a + b \equiv a + c \pmod n
  \end{equation*}
 \item
  Prove using the definition of \(\equiv\) that
  \begin{equation*}
   \Forall n \in \Naturals :
   \Forall a, b, c \in \Integers :
   b \equiv c \pmod n \implies
   ab \equiv ac \pmod n
  \end{equation*}
 \item
  Prove by induction on \(n\) that
  \begin{equation*}
    \Forall m \in \Naturals :
    \Forall a, b \in \Integers :
    \Forall n \in \Naturals :
    a \equiv b \pmod m \implies
    a^n \equiv b^n \pmod m
  \end{equation*}
 \item
  Hence show that \(7^{4n} \equiv 1 \pmod{10} \Forall n \in \Naturals\), and
  \(7^{2n} \equiv 1 \pmod 4 \Forall n \in \Naturals\). Use
  this to find the last digit of
  \begin{equation*}
   {}^4 7 \defeq 7^{7^{7^7}} = 7^{(7^{(7^7)})}
  \end{equation*}
 \item
  Prove the ``Freshman's Dream'' theorem:

  Let \(p\) be a prime, and let \(a, b \in \Integers\). Then
  \begin{equation*}
   (a + b)^p \equiv a^p + b^p \pmod p
  \end{equation*}
 \item
  Hence prove \emph{Fermat's Little Theorem}:

  Let \(p\) be a prime, and \(a \in \Naturals\). Then
  \begin{equation*}
   a^p \equiv a \pmod p
  \end{equation*}
  by induction on \(a\).
 \item
  Prove that, for all \(n \in \Naturals\), for all \(a, b, c \in \Integers\),
  \begin{itemize}
   \item
    \(a \equiv a \pmod n\)
   \item
    \(a \equiv b \pmod n \implies b \equiv a \pmod n\)
   \item
    \(\text{``\(a \equiv b \pmod n\) and \(b \equiv c \pmod n\)''} \implies
      a \equiv c \pmod n\)
  \end{itemize}
  These three properties are \emph{reflexivity}, \emph{symmetry}, and
  \emph{transitivity}, respectively. Together, they mean that \(\equiv\) is an
  \emph{equivalence relation}.
 \item
  Let \(u_i, v_i \in \Reals\) be two finite sequences of real numbers, for
  \(i \in \set{1, 2, \dotsc, n}\) (ie both have length \(n\)).

  By considering the number of possible roots of the quadratic
  \begin{equation*}
   p(x) = \sum_{i = 1}^n (u_i x - v_i)^2
  \end{equation*}
  show that
  \begin{equation*}
   \parens[\Big]{\sum_{i = 1}^n u_i v_i}^2
   \le \parens[\Big]{\sum_{i = 1}^n u_i^2}
       \parens[\Big]{\sum_{i = 1}^n v_i^2}
  \end{equation*}
  and state the condition for equality to hold.

  This is called the \emph{Cauchy-Schwarz inequality}.
 \item
  Let \(p_i\), \(i = 1, \dotsc, 6\) be the probability that a biased die rolls
  the value \(i\). Use the Cauchy-Schwarz inequality to show that the
  probability of rolling the same value twice in two rolls cannot be made lower
  than \(\frac 16\). Hence show that there is only one possible set of
  probabilities for which this chance is exactly \(\frac 16\).
 \item
  For nonnegative real numbers \(x_i \ge 0\), \(i = 1, \dotsc, n\), define
  \begin{align*}
   M_1'(x_1, \dotsc, x_n) &\defeq
   \frac 1n \sum_{k = 0}^n x_k \\
   M_0'(x_1, \dotsc, x_n) &\defeq
   \parens[\Big]{
    \prod_{k = 0}^n x_k
   }^{\frac 1n}
  \end{align*}
  Prove that for all \(x_1, x_2 \in \Reals_0^+\),
  \begin{equation*}
   M_1'(x_1, x_2) \ge M_0'(x_1, x_2)
  \end{equation*}
  and state the condition for equality to hold.
 \item
  Show that if \(x_i, y_i \ge 0\), \(i = 1, \dotsc, n\), then
  \begin{align*}
   M_1'(M_1'(x_1, \dotsc, x_n), M_1'(y_1, \dotsc, x_n)) &=
   M_1'(x_1, \dotsc, x_n, y_1, \dotsc, y_n) \\
   \text{and}\quad
   M_0'(M_0'(x_1, \dotsc, x_n), M_0'(y_1, \dotsc, x_n)) &=
   M_0'(x_1, \dotsc, x_n, y_1, \dotsc, y_n)
  \end{align*}
  and hence deduce that if for some \(n \in \Naturals\),
  \begin{equation*}
   M_1'(x_1, \dotsc, x_n) \ge M_0'(x_1, \dotsc, x_n)
    \Forall x_1, \dotsc, x_n \in \Reals_0^+
  \end{equation*}
  then
  \begin{equation*}
   M_1'(x_1, \dotsc, x_{2n}) \ge M_0'(x_1, \dotsc, x_{2n})
    \Forall x_1, \dotsc, x_{2n} \in \Reals_0^+
  \end{equation*}
 \item
  Show that if \(x_i \ge 0\), \(i = 1, \dotsc, n\), then
  \begin{align*}
   M_1'(M_1'(x_1, \dotsc, x_n), x_1, \dotsc, x_n) &=
   M_1'(x_1, \dotsc, x_n) \\
   \text{and}\quad
   M_0'(M_0'(x_1, \dotsc, x_n), x_1, \dotsc, x_n) &=
   M_0'(x_1, \dotsc, x_n)
  \end{align*}
  and hence deduce that if for some \(n \in \Naturals_{\ge 2}\),
  \begin{equation*}
   M_1'(x_1, \dotsc, x_n) \ge M_0'(x_1, \dotsc, x_n)
    \Forall x_1, \dotsc, x_n \in \Reals_0^+
  \end{equation*}
  then
  \begin{equation*}
   M_1'(x_1, \dotsc, x_{n - 1}) \ge M_0'(x_1, \dotsc, x_{n - 1})
    \Forall x_1, \dotsc, x_{n - 1} \in \Reals_0^+
  \end{equation*}
 \item
  Prove by induction that
  \begin{equation*}
   M_1'(x_1, \dotsc, x_n) \ge M_0'(x_1, \dotsc, x_n)
    \Forall x_1, \dotsc, x_n \in \Reals_0^+
  \end{equation*}
  for all \(n\) of the form \(2^m\) for some \(m \in \Naturals\).

  Carefully argue what the condition for equality will be for each \(n\).

  Explain why we can now deduce that
  \begin{equation*}
   M_1'(x_1, \dotsc, x_n) \ge M_0'(x_1, \dotsc, x_n)
    \Forall x_1, \dotsc, x_n \in \Reals_0^+
  \end{equation*}
  for all \(n \in \Naturals\). This inequality is often referred to as the
  \emph{arithmetic mean-geometric mean} inequality, or ``AM-GM''.
 \item
  Use AM-GM to find the minimum value of
  \begin{equation*}
   \frac xy + \frac{2y}z + \frac{4z}x
  \end{equation*}
  where \(x, y, z \in \Reals^+\), stating values of \(x, y, z\) that attain this
  minimum.

  Also find the maximum value of
  \begin{equation*}
   (1 + x - y)(2 + y - z)(3 + z - x)
  \end{equation*}
  where each bracket is positive, stating values of \(x, y, z\) that attain this
  maximum.

  If we only require that \(x, y, z \in \Reals^+\), show that there is no
  maximum.
  % let z = 1, x = 5.
  % Then have (6 - y)(2 + y - 1)(3 + 1 - 5)
  %         = (y - 6)(y + 1)
  % which grows arbitrarily large as y → ∞
 \item
  % full credit to sterrrrrrrrrs
  Show that if \(x\) satisfies \(ax^2 + bx + c = 0\) where \(a \ne 0\) and
  \(c \ne 0\), then
  \begin{equation*}
   x = \frac{2c}{-b \pm \sqrt{b^2 - 4ac}}
  \end{equation*}
\end{enumerate}

\section{Trigonometry}

\begin{enumerate}
 \item
  Prove the following identities (you should argue from definitions):
  \begin{itemize}
   \item
    \(\sin(\pi - \theta) \equiv \sin \theta\)
   \item
    \(\cos(\pi - \theta) \equiv - \cos \theta\)
   \item
    \(\sin(\theta + 2\pi) \equiv \sin \theta\)
   \item
    \(\cos(\theta + 2\pi) \equiv \cos \theta\)
   \item
    \(\sin(\frac \pi 2 - \theta) \equiv \cos \theta\)

    (Here you should start with the case \(\theta \in \intcc{0, \frac \pi 2}\)
    and then extend the result to \(\Reals\).)
   \item
    \(\tan(-\theta) \equiv -\tan \theta\)
   \item
    \(\tan(\theta + \pi) \equiv \tan \theta\)
   \item
    \(\tan(\frac \pi 2 - \theta) \equiv \dfrac 1{\tan \theta}\)
  \end{itemize}
  Consider the set of identities from this question, and the identities
  describing the parity of \(\sin\) and \(\cos\).

  Find a locally minimal subset of these identities that generates all the
  others (that is, from which you can prove all the others, but if you remove
  any element of the subset you can no longer prove all the others).
 \item
  Prove Pythagoras' Theorem.

  Hence deduce the Pythagorean identity for \(\sin \theta\) and \(\cos \theta\)
  when \(\theta \in \intcc{0, \frac \pi 2}\):
  \begin{equation*}
   \sin^2 \theta + \cos^2 \theta \equiv 1
  \end{equation*}
  It is possible to show that this holds for all \(\theta\) by doing some boring
  stuff where you consider \(\abs{\sin \theta}\) and \(\abs{\cos \theta}\). I
  won't bore you with it here, you may just assume it.
 \item
  Prove that
  \begin{align*}
   \tan^2 \theta + 1 &\equiv \frac 1{\cos^2 \theta} \\
   \frac 1{\tan^2 \theta} + 1 &\equiv \frac 1{\sin^2 \theta}
  \end{align*}
 \item
  Draw a triangle \(\tri{ABC}\) with \(\lseg{AC}\) horizontal, and
  \(\angle{BAC}\) and \(\angle{BCA}\) acute. Let \(\angle{BAC} = \theta\).

  Label the sides in the conventional way:
  \(\lseg{AB} = c\), \(\lseg{BC} = a\), \(\lseg{CA} = b\).

  Drop an altitude\footnote{
   This is a word for the perpendicular to a side of a triangle the passes
   through the opposite vertex.
  } from \(B\) to the point \(D\) on \(\lseg{AC}\).

  Derive the \emph{cosine rule} for angles less than \(\degrees{90}\) by
  finding the length \(\lseg{BC}\).

  You may now assume that this also holds for \(\theta\) obtuse.
 \item
  In the same triangle, derive the \emph{sine area rule} for angles less than
  \(\degrees{90}\) by finding the area \(\area{\tri{ABC}}\).

  Noting that \(\tri{BCA}\) and \(\tri{CAB}\) are also triangles, immediately
  state two similar sine area rules.

  Hence deduce the \emph{sine rule}.

  You may now assume that both also hold for obtuse angles.
 \item
  Now draw a new triangle, \(\tri{ABC}\) where \(\lseg{BC}\) is vertical, with
  both \(\angle{ABC}\) and \(\angle{ACB}\) acute, and drop an altitude from
  \(A\) to the point \(D\) on \(\lseg{BC}\). Let \(\angle{BAD} = \alpha\),
  \(\angle{CAD} = \beta\).

  Also, let the triangle be scaled so that \(\lseg{AD} = 1\).

  By calculating the length \(\lseg{BC}\) in two different ways, determine the
  \emph{compound angle formula for cosine}\footnote{
   for angles \(< \degrees{90}\)
  }:
  \begin{equation*}
   \cos(\alpha + \beta) \equiv \cos \alpha \cos \beta - \sin \alpha \sin \beta
  \end{equation*}
 \item
  In the same triangle, determine the
  \emph{compound angle formula for
  sine}\footnote{
   for angles \(< \degrees{90}\)
  }, by calculating the area of the triangle in two different ways.
 \item
  Suppose that the compound angle formula for cosine holds for all
  \(\alpha, \beta\).

  Give a different proof of the compound angle formula for sine just using the
  parities of trig functions and the identity
  \begin{equation*}
   \sin x \equiv \cos(\tfrac \pi 2 - x)
  \end{equation*}
  You may now assume that both compound angle formulae hold for all
  \(\alpha, \beta\)\footnote{
   It's quite straightforward to show this by just bashing all the different
   extension cases, but not every interesting.}.

  The latter two identities describe the \emph{parity} of \(\cos\) and \(\sin\).
  Particularly, \(\cos\) is an \emph{even} function, and \(\sin\) is an
  \emph{odd} function.
 \item
  Prove the compound angle formula for tangent:
  \begin{equation*}
   \tan(\alpha + \beta) \equiv
   \frac{\tan \alpha + \tan \beta}{1 - \tan \alpha \tan \beta}
  \end{equation*}
 \item
  Use parities to state compound angle formulae for \(\sin(\alpha - \beta)\),
  \(\cos(\alpha - \beta)\), and \(\tan(\alpha - \beta)\)
 \item
  Write \(\sin 2x\) in terms of \(\sin x\) and \(\cos x\).
 \item
  Also write \(\cos 2x\) in terms of \(\sin x\) and \(\cos x\).

  Then use an identity to write \(\cos 2x\) in terms of just \(\sin x\), and
  also write \(\cos 2x\) in terms of just \(\cos x\).
 \item
  Hence write \(\sin^2 x\) as \(A \cos(ax + b) + B \sin(cx + d) + C\).

  Do the same for \(\cos^2 x\).
 \item
  By considering \(\sin(\alpha + \beta)\) and \(\sin(\alpha - \beta)\), write
  \(\sin \alpha \cos \beta\) as a linear combination\footnote{
   A linear combination of objects is a sum of each of the objects, multiplied
   by some scalar (a real number in this case).
  } of trigonometric functions of linear combinations of \(\alpha\) and
  \(\beta\). This is one of the \emph{product-sum} identities.
 \item
  Do the same for \(\sin \alpha \sin \beta\) and
  \(\cos \alpha \cos \beta\).
 \item
  Consider \(\sin \alpha + \sin \beta\). By finding \(\alpha'\) and \(\beta'\)
  such that \(\alpha' + \beta' = \alpha\) and \(\alpha' - \beta' = \beta\),
  write \(\sin \alpha + \sin \beta\) as a product of trigonometric functions of
  linear combinations of \(\alpha\) and \(\beta\), multiplied by some constant.

  Without doing any more algebra, determine a similar formula for
  \(\sin \alpha - \sin \beta\).

  Also find formulae for \(\cos \alpha + \cos \beta\) and
  \(\cos \alpha - \cos \beta\).
 \item
  Draw triangles to determine the values of
  \(\sin \frac \pi 6\), \(\sin \frac \pi 4\), \(\sin \frac \pi 3\),
  \(\cos \frac \pi 6\), \(\cos \frac \pi 4\), \(\cos \frac \pi 3\),
  \(\tan \frac \pi 6\), \(\tan \frac \pi 4\), \(\tan \frac \pi 3\).
 \item
  Find the values of \(\sin \frac{5\pi}{12}\) and \(\cos \frac{5\pi}{12}\) by
  writing \(5/12\) in a different way.

  Hence deduce the values of \(\sin \frac \pi{12}\) and \(\cos \frac \pi{12}\).
 \item
  If \(\theta = \frac \pi 8\), determine the value of
  \(\cos \theta\) by using an identity for \(\cos 2\theta\).

  Calculate the values of \(\tan \frac \pi 8\) and \(\tan \frac{3\pi}8\).

  Verify your answer by partitioning the regular octagon with unit sides into
  rectangles, squares, and right isosceles triangles, and finding the distances
  from the centre to a vertex and the midpoint of an edge.
 \item
  Show that if \(\theta = \frac \pi 5\), then \(\sin 2\theta = \sin 3\theta\).
  Expand both sides and solve for \(\cos \theta\) to determine the value of
  \(\cos \frac \pi 5\).

  Deduce the value of \(\sin \frac \pi 5\).

  Also show that if \(x\) is the length of any diagonal of a regular pentagon
  with unit sides, then \(x^2 = x + 1\).
 \item \label{q_trig_arcsin}
  Show that the function
  \begin{align*}
   f : \intcc{-1, 1} &\to \intcc{0, \pi} \\
   x &\mapsto \tfrac \pi 2 - \arcsin x
  \end{align*}
  satisfies the definition of \(\arccos x\).
 \item
  Do a rough sketch of \(\cos x + \sin x\).

  By guessing that \(\cos x + \sin x\) can be written in the form
  \(R \sin(x + \alpha)\) where \(R\), \(\alpha\) are constants, find
  suitable constants \(R \in \Reals\) and \(\alpha \in \intco{0, 2\pi}\). Hence
  justify the shape of your sketch, and find the minimum and maximum values.

  Also sketch \(\sqrt{12} \sin 2x + 2 - 4\sin^2 x\), stating its minimum and
  maximum values.
 \item
  If \(r_1 \sin x + r_2 \cos x \equiv R \sin(x + \alpha)\), where
  \(R \ge 0\) and \(\alpha \in \intco{0, 2 \pi}\), show that
  \(R^2 = r_1^2 + r_2^2\) and \(\tan \alpha = r_2 / r_2\).

  Can you deduce the values of both \(R\) and \(\alpha\) given this information?
 \item
  Prove the following identities:
  \begin{alignat*}2
   (\mathrm{i})&&\quad\frac 1{\sin^2 x} + \frac 1{\cos^2 x}
   &\equiv \frac 1{\cos^2 x \sin^2 x} \\
   (\mathrm{ii})&&\quad\cos x + \sin x \tan x
   &\equiv \frac 1{\cos x} \\
   (\mathrm{iii})&&\quad\frac 1{\tan x} + \tan x
   &\equiv \frac 2{\sin 2x} \\
   (\mathrm{iv})&&\quad\sin(x - y) \sin(x + y)
   &\equiv (\sin x - \sin y)(\sin x + \sin y)
  \end{alignat*}
 \item
  Use trigonometric identities to sketch the following:
  \begin{itemize}
   \item
    \(\displaystyle y = \frac{\cos 2x}{\sin(x + \frac \pi 2)}\).
   \item
    \(\sin(x + y) = \sin(x - y)\)
   \item
    \(\sin x = \sin y\)

    This last sketch is actually quite interesting: this is a general solution
    that comes up a lot. For instance, when you're looking at
    \(\sin x = \frac 12\), this is really just \(\sin x = \sin \frac \pi 6\),
    which is a special case of an equation of this form.

    Can you therefore characterise all solutions to \(\sin 13x = \sin 5x\)?
  \end{itemize}
 \item
  By considering
  \begin{itemize}
   \item
    The isosceles triangle with unit legs adjacent to an angle of \(\theta\)
   \item
    The sector of a unit circle subtended by an angle of \(\theta\)
   \item
    The right triangle with unit side adjacent to an angle of \(\theta\)
    (\emph{not} having unit hypotenuse)
  \end{itemize}
  (in that order) and drawing a suitable diagram, show that
  \(\sin \theta < \theta < \tan \theta\) for
  \(\theta \in \intoo{0, \frac \pi 2}\).
 \item
  Hence show that
  \begin{equation*}
   \cos x < \frac{\sin x}x < 1
  \end{equation*}
  for \(x \in \intoo{-\frac \pi 2, \frac \pi 2} \setminus \set{0}\).

  An \emph{extremely} useful consequence of this uses ``the squeeze theorem''.
  Since \(\frac{\sin x} x\) is bounded above and below by functions that tend to
  \(1\) as \(x \to 0\) (from either side of \(0\)),
  \begin{equation*}
   \lim_{x \to 0} \frac{\sin x}x = 1
  \end{equation*}
  We have now somewhat formally shown that ``\(\sin x\) looks roughly like \(x\)
  for small \(x\)''. This has many applications in engineering and physics.
 \item \label{q_trig_weierstrass}
  Show that if \(t = \tan \frac 12 x\), then
  \begin{align*}
   \tan x &\equiv \frac{2t}{1 - t^2} \\
   \sin x &\equiv \frac{2t}{1 + t^2} \\
   \cos x &\equiv \frac{1 - t^2}{1 + t^2}
  \end{align*}
  What is the value of \(\tan \frac 12 x\) if \(\sin x = \frac 5{13}\) and
  \(\cos x = \frac{12}{13}\)?

  Are there infinitely many values of \(\theta \in \intco{0, 2\pi}\) for which
  \(\cos \theta\) and \(\sin \theta\) are both rational?
\end{enumerate}

\section{Calculus}

Throughout, you may assume the following properties of a limit:

\begin{tcolorbox}
 For any functions \(r, s : \Reals \to \Reals\),
 \begin{itemize}
  \item
   If \(\lim_{t \to a} r(t) = b\) and \(\lim_{t \to a} s(t) = c\), then
   \(\lim_{t \to a} \bracks{r(t) + s(t)} = b + c\)
  \item
   If \(\lim_{t \to a} r(t) = b\) and \(\lim_{t \to a} s(t) = c\), then
   \(\lim_{t \to a} \bracks{r(t) s(t)} = bc\).
  \item
   If \(\lim_{t \to a} s(t) = b\) and \(\lim_{t \to b} r(t) = c\), then
   \(\lim_{t \to a} r(s(t)) = c\).
 \end{itemize}
 Here when we say \(\lim_{\ldots} \dotsb = \ell\) we mean that
 the limit exists, is finite, and is equal to \(\ell\).
\end{tcolorbox}

\begin{enumerate}
 \item
  Prove from first principles that, if
  \(a\) is a constant and \(f, g\) are differentiable functions,
  \begin{alignat*}2
   && \dv{x}(af(x)) &= af'(x) \\
   &\text{and}\quad&\dv{x}(f(x) + g(x)) &= f'(x) + g'(x)
  \end{alignat*}
  This means that differentiation is a \emph{linear operator}.
 \item
  Use the Binomial Theorem and the first-principles definition of the derivative
  to prove the \emph{power rule}
  \begin{equation*}
   \dv{x}(x^n) \equiv nx^{n - 1}
  \end{equation*}
  for all \(n \in \Naturals\).
 \item
  For dependent variables \(u = f(x)\), \(v = g(x)\) (where \(f\) and \(g\) are
  differentiable functions), this is the
  \emph{product rule} for differentiation:
  \begin{equation*}
   \dv{x}(uv) \equiv u'v + uv'
  \end{equation*}
  Prove the product rule from first principles, by choosing a suitable place to
  add and subtract a term \(f(x + h)g(x)\).
  %   lim [f(x + h)g(x + h) - f(x + h)g(x) + f(x + h)g(x) - f(x)g(x)] / h
  % = lim [f(x + h)(g(x + h) - g(x)) + g(x)(f(x + h) - f(x)] / h
  % = f(x) g'(x) + g(x) f'(x)
 \item
  Give an alternative proof of the power rule using induction and the product
  rule.
 \item
  For dependent variables \(u = f(x)\), \(v = g(x)\) (where \(f\) and \(g\) are
  infinitely differentiable functions), calculate and simplify expressions for
  \(\dv[2]{x}(uv)\), \(\dv[3]{x}(uv)\), \(\dv[4]{x}(uv)\), \(\dv[5]{x}(uv)\).
  Hence guess a formula for \(\dv[n]{x}(uv)\), for all \(n \in \Naturals\).
  Prove that it is correct by induction.

  (This is called the Leibniz rule, which generalises the product rule).
 \item
  Let \(f, g\) be differentiable functions. Let \(y = f(x) / g(x)\). Multiply
  both sides by \(g(x)\), differentiate both sides, and rearrange for
  \(\dv<y>{x}\) to obtain the \emph{quotient rule}:
  \begin{equation*}
   \dv{x}\parens[\Big]{\frac uv}
    \equiv \frac{vu' - uv'}{v^2}
  \end{equation*}
  This formula is probably worth memorising, but that will come through school.
  %          yv = u
  % ⇒ y'v + yv' = u'
  % ⇒        y' = (u' - yv') / v
  %             = (u' - uv'/v) / v
  %             = (u'v - uv') / v²
 \item
  Let \(n \in \Naturals\). Let \(y = x^{-n}\) (which is defined as \(1 / x^n\)).
  By multiplying both sides by \(x^n\) and then differentiating both sides, show
  that the power rule also holds for all \(n \in \Integers\).

  Use the quotient rule to demonstrate the same result instantly.
 \item
  If \(f\) is a differentiable function and \(k\) is a non-zero constant,
  write down the first-principles definition of \(\dv{x}(f(kx))\). By
  multiplying top and bottom by \(k\) and making an appropriate
  substitution\footnote{
   If \(t\) and \(t'\) are related variables so that
   \(t' \to 0 \implies t \to 0\), then
   \(\lim_{t \to 0} s(t) = \lim_{t' \to 0} s(t)\).
  }, show that it is equal to \(kf'(kx)\).

  Show that this also holds if \(k = 0\).
  %   lim_h [f(kx + kh) - f(kx)] / h
  % = lim_t [f(kx + t) - f(kx)] / (t / k)            | where t = kh
  % = kf'(kx)
 \item
  Use a similar proof to show that if \(f\) is a differentiable function and
  \(m\) and \(c\) are constants, then \(\dv{x}(f(mx + c)) \equiv mf'(mx + c)\).
 \item
  If \(g\) is a differentiable function, find the equation of the tangent to the
  graph \(y = g(x)\) at the point where \(x = x_0\), in the form
  \(y = mx + c\). By assuming that as \(x \to x_0\), \(g(x) \approx mx + c\),
  deduce the \emph{chain rule}:
  \begin{equation*}
   \dv{x}(f(g(x))) \equiv g'(x) f'(g(x))
  \end{equation*}
  Some people write it in the following form:
  \begin{equation*}
   \dv<y>{x} \equiv \dv<y>{u} \cdot \dv<u>{x}
  \end{equation*}
  where we let \(u = g(x)\) and \(y = f(u)\), to see that
  \(\dv<y>{x}
    = \dv{x}(u) \cdot f'(u)
    = \dv<u>{x} \cdot \dv{u}(y)\).

  This one is perhaps easier to remember, because it looks like you're just
  ``cancelling the \(\diff u\,\)s''. Of course, this is not formally what is
  happening as these are not fractions.
 \item
  Let \(y = x^{p/q}\) where \(p, q \in \Integers, q \ne 0\). By raising both
  sides to the power \(q\) and then differentiating both sides (using the chain
  rule), show that the power rule also holds for all \(p / q \in \Rationals\).
  %              y^q = x^p
  % ⇒ q y^(q - 1) y' = p x^(p - 1)
  %             ⇒ y' = (p / q) x^(p - 1)  y^(1 - q)
  %                  = (p / q) x^(p - 1)  x^((p / q)(1 - q))
  %                  = (p / q) x^((p / q)(1 - q) + p - 1)
  %                  = (p / q) x^((p(1 - q) + q(p - 1)) / q)
  %                  = (p / q) x^((p - q) / q)
  %                  = (p / q) x^((p / q) - 1)

  Here \(x^{p/q}\), where \(p/q\) is in lowest terms, is defined as ``the real
  number \(t\) such that \(t^q = x^p\)'', which is the only property you need
  for this question.

  If \(x\) is positive and \(p\) is even, there are two such numbers. The
  positive one is then taken.

  If \(x\) is negative and \(q\) is even there is no such number, so the
  operation is then undefined. It can be shown that otherwise there will exist
  such a number in \(\Reals\).
 \item
  If \(y = \sqrt x\), use the chain rule to show that \(y \dv<y>{x}\) is
  constant.
 \item
  If \(y = \sqrt{\sin x}\), calculate \(y \dv<y>{x}\) using the chain rule.

  In general, determine a formula for \(\dv{x}(y^2)\).
 \item
  Let \(f\) a function that is differentiable at \(x_0\), and \(y = f(x)\) be a
  graph of that function.

  Let \(h\) be a constant. Find a formula for the gradient of the line through
  the points where \(x = x_0 - h\) and \(x = x_0 + h\).
 \item
  Use the following facts to find the derivative of \(\sin x\):
  \begin{itemize}
   \item
    \(\sin x\) is differentiable.
   \item
    If \(f\) is a function that is differentiable at \(x_0\), then, on the graph
    \(y = f(x)\), the gradient of the line through the points where
    \(x = x_0 - h\) and \(x = x_0 + h\) approaches the gradient of the tangent
    at \(x_0\) as \(h \to 0\).
   \item
    The compound angle formula for \(\sin\).
   \item
    \(\displaystyle
     \lim_{t \to 0} \frac{\sin t} t = 1
     \)
  \end{itemize}
 \item
  Hence use a trig identity and the chain rule to deduce the derivative of
  \(\cos x\), without evaluating any limits.
 \item
  Find the derivative of \(\tan x\). Write it in terms of \(\tan x\).
 \item \label{q_calc_arcsin}
  If \(y = \arcsin x\), then use implicit differentiation\footnote{
   This is the name of the trick where you apply a function to both sides, so
   you can use the chain rule.
  } to find
  \(\dv<y>{x}\) in terms of \(y\).

  Since \(-\frac \pi 2 \le y \le \frac \pi 2\), \(\cos y \ge 0\). Hence use the
  pythagorean identity to write the derivative in terms of \(\sin y\), and
  therefore write it in terms of \(x\).
  %    sin y = x
  % y' cos y = 1
  %       y' = 1 / cos y
  %          = 1 / sqrt(1 - x²)
 \item
  Use question \ref{q_trig_arcsin} from Trigonometry to immediately deduce the
  derivative of \(\arccos x\).
 \item
  Find by similar methods to question \ref{q_calc_arcsin} the derivative of
  \(\arctan x\).
 \item
  Show that if \(f\) and \(g\) are continuous functions and
  \(a, b, \alpha, \beta \in \Reals\) are constant, then
  \begin{equation*}
   \integ[a]<b>{(\alpha f(x) + \beta g(x))}{x} \equiv
   \alpha \integ[a]<b>{f(x)}{x} +
   \beta  \integ[a]<b>{g(x)}{x}
  \end{equation*}
  by using the Fundamental Theorem of Calculus and considering \(F, G\) such
  that \(F'(x) = f(x)\) and \(G'(x) = g(x)\), and letting
  \(L(x) \defeq \alpha F(x) + \beta G(x)\).
 \item
  Also use the Fundamental Theorem of Calculus to show that if \(f\) is
  continuous and \(a, b, c \in \Reals\) are constant, then
  \begin{equation*}
   \integ[a]<b>{f(x)}{x} + \integ[b]<c>{f(x)}{x} \equiv
   \integ[a]<c>{f(x)}{x}
  \end{equation*}
  This is not how these facts are normally proved (in fact, you would probably
  need them to prove the Theorem itself). If you think about areas, they should
  sound quite reasonable.
 \item \label{q_calc_int_pwr}
  If \(\alpha \in \Rationals\), use the power rule to work out
  \begin{equation*}
   \integ{x^\alpha}{x}
  \end{equation*}
  State the value of \(\alpha\) for which your answer breaks down.
 \item
  If \(f, g\) are continuous functions, show that
  \begin{equation*}
   \integ[a]<b>{g'(x) f(g(x))}{x} \equiv
   \integ[g(a)]<g(b)>{f(x)}{x}
  \end{equation*}
  Hence evaluate
  \begin{itemize}
   \item \(\displaystyle
    \integ[a]<b>{f(kx)}{x}
    \)

    where \(k\) is constant. Convince yourself by means of a sketch that your
    answer is correct.
   \item \(\displaystyle
    \integ[0]<\sqrt \pi>{2x \sin(x^2)}{x}
    \)
   \item \(\displaystyle
    \integ{\frac x{\sqrt{x^2 + 1}}}{x}
    \)
  \end{itemize}
  This technique is called \emph{integration by substitution}.
 \item
  Find the mistake in the reasoning:
  \begin{tcolorbox}
   Let
   \begin{equation*}
    I \defeq \integ[0]<\pi>{\sin x}{x}
   \end{equation*}
   This can be evaluated directly as \(\eval{0}{\pi}{-\cos x} = 2\).

   Also, the function \(\sin x\) can be written as \(g'(x) f(g(x))\) where
   \begin{align*}
    g(x) &\defeq \sin x \\
    f(x) &\defeq \frac x{\cos(\arcsin x)}
   \end{align*}
   so therefore
   \begin{equation*}
    I = \integ[\sin 0]<\sin \pi>{f(x)}{x} = 0
   \end{equation*}
   and we have \(0 = 2\).
  \end{tcolorbox}
 \item
  Abel and Bernstein are practising integration by substitution. They get the
  question:
  \begin{tcolorbox}
   Find
   \begin{equation*}
    \integ{\frac{2\tan x}{\cos^2 x}}{x}
   \end{equation*}
  \end{tcolorbox}
  Here is A's solution:
  \begin{tcolorbox}
   The integrand can be written in the form \(g'(x) f(g(x))\) where
   \begin{alignat*}2
    &&g(x) &\defeq \frac 1{\cos x} \\
    \implies{}&& g'(x) &=
     \frac{\sin x}{\cos^2 x} =
     \frac{\tan x}{\cos x} \\
    &&f(x) &\defeq 2x
   \end{alignat*}
   so the antiderivative is
   \begin{align*}
    \integ[x_0]<x>{\frac{2\tan t}{\cos^2 t}}{t}
     &= \integ[x_0']<1/\cos x>{2t}{t} \\
     &= \eval{x_0'}{1/\cos x}{t^2} \\
     &= \frac 1{\cos^2 x} + C
   \end{align*}
  \end{tcolorbox}
  Here is B's solution:
  \begin{tcolorbox}
   The integrand can be written in the form \(g'(x) f(g(x))\) where
   \begin{alignat*}2
    &&g(x) &\defeq \tan x \\
    \implies{}&& g'(x) &= \frac 1{\cos^2 x} \\
    &&f(x) &\defeq 2x
   \end{alignat*}
   so the antiderivative is
   \begin{align*}
    \integ[x_0]<x>{\frac{2\tan t}{\cos^2 t}}{t}
     &= \integ[x_0']<\tan x>{2t}{t} \\
     &= \eval{x_0'}{\tan x}{t^2} \\
     &= \tan^2 x + C
   \end{align*}
  \end{tcolorbox}
  Who is correct?
 \item
  If \(u\) and \(v\) are variables dependent on \(x\), use the product rule to
  show that
  \begin{equation*}
   \integ[a]<b>{uv'}{x} = \eval{a}{b}{uv} - \integ[a]<b>{u'v}{x}
  \end{equation*}
  Hence show that
  \begin{equation*}
   \lim_{t \to \infty}
    \bracks[\Big]{
     \integ[0]<t>{x^{n + 1} e^{-x}}{x}
    } =
   \lim_{t \to \infty}
    \bracks[\Big]{
     (n + 1)\integ[0]<t>{x^n e^{-x}}{x}
    }
  \end{equation*}
  and hence prove that for \(n \in \Naturals_0\),
  \begin{equation*}
   n! \equiv \lim_{t \to \infty}
    \bracks[\Big]{
     \integ[0]<t>{x^n e^{-x}}{x}
    }
  \end{equation*}
  This technique is called \emph{integration by parts}.
 \item
  This question is motivated by question \ref{q_calc_int_pwr}.

  Define
  \begin{align*}
   \ell : \Reals^+ &\to \Reals\\
   x &\mapsto \integ[1]<x>{\frac 1t}{t}
  \end{align*}
  By means of a substitution, show that for \(a, b \in \Reals^+\) constant,
  \begin{equation*}
   \ell(a) \equiv \integ[b]<ab>{\frac 1t}{t}
  \end{equation*}
  and deduce that \(\ell(a) + \ell(b) \equiv \ell(ab)\).
 \item
  By means of a different substitution, show that for \(a \in \Reals^+\) and
  \(n \in \Rationals\), \(n \cdot \ell(a) \equiv \ell(a^n)\).
 \item
  By rewriting \(-\ell(b)\), find a formula for
  \(\ell(a) - \ell(b)\) involving only one \(\ell\).
 \item
  Explain briefly why \(a < b \implies \ell(a) < \ell(b)\)\footnote{
   This means that \(\ell\) is \emph{monotonic increasing}.
  }.

  Draw a sketch to show that \(\frac 12 \le \ell(2) \le 1\). Hence show that
  \(\frac n2 \le \ell(2^n) \le n\), and deduce that as \(x \to \infty\),
  \(\ell(x) \to \infty\).

  Also show that as \(x \to 0^+\), \(\ell(x) \to -\infty\).
 \item
  By means of a sketch, show that for \(n \in \Naturals\),
  \begin{equation*}
   \ell(n) \le \sum_{k = 1}^n \frac 1k \le 1 + \ell(n)
  \end{equation*}
  and deduce that the harmonic series (the above sum) does not converge.
 \item
  Since \(\ell\) is monotonic increasing, and it goes from \(-\infty\) and
  \(\infty\), it is a bijection\footnote{
   This fact is fairly easy to appreciate, but not so easy to prove. We will
   just assume it here.
  } between \(\Reals^+\) and \(\Reals\). Therefore
  it has an inverse, so let's define \(\xi: \Reals \to \Reals^+\) to be such
  that
  \begin{equation*}
   \ell(\xi(x)) = x \Forall x \in \Reals
  \end{equation*}
  For \(a, b \in \Reals\), show that \(\xi(a + b) \equiv \xi(a) \xi(b)\).

  Here you can use the identity defining \(\xi\) to write \(a\) and \(b\) in a
  different form.

  Also show that for \(n \in \Rationals\), \((\xi(a))^n \equiv \xi(an)\).
 \item
  Show that if \(t = \xi(\frac pq \cdot \ell(x))\) then \(t^q = x^p\).

  This means that \(\xi(n \cdot \ell(x))\) coincides with \(x^n\) for all
  rational \(n\) when \(x\) is positive. Since \(\xi\) and \(\ell\) are both
  quite nice, ``smooth'' functions, it is nice to \emph{define}
  \begin{equation*}
   x^t \defeq \xi(t \cdot \ell(x))
  \end{equation*}
  for \(t \in \Reals\) and \(x\) positive. This agrees with the old definitions,
  and ensures various nice continuity properties.

  If we \emph{define}\footnote{
   One should stress again that this is not the only definition of \(e\)
   (although any correct alternative definition should imply that this one is
   also true)
  } \(e \defeq \xi(1)\), then
  \(e^x \equiv \xi(x \cdot \ell(\xi(1))) \equiv \xi(x)\). So in fact \(\xi\) has
  been the famous ``exponential function'' all this time! This is one of the
  reasons that the number \(e\) is significant. From now on I will write \(e^x\)
  or \(\exp x\) instead of \(\xi(x)\), and \(\ln x\) or
  \(\log x\)\footnote{
   standing for ``natural logarithm'' (logarithmus naturalis) and ``logarithm'',
   respectively.
  }
  instead of \(\ell(x)\).
 \item
  If \(f\) is a positive, differentiable function, determine the derivative of
  \(\ln(f(x))\).
 \item
  Find the derivative of \(e^x\) (aka \(\xi(x)\)), using the definition of
  \(\ln x\) (aka \(\ell(x)\)).
 \item
  Hence find three different functions that are solutions to the
  \emph{differential equation}
  \begin{equation*}
   f'(x) = f(x)
  \end{equation*}
  Find three functions that are solutions to
  \begin{equation*}
   f'(x) = 2f(x)
  \end{equation*}
  Can you also find two solutions \(f_1, f_2\) to the equation
  \begin{equation*}
   f''(x) + 5f'(x) + 6f(x) = 0
  \end{equation*}
  and a constant \(x_0\) such that
  \(f_1(x_0) f_2'(x_0) - f_1'(x_0) f_2(x_0) \ne 0\)?
 \item
  Let \(a \in \Reals^+\) be constant. Find the derivative of \(a^x\) from the
  definition.
 \item
  Let \(t \in \Reals\) be constant. Show that the power rule for \(x^t\) holds
  for all \(t \in \Reals\) when \(x > 0\).

  This is the best we can do for now, since it's not so nice to define
  irrational powers on negative numbers (eg what should \((-1)^\pi\)
  be\footnote{
   In fact, if you were to define it it would almost certainly have to be a
   complex number.
  }?)

  An alternative route to get this result is to just define
  \(x^t\) for \(t \in \Reals\) as
  \begin{equation*}
   \lim_{k \to \infty} x^{(a_k)}
  \end{equation*}
  where \(a_k\) is some rational sequence converging to \(t\). This whole
  approach depends a little more subtly on how you've actually define the real
  numbers. It is basically an appeal to the continuity of the function
  \(x^t\).
 \item
  Show that \(x^a x^b \equiv x^{a + b}\) holds for \(x \in \Reals^+\),
  \(a, b \in \Reals\).

  Also show that \((x^a)^n \equiv x^{an}\) if \(n \in \Naturals\).
 \item
  By writing \(\ln(uv) = \ln u + \ln v\), deduce a weak version of the product
  rule from the chain rule.

  This is a weak version because it is only valid for \(u, v > 0\). Can you use
  it, or use a similar proof, to deduce a weak product rule for
  \(u > 0\) and \(v < 0\)?
 \item
  Suppose that the function \(e^x\) can be written in the form
  \begin{align*}
   e^x
    &= a_0 + a_1 x + a_2 x^2 + \dotsb \\
    &= \sum_{k = 0}^\infty a_k x^k
  \end{align*}
  where \(a_k\) are constants to be determined. This is called a
  \emph{power series}. This particular type of power series is called a
  \emph{Taylor series}, and in fact this is a particular type of Taylor series
  called a \emph{Maclaurin series}.

  By letting \(x = 0\), deduce what \(a_0\) must be.

  By differentiating both sides, deduce what \(a_1\) must be.

  Find a \(k\)th term for \(a_k\).

  Assuming this series converges, show that
  \begin{equation*}
   e =
   \sum_{k = 0}^\infty \frac 1{k!} =
   1 + \frac 1{1!} + \frac 1{2!} + \frac 1{3!} + \dotsb
  \end{equation*}
  This series is in fact a popular alternative definition of \(e^x\) (and
  therefore also \(e\)). There is also the weird product formula
  \begin{equation*}
   e^x =
    \lim_{n \to \infty}
     \parens[\Big]{
      1 + \frac xn
     }^n
  \end{equation*}
  which the author thinks is weird\footnote{
   read: doesn't really understand
  }.
 \item
  Also determine Maclaurin series for \(\sin x\) and \(\cos x\).
 \item
  Determine the Maclaurin series of \((1 + x)^{-1}\).
 \item
  By writing \(t = x^2\), show that the Maclaurin series of \((1 + x^2)^{-1}\)
  is
  \begin{equation*}
   \sum_{k = 0}^\infty (-1)^k x^{2k} = 1 - x^2 + x^4 - x^6 + x^8 + \dotsb
  \end{equation*}
  By assuming that this series converges when \(x \in \intcc{0, 1}\) and
  calculating
  \begin{equation*}
   \integ[0]<1>{\frac 1{1 + x^2}}{x}
  \end{equation*}
  in two different ways, show that
  \begin{equation*}
   \frac \pi 4 = 1 - \frac 13 + \frac 15 - \frac 17 + \frac 19 + \dotsb
  \end{equation*}
  You may assume that to integrate a series you can integrate each term
  separately.
 \item
  Suppose that \(f\) is an infinitely differentiable function. Write down its
  Maclaurin series.
 \item
  Suppose that \(f\) is an infinitely differentiable function, and that it can
  be written in the form
  \begin{align*}
   f(x) &= a_0 + a_1(x - x_0) + a_2(x - x_0)^2 + \dotsb \\
        &= \sum_{k = 0}^\infty a_k(x - x_0)^k
  \end{align*}
  where \(x_0\) is some constant, and \(a_k\) are constants to be determined.
  Determine the constants.

  This is the full \emph{Taylor series} of \(f\).

  Beware that not every function is equal to all of its Taylor series at all
  points.
 \item
  Find \(\integ{\sin^2 x}{x}\).
 \item
  Find \(\integ{\sin 7x \cos 13 x}{x}\).
 \item
  Use the substitution \(g(x) = \cos x\) to determine \(\integ{\tan x}{x}\).
 \item
  Show that if \(t = \tan \frac 12 x\), then
  \begin{equation*}
   \dv<t>{x} = \frac 2{1 + t^2}
  \end{equation*}
  Use this in addition to an identity from question \ref{q_trig_weierstrass}
  in Trigonometry to determine
  \begin{equation*}
   \integ{\frac 1{\sin x}}{x}
  \end{equation*}
  This technique is called the \emph{Weierstrass substitution}. It is generally
  used to transform any integral involving trigonometric functions into an
  integral of a rational function (which unfortunately doesn't always make it
  look much easier).
\end{enumerate}

\section{Complex Numbers}

\begin{enumerate}
 \item
  Show that addition and multiplication in the complex numbers are commutative
  and associative.
 \item
  Show that multiplication distributes over addition in the complex numbers.

  (This means that
  \begin{equation*}
   a(b + c) = ab + ac \Forall a, b, c \in \Complex)
  \end{equation*}
 \item
  Show that if \(x\) is real, then \(x(z_1, z_2) \equiv (x z_1, x z_2)\).
 \item
  Show that if \(x, y \in \Reals\), then \((x, y) \equiv x + i y\).

  This gives rise to the conventional notation \(x + iy\) for an arbitrary
  complex number. That is, if you are given \(z \in \Complex\), it fine to say
  ``let \(x, y \in \Reals\) be such that \(z = x + iy\).''
 \item
  Show that \(i^2 = -1\).
 \item
  Show that if \(z, w \in \Complex\), then \(\abs{zw} = \abs z \abs w\).
 \item
  If \(x, y \in \Reals\) and \(z = x + iy\), find a \(w \in \Complex\) such that
  \(zw = \abs{z}^2\). \(w\) is usually written \(\conj z\).

  If \(z \ne 0\), find a complex number \(z^{-1} \in \Complex\) such that
  \(zz^{-1} = 1\). Write it in terms of \(z\) and \(\conj z\).
 \item
  Given two complex numbers
  \begin{align*}
   z &= r_1(\cos \theta_1 + i \sin \theta_1) \\
   w &= r_2(\cos \theta_2 + i \sin \theta_2)
  \end{align*}
  where \(r_1, r_2, \theta_1, \theta_2 \in \Reals\),
  calculate the values of \(\abs{z}\) and \(\abs{w}\).

  Draw a diagram in the complex plane\footnote{
   The complex plane is a two-dimensional plane with ``real part'' running along
   the horizontal axis, and ``imaginary part'' running along the vertical axis,
   used to represent complex numbers.
  } to interpret the values of
  \(r_1\), \(r_2\), \(\theta_1\), and \(\theta_2\).

  Calculate \(zw\) and write it in the form
  \(R_1 \cos \phi_1 + R_2 \sin \phi_2\). Hence calculate \(\abs{zw}\) and draw
  \(zw\) on your diagram.

  Could \(z\) be any complex number? Write down a pair of values
  \((r_1, \theta_1)\) so that \(z = -1 - \sqrt 3i\).

  Write down another such pair of values for which \(r_1\) is the same but
  \(\theta_1\) is different. Also write down another pair for which
  \(\theta_1\) is the same but \(r_1\) is different.
 \item
  Carefully write down the first eight terms of \(e^{ix}\). By comparing with
  the Maclaurin series of \(\sin x\) and \(\cos x\), write down a sensible
  formula for \(e^{ix}\), of the form \(\alpha \cos x + \beta \sin x\) where
  \(\alpha, \beta \in \Complex\).
 \item
  You may now assume that this formula is correct. You may also assume that
  \((e^z)^w = e^{zw}\) still holds \(\Forall z, w \in \Complex\).

  Write down the values of \(e^{i\pi}\), \(e^{i\pi / 2}\),
  \(e^{i\pi / 4}\), \(e^{i\pi / 3}\), and \(e^{2 i \pi}\).
 \item
  Use the formula to express
  \((\cos \theta + i \sin \theta)^n\) in the form \(\cos \phi + i \sin \phi\).

  This is called ``de Moivre's Theorem''.

  Give a more rigorous proof of de Moivre's Theorem by induction.
 \item
  By considering the real part of
  \(e^{5ix}\), deduce a formula for \(\cos 5x\) in terms of \(\cos x\).

  You may wish to write \(c = \cos x\) for convenience.
 \item
  Write down \(\alpha_1, \alpha_2, \beta_1, \beta_2 \in \Complex\) such that
  \begin{align*}
   e^{ix} &= \alpha_1 \cos x + \beta_1 \sin x \\
   e^{-ix} &= \alpha_2 \cos x + \beta_2 \sin x
  \end{align*}
  Solve this system of equations to find nice formulae for
  \(\sin x\) and \(\cos x\) in terms of \(e^{\pm ix}\).
 \item
  Hence find a solution \(z \in \Complex\) to \(\sin z = 2\).

  Is this the only solution?
 \item
  We write \(\Integers[i]\) to denote the Gaussian integers, which is the set of
  complex numbers with integer real and imaginary parts - ie
  \(\set{a + bi : a, b \in \Integers}\).

  Let \(n \in \Integers[i]\). Show that \(\abs{n}^2 \in \Integers\).

  Explain why \(n^2\) corresponds to a Pythagorean triple.
\end{enumerate}

\section{Matrices}

This section unfortunately does not make any attempt to define anything, as I
can only think of one worthwhile question. When you have had the necessary
definitions and tools defined for you, you may attempt it.
\begin{enumerate}
 \item
  Let \(\vec f_n\) be a sequence of vectors defined as
  \begin{equation*}
   \vec f_n \defeq
   \begin{cases*}
    \begin{pmatrix} 0 \\ 1 \end{pmatrix} &
     if \(n = 0\) \\
    \begin{pmatrix} 0 & 1 \\ 1 & 1 \end{pmatrix}\vec f_{n - 1} &
     for \(n \in \Naturals\)
   \end{cases*}
  \end{equation*}
  Show that
  \(\vec f_{n + 2} = \vec f_{n + 1} + \vec f_n\) for all
  \(n \in \Naturals\), and that \((\vec f_1)_1 = (\vec f_2)_1 = 1\). By
  writing \(\begin{psmallmatrix} 0 & 1 \\ 1 & 1 \end{psmallmatrix}\) as
  \(\mat U \mat D \mat U^{-1}\) where \(\mat D\) is a diagonal matrix,
  find a closed form for the \(n\)th term of the Fibonacci sequence.
\end{enumerate}

\end{document}
