\documentclass[a4paper,12pt]{article}
\author{Izaak van Dongen}
\title{Recurring concepts in STEP}

\usepackage{mysty}
\usepackage{mymaths}

% Embed source files into PDF in case of loss. You can view or extract the
% source files by doing `pdfdetach -list <file.pdf>` or
% `pdfdetach -saveall <file.pdf>`, using pdfdetach from poppler, or some other
% suitable method.
\usepackage[main]{embedall}
\embedfile{mymaths.sty}
\embedfile{mysty.sty}

\begin{document}
\maketitle
\tableofcontents

%TODO: t = tan x/2 for sec and cosec integrals

\section{Introduction}

Here I list some skills/tricks/concepts/fundamentals that I think are useful in
STEP. I provide accompanying actual questions.

The questions here come from a very broad range of years and papers.
Particularly, from
\begin{multicols}3
 \begin{itemize}
  \item 1994, paper I
  \item 1995, paper I
  \item 1996, paper II
  \item 1997, paper I
  \item 1997, paper III
  \item 1998, paper I
  \item 2000, paper I
  \item 2000, paper II
  \item 2002, paper I
  \item 2005, paper I
  \item 2006, paper II
  \item 2006, paper III
  \item 2007, paper I
  \item 2007, paper II
  \item 2007, paper III
  \item 2008, paper I
  \item 2009, paper I
  \item 2010, paper II
  \item 2011, paper II
  \item 2011, paper III
  \item 2012, paper II
  \item 2012, paper III
  \item 2013, paper III
  \item 2015, paper I
  \item 2015, paper II
  \item 2015, paper III
  \item 2017, paper II
  \item 2017, paper III
 \end{itemize}
\end{multicols}
It is worth giving some consideration to which papers you wish to save to use as
a full practice paper (or particularly if any of your teachers had any papers in
mind to use as practice...). However, it is always possible to still do a paper
as a full practice paper even if you've done a question from it. Think of this
as challenge mode.

There is likely some overlap between these questions and the questions that
Clive Johnson likes to set. He won't mind if you've done them already and just
hand them in though, or if you pick some different questions.

As an author of a document relating to STEP, I am obliged to remind the reader
that in STEP it is \emph{crucial} to present a clear argument in your answer,
not just a calculation.

We begin.

\section{Integration}

\subsection{Particular substitutions exploiting symmetry}

This section concerns substitutions such as \(t = x + \sqrt{1 + x^2}\).
(Remember the trick here was to spot that \(1/t\) simplifies nicely!)

This substitution is particularly powerful because we can write \(x\) and
\(\sqrt{1 + x^2}\) as rational functions of \(t\). So if you have any rational
function of \(x\) and \(\sqrt{1 + x^2}\), this substitution transforms its
integral into a rational function of \(t\).

\subsubsection{1994, Paper I, Question 8}
\begin{center}
 \includegraphics[width=0.8\textwidth]{screenshots/1994_I_8.png}
\end{center}

\subsubsection{1998, Paper I, Question 2}
\begin{center}
 \includegraphics[width=0.9\textwidth]{screenshots/1998_I_2.png}
\end{center}

\subsubsection{2006, Paper II, Question 4}
\begin{center}
 \includegraphics[width=0.8\textwidth]{screenshots/2006_II_4.png}
\end{center}

\subsubsection{2009, Paper I, Question 6}
\begin{center}
 \includegraphics[width=0.9\textwidth]{screenshots/2009_I_6.png}
\end{center}

\subsubsection{2010, Paper II, Question 4}
\begin{center}
 \includegraphics[width=\textwidth]{screenshots/2010_II_4.png}
\end{center}

\subsubsection{2012, Paper II, Question 3}
\begin{center}
 \includegraphics[width=0.9\textwidth]{screenshots/2012_II_3.png}
\end{center}

\subsubsection{2015, Paper II, Question 6}
\begin{center}
 \includegraphics[width=0.8\textwidth]{screenshots/2015_II_6.png}
\end{center}

\subsubsection{2015, Paper III, Question 1}
\begin{center}
 \includegraphics[width=\textwidth]{screenshots/2015_III_1.png}
\end{center}

\subsection{The tangent half-angle substitution}

This section concerns the substitution \(t = \tan \tfrac 12 x\).

As the below questions show, this substitution is particularly powerful because
it transforms the integral of any rational function of trigonometric functions
into a rational function of \(t\).

\subsubsection{Deriving formula book equations}

Use the substitution \(t = \tan \tfrac 12 x\) to show directly that
\begin{align*}
 \int \csc x\,\mathrm dx = \ln \abs{\tan \tfrac 12 x} + C,&&
 \int \sec x\,\mathrm dx = 2 \ln \abs*{\frac{1 + t}{1 - t}} + C.
\end{align*}
Prove the trigonometric identities
\begin{equation*}
 \frac{1 + t}{1 - t} = \sec x + \tan x = \tan(\tfrac 12 x + \tfrac 14 \pi)
\end{equation*}
and
\begin{equation*}
 \tan \tfrac 12 x = \csc x - \cot x.
\end{equation*}

\subsubsection{2000, Paper II, Question 6}
\begin{center}
 \includegraphics[width=0.95\textwidth]{screenshots/2000_II_6.png}
\end{center}

\subsubsection{2013, Paper III, Question 1}
\begin{center}
 \includegraphics[width=0.95\textwidth]{screenshots/2013_III_1.png}
\end{center}

\subsection{Taking clever linear combinations}

\subsubsection{1995, Paper I, Question 2}
\begin{center}
 \includegraphics[width=0.9\textwidth]{screenshots/1995_I_2.png}
\end{center}

\subsubsection{2002, Paper I, Question 7}
\begin{center}
 \includegraphics[width=\textwidth]{screenshots/2002_I_7.png}
\end{center}

\subsubsection{2006, Paper III, Question 2}
\begin{center}
 \includegraphics[width=\textwidth]{screenshots/2006_III_2.png}
\end{center}

\subsubsection{2007, Paper I, Question 3}
\begin{center}
 \includegraphics[width=\textwidth]{screenshots/2007_I_3.png}
\end{center}

\subsubsection{2007, Paper III, Question 3}
\begin{center}
 \includegraphics[width=\textwidth]{screenshots/2007_III_3.png}
\end{center}

\subsection{Indirectly establishing equalities between integrals}

\subsubsection{2007, Paper III, Question 7}
\begin{center}
 \includegraphics[width=\textwidth]{screenshots/2007_III_7.png}
\end{center}

\subsubsection{2011, Paper III, Question 6}
\begin{center}
 \includegraphics[width=0.95\textwidth]{screenshots/2011_III_6.png}
\end{center}

\subsubsection{2017, Paper III, Question 6}
\begin{center}
 \includegraphics[width=\textwidth]{screenshots/2017_III_6.png}
\end{center}

\subsection{Sum-product identities}

I've only found one really good question that uses this, but the idea this
question uses to deal with products of trig functions inside integrals is
\emph{really really important}.

\subsubsection{2009, Paper I, Question 7}
\begin{center}
 \includegraphics[width=0.9\textwidth]{screenshots/2009_I_7.png}
\end{center}

\subsection{This question}

I can't spoil the question, but there's a \emph{crucial} property of this
expression for \(f''(x)\) that you would profit greatly from being aware of.

\subsubsection{2011, Paper II, Question 6}
\begin{center}
 \includegraphics[width=\textwidth]{screenshots/2011_II_6.png}
\end{center}

\subsection{Integrating step functions}

\subsubsection{2000, Paper I, Question 3}
\begin{center}
 \includegraphics[width=\textwidth]{screenshots/2000_I_3.png}
\end{center}

\subsection{Miscellaneous}

\subsubsection{1997, Paper I, Question 7}
\begin{center}
 \includegraphics[width=0.9\textwidth]{screenshots/1997_I_7.png}
\end{center}

\section{Fibonacci numbers and induction}

\subsubsection{1996, Paper II, Question 3}
\begin{center}
 \includegraphics[width=\textwidth]{screenshots/1996_II_3.png}
\end{center}

\subsubsection{2010, Paper II, Question 3}
\begin{center}
 \includegraphics[width=\textwidth]{screenshots/2010_II_3.png}
\end{center}

\subsubsection{2012, Paper III, Question 8}
\begin{center}
 \includegraphics[width=\textwidth]{screenshots/2012_III_8.png}
\end{center}

\section{Working with inequalities}

\subsubsection{1997, Paper III, Question 4}
\begin{center}
 \includegraphics[width=\textwidth]{screenshots/1997_III_4.png}
\end{center}

\subsubsection{2007, Paper II, Question 7}
\begin{center}
 \includegraphics[width=\textwidth]{screenshots/2007_II_7.png}
\end{center}

\subsubsection{2017, Paper II, Question 4}
\begin{center}
 \includegraphics[width=\textwidth]{screenshots/2017_II_4.png}
\end{center}

\section{Working with definitions}

\subsubsection{2006, Paper III, Question 8}
\begin{center}
 \includegraphics[width=\textwidth]{screenshots/2006_III_8.png}
\end{center}

\subsubsection{2008, Paper I, Question 1}
\begin{center}
 \includegraphics[width=\textwidth]{screenshots/2008_I_1.png}
\end{center}

\subsubsection{2012, Paper II, Question 2}
\begin{center}
 \includegraphics[width=\textwidth]{screenshots/2012_II_2.png}
\end{center}

\subsubsection{2015, Paper III, Question 2}
\begin{center}
 \includegraphics[width=\textwidth]{screenshots/2015_III_2.png}
\end{center}

\section{Miscellaneous}

\subsection{Irrationality of \(e\)}

\subsubsection{1997, Paper III, Question 7}
\begin{center}
 \includegraphics[width=\textwidth]{screenshots/1997_III_7.png}
\end{center}

\subsection{Telescoping products}

\subsubsection{2005, Paper I, Question 7}
\begin{center}
 \includegraphics[width=\textwidth]{screenshots/2005_I_7.png}
\end{center}

\subsection{Deriving trigonometric special angle formulae}

\subsubsection{2015, Paper I, Question 2}
\begin{center}
 \includegraphics[width=\textwidth]{screenshots/2015_I_2.png}
\end{center}

\end{document}
