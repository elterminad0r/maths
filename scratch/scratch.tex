\documentclass{article}

\usepackage{parskip}
\usepackage{mymaths}

\usepackage{enumitem}

\newcommand*\tensordim[2]{
 \underset{
  \mathllap{#2\ \text{times}
  \ \left\{
   \vphantom{
    \begin{array}{c}
     \hline \hline \vdots \\[1ex] \hline
    \end{array}
   } \right.
  }
  \begin{array}{c}
   \hline \hline \vdots \\[1ex] \hline
  \end{array}
 }
 {\mathrm #1}
}

\begin{document}
\(a, b \in \mathcal M\) are conjugate iff \(\mathop\exists g \in \mathcal M\) such that \(a = g b g^{-1}\)

\(\mathscr B\)

\(\bigcup\limits_{g \in G} g H g^{-1}\)

Let Q be a \((2n+1)\times(2n+1)\) orthogonal matrix

\begin{equation*}
 1 + \tensordim{H}{n} \,\mathbf x
\end{equation*}

8(b). We say that the relation \(S\) "contains" the relation \(R\) if \(aSb\)
whenever \(aRb\).  Let \(R\) be the relation on \(\mathbb Z\) given by ``\(aRb\)
if \(b = a + 3\)''.  How many equivalence relations on \(\mathbb Z\) contain
\(R\)?  [Note: the answer is neither "one" nor "infinitely many".]

8(c). Give eight relations on the set \(\mathbb Z\), one for each subset of
\(\set{\text{reflexive}, \text{symmetric}, \text{transitive}}\).  E.g.: one
obeying all of R, S, T; one obeying R and S but not T; etc.


\(a R b\) contains

\begin{enumerate}
 \item Find the general solutions of
\begin{enumerate}[label=(\roman*)]
 \item \(y_{n + 2} + y_{n + 1} - 6 y_n = n^2\)
 \item \(y_{n + 2} - 3 y_{n + 1} + 2 y_n = n\)
 \item \(y_{n + 2} - 4 y_{n + 1} + 4 y_n = a^n\)
\end{enumerate}

\item Find the general solutions of \\
\begin{tabular}{ll}
  (i) & \(y_{n + 2} + y_{n + 1} - 6 y_n = n^2\) \\
 (ii) & \(y_{n + 2} - 3 y_{n + 1} + 2 y_n = n\) \\
(iii) & \(y_{n + 2} - 4 y_{n + 1} + 4 y_n = a^n\) \\
\end{tabular}

\item Find the general solutions of \\
\(  (\mathrm{i})\ \ \  y_{n + 2} + y_{n + 1} - 6 y_n = n^2\) \\
\( (\mathrm{ii})\ \  y_{n + 2} - 3 y_{n + 1} + 2 y_n = n\) \\
\((\mathrm{iii})\ \ y_{n + 2} - 4 y_{n + 1} + 4 y_n = a^n\)

\item Find the general solutions of \\
 (i)\(\ \ \  y_{n + 2} + y_{n + 1} - 6 y_n = n^2\) \\
 (ii)\(\ \  y_{n + 2} - 3 y_{n + 1} + 2 y_n = n\) \\
 (iii)\(\ \  y_{n + 2} - 4 y_{n + 1} + 4 y_n = a^n\)
\end{enumerate}

\begin{equation*}
 \underbrace{
  1 + 1 + \dotsb + 1
 }_{\text{\(100\) times}}
\end{equation*}

\begin{equation*}
 \mathit{ABCDEFGH}\ ABCDEFGH
\end{equation*}

\(\ncong\)

\Huge ORW
\end{document}

