\documentclass[a4paper,12pt]{article}
\author{Alastair Horn, Izaak van Dongen}
\title{chill pure questions to eat meals to}

\usepackage{mysty}
\usepackage{mymaths}

% Embed source files into PDF in case of loss. You can view or extract the
% source files by doing `pdfdetach -list <file.pdf>` or
% `pdfdetach -saveall <file.pdf>`, using pdfdetach from poppler, or some other
% suitable method.
\usepackage{embedall}
\embedfile{mymaths.sty}
\embedfile{mysty.sty}

\begin{document}
\maketitle

The following was motivated by mealtime discussions. Sometimes prior knowledge
of fairly late Michaelmas content is assumed.

\begin{enumerate}
 \item
  The relation ``\(\isom\)'' is an equivalence relation on any set of groups.
  Determine (with proof) the equivalence classes up to isomorphism of
  % unfortunate consequences of choice:
  % Q * R/Z =~ R * Q/Z =~ R
  % R * R/Z =~ R/Z
  % R/Z * Q/Z =~ R/Z * R/Z
  % TODO: figure out whether or not GL_2(R) =~ GL_3(R), add that.
  \allowdisplaybreaks
  \begin{align*}
   \set{  & \Z,\ \Q,\ \R,\ \Z \dprod \Z,\ \Q \dprod \Q,
        \\& \Q / \Z,\ \R / \Z,\ \Sym(\Z),\ \Sym(\R),
        \\& S^1,\ \R^\times,\ \R_{>0}^\times,\ \Q^\times,\ \Q_{>0}^\times,
        \\& S^1 \dprod S^1,\ \R^\times \dprod \R^\times,
            \ \Q^\times \dprod \Q^\times,\ \Sym(\Z) \dprod \Sym(\Z),
        \\& S_\infty,\ A_\infty,\ D_\infty,\ \Z^\N,\ \Z^\Z,\ \Z^\R,\ \Isom(\R),
          \ C_2^\N,\ C_2^\R,
        \\& \R \dprod \Z,\ \Q \dprod \Z,\ \Q \dprod (\Q/\Z),
            \ \Z \dprod (\R/\Z),\ \Z \dprod (\Q/\Z),
        \\& (\R/\Z) \dprod (\R/\Z),\ \ (\Q/\Z) \dprod (\Q/\Z),
        \\& \R \dprod C_2,\ \Q \dprod C_2,\ \Z \dprod C_2,
            \ (\R/\Z) \dprod C_2,\ (\Q/\Z) \dprod C_2,
        \\& \GL_2(\Q),\ \GL_3(\Q),
          \ \GL_4(\Q),\ \GL_\infty(\Q),
          \ \GL_2(\R),\ \GL_3(\R),
          \ \GL_4(\R),\ \GL_\infty(\R),
        \\& \GL_2(\C),\ \GL_3(\C),
          \ \GL_4(\C),\ \GL_\infty(\C),
          \ \GL_2(\Z),\ \GL_3(\Z),
          \ \GL_4(\Z),\ \GL_\infty(\Z),
        \\& \SL_2(\Q),\ \SL_3(\Q),
          \ \SL_4(\Q),\ \SL_\infty(\Q),
          \ \SL_2(\R),\ \SL_3(\R),
          \ \SL_4(\R),\ \SL_\infty(\R),
        \\& \SL_2(\C),\ \SL_3(\C),
          \ \SL_4(\C),\ \SL_\infty(\C),
          \ \SL_2(\Z),\ \SL_3(\Z),
          \ \SL_4(\Z),\ \SL_\infty(\Z),
        \\& \PGL_2(\Q),\ \PGL_3(\Q),
          \ \PGL_4(\Q),\ \PGL_\infty(\Q),
          \ \PGL_2(\R),\ \PGL_3(\R),
          \ \PGL_4(\R),\ \PGL_\infty(\R),
        \\& \PGL_2(\C),\ \PGL_3(\C),
          \ \PGL_4(\C),\ \PGL_\infty(\C),
          \ \PGL_2(\Z),\ \PGL_3(\Z),
          \ \PGL_4(\Z),\ \PGL_\infty(\Z),
        \\& \PSL_2(\Q),\ \PSL_3(\Q),
          \ \PSL_4(\Q),\ \PSL_\infty(\Q),
          \ \PSL_2(\R),\ \PSL_3(\R),
          \ \PSL_4(\R),\ \PSL_\infty(\R),
        \\& \PSL_2(\C),\ \PSL_3(\C),
          \ \PSL_4(\C),\ \PSL_\infty(\C),
          \ \PSL_2(\Z),\ \PSL_3(\Z),
          \ \PSL_4(\Z),\ \PSL_\infty(\Z),
        \\& \O_2(\Q),\ \O_3(\Q),
          \ \O_4(\Q),\ \O_\infty(\Q),\ \SO_2(\Q),
          \ \SO_3(\Q),\ \SO_4(\Q),\ \SO_\infty(\Q),
        \\& \O_2(\R),\ \O_3(\R),
          \ \O_4(\R),\ \O_\infty(\R),\ \SO_2(\R),
          \ \SO_3(\R),\ \SO_4(\R),\ \SO_\infty(\R),
        \\& \O_2(\C),\ \O_3(\C),
          \ \O_4(\C),\ \O_\infty(\C),\ \SO_2(\C),
          \ \SO_3(\C),\ \SO_4(\C),\ \SO_\infty(\C),
        \\& \O_2(\Z),\ \O_3(\Z),
          \ \O_4(\Z),\ \O_\infty(\Z),\ \SO_2(\Z),
          \ \SO_3(\Z),\ \SO_4(\Z),\ \SO_\infty(\Z),
        \\& \U(2),\ \U(3),
          \ \U(4),\ \U(\infty),\ \SU(2),
          \ \SU(3),\ \SU(4),\ \SU(\infty),
        \\& \RL_2^{\cycsgp{-1}}(\Q),
          \ \RL_3^{\cycsgp{-1}}(\Q),
          \ \RL_4^{\cycsgp{-1}}(\Q),
          \ \RL_2^{\cycsgp{-1}}(\R),
          \ \RL_3^{\cycsgp{-1}}(\R),
          \ \RL_4^{\cycsgp{-1}}(\R),
        \\& \RL_2^{\cycsgp{-1}}(\C),
          \ \RL_3^{\cycsgp{-1}}(\C),
          \ \RL_4^{\cycsgp{-1}}(\C),
          \ \RL_2^{\cycsgp i}(\C),
          \ \RL_3^{\cycsgp i}(\C),
          \ \RL_4^{\cycsgp i}(\C),
        \\& \RL_2^{S^1}(\C),
          \ \RL_3^{S^1}(\C),
          \ \RL_4^{S^1}(\C),
          \ \RL_2^{\Q_{>0}^\times}(\Q),
          \ \RL_3^{\Q_{>0}^\times}(\Q),
          \ \RL_4^{\Q_{>0}^\times}(\Q),
        \\& \RL_2^{\R_{>0}^\times}(\R),
          \ \RL_3^{\R_{>0}^\times}(\R),
          \ \RL_4^{\R_{>0}^\times}(\R),
          \ \RL_2^{\R_{>0}^\times}(\C),
          \ \RL_3^{\R_{>0}^\times}(\C),
          \ \RL_4^{\R_{>0}^\times}(\C),
   }
  \end{align*}
  Here
  \begin{itemize}
   \item
    \(S^1 = \set{z \in \C : \abs z = 1}\) is the group of complex numbers of
    modulus \(1\) under multiplication.
   \item
    For any set \(S\),
    \(\Sym(S) = \set{f\colon S \to S : \text{\(f\) is a bijection}}\) is the
    group of \emph{symmetries} or \emph{permutations} of \(S\), under
    composition.
   \item
    eg \(\R^\times = (\R \setminus \set{0}, \cdot)\) is the group of nonzero
    reals under multiplication.
   \item
    eg \(\R_{>0}^\times = (\R_{>0}, \cdot)\) is the group of positive reals
    under multiplication.
   \item
    \(S_\infty = \bigcup_n S_n\) is the group of permutations of \(\N\)
    affecting only finitely many elements.
   \item
    \(A_\infty = \bigcup_n A_n\) is the group of even permutations of \(\N\)
    affecting only finitely many elements. This is a subgroup of \(S_\infty\).
   \item
    \(D_\infty = \Isom(\Z)\) is the group of isometries of \(\Z\):
    \begin{equation*}
     D_\infty
       = \set{f\colon \Z \to \Z: \forall x, y, \abs{f(x) - f(y)} = \abs{x - y}}
    \end{equation*}
   \item
    For a group \(G\) and a nonempty set \(S\), \(G^S\) is the group of
    functions from \(S\) to \(G\) under pointwise operations
    (ie \(\forall x, (f \cdot_{(G^S)} g)(x) = f(x) \cdot_G g(x)\)).
   \item
    If \(R\) is a commutative unital ring, then
    \begin{itemize}
     \item
      \(\GL_n(R)\) is the group of invertible \(n \times n\) matrices
      with components in \(R\). The ``general linear group''.

      \(\GL_\infty(R) = \bigcup_n \GL_n(R)\), with the natural
      inclusion. This is a mostly custom definition.
     \item
      \(\SL_n(R)\) is the subgroup of \(\GL_n(R)\) of matrices
      with determinant \(1\). The ``special linear group''.

      \(\SL_\infty(R) = \bigcup_n \SL_n(R)\) with the natural
      inclusion. This is isomorphic to the subgroup of
      \(\GL_\infty(R)\) of matrices with determinant \(1\), after
      noting that the determinant can lift to this union.
      This is a mostly custom definition.
     \item
      \(\PGL_n(R)\) is the quotient
      \(\GL_n(R) / Z\), where \(Z \nsgp \GL_n(R)\) is the
      subgroup of scalar matrices
      \(Z = \set{\lambda \mat I: \lambda \in R^\times}\).
      The ``projective general linear group''.

      \(\PGL_\infty(R) = \bigcup_n \PGL_n(R)\). The existence of this inclusion
      a bit more involved. This is a mostly custom definition.
     \item
      \(\PSL_n(R) = SL_n(R) / (SL_n(R) \cap Z)\).
      The ``projective special linear group''.

      \(\PSL_\infty(R) = \bigcup_n \PSL_n(R)\). The existence of this inclusion
      a bit more involved. This is a mostly custom definition.
     \item
      \(\O_n(R)\) is the group of matrices that are orthogonal -
      ie, such that \(\mat M \tran{\mat M} = \mat I\).
      The ``orthogonal group''.

      The fact that this is defined over any ring and not just \(\R\) is custom.

      \(\O_\infty(R) = \bigcup_n \O_n(R)\), with the natural inclusion. This is
      a mostly custom definition.
     \item
      \(\SO_n(R)\) is the subgroup of \(\O_n(R)\) of orthogonal matrices of
      determinant \(1\). The ``special orthogonal group''.

      \(\SO_\infty(R) = \bigcup_n \SO_n(R)\), with the natural inclusion. This
      is a mostly custom definition.
     \item
      \(\U(n)\) is the subgroup of \(\GL_n(\C)\) of matrices that are unitary -
      ie such that \(\mat M \herm{\mat M} = \mat I\).
      The ``unitary group''.

      \(\U(\infty) = \bigcup_n \U(n)\) with the natural inclusion. This is a
      mostly custom definition.
     \item
      If \(G \sgp R^\times\), then \(\RL_n^G(R)\) is the subgroup of
      \(\GL_n(R)\) with determinants in \(G\). This generalises \(\SL\) and
      \(\GL\), since \(\SL_n(R) = \RL_n^{\set 1}(R)\), and
      \(\GL_n(R) = \RL_n^{R^\times}(R)\).
      The ``restricted linear group''. This definition is completely made up.
    \end{itemize}
  \end{itemize}
  This question is not supposed to be unpleasantly long and depressing.
  Rather, it is a very pleasant source of interesting sub-questions to think
  about over a long period of time.
  Be advised that this question can be done performing
  fewer direct comparisons than \(\tfrac 12 n(n - 1)\).
 \item
  If \(A, B, G\) are groups, we say that \(A \dprod B\) is a
  \emph{decomposition} of \(G\) if \(G \isom A \dprod B\), and neither of \(A\)
  and \(B\) is trivial. We say that a group \(G\) is \emph{decomposable} if
  there exists a decomposition of \(G\).

  Which of the following groups are decomposable?
  \begin{multicols}{4}
   \begin{itemize}
    \item
     \(\Z\)
    \item
     \(\Q\)
    \item
     \(\C\)
    \item
     \(\Q^\times\)
    \item
     \(\R^\times\)
    \item
     \(\C^\times\)
    \item
     \(\Q_{>0}^\times\)
    \item
     \(\Q / \Z\)
   \end{itemize}
  \end{multicols}
 \item
  Are there infinitely many pairwise non-isomorphic groups of infinite order?

  Are there uncountably many?
 \item
  Here is a strengthening of Cayley's theorem in the case of finite groups.
  \begin{tcolorbox}
   If \(G\) is a group of finite order, then \(G\) is isomorphic to a subgroup
   of \(A_n\) for some \(n\).
  \end{tcolorbox}
  Is it true?
 \item
  Does \(\R\) have any proper subgroups of finite index?
 \item
  If the Axiom of Choice is assumed, it is a famous result than then every
  vector space has a basis. Particularly, \(\R\) and \(\C\) both have bases as
  \(\Q\)-vector spaces. These bases must have the same cardinality, so in fact
  they induce an isomorphism of additive groups, \(\R \isom \C\).

  Prove that if \(\R \isom \C\), then \(S^1 \isom \C^\times\).

  By assuming further that \(\R \isom \R \dprod \Q\), show that
  \begin{itemize}
   \item
    \(\Q \dprod (\R/\Z) \isom \R \dprod (\Q/\Z) \isom
      \R \dprod (\R/\Z) \isom (\R/\Z)\)
   \item
    \((\R/\Z) \dprod (\Q/\Z) \isom (\R/\Z) \dprod (\R/\Z)\)
   \item
    \(\R/\Q \isom \R\)
    (be careful here! \(A \isom B\) and \(C \isom D\) don't necessarily imply
    \(A/C \isom B/D\))
  \end{itemize}
 \item
  Let \(G_1 \sgp G_2 \sgp G_3 \sgp ...\) be a chain of groups ordered by
  inclusion. Let \(G = \bigcup_n G_n\). Show that \(G\) is a group with the
  natural operation ``given \(a, b \in G\), let \(i\) be such that
  \(a, b \in G_i\). Then \(a \cdot_G b \defeq a \cdot_{G_i} b\)''.

  Show that if each \(G_i\) is simple, then \(G\) is simple.
 \item
  Let \(G\) be a group, and let \(a, b \in G\). Must \(ab\) have the same order
  as \(ba\)?
 \item
  Find a natural operation \(\ast\) on \(\N\) that is associative, but not
  commutative\footnote{If your answer mentioned the word ``bijection'', go and
  stand in the corner, think about what you've done, and think of an operation
  that is associative and \emph{anticommutative}, ie
  \(a \ast b = b \ast a \iff a = b\)}.
 \item
  Is every \(x \in \R\) the root of some nonzero power series with coefficients
  in \(\Q\)?

  Is every \(z \in \C\) the root of some nonzero power series with coefficients
  in \(\Q\)?
 \item
  Complete \url{https://wwwf.imperial.ac.uk/~buzzard/xena/natural_number_game/}
  \texttt{:)}.

  This is a great resource to learn about both how interactive theorem proving
  works (which is likely a big part of the future of maths), and to learn how
  to actually build up from Peano's axioms, which is not completely
  straightforward, despite being boldly skipped by Numbers and Sets.
 \item
  Let \(\vec f_n\) be a sequence of vectors defined by
  \begin{align*}
   \vec f_0 \defeq
    \begin{pmatrix} 0 \\ 1 \end{pmatrix}
   &&
   \vec f_{n + 1} \defeq
    \begin{pmatrix} 0 & 1 \\ 1 & 1 \end{pmatrix}\vec f_n
  \end{align*}
  Show that
  \(\vec f_{n + 2} = \vec f_{n + 1} + \vec f_n\) for all
  \(n \in \N\), and that \((\vec f_1)_1 = (\vec f_2)_1 = 1\). By
  writing \(\begin{psmallmatrix} 0 & 1 \\ 1 & 1 \end{psmallmatrix}\) as
  \(\mat U \mat D \mat U^{-1}\) where \(\mat D\) is a diagonal matrix,
  find a closed form for the \(n\)th term of the Fibonacci sequence, without
  resorting to any applied nonsense or guesswork.
\end{enumerate}

\end{document}
