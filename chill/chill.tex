\documentclass[a4paper,12pt]{article}
\author{Alastair Horn, Izaak van Dongen}
\title{chill pure questions to eat meals to}

\usepackage{mysty}
\usepackage{mymaths}

% Embed source files into PDF in case of loss. You can view or extract the
% source files by doing `pdfdetach -list <file.pdf>` or
% `pdfdetach -saveall <file.pdf>`, using pdfdetach from poppler, or some other
% suitable method.
\usepackage{embedall}
\embedfile{mymaths.sty}
\embedfile{mysty.sty}

\begin{document}
\maketitle

The following was motivated by mealtime discussions. Sometimes prior knowledge
is assumed.

\begin{enumerate}
 \item
  The relation ``\(\isom\)'' is an equivalence relation on any set of groups.
  Determine (with proof) the equivalence classes up to isomorphism of
  \begin{align*}
   \{  & \Z,\ \Q,\ \R,\ \Z \dprod \Z,\ \Q \dprod \Q,
     \\& \Q / \Z,\ \R / \Z,\ \Sym(\Z),\ \Sym(\R),
     \\& S^1,\ \R^\times,\ \R_{>0}^\times,\ \Q^\times,\ \Q_{>0}^\times,
     \\& S^1 \dprod S^1,\ \R^\times \dprod \R^\times,
         \ \Q^\times \dprod \Q^\times,\ \Sym(\Z) \dprod \Sym(\Z)
     \\& S_\infty, A_\infty, D_\infty, \Z^\N, \Z^\Z, \Z^\R
     \}
  \end{align*}
  Here
  \begin{itemize}
   \item
    \(S^1 = \{z \in \C : \abs z = 1\}\).
   \item
    For any set \(S\),
    \(\Sym(S) = \{f\colon S \to S : \text{\(f\) is a bijection}\}\) is the group
    of \emph{symmetries} or \emph{permutations} of \(S\).
   \item
    eg \(\R^\times = (\R \setminus \{0\}, \cdot)\) is the group of nonzero
    reals under multiplication.
   \item
    eg \(\R_{>0}^\times = (\R_{>0}, \cdot)\) is the group of positive reals
    under multiplication.
   \item
    \(S_\infty = \bigcup_n S_n\) is the group of permutations of \(\N\)
    affecting only finitely many elements.
   \item
    \(A_\infty = \bigcup_n A_n\) is the group of even permutations of \(\N\)
    affecting only finitely many elements. This is a subgroup of \(S_\infty\).
   \item
    \(D_\infty\) is the group of isometries of \(\Z\):
    \(D_\infty
      = \{f\colon \Z \to \Z : \forall x, y. \abs{f(x) - f(y)} = \abs{x - y}\}\).
   \item
    For a group \(G\) and a nonempty set \(S\), \(G^S\) is the group of functions
    from \(S\) to \(G\) under pointwise operations
    (ie \(\forall x. (f \cdot_{(G^S)} g)(x) = f(x) \cdot_G g(x)\)).
  \end{itemize}
  You are free to prove that any of these are groups if you don't believe me.
 \item
  If \(A, B, G\) are groups, we say that \(A \dprod B\) is a
  \emph{decomposition} of \(G\) if \(G \isom A \dprod B\), and neither of \(A\)
  and \(B\) is trivial. We say that a group \(G\) is \emph{decomposable} if
  there exists a decomposition of \(G\).

  Which of the following groups are decomposable?
  \begin{multicols}{4}
   \begin{itemize}
    \item
     \(\Z\)
    \item
     \(\Q\)
    \item
     \(\C\)
    \item
     \(\Q^\times\)
    \item
     \(\R^\times\)
    \item
     \(\C^\times\)
    \item
     \(\Q_{>0}^\times\)
    \item
     \(\Q / \Z\)
   \end{itemize}
  \end{multicols}
 \item
  Here is a strengthening of Cayley's theorem in the case of finite groups.
  \begin{tcolorbox}
   If \(G\) is a group of finite order, then \(G\) is isomorphic to a subgroup
   of \(A_n\) for some \(n\).
  \end{tcolorbox}
  Is it true?
 \item
  Does \(\R\) have any subgroups of index \(2\)? Of finite index other than
  \(1\)?
 \item
  If the Axiom of Choice is assumed, it is a famous result than then every
  vector space has a basis. Particularly, \(\R\) and \(\C\) both have bases as
  \(\Q\)-vector spaces. These bases must have the same cardinality, so in fact
  they induce an isomorphism of additive groups, \(\R \isom \C\).

  Prove that if \(\R \isom \C\), then \(\R/\Z \isom \C^\times\).
 \item
  Let \(G_1 \sgp G_2 \sgp G_3 \sgp ...\) be a chain of groups ordered by
  inclusion. Let \(G = \bigcup_n G_n\). Show that \(G\) is a group with the
  natural operation ``given \(a, b \in G\), let \(i\) be such that
  \(a, b \in G_i\). Then \(a \cdot_G b \defeq a \cdot_{G_i} b\)''.

  Show that if each \(G_i\) is simple, then \(G\) is simple.
 \item
  Let \(G\) be a group, and let \(a, b \in G\). Must \(ab\) have the same order
  as \(ba\)?
 \item
  Find a natural operation \(\ast\) on \(\N\) that is associative, but not
  commutative\footnote{If your answer mentioned the word ``bijection'', go and
  stand in the corner, think about what you've done, and think of an operation
  that is associative and \emph{anticommutative}, ie
  \(a \ast b = b \ast a \iff a = b\)}.
 \item
  Is every \(x \in \R\) the root of some nonzero power series with coefficients
  in \(\Q\)?

  Is every \(z \in \C\) the root of some nonzero power series with coefficients
  in \(\Q\)?
 \item
  Complete \url{https://wwwf.imperial.ac.uk/~buzzard/xena/natural_number_game/}.
 \item
  Let \(\vec f_n\) be a sequence of vectors defined by
  \begin{align*}
   \vec f_0 \defeq
    \begin{pmatrix} 0 \\ 1 \end{pmatrix}
   &&
   \vec f_{n + 1} \defeq
    \begin{pmatrix} 0 & 1 \\ 1 & 1 \end{pmatrix}\vec f_n
  \end{align*}
  Show that
  \(\vec f_{n + 2} = \vec f_{n + 1} + \vec f_n\) for all
  \(n \in \N\), and that \((\vec f_1)_1 = (\vec f_2)_1 = 1\). By
  writing \(\begin{psmallmatrix} 0 & 1 \\ 1 & 1 \end{psmallmatrix}\) as
  \(\mat U \mat D \mat U^{-1}\) where \(\mat D\) is a diagonal matrix,
  find a closed form for the \(n\)th term of the Fibonacci sequence, without
  resorting to any applied nonsense or guesswork.
\end{enumerate}

\end{document}
