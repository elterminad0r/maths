% Compiling with
% latexmk -halt-on-error -shell-escape -synctex=1 -pdf
% (Recommend using a latexmkrc file so as just to run latexmk -pvc, for example)
% Probably you can achieve the same with an inordinate number of invocations of
% pdflatex -halt-on-error -shell-escape -synctex=1

% fleqn aligns equations to the left, a4 paper size, 11pt font, article class
\documentclass[fleqn,a4paper,11pt]{article}
% for some reason it won't work properly until later
\title{There does not exist a function from R to R that is continuous
       precisely on Q.}
\author{\texorpdfstring{Izaak van Dongen (\texttt{imv26})}
                       {Izaak van Dongen (imv26)}}

\usepackage{mymaths}
\usepackage{mystyle}

\begin{document}
\title{\texorpdfstring{There does not exist a function \(\R \to \R\) that is
                       continuous precisely on \(\Q\)}
                      {There does not exist a function from R to R that is
                       continuous precisely on Q}}
 \maketitle\thispagestyle{empty} % no page number under title

 \section*{Abstract}

 We aim to prove that there does not exist a function \(\R \to \R\) that is
 continuous precisely on \(\Q\).

 \section*{\texorpdfstring{There does not exist a function
                           \boldmath\(\R \to \R\) that is continuous precisely
                           on \boldmath\(\Q\)}
                          {There does not exist a function from R to R that is
                           continuous precisely on Q}}

 \begin{lemma} \label{lemma_closed_intersection}
  If \((a_n)\), \((b_n)\) are sequences in \(\R\) such that \(a_n \le b_n\) and
  \begin{equation*}
   \intcc{a_1, b_1} \supseteq \intcc{a_2, b_2} \supseteq \intcc{a_3, b_3}
   \supseteq \dotsb
  \end{equation*}
  then \(\bigcap_{n = 1}^\infty \intcc{a_n, b_n} \ne \emptyset\).
 \end{lemma}
 \begin{proof}
  For convenience, say \(S \defeq \bigcap_{n = 1}^\infty \intcc{a_n, b_n}\).

  Since these intervals are nested,
  \(a_{n + 1} \in \intcc{a_n, b_n}\) and particularly \(a_{n + 1} \ge a_n\) for
  all \(n\), so \((a_n)\) is monotone increasing. Furthermore,
  \(a_n \in \intcc{a_1, b_1}\) for all \(n\), so \(a_n \le b_1\) and
  \((a_n)\) is bounded above. Hence we can set
  \(a = \lim_{n \to \infty} a_n\).

  Now to prove \(a \in S\) (which is sufficient to show that
  \(S \ne \emptyset\)), we need \(a \ge a_n\) for all \(n\), which is clear,
  since if we had some \(N\) such that \(a < a_N\), then we would have
  \(\abs{a - a_n} \ge \epsilon\) for all \(n \ge N\), where
  \(\epsilon = a_N - a > 0\), since \(a_n \ge a_N\). This would imply
  \(a_n \nto a\), a contradiction.

  We also need \(a \le b_n\) for all \(n\). This is true since if we had an
  \(N\) with \(a > b_N\), then by convergence of \(a_n\), we could pick
  \(M\) such that \(a - a_M < a - b_N\), so \(a_M > b_N\). But then
  %              Max
  letting \(K = \max\set{N, M}\), \(K \ge M \implies a_K \ge a_M\), as \((a_n)\)
  is increasing, and \(K \ge N \implies b_K \le b_N\), as \((b_n)\) is
  decreasing by a similar argument to earlier. Hence \(a_K > b_K\), which is a
  contradiction.

  Hence \(a \in S\) and \(S \ne \emptyset\).
 \end{proof}

 \begin{theorem}
  There does not exist a function \(\R \to \R\) that is continuous precisely on
  \(\Q\).
 \end{theorem}
 \begin{proof}
  Suppose that there is such a function, \(f: \R \to \R\).

  By the countability of \(\Q\), let \((x_n)\) be an enumeration of the
  rationals.

  Let \((\epsilon_n)\) be a sequence with
  \(\epsilon_n > 0, \epsilon_n \to 0\), for example \(\epsilon_n = 1/n\).

  Now by continuity of \(f\), choose an \emph{irrational}
  \(\delta_1 > 0\) such that
  \begin{equation*}
   \abs{x - x_1} \le \delta_1 \implies \abs{f(x) - f(x_1)} < \epsilon_1
  \end{equation*}
  Note the non-strict inequality - this can be achieved by first picking
  \(\delta_1'\) such that this holds for the strict inequality, and then
  choosing a suitable smaller \(\delta_1\) - particularly, any
  \(\delta_1 \in \intoo{0, \delta_1'} \setminus \Q\).

  Now for convenience, discard all the rationals not in
  \(\intcc{x_1 - \delta_1, x_1 + \delta_1}\) from our enumeration, so
  \(x_2 \in \intcc{x_1 - \delta_1, x_1 + \delta_1}\).

  Since we picked \(\delta_1 \notin \Q\), and by definition \(x_1 \in \Q\), we
  have \(x_1 \pm \delta_1 \notin \Q\), so in fact \(x_2\) is neither of the
  endpoints and \(x_2 \in \intoo{x_1 - \delta_1, x_1 + \delta_1}\).

  Now pick an irrational \(\delta_2 > 0\) such that
  \begin{equation*}
   \abs{x - x_2} \le \delta_1 \implies \abs{f(x) - f(x_2)} < \epsilon_2
  \end{equation*}
  \emph{and}
  \(x_1, x_1 \pm \delta_1 \notin \intcc{x_2 - \delta_2, x_2 + \delta_2}\) -
  ie the new interval should not contain \(x_1\), or either of the endpoints.
  This can be done since we can again first use continuity to select a
  \(\delta_2'\) and then pick
  \(\delta_2 \in \intoo{0,
                        \min
                         \set{\delta_2',
                              \abs{x_2 - x_1},
                              \abs{x_2 - (x_1 \pm \delta_1)}}} \setminus \Q\).

  Repeat this procedure to obtain a sequence \((\delta_n)\) with
  \begin{equation*}
   \abs{x - x_n} \le \delta_n \implies \abs{f(x) - f(x_n)} < \epsilon_n
  \end{equation*}
  and
  \(x_n, x_n \pm \delta_n
    \notin \intcc{x_{n + 1} - \delta_{n + 1}, x_{n + 1} + \delta_{n + 1}}\).

  Particularly, we get a \emph{nested} sequence
  \begin{equation*}
   \intcc{x_1 - \delta_1, x_1 + \delta_1} \supsetneq
   \intcc{x_2 - \delta_2, x_2 + \delta_2} \supsetneq
   \intcc{x_3 - \delta_3, x_3 + \delta_3} \supsetneq \dotsb
  \end{equation*}
  where \(f\) is getting ``more and more continuous'' as we go along.

  By Lemma \ref{lemma_closed_intersection}, we can pick a real
  \(\alpha \in \bigcap_{n = 1}^\infty \intcc{x_n - \delta_n, x_n + \delta_n}\).
  Note that \(\alpha \ne x_n \pm \delta_n\) for any \(n\), since we made the
  intervals strictly nested.

  By the fact that
  \(x_n \notin \intcc{x_{n + 1} - \delta_{n + 1}, x_{n + 1} + \delta_{n + 1}}\),
  \(\alpha\) must be irrational, since \(\Q\) is countable and this construction
  therefore ``dodges'' all of the rationals.

  But also, given \(\epsilon > 0\), we can pick an \(n\) such that
  \(\epsilon_n < \frac 12 \epsilon\) (since \(\epsilon_n \to 0\)). Now by
  construction,
  \begin{equation*}
   \abs{x - x_n} < \delta_n \implies \abs{f(x) - f(x_n)} < \epsilon_n
  \end{equation*}
  Since \(\alpha \in \intcc{x_n - \delta_n, x_n + \delta_n}\), we have
  \(\abs{f(\alpha) - f(x_n)} < \epsilon_n\), so letting
  \(\delta = \min\set{\abs{\alpha - (x_n \pm \delta_n)}}\)
  (so \(\delta > 0\) and \(\intoo{\alpha - \delta, \alpha + \delta} \subset
                           \intcc{x_n - \delta_n, x_n + \delta_n}\)),
  \begin{align*}
   \abs{x - \alpha } < \delta \implies
   \abs{f(x) - f(\alpha)}
   &= \abs{f(x) - f(x_n) + f(x_n) - f(\alpha)} \\
   &\le \abs{f(x) - f(x_n)} + \abs{f(x_n) - f(\alpha)} \\
   &\le \epsilon_n + \epsilon_n \\
   &< \epsilon
  \end{align*}
  So \(f\) is continuous at \(\alpha\).
 \end{proof}
 In fact, this proof shows that a function \(\R \to \R\) cannot be continuous at
 precisely any set that is countable and dense in \(\R\).
\end{document}
