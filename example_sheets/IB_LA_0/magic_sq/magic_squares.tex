% Compiling with
% latexmk -halt-on-error -shell-escape -synctex=1 -pdf <file.tex>
% (Recommend using a latexmkrc file so as just to run latexmk -pvc, for example)
% Probably you can achieve the same with an inordinate number of invocations of
% pdflatex -halt-on-error -shell-escape -synctex=1 <file.tex>

% fleqn aligns equations to the left, a4 paper size, 11pt font, article class
\documentclass[fleqn,a4paper,11pt]{article}
\title{Notes on Magic Square Spaces}
\author{\texorpdfstring{Izaak van Dongen (\texttt{imv26})}
                       {Izaak van Dongen (imv26)}}
\date{}

\usepackage{mymaths}
\usepackage{mystyle}

% Embed source files into PDF in case of loss. You can view or extract the
% source files by doing `pdfdetach -list <file.pdf>` or
% `pdfdetach -saveall <file.pdf>`, using pdfdetach from poppler, or some other
% suitable method.
\usepackage{embedall}
\embedfile{mymaths.sty}
\embedfile{mystyle.sty}

\begin{document}
\maketitle

We write \(\mathfrak M_3(\R)\) for the space of \(3 \times 3\) magic squares
over \(\R\), ie the set of \(3 \times 3\) matrices
\begin{equation*}
 \begin{array}{c|c|c}
  a & b & c \\ \hline
  d & e & f \\ \hline
  g & h & i \\
 \end{array} \quad \text{such that} \quad
 \begin{alignedat}{3}
  a + b + c &= d + e + f &&{}= g + h + i &&{}= \\
  a + d + g &= b + e + h &&{}= c + f + i &&{}= \\
  a + e + i &= c + e + g.
 \end{alignedat}
\end{equation*}
This is a vector space over \(\R\) with elementwise addition and scalar
multiplication.
\begin{theorem}
 \(\dim \mathfrak M_3(\R) = 3\).
\end{theorem}
\begin{proof}
 Note that if
 \begin{equation*}
  \begin{array}{c|c|c}
   a & b & c \\ \hline
   d & \ast_1 & \ast_2 \\ \hline
   \ast_3 & \ast_4 & \ast_5 \\
  \end{array} \in \mathfrak M_3(\R),
 \end{equation*}
 then the whole square is determined, since (writing \(T = a + b + c\)), we must
 have
 \begin{align*}
  \ast_3 &= T - (a + d) && \text{by considering the first column \(C_1\)} \\
  \ast_1 &= a + d - c && \text{by considering the trailing diagonal} \\
  \ast_2 &= T - (a + 2d - c) && \text{by considering \(R_2\)} \\
  \ast_4 &= T - (a + b + d - c) && \text{by considering \(C_2\)} \\
  \ast_5 &= a + 2d - 2c && \text{by considering \(C_3\)} \\
         &= T - (2a + d - c) && \text{by considering the leading diagonal} \\
         &= -a + b + 2c - d && \text{defn. of \(T\)}
 \end{align*}
 But we can solve the two equations for \(\ast_5\) to get
 \(d = \tfrac 13 (-2a + b + 4c)\).

 Noting that
 \begin{equation*}
  \begin{array}{c|c|c}
   3 & 0 & 0 \\ \hline
   -2 & 1 & 4 \\ \hline
   2 & 2 & -1 \\
  \end{array}_,\quad
  \begin{array}{c|c|c}
   0 & 3 & 0 \\ \hline
   1 & 1 & 1 \\ \hline
   2 & -1 & 2 \\
  \end{array}_,\quad
  \begin{array}{c|c|c}
   0 & 0 & 3 \\ \hline
   4 & 1 & -2 \\ \hline
   -1 & 2 & 2 \\
  \end{array}_,\quad
 \end{equation*}
 are indeed magic squares, they must span \(\mathfrak M_3(\R)\) by the fact that
 \(a, b, c\) determine the whole matrix. Furthermore, they are linearly
 independent, eg because their first rows are linearly independent in \(\R^3\).
\end{proof}

\end{document}
