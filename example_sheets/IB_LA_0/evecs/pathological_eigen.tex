% Compiling with
% latexmk -halt-on-error -shell-escape -synctex=1 -pdf <file.tex>
% (Recommend using a latexmkrc file so as just to run latexmk -pvc, for example)
% Probably you can achieve the same with an inordinate number of invocations of
% pdflatex -halt-on-error -shell-escape -synctex=1 <file.tex>

% fleqn aligns equations to the left, a4 paper size, 11pt font, article class
\documentclass[fleqn,a4paper,11pt]{article}
\title{A Short Exposition of Linear Maps with pathological Eigenbehaviour}
\author{\texorpdfstring{Izaak van Dongen (\texttt{imv26})}
                       {Izaak van Dongen (imv26)}}
\date{}

\usepackage{mymaths}
\usepackage{mystyle}

% Embed source files into PDF in case of loss. You can view or extract the
% source files by doing `pdfdetach -list <file.pdf>` or
% `pdfdetach -saveall <file.pdf>`, using pdfdetach from poppler, or some other
% suitable method.
\usepackage{embedall}
\embedfile{mymaths.sty}
\embedfile{mystyle.sty}

\begin{document}
\maketitle

Let \(V\) be an \(\R\)-vector space, and let \(\alpha: V \to V\) be a linear
map.
\begin{itemize}
 \item
  \adjustbox{valign=t}{
   \begin{tcolorbox}
    Can \(\alpha\) have no eigenvalues over \(\R\)?
   \end{tcolorbox}
  }

  Yes. Let \(V = \R^2\), and let \(\alpha\) have matrix
  \(A = \begin{psmallmatrix} 0 & -1 \\ 1 & 0 \end{psmallmatrix}\) in the standard
  basis. Then \(\chi_A(t) = t^2 + 1\), which has no real roots.

  An infinite-dimensional example is \(V = \R^\N\), and
  \(\alpha((x_1, x_2, x_3, \dotsc)) = (0, x_1, x_2, \dotsc)\).

  Then if \(\alpha((x_n)) = \lambda (x_n)\), we have
  \(x_1 = \lambda \cdot 0 = 0\), and
  \(x_{n + 1} = \lambda x_n\). By induction, then, \(x_n = 0\) for all \(n\), so
  \(\alpha\) has no nontrivial eigenspaces.
 \item
  \adjustbox{valign=t}{
   \begin{tcolorbox}
    Is it possible that all \(\lambda \in \R\) are an eigenvalue of \(\alpha\)?
   \end{tcolorbox}
  }

  Yes. Let \(V = \R^\R\), and define
  \begin{align*}
   \alpha: \R^\R &\to \R^\R \\
               f &\mapsto (x \mapsto x f(x))
  \end{align*}
  \(\alpha\) is linear, since
  \begin{equation*}
   \alpha(\lambda f + \mu g)(x)
    = x \cdot (\lambda f + \mu g)(x)
    = \lambda x f(x) + \mu x g(x)
    = \lambda \alpha(f)(x) + \mu \alpha(g)(x)
  \end{equation*}
  But for all \(\lambda \in \R\), the indicator function
  \(\mathbf 1_{\set\lambda}\) is then an eigenvector of \(\alpha\) with
  eigenvalue \(\lambda\).

  For a smaller example, this also works for the direct sum of
  \(\R\) copies of \(\R\).
 \item
  \adjustbox{valign=t}{
   \begin{tcolorbox}
    Is it possible that precisely the \(\lambda \in \Q\) are an eigenvalue of
    \(\alpha\)?
   \end{tcolorbox}
  }

  Yes. Let \(V = \R^\Q\), and define
  \begin{align*}
   \alpha: \R^\Q &\to \R^\Q \\
               f &\mapsto (x \mapsto x f(x))
  \end{align*}
  \(\alpha\) is again linear for very similar reasons to last time.

  Then clearly for any \(\lambda \in \Q\), \(\mathbf 1_{\set \lambda}\) is again
  an eigenvector of \(\alpha\) with eigenvalue \(\lambda\).

  Now suppose \(0 \ne f \in \R^\Q\) and \(\lambda \in \R\) are such that
  \(\alpha(f) = \lambda f\).

  Then \(f\) is nonzero at some \(x\). Since
  \(x f(x) = \lambda f(x) \implies x = \lambda\),
  \(\lambda\) was rational.

  So the only eigenvalues of \(\alpha\) are rational.

  This also works for the direct sum of \(\Q\) copies of \(\R\).
\end{itemize}


\end{document}
