% Compiling with
% latexmk -halt-on-error -shell-escape -synctex=1 -pdf
% (Recommend using a latexmkrc file so as just to run latexmk -pvc, for example)
% Probably you can achieve the same with an inordinate number of invocations of
% pdflatex -halt-on-error -shell-escape -synctex=1

% fleqn aligns equations to the left, a4 paper size, 11pt font, article class
\documentclass[fleqn,a4paper,11pt]{article}
% for some reason it won't work properly until later
\title{Cardinalities of sets of functions}
\author{\texorpdfstring{Izaak van Dongen (\texttt{imv26})}
                       {Izaak van Dongen (imv26)}}

\usepackage{mymaths}
\usepackage{mystyle}

% Embed source files into PDF in case of loss. You can view or extract the
% source files by doing `pdfdetach -list <file.pdf>` or
% `pdfdetach -saveall <file.pdf>`, using pdfdetach from poppler, or some other
% suitable method.
\usepackage{embedall}
\embedfile{mymaths.sty}
\embedfile{mystyle.sty}

\begin{document}
 \section*{Abstract}

 We present some results on cardinalities of functions involving \(\R\)
 satisfying certain properties in analysis.

 \section*{Notation and basic results}

 We denote ``there exists a bijection between the sets \(A\) and \(B\)'' by
 \(A \equip B\).

 \(A^B\) denotes the set of functions from the set \(B\) to the set \(A\).

 \(\powerset(A)\)

 We will use some of the following basic results about cardinality:
 \begin{itemize}
  \item
   \(\equip\) is an equivalence relation on the class of sets
  \item
   \(\N \equip \Q\)
  \item
   \(\N \times \N \equip \N\)
  \item
   \(\R \equip \powerset(\N)\)
  \item
   If \(I \subseteq \R\) is an interval of nonzero length, then
   \(I \sim \R\).
  \item
   For a set \(A\), \(\powerset(A) \equip 2^A\), the set of indicator functions
   on \(A\).
  \item
   \(\R^n \equip \R \times \dotsb \times \R \equip \R\) for \(n \ge 1\).
  \item
   \(\R^\N \equip \R\)
  \item
   Cantor-Schr\"oder-Bernstein Theorem: For two sets \(A\) and \(B\), then
   if there exists an injection \(A \to B\) and an injection \(B \to A\), then
   \(A \equip B\).
  \item
   If \(A, B\) are sets, then \(A \times B \equip B \times A\)
  \item
   Various substitution properties: eg if \(A, B, C\) are sets, and
   \(A \equip B\), then
   \begin{itemize}
    \item
     \(A^C \equip B^C\)
    \item
     \(C^A \equip C^B\).
    \item
     \(A \times C \equip B \times C\)
   \end{itemize}
 \end{itemize}

 \section*{The Maths}

 \begin{theorem} \label{thm_r_to_r}
  \(\R^\R \equip \powerset(\R)\)
 \end{theorem}
 \begin{proof}
  By Cantor-Schr\"oder-Bernstein.

  Since a function on a set is a kind of relation on that set,
  \(\R^\R \subseteq \powerset(\R^2) \sim \powerset(\R)\), giving the first
  injection.

  But also, a subset \(S \subseteq \R\) can be identified by the indicator
  function of that subset:
  \begin{align*}
   1_S\colon \R &\to \R \\
   x &\mapsto
   \begin{cases}
    1 & x \in S \\
    0 & x \notin S
   \end{cases}
  \end{align*}
  giving the reverse injection \(S \mapsto 1_S\).
 \end{proof}

 \begin{theorem} \label{thm_cont}
  \(\set{f\colon \R \to \R \mid \text{\(f\) is continuous}}
     \equip \R\)
 \end{theorem}
 \begin{proof}
  By Cantor-Schr\"oder-Bernstein.

  A continuous function \(f\colon \R \to \R\) is identified by its restriction
  to \(\Q\), since for any real \(x\), \(f(x)\) can be recovered by letting
  \(x_n\) be some sequence in \(\Q\) with \(x_n \to x\) (eg decimal expansions)
  and using \(f(x) = f(\lim_{n \to \infty} x_n) = \lim_{n \to \infty} f(x_n)\).
  This gives the first injection, since we have an injection
  \(\set{f\colon \R \to \R \mid \text{\(f\) is continuous}}
    \to \R^\Q\), and \(\R^\Q \sim \R^\N \sim \R\).

  The second injection can be given by \(x \mapsto (y \mapsto x)\) - mapping
  reals to constants functions taking that value.
 \end{proof}

 \begin{theorem}
  \(\set{f\colon \R \to \R \mid \text{\(f\) is monotone}}
     \equip \R\)
 \end{theorem}
 \begin{proof}
  By Cantor-Schr\"oder-Bernstein.

  We employ a similar approach to the proof of Theorem \ref{thm_cont}.

  We know that the set of discontinuities of a monotone function is countable
  (by an example sheet). Therefore when we restrict a monotone function
  \(f\colon \R \to \R\) to \(\Q\), this lets us recover all but countably many
  values of \(f(x)\).

  So let's define an injection as follows:
  \begin{align*}
   g\colon \set{f\colon \R \to \R \mid \text{\(f\) is monotone}}
    &\to \R^\Q \times (\R \times \R)^\N \\
    f &\mapsto ((x \mapsto f(x)), h(f))
  \end{align*}
  where
  \(h: \set{f\colon \R \to \R \mid \text{\(f\) is monotone}}
     \to (\R \times \R)^\N\) is
  defined as follows:

  Given \(f: \R \to \R\) which is monotone, let the sequence \(a_n\) in \(\R\)
  enumerate the input values at which \(f\) is discontinuous. If there are
  finitely many, then once the enumeration is finished, let \(a_n\) be
  identically \(0\) (this is a bit of a formality - it doesn't matter what
  happens after this point, since I won't use it).

  Then for all \(n\), define \(h(f)(n) = (a_n, f(a_n))\).

  Then \(g\) is an injection, because it allows the recovery of all values of
  \(f(x)\). If \(f\) is continuous at \(x\), then \(f(x)\) can be recovered in
  the same way as before, by taking the limit of a sequence in \(\Q\) tending to
  \(x\).

  If \(x\) is a point of discontinuity, then the value of \(f(x)\) can be
  recovered by looking up the value of \(f(x)\) in the right component of
  \(g(f)\), the first time it occurs, since any discontinuity will be listed
  somewhere by \(h(f)\).

  Then we get an injection
  \(\set{f\colon \R \to \R \mid \text{\(f\) is monotone}} \to \R\), since \\
  \(\R^\Q \times (\R \times \R)^\N
     \equip \R^\N \times \R^\N
     \equip \R \times \R
     \equip \R\).

  The reverse injection can be given by \(x \mapsto (y \mapsto y + x)\), mapping
  real numbers to the monic linear polynomial shifted by the real as a constant
  term. These are strictly increasing.
 \end{proof}

 \begin{theorem}
  \(\set{f\colon \R \to \R \mid \text{\(f\) is Riemann integrable}}
     \equip \powerset(\R)\)
 \end{theorem}
 \begin{proof}
  By Cantor-Schr\"oder-Bernstein.

  The first injection can be obtained immediately by composing the inclusion map
  into \(\R^\R\) with the bijection from Theorem \ref{thm_r_to_r}. It remains to
  obtain the reverse injection.

  We go on somewhat of a tangent. The main idea we employ is to try and
  construct an uncountable set that takes up no space, based on a picture we
  once saw of the Cantor set. Then we can waddle a function from \(\R \to \R\)
  nicely over to that set, and show that its integral is zero, so it's
  integrable.

  Define the function \(C\) on the set of binary sequences by
  \begin{equation*}
   C((a_n)) = \sum_{n = 1}^\infty a_n \cdot 3^{-n}
  \end{equation*}
  This sum always converges, by comparison with \(\sum_n 3^{-n} = \frac 12\).

  We claim further that \(C\) is injective%
  \footnote{
   This is really because of intuition about ternary expansions, but I thought
   it would be good manners to prove it.
  }.

  Suppose there are two such distinct sequences \((a_n)\) and \((b_n)\), and
  that \(C((a_n)) = C((b_n))\). Let \(k \in \N\) be the smallest number such
  that \(a_k \ne b_k\). Without loss of generality, suppose that \(a_k = 0\) and
  \(b_k = 1\). Now, by cancelling all terms prior to \(k\), we have
  \begin{equation*}
   \sum_{\mathclap{n = k + 1}}^\infty a_n \cdot 3^{-n}
    = 3^{-k} + \sum_{\mathclap{n = k + 1}}^\infty b_n \cdot 3^{-n}
  \end{equation*}
  But
  \begin{equation*}
   \sum_{\mathclap{n = k + 1}}^\infty a_n \cdot 3^{-n}
    \le \sum_{\mathclap{n = k + 1}}^\infty 3^{-n}
    = \tfrac 12 \cdot 3^{-k}
  \end{equation*}
  so the LHS is \(\le \tfrac 12 \cdot 3^{-k}\). But each term in the sum in the
  RHS is nonnegative, so the RHS is \(\ge 3^{-k}\). This is absurd, since
  \(3^{-k} > 0\), so we conclude that \(a_n = b_n\) for all \(n\), and hence
  \(C\) is injective.

  Since \(2^\N \equip \R\), the image of \(C\) bijects with \(\R\):
  \(C(2^\N) \equip \R\).

  By splitting up the sum and using the same bound, we have that for all \(k\),
  \begin{equation*}
   \sum_{n = 1}^k a_n \cdot 3^{-n}
    \le C((a_n))
    \le \tfrac 12 \cdot 3^{-k} + \sum_{n = 1}^k a_n \cdot 3^{-n}
  \end{equation*}
  But
  \begin{equation*}
   \sum_{n = 1}^k a_n \cdot 3^{-n}
    = 3^{-k} \sum_{n = 1}^k a_n \cdot 3^{k - n}
  \end{equation*}
  so letting \(M = \sum_n a_n \cdot 3^{k - n}\) (note that \(M \in \N)\), we get
  that for all \(k\),
  \begin{equation*} \label{eq_C_even_odd}
   \tfrac 12(2M) \cdot 3^{-k} \le C((a_n)) \le \tfrac 12(2M + 1) \cdot 3^{-k}
   \tag{\(\ast\)}
  \end{equation*}
  for some \(M\). Particularly, we cannot have
  \begin{equation*}
   \tfrac 12(2M + 1) \cdot 3^{-k} < C((a_n)) < \tfrac 12(2M + 2) \cdot 3^{-k}
  \end{equation*}
  for any \(M\), as this open interval is disjoint from all the intervals of the
  form given in (\ref{eq_C_even_odd}).

  Now, at last, we begin to construct our injection. By Theorem
  \ref{thm_r_to_r}, it suffices to give an injection from \(\R^\R\). By the
  discussion about the image of \(C\), it suffices to give an injection from
  \(\R^{C(2^\N)}\), and by cardinality of intervals, it suffices to give an
  injection from \(\intcc{0, 1}^{C(2^\N)}\). I claim that this injection is
  given by ``unrestriction'':

  Given \(f: C(2^\N) \to \intcc{0, 1}\), consider its unrestriction to \(\R\).
  This is the function
  \begin{align*}
   g\colon \R &\to \R \\
   x &\mapsto
   \begin{cases}
    f(x) & x \in C(2^\N) \\
    0 & x \notin C(2^\N)
   \end{cases}
  \end{align*}
  The mapping \(f \mapsto g\) is clearly an injection, since we can rerestrict
  \(g\) to \(C(2^\N)\) to recover \(f\).

  Now it remains only to show that this injection is well-defined - ie \(g\) is
  integrable. By the general bound \(0 \le C((a_n)) \le \tfrac 12\), it suffices
  to show that \(g\) is integrable on \(\intcc{0, \tfrac 12}\), as it's zero
  everywhere else.

  I claim that for all \(g\),
  \begin{equation*}
   \int_0^{\tfrac 12} g(x) dx = 0
  \end{equation*}
  Proof by the \(\epsilon\) characterisation:

  Given \(\epsilon > 0\), choose \(k\) so that
  \(\tfrac 12 \cdot(\tfrac 23)^k < \epsilon\), and then let
  \newcommand*\dissD{\mathcal D}%
  \(\dissD = \set{a_0, \dotsc, a_{2n - 1}}\), where \(n = 2^k\), and \(a_i\)
  will be explained shortly. For each \(i = 0, \dotsc, n\), let \(B(i)\) be the
  big-endian (ie, normal) binary expansion of \(i\), as a function from index to
  bits. That is, \(B\) should be such that for each \(i = 0, \dotsc, n\),
  \begin{equation*} \label{eq_binary}
   i = \sum_{j = 0}^{k - 1} B(i)(j) \cdot 2^{k - 1 - j}
   \tag{\(\dag\)}
  \end{equation*}
  It can be checked that for \(0 \le i < 2^k\) and \(0 \le j < k\), the
  recurrences
  \begin{align*}
   B(0)(j) &= 0 \\
   B(i)(j) &=
   \begin{cases}
    i \bmod 2 & j = k - 1 \\
    B(\floor{i/2})(j + 1) & j \ne k - 1
   \end{cases}
  \end{align*}
  are well-founded and define such a \(B\).

  Then, define the even and odd \(a_i\) separately as follows:
  \begin{align*}
   a_{2i} &= \tfrac 13 \sum_{j = 0}^{k - 1} B(i)(j) \cdot 3^{-j} \\
   a_{2i + 1} &= a_{2i} + \tfrac 12 \cdot 3^{-k}
  \end{align*}
  Certainly we have that \(a_0 = 0\) and
  \begin{align*}
   a_{2n - 1}
    &= \tfrac 12 \cdot 3^{-k}
     + \tfrac 13 \sum_{j = 0}^{k - 1} B(2^k - 1)(j) \cdot 3^{-j} \\
    &= \tfrac 12 \cdot 3^{-k} + \tfrac 13 \sum_{j = 0}^{k - 1} 3^{-j} \\
    &= \tfrac 12 \cdot 3^{-k} + \tfrac 12 (1 - 3^{-k}) \\
    &= \tfrac 12
  \end{align*}
  I will now discuss further why this gives rise to a well-defined dissection.

  If \(i_1 < i_2\), then for the first \(J\) where \(B(i_1)(J) \ne B(i_2)(J)\),
  we must have \(0 = B(i_1)(J) < B(i_2)(J) = 1\). Otherwise, we would get
  \begin{align*}
   i_2 - i_1
    &= -2^{k - 1 - J} +
        \sum_{\mathclap{j = J + 1}}^{k - 1}
         2^{k - 1 - j}[B(i_2)(j) - B(i_1)(j)] \\
    &\le -2^{k - 1 - J} +
        \sum_{\mathclap{j = J + 1}}^{k - 1}
         2^{k - 1 - j} \cdot B(i_2)(j) \\
    &\le -2^{k - 1 - J} + \sum_{\mathclap{j = J + 1}}^{k - 1} 2^{k - 1 - j} \\
    &= -2^{k - 1 - J} + 2^{k - 1 - J} - 1 \\
    &= -1
  \end{align*}
  contradicting \(i_1 < i_2\). There certainly is such a \(J\), as otherwise it
  would be the case that \(i_1 = i_2\), by (\ref{eq_binary}). Therefore, we have
  \begin{align*}
   3(a_{2i_2} - a_{2i_1})
    &= 3^{-J} +
        \sum_{\mathclap{j = J + 1}}^{k - 1} 3^{-j}[B(i_2)(j) - B(i_1)(j)] \\
    &\ge 3^{-J} - \sum_{\mathclap{j = J + 1}}^{k - 1} 3^{-j} \cdot B(i_1)(j) \\
    &\ge 3^{-J} - \sum_{\mathclap{j = J + 1}}^{k - 1} 3^{-j} \\
    &= 3^{-J} - 3^{-(J + 1)} \cdot
        \sum_{\mathclap{j = 0}}^{\mathclap{k - J - 2}} 3^{-j} \\
    &= 3^{-J} - 3^{-(J + 1)}(3^{-(k - J - 1)} - 1) / \tfrac 23 \\
    &= 3^{-J} - \tfrac 12 (3^{-k + 1} - 3^{-J}) \\
    &= \tfrac 12 (3^{-(J - 1)} - 3^{-(k - 1)})
  \end{align*}
  However, the \(J\) for \(2i_1\) and \(2i_2\) cannot be \(k - 1\), since
  \(B(2i + 2)(k - 1)
    \equiv 2i + 2 \equiv 2i \equiv B(2i)(k - 1) \equiv 0 \pmod 2\),
  by taking (\ref{eq_binary}) modulo 2, so \(B(2i_1)(k) = B(2i_2)(k) = 0\).
  Therefore \(J - 1  < k - 1\), so \(a_{2i_2} > a_{2i_1}\), ie the \(a_i\) of
  even index are in the right order.

  It remains to show that \(a_{2i} < a_{2i + 1} < a_{2i + 2}\). The first
  inequality follows immediately from the definition of \(a_{2i + 1}\). The
  second can be obtained by recycling our work above:
  \(3(a_{2i + 2} - a_{2i}) \ge \tfrac 12 (3^{-(J - 1)} - 3^{-(k - 1)})\) with
  \(J\) as above, so
  \begin{align*}
   3(a_{2i + 2} - a_{2i + 1})
    &= 3(a_{2i + 2} - a_{2i}) - \tfrac 32 \cdot 3^{-k} \\
    &\ge \tfrac 12 (3^{-(J - 1)}
     - 3^{-(k - 1)}) - \tfrac 12 \cdot 3^{-(k - 1)} \\
    &= \tfrac 12 (3^{-(J - 1)} - 2 \cdot 3^{-(k - 1)}) \\
    &> \tfrac 12 (3^{-(J - 1)} - 3 \cdot 3^{-(k - 1)}) \\
    &= \tfrac 12 (3^{-(J - 1)} - 3^{-(k - 2)}) \\
    &\ge 0 \quad \text{as \(J - 1 \le k - 2\), as discussed.}
  \end{align*}
  It follows that \(a_{2i + 2} > a_{2i + 1}\), and hence \(\dissD\) is a
  well-defined dissection.

  Now let \(Z(i)\) denote the largest number less than or equal to \(k\) such
  that \(B(i)(j) = 0\) for all \(k - Z(i) \le j < k\).%
  \footnote{
   This is essentially the largest number such that
   \(2^{Z(i)}\) divides \(i\).
  }

  Then for all \(i < k - 1\), we have \(k - 1 - Z(i + 1) \ge 0\) and
  \(B(i + 1)(k - 1 - Z(i + 1)) = 1\), since \(i + 1 > 0\), so its binary
  expansion cannot all be zeroes.

  Writing \(Z = Z(i + 1)\) and noting that
  \begin{equation*}
   2^{(k - 1) - (k - 1 - Z)}
    = 1 + \sum_{\mathclap{j = k - Z}}^{k - 1}
               2^{k - 1 - j}
  \end{equation*}
  we must have that \(B(i)(k - 1 - Z) = 0\),
  \(B(i)(j) = 1\) for all \(k - Z \le j < k\), and \(B(i)(j) = B(i + 1)(j)\) for
  all \(0 \le j < k - 1 - Z\), since this binary expansion is one less than the
  expansion for \(i + 1\), and by uniqueness of binary expansions. Hence,
  \begin{align*}
   a_{2i + 2}
    &= \tfrac 13 \sum_{j = 0}^{k - 1} B(i + 1)(j) \cdot 3^{-j} \\
    &= \tfrac 13 \cdot 3^{-(k - 1 - Z)}
     + \tfrac 13 \sum_{j = 0}^{\mathclap{k - 2 - Z}} B(i + 1)(j) \cdot 3^{-j} \\
    &= 3^{-(k - Z)}
     + \tfrac 13 \sum_{j = 0}^{\mathclap{k - 2 - Z}} B(i + 1)(j) \cdot 3^{-j} \\
    &= \tfrac 12 (2M + 2) \cdot 3^{-(k - Z)} \\
   a_{2i + 1}
    &= \tfrac 12 \cdot 3^{-k}
     + \tfrac 13 \sum_{j = 0}^{k - 1} B(i)(j) \cdot 3^{-j} \\
    &= \tfrac 12 \cdot 3^{-k}
     + \tfrac 13 \sum_{j = 0}^{\mathclap{k - 2 - Z}} B(i + 1)(j) \cdot 3^{-j}
     + \tfrac 13 \sum_{\mathclap{j = k - Z}}^{k - 1} 3^{-j} \\
    &= \tfrac 12 \cdot 3^{-k} + a_{2i + 2} - 3^{-(k - Z)}
     + \tfrac 13 \sum_{\mathclap{j = k - Z}}^{k - 1} 3^{-j} \\
    &= \tfrac 12 \cdot 3^{-k} + a_{2i + 2} - 3^{-(k - Z)}
     + 3^{Z - k - 1} \cdot \sum_{\mathclap{j = 0}}^{Z - 1} 3^{-j} \\
    &= \tfrac 12 \cdot 3^{-k} + a_{2i + 2} - 3^{-(k - Z)}
     + 3^{Z - k - 1} \cdot (1 - 3^{-Z})/\tfrac 23 \\
    &= \tfrac 12 \cdot 3^{-k} + a_{2i + 2} - 3^{-(k - Z)}
     + \tfrac 12 (3^{Z - k} - 3^{-k}) \\
    &= a_{2i + 2} - \tfrac 12 \cdot 3^{Z - k} \\
    &= \tfrac 12 (2M + 2) \cdot 3^{-(k - Z)} - \tfrac 12 \cdot 3^{Z - k} \\
    &= \tfrac 12 (2M + 1) \cdot 3^{-(k - Z)}
  \end{align*}
  where
  \begin{align*}
   M = 3^{k - 1 - Z} \cdot
        \sum_{j = 0}^{\mathclap{k - 2 - Z}} B(i + 1)(j) \cdot 3^{-j}
  \end{align*}
  is certainly an integer, as \(k - 1 - Z\) is always larger than \(j\), so each
  term in the sum is an integer.

  This means that in all the intervals \(\intoo{a_{2i + 1}, a_{2i + 2}}\), there
  are no points in \(C(2^\N)\), so \(g\) is constantly zero in these intervals.

  Now we can use all of this by splitting the upper sum into odds and evens:
  \begin{align*}
   U(g, \dissD)
    &= \sum_{i = 1}^{2n - 1} (a_i - a_{i - 1}) \sup g(\intoo{a_{i - 1}, a_i}) \\
    &= \sum_{i = 0}^{n - 2}
            (a_{2i + 2} - a_{2i + 1}) \sup g(\intoo{a_{2i + 1}, a_{2i + 2}})
     + \sum_{i = 0}^{n - 1}
            (a_{2i + 1} - a_{2i}) \sup g(\intoo{a_{2i}, a_{2i + 1}}) \\
    &= \sum_{i = 0}^{n - 2}
            (a_{2i + 2} - a_{2i + 1}) \cdot 0
     + \sum_{i = 0}^{n - 1}
            \tfrac 12 \cdot 3^{-k} \cdot \sup g(\intoo{a_{2i}, a_{2i + 1}}) \\
    &\le \sum_{i = 0}^{n - 1}
              \tfrac 12 \cdot 3^{-k} \cdot 1 \\
    &= \tfrac 12 \cdot 3^{-k} \cdot n \\
    &= \tfrac 12 \cdot (\tfrac 23)^k \\
    &< \epsilon
  \end{align*}
  and certainly \(L(g, \dissD) \ge 0 > -\epsilon\), as \(g\) is nonnegative. (In
  fact, \(L(g, \dissD) = 0\) because the complement of \(C(2^\N)\) is dense in
  \(\R\), but that's overkill).

  Hence \(g\) is integrable, so we have our injection from \(\powerset(\R)\) to
  the set of integrable functions.
 \end{proof}

\end{document}
