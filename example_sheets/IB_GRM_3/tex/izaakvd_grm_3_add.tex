% Compiling with
% latexmk -halt-on-error -shell-escape -synctex=1 -pdf <file.tex>
% (Recommend using a latexmkrc file so as just to run latexmk -pvc, for example)
% Probably you can achieve the same with an inordinate number of invocations of
% pdflatex -halt-on-error -shell-escape -synctex=1 <file.tex>

% fleqn aligns equations to the left, a4 paper size, 11pt font, article class
\documentclass[fleqn,a4paper,11pt]{article}
\title{Some Remarks about Algebra}
\author{\texorpdfstring{Izaak van Dongen (\texttt{imv26})}
                       {Izaak van Dongen (imv26)}}

\usepackage{mymaths}
\usepackage{mystyle}

% Embed source files into PDF in case of loss. You can view or extract the
% source files by doing `pdfdetach -list <file.pdf>` or
% `pdfdetach -saveall <file.pdf>`, using pdfdetach from poppler, or some other
% suitable method.
\usepackage{embedall}
\embedfile{mymaths.sty}
\embedfile{mystyle.sty}

\begin{document}
\maketitle

\section[Question 14, GRM Sheet 3]{Question 14, GRM Sheet 3\footnote{%
 This answer was developed in collaboration with Alastair Horn, although any
 mistakes are certainly my own.}}

\begin{tcolorbox}
 ``If a UFD has at least one irreducible, must it have infinitely many (pairwise
 non-associate) irreducibles?''
\end{tcolorbox}

% TODO: integral domain, X non-unit

Consider the ring \(\Q\dbrack X\). Note that \(1 + X \in \Q\dbrack X\) is a
unit, since
\begin{equation*}
 (1 + X)(1 - X + X^2 - X^3 + \dotsb + (-1)^n X^n + \dotsb) = 1
\end{equation*}
since the coefficient in \(X^n\) for \(n > 0\) in the product is
\((-1)^n + (-1)^{n + 1} = 0\). In fact, suppose that
\(f = \sum_n a_n X^n \in \Q\dbrack X\) has \(a_0 = 0\) (ie, \(X \divides f\)).
Then we can define
\begin{equation*}
 1 - f + f^2 - f^3 + \dotsb + (-1)^n f^n + \dotsb \in \Q\dbrack X
\end{equation*}
as the formal power series \(\sum_n a_n' X^n\) where
\(a_n' = \sum_{i = 0}^\infty ((-1)^i f^i)_n\), where \((g)_n\) denotes ``the
coefficient in \(X^n\) in \(g\)''. This sum is convergent and well-defined in
\(\Q\) precisely because \(X \divides f\), so \(X^i \divides f^i\), ie
\((f^i)_n = 0\) for any \(n < i\), so really
\(a_n' = \sum_{i = 0}^n ((-1)^i f^i)_n\). Then
\begin{equation*}
 (1 + f)(1 - f + f^2 - f^3 + \dotsb + (-1)^n f^n + \dotsb) = 1
\end{equation*}
since
\begin{align*}
 (1 + f)(1 - f + f^2 - \dotsb)
  &= (1 - f + f^2 - \dotsb) + f(1 - f + f^2 - \dotsb) \\
  &= (1 - f + f^2 - \dotsb) + (f - f^2 + f^3 - \dotsb) \\
  &= 1
\end{align*}
where these manipulations are allowable due to similar considerations about
\(X^n\) only appearing finitely often - eg
\begin{align*}
 (f(1 - f + f^2 - \dotsb))_n
  &= (f(1 - f + f^2 - \dotsb + (-1)^n f^n))_n \\
  &= (f - f^2 + f^3 - \dotsb + (-1)^n f^{n + 1})_n \\
  &= (f - f^2 + f^3 - \dotsb)_n
\end{align*}
So in fact any element of \(\Q\dbrack X\) of the form \(1 + f\) for such an
\(f\) is a unit. If \(0 \ne a \in \Q\), the constant series
\(a \in \Q\dbrack X\) is also a unit, inverted by the inverse \(a^{-1}\) of
\(a\) in \(\Q\). Therefore it is in fact the case that any element of
\(\Q\dbrack X\) with a nonzero constant term \(a\) is an associate of a series
with constant term \(1\), so is a unit.

Now let \(f = \sum_n a_n X^n\), \(g = \sum_n b_n X^n\) be arbitrary nonzero
elements of \(\Q\dbrack X\). Let \(i\) and \(j\) be the least respective
naturals such that \(a_i \ne 0\), and \(b_j \ne 0\). Then
\((fg)_{ij} = a_i b_j \ne 0\), and for all \(k < ij\), \((fg)_k = 0\). This
shows that \(\Q\dbrack X\) is an integral domain, but also that the degree of
the least significant terms is additive when multiplying in \(\Q\dbrack X\).


Therefore, for any \(f \in \Q\dbrack X\), we have that the degree of the least
significant term of \(Xf\) is at least \(1\). Therefore, \(X\) is not a unit,
and in fact, \(X^n\) for \(n > 0\) is not a unit. If \(f\) is an arbitrary
nonzero element with \(i\) as before, then factoring out \(X^i\) - say
\(f = X^i h\) - leaves a power series \(h\) with a nonzero constant term. So
\(h\) is a unit, so in fact every element \(f\) of \(\Q\dbrack X\) is an
associate of \(X^i\), where \(i\) is the degree of the least significant term of
\(f\).

So the ``associate classes'' in \(\Q\dbrack X\) are exactly the classes of
associates of \(X^n\), for \(n \in \N\).

Suppose then that \(I \ideal \Q\dbrack X\). Let \(n\) be the least \(n\) such
that \(X^n \in I\) (ie, such that the class of associates of \(X^n\) intersects
\(I\)). Then \(I = (X^n)\), since it must contain \(X^n\), and if
\(I\) contained an element not in \((X^n)\), that element would have to be an
associate of \(X^k\) with \(k < n\), contradicting minimality of \(n\). So
\(\Q\dbrack X\) is a PID, and hence a UFD.

But, up to associates, the non-units in \(\Q\dbrack X\) are \(\{X^n\}\). For
\(n > 1\), \(X^n = X X^{n - 1}\) is a decomposition into non-units, so \(X^n\)
is not irreducible. \(X\), however, \emph{is} irreducible, as the only possible
way to factor it requires at least one factor to have a nonzero constant term,
ie one factor must be a unit. So \(\Q\dbrack X\) contains exactly one
irreducible, up to associates.

\section{Question 13, GRM Sheet 3}

Define \(\conj{a + b\sqrt 2} \defeq a - b\sqrt 2\), and
\begin{alignat*}2
 N :&& \Z[\sqrt 2] &\to \N \\
     &&a + b\sqrt 2 &\mapsto
         \abs{(a + b \sqrt 2)\conj{(a + b\sqrt 2)}}
         = \abs{a^2 - 2b^2}
\end{alignat*}
Noting that
\begin{align*}
 \conj{(a + b\sqrt 2)(c + d\sqrt 2)}
 &= ac + 2bd - (ad + bc)\sqrt 2 \\
 &= \conj{(a + b\sqrt 2)} \cdot \conj{(c + d\sqrt 2)}
\end{align*}
we must have \(N(xy) = N(x)N(y)\), so \(N\) satisfies the first condition to be
a Euclidean function.

Now let \(x = a + b\sqrt 2\) and \(0 \ne y = c + d \sqrt 2\) be arbitrary.
Calculating in \(\Q[\sqrt 2]\),
\begin{align*}
 \frac xy
  &= \frac{ac - 2bd + (bc - ad)\sqrt 2}{c^2 - 2d^2} \\
  &= \frac{ac - 2bd}{c^2 - 2d^2} + \frac{bc - ad}{c^2 - 2d^2} \cdot \sqrt 2
\end{align*}
But both of these two fractions are within at most \(\tfrac 12\) of an integer.
Say \(q_1, q_2 \in \Z\) with
\begin{equation*}
 \abs[\bigg]{q_1 - \frac{ac - 2bd}{c^2 - 2d^2}} \le \tfrac 12, \qquad
 \abs[\bigg]{q_2 - \frac{bc - ad}{c^2 - 2d^2}} \le \tfrac 12
\end{equation*}
Say \(q = q_1 + q_2 \sqrt 2\). Then letting \(r = x - yq\), we get
\(x = yq + r\), where
\begin{align*}
 N(r)
  &= N(x - yq) \\
  &= N(y)N(x/y - q),
    \quad \text{using the obvious extension of \(N\) to \(\Q[\sqrt 2]\)} \\
  &= N(y) \abs[\Bigg]{\parens[\bigg]{q_1 - \frac{ac - 2bd}{c^2 - 2d^2}}^2
            - 2\parens[\bigg]{q_2 - \frac{bc - ad}{c^2 - 2d^2}}^2} \\
  &\le N(y) \bracks[\Bigg]{\abs[\bigg]{q_1 - \frac{ac - 2bd}{c^2 - 2d^2}}^2
            + 2\abs[\bigg]{q_2 - \frac{bc - ad}{c^2 - 2d^2}}^2} \\
  &\le N(y) \bracks{(\tfrac 12)^2 + 2(\tfrac 12)^2} \\
  &< N(y)
\end{align*}
So \(N\) is a proper Euclidean function, since also \(N(1) = 1\).

Now \(a + b\sqrt 2\) is a unit iff its norm is \(1\), ie \(a^2 - 2b^2 = \pm 1\),
since \(N\) is a Euclidean function. This is very reminiscent of a Pell
equation.

Note that \(1 + \sqrt 2\) certainly is a unit, with inverse \(-1 + \sqrt 2\).

Now suppose \(a + b\sqrt 2 > 1\) is a unit with the usual \(>\) in \(\R\).
Then its inverse is \(\pm(a - b\sqrt 2)\), so \(\abs{a - b\sqrt 2} < 1\), so
\(-1 < a - b\sqrt 2\) and \(-1 < -a + b \sqrt 2\). Therefore \(0 < a + b\sqrt 2
+ a - b\sqrt 2 = 2a\) and similarly \(0 < 2b\sqrt 2\), so \(a\) and \(b\) are
both positive. But \(1 + \sqrt 2\) is clearly the smallest element of
\(\Z[\sqrt 2]\) with both coefficients positive. So it is the least unit that's
at least \(1\).

Now suppose that \(x \in \Z[\sqrt 2]\) is a positive unit.
\((1 + \sqrt 2)^n \to \infty\) monotonically as \(n \to \infty\) and
\((1 + \sqrt 2)^n \to 0\) monotonically as \(n \to -\infty\). Therefore there is
some \(n \in \Z\) with \((1 + \sqrt 2)^n \le x < (1 + \sqrt 2)^{n + 1}\). Then
multiplying by \((1 + \sqrt 2)^{-n}\), we get
\(1 \le x(1 + \sqrt 2)^{-n} < 1 + \sqrt 2\). Since there is no unit strictly
contained between \(1\) and \(1 + \sqrt 2\), \(x(1 + \sqrt 2)^{-n} = 1\), so
\(x = (1 + \sqrt 2)^n\), for some \(n \in \Z\).

If \(x\) is instead a negative unit, then \(-x\) is a positive unit. So
\(x = -(1 + \sqrt 2)^n\) for some \(n \in \Z\).

Of course, since \((1 + \sqrt 2)^{-1} = (-1 + \sqrt 2)\), when \(n < 0\), we can
write \((1 + \sqrt 2)^n = (-1 + \sqrt 2)^{-n} = (-1)^n (1 - \sqrt 2)^{-n}\). So
indeed the units are exactly \(\pm(1 \pm \sqrt 2)^n\).

\section{Question 13, Sheet 3, 2013}

\begin{tcolorbox}
 Let \(R\) be a ring on ground-set \(\Z\) whose multiplication is the same as
 the usual multiplication on \(\Z\). Must its addition be the same as the usual
 addition on \(\Z\)?
\end{tcolorbox}

Suppose \(R = (\Z, \oplus, \cdot)\) is a ring, where \(\cdot\) is the usual
multiplication, and \(\oplus\) is some operation that satisfies the ring axioms.

\(R\) still has multiplicative identity \(1\). The additive identity must also
be \(0\), since if \(0_R\) is an additive identity, the map
\(x \mapsto 0_R \cdot x\) is not injective, but multiplication by anything
other than \(0\) in \(\Z\) is an injective function.

Note that
\(1 \oplus (-1) = (-1)^2 \oplus (-1) = (-1)((-1) \oplus 1)\), so we must have
\(1 \oplus (-1) = 0\), as \(R\) is an integral domain and \(-1 \ne 1\). So the
additive inverse of \(1\) in \(R\) is \(-1\), and therefore in general the
additive inverse of \(n\) is \(-n\).

Suppose \(R\) has finite characteristic. Then its characteristic must be some
prime \(p\), as \(R\) is still an integral domain, and if it had characteristic
\(c = ab\) where \(a, b < c\), then \(a\) and \(b\) (interpreted as multiples of
\(1_R\)) would be zero divisors. \(R\) also doesn't have characteristic \(1\) as
it is non-trivial.

Then the subring generated by \(1\) has order \(p\), so is isomorphic to
\(\Z/p\Z\), so is a field. All of its nonzero elements are units, but \(\Z\)
only has two units, and both of them are generated by \(1\) in \(R\), so in fact
\(p = 3\).

\end{document}
