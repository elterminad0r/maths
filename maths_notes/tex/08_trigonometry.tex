\section{Trigonometry}

\subsection{Definitions} \label{sec_trig_definitions}

%FIXME add diagram

\begin{align*}
 \sin \theta &= \frac OH \\
 \cos \theta &= \frac AH \\
 \tan \theta &= \frac OA = \frac{\sin \theta}{\cos \theta}
\end{align*}

\subsubsection{Periodicity} \label{sec_trig_periodic}

%FIXME add more identities, diagrams (or refer to previous diagrams).

Because they are in the same right triangle, and the sum of angles in a
triangle is \(\pi\) (\(\ang{180}\)) (\ref{sec_geom_polygon_angles}), the
angle other than \(\theta\) must be \(\frac 12 \pi - \theta\). Therefore,
\begin{align*}
 \sin \theta &\equiv \cos(\frac 12 \pi - \theta) \\
 \cos \theta &\equiv \sin(\frac 12 \pi - \theta)
\end{align*}
From their parity \ref{sec_trig_parity} we may deduce
\begin{align*}
 \sin(\theta - \frac 12 \pi) &\equiv -\cos \theta \\
 \cos(\theta - \frac 12 \pi) &\equiv \sin \theta
\end{align*}
implying
\begin{align*}
 \sin(\theta + \pi) &\equiv -\sin \theta \\
 \cos(\theta + \pi) &\equiv -\cos \theta
\end{align*}
From this, we see that
\begin{alignat}2
 \sin(\theta + 2\pi) &\equiv \sin(\theta + \pi + \pi) &&\equiv \sin \theta \\
 \cos(\theta + 2\pi) &\equiv \cos(\theta + \pi + \pi) &&\equiv \cos \theta \\
 \tan(\theta + \pi) &\equiv \frac{\sin(\theta + \pi)}{\cos(\theta + \pi)}
     &&\equiv \frac{-\sin \theta}{-\cos \theta} \equiv \tan \theta
\end{alignat}
and in fact,
\begin{alignat}2
 \sin(\theta + 2\pi n) &\equiv \sin \theta &&\iff n \in \Integers \\
 \cos(\theta + 2\pi n) &\equiv \cos \theta &&\iff n \in \Integers \\
 \tan(\theta + \pi n) &\equiv \tan \theta &&\iff n \in \Integers
\end{alignat}
Therefore, the \emph{period} of \(\sin \theta\) and \(\cos \theta\) is
\(2\pi\), whereas the period of \(\tan \theta\) is \(\pi\).

\subsubsection{Parity} \label{sec_trig_parity}

From \ref{sec_trig_definitions}, we see that \(\sin \theta\) and
\(\tan \theta\) are \emph{odd} functions:
\(\sin -x = -\sin x\) and \(\tan -x = -\tan x\),
whereas \(\cos \theta\) is an \emph{even} function:
\(\cos -x = \cos x\).

\subsubsection{Reciprocal functions} \label{sec_trig_reciprocal}

%fixme some graphs

The reciprocals of sine, cosine and tangent are cosecant, secant and
cotangent respectively. These are written
\begin{align*}
 \sec \theta &= \frac 1{\cos \theta} \\
 \operatorname{cosec} \theta = \csc \theta
     &= \frac 1{\sin \theta} \\
 \cot \theta &= \frac 1{\tan \theta}
     = \frac{\cos \theta}{\sin \theta}
\end{align*}

\subsection{Special Angles}

%FIXME add diagrams, derive other angles

From geometric construction of triangles in addition to Pythagoras' Theorem
(\ref{sec_pythagoras}), we can determine that the following values of
trigonometric functions at certain angles hold.

\begin{table}[H]
 \centering
 \begin{tabular}{*4M}
  \toprule
  \text{\boldmath\(\theta\)} & \text{\boldmath\(\sin \theta\)}
  & \text{\boldmath\(\cos \theta\)} & \text{\boldmath\(\tan \theta\)} \\
  \midrule
  0 & 0 & 1 & 0 \\[1ex]
  \frac \pi 6 & \frac 12 & \frac{\sqrt 3} 2 & \frac {\sqrt 3} 3 \\[3ex]
  \frac \pi 4 & \frac {\sqrt 2} 2 & \frac {\sqrt 2} 2 & 1 \\[3ex]
  \frac \pi 3 & \frac{\sqrt 3} 2 & \frac 12 & \sqrt 3 \\[3ex]
  \frac \pi 2 & 1 & 0 & \infty\\[2ex]
  \bottomrule
 \end{tabular}
 \caption{The basic trigonometric constants}
\end{table}

Strictly, \(\tan \frac 12 \pi\) is undefined.

The symmetry of sine and cosine going opposite ways across this table is due
to the fact that \(\sin \theta \equiv \cos(\frac 12 \pi - \theta)\).

We need only discuss special angles in \(\intcc{0, \frac 12 \pi}\) because
the corresponding angles in the other quadrants can be found from symmetries
and periodicities. This is convenient as it means all values of trig
functions that we discuss here are positive, so we can take positive square
roots without remorse.

We can always calculate half-angles in the first quadrant using the
half-angle formulae (Theorem \ref{thm_trig_half_angle}). We can also use
compound angle formulae to build some angles, such as
\(\frac \pi {12} = \frac \pi 3 - \frac \pi 4\).

A small number of more esoteric angles are presented below. Many more can be
found \cite{WikiTrigConstants}.

\begin{table}[H]
 \centering
 \begin{tabular}{*4M}
  \toprule
  \text{\boldmath\(\theta\)} & \text{\boldmath\(\sin \theta\)}
  & \text{\boldmath\(\cos \theta\)} & \text{\boldmath\(\tan \theta\)} \\
  \midrule
  \frac \pi {12} & \frac{\sqrt 6 - \sqrt 2} 4 & \frac{\sqrt 6 + \sqrt 2} 4
  & 2 - \sqrt 3 \\[3ex]
  \frac \pi 8 & \frac 12 \sqrt{2 - \sqrt 2} & \frac 12 \sqrt{2 + \sqrt 2}
  & \sqrt 2 - 1 \\[3ex]
  \frac \pi 5 & \frac 14 \sqrt{10 - 2\sqrt 5} & \frac{\sqrt 5 + 1} 4
  & \sqrt{5 - 2\sqrt 5} \\[3ex]
  \frac{3\pi}{10} & \frac{\sqrt 5 + 1} 4 & \frac 14 \sqrt{10 - 2\sqrt 5}
  & \frac 15 \sqrt{25 + 10\sqrt 5} \\[3ex]
  \frac{3\pi} 8 & \frac 12 \sqrt{2 + \sqrt 2} & \frac 12 \sqrt{2 - \sqrt 2}
  & \sqrt 2 + 1 \\[3ex]
  \frac{5\pi}{12} & \frac{\sqrt 6 + \sqrt 2} 4 & \frac{\sqrt 6 - \sqrt 2} 4
  & 2 + \sqrt 3  \\[2ex]
  \bottomrule
 \end{tabular}
 \caption{More advanced trigonometric constants}
\end{table}

The second half of the table can be derived from the first half.

The result for \(\frac 1{12} \pi\) can be derived by using
\(\frac 1{12} \pi = \frac 13 \pi - \frac 14 \pi\).

The result for \(\frac 18 \pi\) can be derived by using
\(\frac 18 \pi = \frac 12 \cdot \frac 14 \pi\).

To get the result for \(\frac 15 \pi\) we can let
\(\varphi = \frac 15 \pi\).  Noting that
\(\sin 2\varphi = \sin 3\varphi\), we have
\begin{alignat*}2
 && 2\sin \varphi \cos \varphi
     &= \sin \varphi (2\cos^2 \varphi - 1)
     + \cos \varphi \cdot 2\sin \varphi \cos \varphi \\
 \iff{}&& 0 &= \sin \varphi (2\cos^2 \varphi - 1
                         + 2\cos^2 \varphi - 2\cos \varphi) \\
 \iff{}&& 0 &= 4\cos^2 \varphi - 2\cos \varphi - 1
     \quad \text{as \(\sin \varphi \ne 0\)} \\
 \iff{}&& \cos \varphi &= \frac{1 \pm \sqrt 5} 4 \\
 \iff{}&& \cos \varphi &= \frac{1 + \sqrt 5} 4
     \quad \text{as \(\cos \varphi > 0\)}
\end{alignat*}
We can then proceed to derive sine, and therefore also tangent, by applying
the Pythagorean identity and simplifying.

\subsection{Cosine Rule}

\subsection{Sine Rule}

\subsection{Sine Area Rule}

\subsection{Pythagorean Identities} \label{sec_trig_pythag}

It follows from Pythagoras' Theorem (\ref{sec_pythagoras})
and subsequently from the definition of
\(\tan\), \(\cot\), \(\sec\), \(\csc\) (\ref{sec_trig_reciprocal}) that
\begin{align*}
 \cos^2 \theta + \sin^2 \theta &\equiv 1 \\
 \implies \tan^2 \theta + 1 &\equiv \sec^2 \theta \\
 \cot^2 \theta + 1 &\equiv \csc^2 \theta
\end{align*}

\subsection{Compound Angle Identities} \label{sec_comp_angle}

%FIXME add diagram

\begin{theorem}[Addition formulae for trig functions]
 Trigonometric functions of a sum or difference of angles can be expanded into
 trigonometric functions of each angle.
 \begin{align*}
  \sin(\alpha + \beta) &\equiv
     \sin \alpha \cos \beta +  \sin \beta \cos \alpha \\
  \cos(\alpha + \beta) &\equiv
     \cos \alpha \cos \beta - \sin \alpha \sin \beta \\
  \tan(\alpha + \beta) &\equiv
      \frac{\tan \alpha + \tan \beta}{1 - \tan \alpha \tan \beta}
 \end{align*}
\end{theorem}
\begin{proof}
 The formula for sine can be shown geometrically.

 The formula for cosine could also be shown with a similar geometrical argument,
 but we can simply use section \ref{sec_trig_periodic} to show
 \begin{align*}
  \cos(\alpha + \beta) &\equiv
      \sin(\frac 12 \pi - \alpha - \beta) \equiv
      \sin(\frac 12 \pi - \alpha)\cos \beta +
          \sin \beta \cos(\frac 12 \pi - \alpha) \\
  &\equiv
      \cos \alpha \cos \beta + \sin \beta \sin \alpha \qedhere
 \end{align*}
\end{proof}
\begin{proof}
 \begin{align*}
  \tan(\alpha + \beta) &\equiv
      \frac{\sin(\alpha + \beta)}{\cos(\alpha + \beta)} \equiv
      \frac{\sin \alpha \cos \beta + \sin \beta \cos \alpha}
           {\cos \alpha \cos \beta + \sin \alpha \sin \beta} \\
      &\equiv \frac{
            \frac{\sin \alpha \cos \beta}{\cos \alpha \cos \beta} +
            \frac{\sin \alpha \sin \beta}{\cos \alpha \cos \beta}}
           {\frac{\cos \alpha \cos \beta}{\cos \alpha \cos \beta} +
            \frac{\sin \alpha \sin \beta}{\cos \alpha \cos \beta}}
           \equiv
      \frac{\tan \alpha + \tan \beta}{1 - \tan \alpha \tan \beta} \qedhere
 \end{align*}
\end{proof}

\begin{theorem}[Compound angle formulae] \label{thm_trig_compound}
 Trigonometric functions of a sum or difference of angles can be expanded into
 trigonometric functions of each angle.
 \begin{align*}
  \sin \alpha \pm \beta &\equiv
     \sin \alpha \cos \beta \pm \sin \beta \cos \alpha \\
  \cos \alpha \pm \beta &\equiv
     \cos \alpha \cos \beta \mp \sin \alpha \sin \beta \\
  \tan \alpha \pm \beta &\equiv
      \frac{\tan \alpha \pm \tan \beta}{1 \mp \tan \alpha \tan \beta}
 \end{align*}
\end{theorem}
\begin{proof}
 Results for a function of a difference can be derived from the parity of
 each function (\ref{sec_trig_parity}), and rewriting \(\alpha - \beta\) as
 \(\alpha + (-\beta)\).
\end{proof}
One might also just assume it's probably OK to flip signs.

\subsubsection{Double Angle Formulae} \label{sec_trig_double_angle}
\begin{theorem}[Double angle formulae] \label{thm_trig_double_angle}
 Trigonometric functions of double angles can be expressed in terms of
 trigonometric functions of the angle.
 \begin{align*}
  \sin 2\theta &\equiv
     2\sin \theta \cos \theta \\
  \cos 2\theta &\equiv
     \cos^2 \theta - \sin^2 \theta \equiv
     2\cos^2 \theta - 1 \equiv 1 - 2\sin^2 \theta
     \ \text{(by \ref{sec_trig_pythag})} \\
  \tan 2\theta &\equiv
      \frac{2\tan \theta}{1 - \tan^2 \theta}
 \end{align*}
\end{theorem}
\begin{proof}
 These follow from \(2\theta \equiv \theta + \theta\) in combination with
 Theorem \ref{thm_trig_compound}.
\end{proof}
It follows, by rearrangement of identities in Theorem
\ref{thm_trig_double_angle}, that
\begin{theorem}[Squared half-angle identities]
 Squared trigonometric functions can be rewritten in terms of trigonometric
 functions of the double angle.
 \begin{alignat*} 3
  \sin^2 \theta &\equiv
      \frac{1 - \cos 2 \theta} 2 \\
  \cos^2 \theta &\equiv
      \frac{1 + \cos 2 \theta} 2
  \intertext{and}
  \tan^2 \theta &\equiv \frac{\sin^2 \theta}{\cos^2 \theta}
      \equiv \frac{1 - \cos 2 \theta}{1 + \cos 2 \theta}
      &&\equiv \frac{1 - \cos^2 2 \theta}{(1 + \cos 2 \theta)^2}
      \equiv \parens[\Big]{\frac{\sin 2 \theta}{1 + \cos 2 \theta}}^2 \\
  &   &&\equiv \frac{(1 - \cos 2 \theta)^2}{1 - \cos^2 2 \theta}
      \equiv \parens[\Big]{\frac{1 - \cos 2 \theta}{\sin 2 \theta}}^2
 \end{alignat*}
\end{theorem}
\begin{theorem}[Half angle identities] \label{thm_trig_half_angle}
 Trigonometric functions of half-angles can be expressed in terms of
 trigonometric functions of the full angle.
 \begin{alignat*}2
  &&\abs*{\sin \theta} &\equiv
      \sqrt{\frac{1 - \cos 2 \theta} 2} \\
  \iff{}&& \abs*{\sin \tfrac 12 \theta} &\equiv
      \sqrt{\frac{1 - \cos \theta} 2} \\
  &&\abs*{\cos \theta} &\equiv
      \sqrt{\frac{1 + \cos 2 \theta} 2} \\
  \iff{}&& \abs*{\cos \tfrac 12 \theta} &\equiv
      \sqrt{\frac{1 + \cos \theta} 2} \\
  &&\tan \theta &\equiv
      \frac{\sin 2 \theta}{1 + \cos 2 \theta}
      \equiv \frac{1 - \cos 2 \theta}{\sin 2 \theta} \\
  \iff{}&& \tan \tfrac 12 \theta &\equiv
      \frac{\sin \theta}{1 + \cos \theta}
      \equiv \frac{1 - \cos \theta}{\sin \theta}
 \end{alignat*}
 The modulus signs here are due to the sign changes between quadrants. They can
 often by eliminated by consideration of the domain of \(\theta\).
\end{theorem}

\subsubsection{Triple Angles and Beyond}

\subsection{Sum-Product Identities} \label{sec_trig_sum_product}

\begin{theorem}[Sum-Product Identities]
 The product of trigonometric functions can be rewritten as a sum:
 \begin{align*}
  \sin \alpha \cos \beta &\equiv
      \frac{\sin(\alpha + \beta) + \sin(\alpha - \beta)} 2 \\
  \cos \alpha \cos \beta &\equiv
   \frac{\cos(\alpha + \beta) + \cos(\alpha - \beta)} 2 \\
  \sin \alpha \sin \beta &\equiv
   \frac{\cos(\alpha - \beta) - \cos(\alpha + \beta)} 2
 \end{align*}
\end{theorem}
\begin{proof}
 In each case we can derive by applying compound angle formulae to the RHS:
 \begin{align*}
  \frac{\sin(\alpha + \beta) + \sin(\alpha - \beta)} 2 &\equiv
   \frac{\sin \alpha \cos \beta + \sin \beta \cos \alpha +
         \sin \alpha \cos \beta - \sin \beta \cos \alpha} 2 \\
  &\equiv \frac{2 \sin \alpha \cos \beta}
   2 \equiv \sin \alpha \cos \beta \\[3ex]
  \frac{\cos(\alpha + \beta) + \cos(\alpha - \beta)} 2 &\equiv
   \frac{\cos \alpha \cos \beta - \sin \alpha \sin \beta +
         \cos \alpha \cos \beta + \sin \alpha \sin \beta} 2 \\
  &\equiv \frac{2 \cos \alpha \cos \beta}
   2 \equiv \cos \alpha \cos \beta \\[3ex]
  \frac{\cos(\alpha - \beta) - \cos(\alpha + \beta)} 2 &\equiv
   \frac{\cos \alpha \cos \beta + \sin \alpha \sin \beta -
         \cos \alpha \cos \beta + \sin \alpha \sin \beta} 2 \\
  &\equiv \frac{2 \sin \alpha \sin \beta} 2 \equiv \sin \alpha \sin \beta
      \qedhere
 \end{align*}
\end{proof}

\subsubsection{Product-Sum Identities} \label{sec_trig_product_sum}

%FIXME: annotate sections used

\begin{theorem}[Sum-Product Identities]
 Similarly, the sums and differences of trigonometric functions can be rewritten
 as a product:
 \begin{align*}
  \sin \alpha + \sin \beta &\equiv
      2 \sin \frac{\alpha + \beta}2 \cos \frac{\alpha - \beta}2 \\
  \sin \alpha - \sin \beta &\equiv
      2 \sin \frac{\alpha - \beta} 2 \cos\frac{\alpha + \beta} 2 \\
  \cos \alpha + \cos \beta &\equiv
      2 \cos\frac{\alpha + \beta} 2 \cos \frac{\alpha - \beta} 2 \\
  \cos \alpha - \cos \beta &\equiv
   -2 \sin \frac{\alpha + \beta} 2 \sin \frac{\alpha - \beta} 2
 \end{align*}
 \end{theorem}
\begin{proof}
 We can derive the sums by considering \(\alpha' = \frac 12 (\alpha + \beta)\)
 and \(\beta' = \frac 12 (\alpha - \beta)\). Then,
 \begin{align*}
  \sin \alpha + \sin \beta &\equiv
   \sin(\alpha' + \beta') + \sin(\alpha' - \beta') \equiv
   2\sin \alpha' \cos \beta'\quad \text{(from \ref{sec_trig_sum_product})}
  \\&\equiv 2 \sin \frac{\alpha + \beta}2 \cos \frac{\alpha - \beta} 2 \\[3ex]
   \cos \alpha + \cos \beta &\equiv
   \cos(\alpha' + \beta') + \cos(\alpha' - \beta') \equiv
   2\cos \alpha' \cos \beta'\quad \text{(from \ref{sec_trig_sum_product})}
   \\&\equiv 2 \cos\frac{\alpha + \beta} 2 \cos \frac{\alpha - \beta} 2
 \end{align*}
 The difference of sines is quite straightforward to show using
 (\ref{sec_trig_parity}).
 \begin{equation*}
  \sin \alpha - \sin \beta =
      \sin \alpha + \sin -\beta =
      2 \sin \frac{\alpha - \beta} 2 \cos\frac{\alpha + \beta} 2
 \end{equation*}
 For the difference of cosines, can exploit the periodicity of cosine
 (\ref{sec_trig_periodic}).
 \begin{align*}
  \cos \alpha - \cos \beta &\equiv
  \cos \alpha + \cos (\beta + \pi)  \equiv
  2 \cos \frac{\alpha + \beta + \pi} 2 \cos \frac {\alpha - \beta - \pi} 2
  \\&\equiv
  2 \cos \parens[\Big]{\frac \pi 2 - \parens[\Big]{-\frac{\alpha + \beta} 2}}
    \cos \parens[\Big]{\frac \pi 2 - \frac{\alpha - \beta} 2} \equiv
  2 \sin -\frac{\alpha + \beta} 2 \sin \frac{\alpha - \beta} 2 \\&\equiv
  -2 \sin \frac{\alpha + \beta} 2 \sin \frac{\alpha - \beta} 2 \qedhere
 \end{align*}
\end{proof}

\subsection[The Weierstrass substitution
            \texorpdfstring{(\(\tan(\theta / 2)\))}{(tangent half-angle)}]
   {The Weierstrass substitution \boldmath\(\parens[\Big]{\tan(\tfrac 12 \theta)}\)}

\begin{theorem}[The Weierstrass substitution]
 If we let \(t = \tan \tfrac 12 \theta\), such that \(t\) is defined (ie
 \(\NExists n \in \Integers \colon \theta = (2n + 1)\pi\)) then we have:
 \begin{align*}
  \sin \theta &\equiv \frac{2t}{1 + t^2} \\
  \cos \theta &\equiv \frac{1 - t^2}{1 + t^2} \\
  \tan \theta &\equiv \frac{2t}{1 - t^2} \\
  \dv<\theta>{t} &\equiv \frac 2{1 + t^2}
 \end{align*}
\end{theorem}
\begin{proof}
 The identity for the tangent is an immediate consequence of the double
 angle formula for the tangent (\ref{sec_trig_double_angle}), where we're
 considering \(\tan{(2 \cdot \frac 12 \theta)}\).

 The identity for the sine can be derived by considering the double angle
 formula for sine, and using a smattering of identities from sections
 \ref{sec_trig_definitions}, \ref{sec_trig_pythag}:
 \begin{align*}
  \sin \theta &\equiv 2\sin \tfrac 12 \theta \cos\tfrac 12 \theta
      \equiv 2\tan \tfrac 12 \theta \cos^2 \tfrac 12 \theta \\
      &\equiv \frac{2\tan \tfrac 12 \theta}{\sec^2 \tfrac 12 \theta}
      \equiv \frac{2t}{1 + t^2}
 \end{align*}
 Similarly for the cosine:
 \begin{align*}
  \cos \theta
      &\equiv \cos^2 \tfrac 12 \theta - \sin^2 \tfrac 12 \theta
      \equiv \cos^2 \tfrac 12 \theta
          (1 - \tan^2 \tfrac 12 \theta) \\
      &\equiv \frac{1 - \tan^2 \tfrac 12 \theta}
                   {\sec^2 \tfrac 12 \theta}
      \equiv \frac{1 - t^2}{1 + t^2} \qedhere
 \end{align*}
\end{proof}

This fact is useful for a number of reasons. Firstly, when integrating a
rational function of trig functions, we can apply this trick to transform
the integral into a rational function of \(t\), as all trig functions and
also \(\diff \theta\) become rational functions of \(t\).

It can also be used as an alternative parametrisation of a circle: instead
of
\begin{equation*}
 (x, y) = (\cos\theta, \sin\theta) \colon \theta \in \intco{0, 2\pi}
\end{equation*}
we can use
\begin{equation*}
 (x, y) = \parens[\Big]{\frac{1 - t^2}{1 + t^2},
                        \frac{2t}{1 + t^2}} \colon t \in \Reals
\end{equation*}
This is useful as it is a rational function, so for rational \(t\) it gives
a rational point on the unit circle, which can be used, for example, to find
Pythagorean triples. The only point it misses is \((0, 1)\).

\subsubsection{Heuristic for sine and cosine}

%FIXME add graphic.

The formula involving the tangent can be easily derived from the double
angle formula. Having this, we can draw a right triangle with angle
\(\theta\), opposite side \(2t\) and adjacent side \(1 - t^2\). We can
deduce from Pythagoras' theorem (\ref{sec_pythagoras}) that the length of
the hypotenuse is
\begin{equation*}
 \sqrt{(2t)^2 + (1 - t^2)^2} = \sqrt{1 + 2t^2 + t^4} = 1 + t^2
\end{equation*}
and then deduce the magnitudes of sine and cosine from their definitions.
Fortuitously, it turns out we can immediately drop any absolute value
signs, making this trick quite useful.
