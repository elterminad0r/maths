\section{Curiosities}

\subsection[\texorpdfstring{\(e^\pi\) vs \(\pi^e\)}{e\^{}pi vs pi\^{}e}]
           {\boldmath\(e^\pi\) vs \(\pi^e\)}

This is a classic problem. Is \(e^\pi\) or \(\pi^e\) greater? A nice way to
tackle it is by noticing that you can rewrite it as an equivalent inequality, by
raising to the power \(1 / (\pi e)\), as both sides are positive.
\begin{alignat*}2
 && e^\pi &> \pi^e \\
 \iff{}&& e^{\frac 1e} &> \pi^{\frac 1\pi}
\end{alignat*}
This is easier to tackle, as by separating the two constants to either side like
this, we have basically turned the question into one about the shape of the
graph \(y = x^{\frac 1x}\). Note that
\begin{equation*}
 \dv<y>{x} = \parens[\Big]{\frac 1{x^2} - \frac 1{x^2}\ln x} x^{\frac 1x}
           = \frac 1{x^2} (1 - \ln x) x^{\frac 1x}
\end{equation*}
so in fact, for \(x < e\), \(\dv<y>{x} > 0\), and for
\(x > e\), \(\dv<y>{x} < 0\), so the graph has a local maximum at \(x = e\) and
is decreasing from there on out. We conclude that
% TODO draw the actual graph
\begin{equation*}
 e^{\frac 1e} > \pi^{\frac 1\pi}
\end{equation*}
and therefore,
\begin{equation*}
 e^\pi > \pi^e
\end{equation*}
