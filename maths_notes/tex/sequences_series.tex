\section{Sums, Sequences and Series Galore}

\subsection{Binomial Theorem}

\begin{theorem}[Binomial theorem] Where \(n \in \Naturals\),
\begin{equation*}
(a + b)^n \equiv \sum_{r = 0}^n \binom nr a^r b^{n - r}
\end{equation*}
where the binomial coefficients are given by
\begin{equation*}
\binom nr = \nCr nr \defeq \frac{n!}{r!\cdot(n - r)!}
\end{equation*}
\end{theorem}
\begin{proof}
We can prove that this is the case by induction. If \(P(n)\) denotes the
binomial theorem for exponent \(n, \forall n \in \Integers^+\), then we
consider \(P(1)\):
\begin{equation*}
(a + b)^1 \equiv a + b
    \equiv \frac{0!}{0!\cdot 1!} a^0b^1 + \frac{1!}{1!\cdot 0!} a^1b^0
\end{equation*}
so \(P(1)\) is true. We now consider \(P(n + 1)\), supposing that \(P(n)\)
is true.
\begin{align*}
(a + b)^{n + 1} &\equiv (a + b)(a + b)^n \\
    &\equiv (a + b)\sum_{r = 0}^n \binom nr a^r b^{n - r}
        \impliedby P(n) \\
    &\equiv a\sum_{r = 0}^n \binom nr a^r b^{n - r}
          + b\sum_{r = 0}^n \binom nr a^r b^{n - r} \\
    &\equiv \sum_{r = 0}^n \binom nr a^{r + 1} b^{n - r}
          + \sum_{r = 0}^n \binom nr a^r b^{n - r + 1} \\
    &\equiv \sum_{r = 1}^{n + 1} \binom n{r - 1} a^{r} b^{n - r + 1}
          + \sum_{r = 0}^n \binom nr a^r b^{n - r + 1} \\
    &\equiv a^{n + 1} + b^{n + 1}
          + \sum_{r = 1}^n \bqty{\binom n{r - 1} + \binom nr}
            a^r b^{n - r + 1} \\
    &\equiv a^{n + 1} + b^{n + 1}
          + \sum_{r = 1}^n \binom{n + 1}r a^r b^{n - r + 1}
            \qquad \text{due to Lemma \ref{lem_bin_coef_sum}} \\
    &\equiv \sum_{r = 0}^{n + 1} \binom{n + 1}r a^r b^{(n + 1) - r} \\
\end{align*}
Therefore, \(P(n + 1)\) is true if \(P(n)\) is true, and therefore \(P(n)\) is
true for \(\forall n \in \Integers^+\) by the principle of mathematical
induction.
\end{proof}

\begin{lemma}[Sum of adjacent binomial coefficients] \label{lem_bin_coef_sum}
\begin{equation*}
\binom n{r - 1} + \binom nr \equiv \binom{n + 1} r
\end{equation*}
\end{lemma}
\begin{proof}
\begin{align*}
\binom n{r - 1} + \binom nr &\equiv \frac{n!}{(r - 1)! \cdot (n - r + 1)!}
                                 + \frac{n!}{r! \cdot (n - r)!}
                                    \qquad \text{by definition} \\
    &\equiv \frac{n!\cdot(r + (n - r + 1))}{r! \cdot (n - r + 1)!} \\
    &\equiv \frac{n! \cdot (n + 1)}{r! \cdot ((n + 1) - r)!} \\
    &\equiv \frac{(n + 1)!}{r! \cdot ((n + 1) - r)!} \\
    &\equiv \binom{n + 1} r \qquad \text{by definition} \tag*{\qedhere}
\end{align*}
\end{proof}

\subsection{Geometric Progressions} \label{sec_seq_GP}

A geometric progression is a sequence \(U_i\) where each progressive term is
given by the previous term multiplied by some constant \emph{common ratio},
denoted \(r\). The first term is denoted \(a\), such that
\begin{gather*}
U_1 = a, \qquad U_2 = ar, \qquad U_3 = ar^2, \qquad \dots\\
U_n = a\underbrace{r \cdot r \cdot \dotsb \cdot r}_{\text{\(n\) times}}
    = ar^{n - 1}
\end{gather*}

\subsection{Fibonacci Sequence}

\begin{theorem}[Fibonacci \(n\)th term]
The Fibonacci numbers \(F_n: n \in \Naturals\) are such that
\(F_1 = F_2 = 1\) and \(F_n = F_{n - 1} + F_{n - 2}\).  \(F_n\) can be given
by the closed form
\begin{equation*}
F_n = \frac{\varphi^n - \psi^n}{\sqrt 5}
\ \text{where}\ \varphi = \frac{1 + \sqrt 5} 2
\ \text{and}\ \psi = \frac{1 - \sqrt 5} 2
\end{equation*}
\end{theorem}
\begin{proof}
Consider the power series
\begin{equation*}
f(x) \defeq \sum_{k=1}^\infty F_k x^k = x + x^2 + 2x^3 + 3x^4 + 5x^5
    + \dotsb
\end{equation*}
Note that this series has a radius of convergence, as it is strictly less
than the series
\begin{equation*}
g(x) \defeq \sum_{k=1}^\infty 2^k x^k
\end{equation*}
This can be proven by induction. First we note that \(F_1 < 2^1\) and
\(F_2 < 2^2\), and that \\
\(F_{n + 2} = F_{n + 1} + F_n < 2^{n + 1} + 2^n < 2^{n + 1} + 2^{n + 1}
    = 2^{n + 2}\). So \(F_n < 2^n\) for all \(n\). Therefore, for
\(\abs{x} < \frac 12\), \(f(x)\)  must be convergent, as \(g(x)\) is
convergent for \(\abs{x} < \frac 12\). This means \(f(x)\) is well defined
and we can manipulate it.

Consider now:
\begin{alignat*}{9}
&& f(x) &= \sum_{k = 1}^\infty F_k x^k
    &&=&\; x\; &+&\; x^2\; &+&\; 2x^3\; &+&\;
        3x^4\; &+&\; 5x^5\; &+&\; 8x^6\; + \dotsb \\
&& x f(x) &= \sum_{k = 2}^\infty F_{k - 1} x^k
    &&=&\; &&x^2\; &+&\; x^3\; &+&\; 2x^4\; &+&\;
        3x^5\; &+&\; 5x^6\; + \dotsb \\
&& x^2 f(x) &= \sum_{k = 3}^\infty F_{k - 2}x^k
    &&=&\; &&&& x^3\; &+&\; x^4\; &+&\; 2x^5\; &+&\; 3x^6\; + \dotsb \\
&\implies& (1 - x - x^2) f(x) &= x \\
&\implies& f(x) &= \frac x{1 - x - x^2}
\end{alignat*}
Now we perform a partial fraction decomposition on \(f(x)\). We do this in a
somewhat tricky way in order to make our lives easier. Note that
\begin{equation*}
1 - x - x^2 = x^2 \pqty{\frac 1x^2 - \frac 1x - 1}
    = x^2\pqty{\frac 1x - \varphi}\pqty{\frac 1x - \psi}
    = (1 - \varphi x)(1 - \psi x)
\end{equation*}
where \(\varphi, \psi = \dfrac{1 \pm \sqrt 5}2\), ie the roots of
\(x^2 - x - 1\), ie the roots of the golden ratio. Now,
\begin{alignat*}{2}
&& f(x) = \frac x{(1 - \varphi x)(1 - \psi x)}
    &\equiv \frac A{1 - \varphi x} + \frac B{1 - \psi x} \\
&\implies& A(1 - \psi x) + B(1 - \varphi x) &\equiv x \\
&\implies& \left\{
    \begin{aligned}
        && A + B &= 0 \\
        &\implies& B &= -A  \\
        && -\psi A - \varphi B &= 1
    \end{aligned} \right. \\
&\implies& A(\psi - \varphi) &= 1 \\
&\implies& A(\sqrt 5) &= 1 \\
&\implies& A &= \frac 1{\sqrt 5}, B = -\frac 1{\sqrt 5}
\end{alignat*}
Now we can use a binomial series expansion to obtain an equivalent series.
\begin{align*}
f(x) &= \frac 1{\sqrt 5}\pqty{\frac 1{1 - \varphi x}
                            - \frac 1{1 - \psi x}} \\
&= \frac 1{\sqrt 5}\pqty{\sum_{k = 0}^\infty (\varphi x)^k
                       - \sum_{k = 0}^\infty (\psi x)^k} \\
&= \sum_{k = 0}^\infty \frac 1{\sqrt 5}(\varphi^k - \psi^k)x^k
= \sum_{k = 1}^\infty F_k x^k
\end{align*}
So we have \(F_n = \dfrac 1{\sqrt 5}(\varphi^n - \psi^n)\). Note that the
constant term is indeed zero, as \(\varphi^0 - \psi^0 = 0\).
\end{proof}
Combined with Exponentation by Squaring (\ref{sec_exp_by_squaring}), and
some simple surd arithmetic, this should be a particularly fast way to
calculate \(F_n\), as compared to utterly na\"ive recursion, or somewhat
faster iteration or memoized recursion.

\subsection{Taylor Series}

The Maclaurin series is the Taylor series around \(0\).
\begin{equation}
f(x) = f(0) + f'(0) x + \frac{f''(0)} 2 x^2 + \frac{f'''(0)}{6}x^3 +\dotsb
  = \sum_{k=0}^\infty \frac{f^{(k)}(0)}{k!}x^k
\end{equation}

\subsection{Common series}
