\section{Linear Algebra}

\subsection{Matrices}

An \(m \times n\) matrix is a rectangular array of numbers, with \(m\) rows and
\(n\) columns:
\begin{equation*}
 \mat M \in \fld K^{m \times n} \iff \mat M =
 \begin{pmatrix}
  a_{11} & a_{12} & a_{13} & \cdots & a_{1n} \\
  a_{21} & a_{22} & a_{23} & \cdots & a_{2n} \\
  a_{31} & a_{32} & a_{33} & \cdots & a_{3n} \\
  \vdots & \vdots & \vdots & \ddots & \vdots \\
  a_{m1} & a_{m2} & a_{m3} & \cdots & a_{mn}
 \end{pmatrix}
 \quad \text{where \(a_{ij} \in \fld K\)}
\end{equation*}
Perhaps it bears repeating that this means the \emph{``height''} of an \(m
\times n\) matrix is \(m\), while the \emph{``width''} is \(n\).
\begin{equation*}
 \rotatebox[origin=c]{90}{
  \scriptsize{\(m\) rows}
  }
 \left\{\vphantom{\begin{matrix}
  \vphantom{} \\ \vphantom{} \\ \vphantom{} \\
  \vphantom{} \\ \vphantom{} \\
 \end{matrix}}\right.
 \overbrace{
  \begin{pmatrix}
   a_{11} & a_{12} & a_{13} & \cdots & a_{1n} \\
   a_{21} & a_{22} & a_{23} & \cdots & a_{2n} \\
   a_{31} & a_{32} & a_{33} & \cdots & a_{3n} \\
   \vdots & \vdots & \vdots & \ddots & \vdots \\
   a_{m1} & a_{m2} & a_{m3} & \cdots & a_{mn}
  \end{pmatrix}
 }^{\text{\(n\) columns}}
\end{equation*}
This is nice for instance because when you multiply an \(m \times n\) matrix by
an \(n \times p\) matrix you get an \(m \times p\) matrix.
