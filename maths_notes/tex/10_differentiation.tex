\section{Differentiation}

%FIXME: derivations from first principles

\subsection{Definition of the derivative}

The derivative of \(f(x)\) gives the gradient of the curve \(y = f(x)\) at
\(x\). It is defined as
\begin{equation}
 f'(x) = \lim_{h \to 0} \frac{f(x + h) - f(x)}h
\end{equation}
Other notations include, if \(y = f(x)\), then
\begin{equation}
 f'(x) = \dv{x}(f(x)) = \dv<y>{x} = y'
\end{equation}

There is also various notation for repeated differentiation. If
\(g(x) = f'(x)\), then \(g'(x) = f''(x)\).
\begin{equation*}
 f''(x) = \dv[2]<y>{x} = \dv[2]{x}(f(x))
\end{equation*}
In general, differentiating \(y = f(x)\) \(n\) times, we have the \(n\)th
derivative
\begin{equation*}
 f^{\overbrace{\prime\prime\prime\dotsb}^{\text{\(n\) times}}}
     = f^{(n)}(x) = \dv[n]<y>{x} = \dv[n]{x}(f(x))
\end{equation*}

\subsection{Basic properties of the derivative}
\label{sec_calc_derivative_properties}

Some basic properties of limits result in corresponding basic properties of
the derivative.
\begin{equation*}
 \dv{x}(a f(x)) = af'(x) \quad \text{where \(a\) is a constant}
\end{equation*}
\begin{equation*}
 \dv{x}(f(x) + g(x)) = f'(x) + g'(x)
\end{equation*}

\subsection{Chain Rule} \label{sec_calc_chain}

\subsection{Product Rule} \label{sec_calc_product}

\begin{theorem}[Product rule]
 For two differentiable functions \(f, g\),
 \begin{equation*}
  \dv{x} (f(x)g(x)) = f(x)g'(x) + f'(x)g(x)
 \end{equation*}
\end{theorem}
\begin{proof}[Proof from first principles]
 This can be derived from the definition of the derivative. Let
 \begin{equation*}
  h(x) = f(x)g(x)
 \end{equation*}
 Then, by definition,
 \begin{align*}
  h'(x) &= \lim_{t \to 0} \frac{h(x + t) - h(x)} t \\
        &= \lim_{t \to 0} \frac{f(x + t)g(x + t) - f(x)g(x)} t \\
        &= \lim_{t \to 0} \frac{f(x + t)g(x + t) - f(x + t)g(x)
                              + f(x + t)g(x) - f(x)g(x)} t \\
        &= \lim_{t \to 0} \bracks[\bigg]{
            f(x + t) \frac{g(x + t) - g(x)} t
          + g(x) \frac{f(x + t) - f(x)} t} \\
        &= \lim_{t \to 0} \bracks[\bigg]{
            f(x + t)g'(x)
          + g(x) f'(x)} \quad \text{by definition} \\
        &= f(x)g'(x) + g(x)f'(x)
  \qedhere
 \end{align*}
\end{proof}
\begin{proof}[Proof from the chain rule]
 We can alternatively apply the chain rule after completing the square of
 \(f(x)g(x)\) by writing
 \begin{alignat*}{2}
  && f(x)g(x) &= \frac 12 \bracks[\Big]
                  {(f(x))^2 + (g(x))^2 - (f(x) - g(x))^2} \\
  &\implies{}& \dv{x} (f(x)g(x))
              &= \frac 12 \dv{x} \bracks[\Big]
                  {(f(x))^2 + (g(x))^2 - (f(x) - g(x))^2} \\
  &&          &= \frac 12 \bracks[\Big]
                  {2f(x)f'(x) + 2g(x)g'(x)
                 - 2(f(x) - g(x))(f'(x) - g'(x))} \\
  &&          &= f(x)f'(x) + g(x)g'(x)
               - (f(x)f'(x) + g(x)g'(x) - f'(x)g(x) - f(x)g'(x)) \\
  &&          &= f'(x)g(x) + f(x)g'(x) \qedhere
 \end{alignat*}
\end{proof}
\begin{proof}[Proof by exponentials]
 Yet another way to show this is to rewrite \(f(x)g(x)\) in terms of
 exponentials, eliminating the product so we can again apply the chain rule.
 \begin{alignat*}{2}
  && f(x)g(x) &= e^{\ln f(x) + \ln g(x)} \\
  &\implies{}& \dv{x} (f(x)g(x))
              &= \dv{x} e^{\ln f(x) + \ln g(x)} \\
  &&          &= \parens[\bigg]{\frac{f'(x)}{f(x)} + \frac{g'(x)}{g(x)}}
                   e^{\ln f(x) + \ln g(x)} \\
  &&          &= \parens[\bigg]{\frac{f'(x)}{f(x)} + \frac{g'(x)}{g(x)}}
                   f(x)g(x) \\
  &&          &= f'(x)g(x) + f(x)g'(x)
 \end{alignat*}
 Now a little more care must be taken, as this does assume that \(\ln f(x)\)
 and \(\ln g(x)\) exist. This will be true if both \(f\) and \(g\) are
 positive at \(x\). If either or both is negative, we can factor out a
 \(-1\), and instead continue with the proof using \(-f(x)\) or \(-g(x)\).
 This minus will then eliminate. Suppose wlog that \(f(x) < 0\):
 \begin{alignat*}{2}
  && f(x)g(x) &= -e^{\ln(-f(x)) + \ln g(x)} \\
  &\implies{}& \dv{x} (f(x)g(x))
              &= -\dv{x} e^{\ln(-f(x)) + \ln g(x)} \\
  &&          &= -\parens[\bigg]{\frac{-f'(x)}{-f(x)}
                               + \frac{g'(x)}{g(x)}}
                   e^{\ln(-f(x)) + \ln g(x)} \\
  &&          &= -\parens[\bigg]{\frac{f'(x)}{f(x)} + \frac{g'(x)}{g(x)}}
                   \parens[\big]{-f(x)g(x)} \\
  &&          &= \parens[\bigg]{\frac{f'(x)}{f(x)} + \frac{g'(x)}{g(x)}}
                   (f(x)g(x)) \\
  &&          &= f'(x)g(x) + f(x)g'(x) \qedhere
 \end{alignat*}
\end{proof}

\subsubsection{Quotient Rule} \label{sec_calc_quotient}

\begin{theorem}[Quotient rule]
 For two differentiable functions \(f, g\),
 \begin{equation*}
  \dv{x}\parens[\bigg]{\frac{f(x)}{g(x)}} = \frac{g(x)f'(x) - f(x)g'(x)}{(g(x))^2}
 \end{equation*}
\end{theorem}
There are a few ways to prove this. We could reappropriate some of the arguments
made in the proof of the product rule (\ref{sec_calc_product}), but it is much
more straightforward to use the product rule as a stepping stone.
\begin{proof}[Proof by explicit differentiation]
 Let
 \begin{alignat*}{2}
  && h(x) &= \frac{f(x)}{g(x)} \\
  &&      &= f(x) \cdot \frac 1{g(x)} \\
  &\implies{}& h'(x)
          &= f'(x) \cdot \frac 1{g(x)}
              + f(x) \cdot \parens[\Bigg]{\frac{-g'(x)}{(g(x))^2}}
              \quad \text{from the product and chain rules} \\
  &&      &= \frac{f'(x)}{g(x)} - \frac{f(x)g'(x)}{(g(x))^2} \\
  &&      &= \frac{f'(x)g(x) - f(x)g'(x)}{(g(x))^2} \qedhere
 \end{alignat*}
\end{proof}
\begin{proof}[Proof by implicit differentiation]
 Let
 \begin{alignat*}{2}
  && h(x) &= \frac{f(x)}{g(x)} \\
  &\implies{}& h(x)g(x) &= f(x) \\
  &\implies{}& h(x)g'(x) + h'(x)g(x)
          &= f'(x) \quad \text{from the product and chain rules} \\
  &\implies{}& h'(x)
          &= \frac{f'(x) - h(x)g'(x)}{g(x)} \\
  &&      &= \frac{f'(x) - \frac{f(x)}{g(x)}g'(x)}{g(x)}
              \quad \text{by definition of \(h\)} \\
  &&      &= \frac{f'(x)g(x) - f(x)g'(x)}{(g(x))^2} \qedhere
 \end{alignat*}
\end{proof}

\subsection{Common derivatives} \label{calc_common}

Differentiation is, informally speaking, an easy problem. Any elementary
function, which is to say any of the functions in table
\ref{tab_calc_derivatives}, or any composition of elementary functions, is
differentiable to an elementary function, and basically the algorithm to do
so just consists of identifying the form of a function as some composition
of other subexpressions, finding the derivative in terms of these
subexpressions and their derivatives, and continuing to recurse down
subexpressions until they are exhausted.

Some of the more common rules for taking the derivative of a composition of
unspecified functions are shown in table
\ref{tab_calc_functions_derivatives}. The most essential ones are marked
with \(\note\). All the other ones can be derived using only these
rules, and the derivatives of some functions in table
\ref{tab_calc_derivatives}.

\begin{longtable}{*4M c}
 \toprule
 \text{\boldmath\(f(x)\)} & \text{\boldmath\(f'(x)\)}
     & \text{\bfseries Alternatives/Notes} & \note & \bfseries ref \\
 \midrule
 \endhead
 g(x) + h(x) & g'(x) + h'(x) & & \note
  & \ref{sec_calc_derivative_properties} \\[1ex]
 a g(x) & a g'(x) & & \note
  & \ref{sec_calc_derivative_properties} \\[1ex]
 g(x) h(x) & g'(x) h(x) + g(x) h'(x)
  & & \note & \ref{sec_calc_product} \\[1ex]
 g(h(x)) & h'(x) g'(h(x)) & & \note & \ref{sec_calc_chain} \\[1ex]
 \frac{g(x)}{h(x)} & \frac{h(x) g'(x) - g(x) h'(x)}{(h(x))^2}
  &&& \ref{sec_calc_quotient} \\[3ex]
 g(ax + b) & a g'(ax + b) \\[3ex]
 g(x)^{h(x)}
  & \parens[\bigg]{\frac{h(x)g'(x)}{g(x)} + h'(x)\ln(g(x))}g(x)^{h(x)}
  & g(x) > 0 && \ref{sec_calc_powers} \\[3ex]
 \sqrt{g(x)} & \frac{g'(x)}{2\sqrt{g(x)}} & g(x) \ge 0 \\[3ex]
 \ln(g(x)) & \frac{g'(x)}{g(x)} & g(x) > 0 \\[3ex]
 \log_a(g(x)) & \frac{g'(x)}{g(x) \ln a} & g(x) > 0 \\[3ex]
 \log_{g(x)} a & \frac{-\ln a}{g(x) (\ln(g(x)))^2}
  & \text{const \(a > 0\), \(g(x) > 0\)} \\[3ex]
  \log_{g(x)} h(x) & \frac 1{(\ln(g(x)))^2}\parens[\bigg]{\frac{\ln(g(x))}{h(x)}
                                           - \frac{\ln(h(x))}{g(x)}}
  & g(x), h(x) > 0 \\[3ex]
 \bottomrule
 \caption{General derivatives of compositions of functions}
 \label{tab_calc_functions_derivatives}
\end{longtable}

In table \ref{tab_calc_derivatives} is a set of common derivatives of
functions. Note that by FTC (\ref{sec_calc_FTC}), these also give a number
of common integrations, although these are reiterated in table
\ref{tab_calc_int_funcs}.

A large number of these can be readily derived from preceding entries. The
most essential ones are marked with a \(\note\). The further down the table
you get, the more obscure the functions get. For most, it can save time to
know the derivatives, but people obviously have their priorities and often
it's more worthwile to understand how to derive the derivative, anyway.

\begin{longtable}{*4M c}
 \toprule
 \text{\boldmath\(f(x)\)} & \text{\boldmath\(f'(x)\)}
     & \text{\bfseries Alternatives/Notes} & \note & \bfseries ref \\
 \midrule
 \endhead
 a & 0 & \text{\(a\) is constant} & \note \\[1ex]
 x^n & nx^{n-1} & \text{\(n\) is constant} & \note
  & \ref{sec_calc_powers} \\[1ex]
 e^x & e^x & \text{\(e\) is Euler's number} & \note & \ref{sec_e} \\[1ex]
 a^x & \ln a \cdot a^x & \text{\(a\) is constant}
  && \ref{sec_calc_powers}\\[1ex]
 \ln x & \frac 1{x} && \note \\[3ex]
 \log_a x & \frac 1{x\ln a} & \text{\(a\) is a constant} \\[3ex]
 \sin x & \cos x && \note \\[1ex]
 \cos x & -\sin x && \note & \ref{sec_calc_trig_basic} \\[1ex]
 \tan x & \sec^2 x & \tan^2 x + 1 && \ref{sec_calc_trig_basic} \\[1ex]
 \sec x & \sec x \tan x &&& \ref{sec_calc_trig_basic} \\[1ex]
 \csc x & -\csc x \cot x &&& \ref{sec_calc_trig_basic} \\[1ex]
 \cot x & -\csc^2 x & -(\cot^2 x + 1) && \ref{sec_calc_trig_basic} \\[1ex]
 \arcsin x & \frac 1{\sqrt{1 - x^2}} && \note & \ref{sec_calc_trig_inv} \\[3ex]
 \arccos x & \frac{-1}{\sqrt{1 - x^2}} && \note & \ref{sec_calc_trig_inv} \\[3ex]
 \arctan x & \frac 1{x^2 + 1} && \note & \ref{sec_calc_trig_inv} \\[3ex]
 \arcsec x & \frac 1{\abs x \sqrt{x^2 - 1}}
  & \frac 1{x^2 \sqrt{1 - x^{-2}}} && \ref{sec_calc_trig_inv} \\[3ex]
 \arccsc x & \frac{-1}{\abs x \sqrt{x^2 - 1}}
  & \frac{-1}{x^2 \sqrt{1 - x^{-2}}} && \ref{sec_calc_trig_inv} \\[3ex]
 \arccot x & -\frac 1{x^2 + 1} &&& \ref{sec_calc_trig_inv} \\[3ex]
 \sinh x & \cosh x \\[1ex]
 \cosh x & \sinh x \\[1ex]
 \tanh x & \sech^2 x & -\tanh^2 x + 1 \\[1ex]
 \sech x & -\tanh x \sech x \\[1ex]
 \csch x & -\coth x \csch x \\[1ex]
 \coth x & -\csch^2 x & -(\coth^2 x + 1) \\[1ex]
 \arcsinh x & \frac 1{\sqrt{x^2 + 1}} \\[3ex]
 \arccosh x & \frac 1{\sqrt{x^2 - 1}} \\[3ex]
 \arctanh x & \frac 1{1 - x^2} \\[3ex]
 \arcsech x & \frac{-1}{x\sqrt{1 - x^2}} \\[3ex]
 \arccsch x & \frac{-1}{\abs x\sqrt{1 + x^2}} \\[3ex]
 \arccoth x & \frac 1{1 - x^2} \\[3ex]
 \bottomrule
 \caption{Common derivatives} \label{tab_calc_derivatives}
\end{longtable}

\subsubsection{Derivatives with powers} \label{sec_calc_powers}

Often when being introduced to differentiation, derivatives like that of
\(x^2\) are demonstrated from first principles, as follows:
\begin{align*}
 \dv{x}(x^2) &= \lim_{h \to 0} \frac{(x + h)^2 - x^2} h \\
             &= \lim_{h \to 0} \frac{2hx + h^2} h \\
             &= \lim_{h \to 0} (2x + h) \\
             &= 2x
\end{align*}
We can do this for a general positive integer power of \(x\),
\(n \in \Integers^+\) by using the binomial theorem:
\begin{align*}
 \dv{x}(x^n) &= \lim_{h \to 0} \frac{(x + h)^n - x^n} h \\
             &= \lim_{h \to 0} \bracks[\bigg]{\frac 1h \parens[\Big]{-x^n
                     + \sum_{r = 0}^n \binom nr x^r h^{n - r}}} \\
             &= \lim_{h \to 0} \bracks[\bigg]{\frac 1h \parens[\Big]{
                       -x^n + \binom n0 x^n + \binom n1 hx^{n - 1}
                       + \sum_{r = 0}^{n - 2}
               \binom nr x^r h^{n - r}}} \\
             &= \lim_{h \to 0} \bracks[\bigg]{\frac 1h \parens[\Big]{-x^n + x^n
                         + hnx^{n - 1} + h^2 \sum_{r = 0}^{n - 2}
               \binom nr x^r h^{n - (r + 2)}}} \\
             &= \lim_{h \to 0} \bracks[\bigg]{nx^{n - 1} + h \sum_{r = 0}^{n - 2}
               \binom nr x^r h^{n - (r + 2)}} \\
             &= nx^{n-1}\quad
               \text{as the smallest power of \(h\) inside the sum is 0}
\end{align*}
Now if we have some negative integer exponent \(-n\) where
\(n \in \Integers^+\), we can use implicit differentiation to show the rule
still holds. Letting \(y = x^{-n}\),
\begin{alignat*}{2}
 &\implies{}& yx^n &= 1 \\
 &\implies{}& nx^{n - 1}y + x^n \dv<y>{x} &= 0 \\
 &\implies{}& \dv<y>{x} &= -nx^{n - 1}yx^{-n}
     = -nx^{-1}x^{-n} = -nx^{-n - 1}
\end{alignat*}
We can then use a little implicit differentiation to carry this onto any
rational exponent \(p / q \in \Rationals\). We let \(y = x^{p / q}\)
\begin{alignat*}{2}
 &\implies{}& y^q &= x^p \\
 &\implies{}& qy^{q - 1}\dv<y>{x} &= px^{p - 1} \\
 &\implies{}& \dv<y>{x} &= \frac pq x^{p - 1}y^{1 - q}
     = \frac pq x^{p - 1}x^{(p / q)(1 - q)} \\
 &&  &= \frac pq x^{p - 1 + (p / q) - p}
     = \frac pq x^{(p / q) - 1}
\end{alignat*}

We can show that this power rule generalizes to any real exponent \(k \in
\Reals\), for \(x > 0\), using the chain rule and the derivative of \(e^x\).
\begin{equation*}
 \dv{x}(x^k) = \dv{x}(e^{k\ln x}) = \frac kx e^{k\ln x}
             = \frac kx x^k = k x^{k - 1}
\end{equation*}
Unfortunately there's no straightforward way to even define exponentiation as a
differentiable function with a negative base for a general real exponent.

To differentiate an exponential function of a positive constant
\(a \in \Reals^+\), we can rewrite it as a power of \(e\), and use the chain
rule.
\begin{equation*}
 \dv{x}(a^x) = \dv{x}(e^{(\ln a)x}) = \ln a \cdot e^{(\ln a)x}
             = \ln a \cdot a^x
\end{equation*}
Lastly, if we have a positive function to the power of another function,
\(g(x)^{h(x)}\) where \(g(x) > 0\), we do a similar rewrite in terms of
\(e\).
\begin{align*}
 \dv{x}\parens[\Big]{g(x)^{h(x)}}
                       &= \dv{x}\parens[\Big]{e^{(\ln(g(x)))h(x)}} \\
                       &= \parens[\Big]{\frac{h(x)g'(x)}{g(x)}
                              + h'(x)\ln(g(x))}e^{(\ln(g(x)))h(x)} \\
                       &= \parens[\Big]{\frac{h(x)g'(x)}{g(x)}
                              + h'(x)\ln(g(x))}g(x)^{h(x)}
\end{align*}

\subsubsection{Basic trigonometric functions} \label{sec_calc_trig_basic}

%FIXME decide how to get around this

For now, we assume that \(\dv{x}(\sin x) = \cos x\). This can
be shown in a number of ways. There are geometric arguments, but we could
also appeal to the definition of sine and cosine in terms of complex
exponentials.

We can deduce from the chain rule that
\begin{equation*}
 \dv{x}(\cos x) = \dv{x}(\sin(\tfrac 12 \pi - x))
                = -\cos(\tfrac 12 \pi - x) = -\sin(x)
\end{equation*}
Now we can use the quotient rule for the tangent.
\begin{equation*}
 \dv{x}(\tan x) = \dv{x}\parens[\Big]{\frac{\sin x}{\cos x}}
                = \frac{\cos x \cos x + \sin x \sin x}{\cos^2 x}
                = \frac 1{\cos^2 x} = \sec^2 x
\end{equation*}
We can use the chain rule for the reciprocal functions.
\begin{equation*}
 \dv{x}(\sec x) = \dv{x}\parens[\Big]{\frac 1{\cos x}}
                = -\frac{-\sin x}{\cos^2 x} = \tan x \sec x
\end{equation*}
\begin{equation*}
 \dv{x}(\csc x) = \dv{x}\parens[\Big]{\frac 1{\sin x}}
                = -\frac{\cos x}{\sin^2 x} = -\cot x \csc x
\end{equation*}
The quotient rule can again be applied for the cotangent.
\begin{equation*}
 \dv{x}(\cot x) = \dv{x}\parens[\Big]{\frac{\cos x}{\sin x}}
                = \frac{-\sin x \sin x - \cos x \cos x}{\sin^2 x}
                = -\frac 1{\sin^2 x} = -\csc^2 x
\end{equation*}

\subsubsection{Inverse trigonometric functions} \label{sec_calc_trig_inv}

We can use implicit differentiation and some identities to tackle
derivatives of inverse trig functions. If we let \(y = \arcsin x\), then
\begin{equation*}
 \sin y = x \implies \dv<y>{x} \cos y = 1
     \implies \dv{x}(\arcsin x) = \frac 1{\cos y} = \frac 1{\sqrt{1 - x^2}}
\end{equation*}
This is allowed because the domain of \(\arcsin x\) is restricted to
\(\intcc{-1, 1}\), meaning that the range of \(y\) is
\(\intcc{-\frac 12 \pi, \frac 12 \pi}\) and \(\cos y > 0\)
(so \(\cos y = \abs{\cos y} = \sqrt{1 - \sin^2 y}\)).

Now, for the inverse cosine, rather than repeating this whole song and
dance, we can use the fact that
\(\arccos x \equiv \frac 12 \pi - \arcsin x\).
\begin{equation*}
 \dv{x}(\arccos x) = \dv{x}(\frac \pi 2 - \arcsin x)
     = -\frac 1{\sqrt{1 - x^2}}
\end{equation*}
For the inverse tangent, we again let \(y = \arctan x\), so that
\begin{equation*}
 \tan y = x \implies \dv<y>{x}(\tan^2 y + 1) = 1
     \implies \dv{x}(\arctan x) = \frac 1{1 + \tan^2 y}
         = \frac 1{1 + x^2}
\end{equation*}
For the inverse secant, we let \(y = \arcsec x\), so that
\begin{equation*}
 \sec y = x \implies \dv<y>{x} \sec y \tan y = 1
     \implies \dv{x}(\arcsec x) = \frac 1{x \tan y}
\end{equation*}
Unfortunately, we can't just take \(\tan y = \sqrt{x^2 - 1}\) as
\(x \in \Reals \setminus \intoo{-1, 1}\), and
\(y \in \intcc{0, \pi} \setminus \set{\frac 12 \pi}\). This means that
\(\tan y > 0\) for
\(y \in \intco{0, \frac 12 \pi} \iff x \in \intco{1, \infty}\), but also
that \(\tan y < 0\) for
\(y \in \intoc{\frac 12 \pi, \pi} \iff x \in \intoc{-\infty, -1}\).
Therefore, \(\tan y\) has the same sign as \(x\), so we can rewrite
\begin{equation*}
 \dv{x}(\arcsec x) = \frac 1{x \tan y} = \frac 1{\abs{x \tan y}}
     = \frac 1{\abs x\abs{\tan y}} = \frac 1{\abs x\sqrt{x^2 - 1}}
\end{equation*}
This is perhaps a little tricky, and it is in some cases more obvious to
just go about it piecewise:
\begin{equation*}
 \dv{x}(\arcsec x) =
 \begin{cases}
  \dfrac 1{x\sqrt{x^2 - 1}} & x \ge 0 \\[3ex]
  \dfrac {-1}{x\sqrt{x^2 - 1}} & x < 0
 \end{cases}
\end{equation*}
The other equivalent can be found by taking out a factor of \(\abs x^2\)
from the square root, to cancel the absolute sign by squaring it. This can
be useful sometimes.
\begin{equation*}
 \dv{x}(\arcsec x) = \frac 1{\abs x\sqrt{x^2 - 1}}
     = \frac 1{\abs x\sqrt{x^2}\sqrt{1 - x^{-2}}}
     = \frac 1{x^2\sqrt{1 - x^{-2}}}
\end{equation*}
To attack the inverse cosecant, it will be order of magnitude easier to just
make use of the fact that \(\arccsc x = \frac 12 \pi - \arcsec x\),
so that
\begin{equation*}
 \dv{x}(\arccsc x) = \dv{x}\parens[\Big]{\frac \pi 2 - \arcsec x}
                   = -\dv{x}(\arcsec x)
\end{equation*}
and then taking your pick of your favourite inverse secant derivative.

Similarly, \(\arccot x = \frac 12 \pi - \arctan x\), so
\begin{equation*}
 \dv{x}(\arccot x) = \dv{x}\parens[\Big]{\frac \pi 2 - \arctan x}
                   = -\frac 1{1 + x^2}
\end{equation*}

\subsection{Implicit differentiation}

% fix this

%FIXME finish this
