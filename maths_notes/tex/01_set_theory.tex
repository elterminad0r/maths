\section{Set Theory}

\subsection{Set Operations}

\subsection{Common sets}

%FIXME: sets of congruence classes
%       sets of polynomial rings: apply set setstyle A

\begin{longtable}{M p{0.7\textwidth}}
 \toprule
 \text{\bfseries Set} & \bfseries Description \\
 \midrule
 \endhead
 \bottomrule
 \endfoot
 \endlastfoot
 \emptyset, \varnothing & The empty set \(\set{}\). \\
 \Naturals & The set of natural numbers, \(\set{1, 2, 3, \dotsc}\).
             May or may not include 0, so sometimes the next few presented
             alternatives are preferable. \\
 \Integers & The set of integers
             \(\set{\dotsc, -2, 1, 0, 1, 2, \dotsc}\) \\
 \Integers^+, \Integers_{> 0} & The set of strictly positive integers
             \(\set{1, 2, 3, \dotsc}\). \\
 \Integers^+_0, \Integers_{\ge 0} &
             The set of strictly nonnegative integers
             \(\set{0, 1, 2, \dotsc}\). \\
 \Integers^-, \Integers_{< 0} & The set of strictly negative integers
             \(\set{-1, -2, -3, \dotsc}\). \\
 \Integers^-_0, \Integers_{\le 0} &
             The set of strictly nonpositive integers
             \(\set{0, -1, -2, \dotsc}\). \\
 \Rationals & The set of rational numbers
             \(\set{\frac ab \mid a, b \in \Integers \land b \neq 0}\).\\
 \setstyle A & The set of algebraic numbers, ie numbers that are roots of
             polynomials in \(\Integers[x]\). \\
 \Reals & The set of real numbers, which may be constructed as in
         \ref{sec_dedekind_cut}. \\
 \Complex & The set of complex numbers
             \(\set{a + bi \mid a, b \in \Reals}\),
             where \(i^2 = 1\).\\
 \intoo{a, b} & The open interval
                 \(\set{x \in \Reals \mid a < x < b}.\)\\
 \intcc{a, b} & The closed interval
                 \(\set{x \in \Reals \mid a \le x \le b}\).\\
 \intco{a, b} & The half-open interval
                 \(\set{x \in \Reals \mid a \le x < b}\).\\
 \intoc{a, b} & The half-open interval
                 \(\set{x \in \Reals \mid a < x \le b}\).\\
 \bottomrule
 \caption{Common sets}
\end{longtable}

\subsection{Closed, Open, Clopen}

\subsection{Axiom of Choice}

\subsection{ZFC}

\subsection{Construction of Sets}

\subsubsection{Dedekind Cuts} \label{sec_dedekind_cut}

\subsection{Cardinality}

% powerset of integers, diagonal argument, schroeder-bernstein, interval
% cardinality

Cardinality is a way to think about the ``size'' of sets. Cardinality is really
a kind of equivalence relation on the class of sets, where two sets have the
same cardinality iff there exists a bijection between them - ie there is a way
to produce a one-to-one mapping between the two sets. If two sets have the same
cardinality they are said to be equipotent.

Any two finite sets obviously have the same cardinality iff they have the same
number of elements. The cardinality of a finite set is usually just given as the
number of elements it has. For example, \(\abs{\emptyset} = 0\) and
\(\abs{\set{1, 3, 2}} = 3\), etc.

We also denote the cardinality of the natural numbers
\(\abs{\Naturals} = \aleph_0\). This is the Hebrew letter ``aleph'' or ``alef'',
with subscript naught.

It is fairly straightforward to show, for example, that we also have
\(\abs{\Integers} = \aleph_0\), with for example the function \(f\) such that we
have
\begin{equation*}
 \begin{array}{c|r|r|r|r|r|r|r|r|r|r}
  n & 1 & 2 & 3 & 4 & 5 & 6 & 7 & 8 & 9 & \cdots \\
  \hline
  f(n) & 0 & 1 & -1 & 2 & -2 & 3 & -3 & 4 & -4 & \cdots \\
 \end{array}
\end{equation*}
which is to say
\begin{align*}
 f(n) &=
  \frac 12 \left\{
  \begin{array}{cl}
   n & \text{\(n\) is even} \\
   1 - n & \text{\(n\) is odd}
  \end{array}
  \right. \\
 f^{-1}(n) &=
  \begin{cases}
   \hfil 2n & n \ge 1 \\
   \hfil -2n + 1 & n \le 0
  \end{cases}
\end{align*}
Similarly, the cardinality of for example the odd numbers is the same as that of
the naturals.

It can also be shown that there exists a bijection between the natural numbers
and the rationals. Explicitly, you can enumerate the rationals in a large grid,
and then simply traverse the grid by spiralling outwards. The bijection here is
that \(n \in \Naturals\) corresponds the \(n\)th rational you reach.

% TODO: pretty pictures
Another approach is to leverage the Cantor-Schr\"oder-Bernstein Theorem, which
states that there exists a bijection between the sets \(A\) and \(B\) iff there
exists an injection from \(A\) to \(B\) and there exists an injection from \(B\)
to \(A\).

For example, we can take
\begin{align*}
 f \colon \Integers &\to \Rationals \\
          n &\mapsto \frac n1
\end{align*}
and
\begin{align*}
 g \colon \Rationals &\to \Integers \\
          \frac pq &\mapsto 2^p 3^q
\end{align*}
which is guaranteed by the Fundamental Theorem of Arithmetic to be injective for
distinct \(p, q\).

The nice thing about this approach is that it is quite neat, and it easily
generalises to sets of objects that can be identified by finite sequences of
natural numbers (or integers). This is (or at least resembles) a technique
called G\"odel coding, which can be applied to statements in formal systems. See
G\"odels incompleteness theorems.

\begin{theorem}[Cantor's Theorem]
 For any set \(S\), there will be no surjection from \(S\) to its powerset
 \(\PowerSet(S)\), which, to be clear, is defined as
 \begin{equation*}
  \PowerSet(S) \defeq \set{X \mid X \subseteq S}
 \end{equation*}
 Clearly then there is also no bijection.
\end{theorem}
\begin{proof}
 Suppose to the contrary that for some set \(S\) there does exist some
 surjective function \(f \colon S \to \PowerSet(s)\). Consider the set
 \(R = \set{x \in S \mid x \notin f(x)}\), that is all elements of \(S\) that
 are not in their own image under \(f\). As \(f\) is surjective onto
 \(\PowerSet(S)\) and \(R\) must be a subset of \(S\), we can now consider the
 preimage of \(R\) under \(f\) - there must exist some \(\zeta\) such that
 \(f(\zeta) = R\). But then \(\zeta \in R \iff \zeta \notin R\), due to the
 definition of \(R\). This is a contradiction.

 % TODO: is this the law of excluded middle?
 Elaborating, we first of all know that either \(\zeta \in R\) or
 \(\zeta \notin R\), as they are logically opposite statements that must be
 either true or false.

 If \(\zeta \in R\), then by the definition of \(R\),
 \(\zeta \notin R\) because \(R = f(\zeta)\), so
 \(\zeta \in R \iff \zeta \in f(\zeta)\).

 If on the other hand, \(\zeta \notin R\), then \(\zeta \in R\) by the
 construction of R, following the same argument. So
 \(\zeta \in R \implies \zeta \notin R\) and
 \(\zeta \notin R \implies \zeta \in R\).
\end{proof}

This theorem can be leveraged to give a nice proof that the reals have strictly
greater cardinality than the naturals, by showing that they are equipotent with
the powerset of the naturals.

Firstly, we construct an injection \(f\) from \(\Reals\) to
\(\PowerSet(\Rationals)\). For some real number \(x\), we define \(f(x)\) as the
set of all rationals less than or equal to \(x\).
\begin{align*}
 f \colon \Reals &\to \PowerSet(\Rationals) \\
          x &\mapsto \set{r \in \Rationals \mid r \le x}
\end{align*}
This is the Dedekind cut corresponding to \(x\), and uniquely identifies each
\(x\). Recalling that the rationals have the same cardinality as the natural
numbers, clearly there must then exist an injection from \(\Reals\) to
\(\PowerSet(\Naturals)\), by the composition of some bijection
\(\Rationals \to \Naturals\) applied to each element of a subset of
\(\Naturals\) with \(f\).

Now we construct an injection \(g\) from \(\PowerSet(\Naturals)\) to \(\Reals\).
Note \(\PowerSet(\Naturals)\) is isomorphic to the set of all infinite sequences
of binary digits. Particularly, for some subset
\(S \in \PowerSet(\Naturals) \iff S \subseteq \Naturals\), the \(n\)th digit of
the binary sequence corresponding to \(S\) may be given by \texttt 1 if
\(n \in S\), and \texttt 0 otherwise.

Now for some \(S \in \PowerSet(\Naturals)\), generate the binary sequence, then
read it back as a ternary expansion of a real number. That is to say, if the
digits are given by \(U_i, i \in \Naturals\), then \(g\) will map \(s\) to the
real number
\begin{equation*}
 x = \sum_{r \in \Naturals} \frac{U_r}{3^r}
\end{equation*}
This sum will always converge, as it is bounded by
\begin{equation*}
 x = \sum_{r \in \Naturals} \frac 1{3^r} = \frac 12
\end{equation*}
Furthermore, none of these sequences conflict with one another, as we have
interpreted them in ternary, which avoids any infinite sequences of \(2\)s,
which could be replaced by a single zero. For example, had we read them in
binary, then the sequences \texttt{0.011111}\ldots and \texttt{0.1000000}\ldots
would both correspond to the real number \(\frac 12\).

So we have our injection, and then by the Cantor-Schr\"oder-Bernstein Theorem,
\(\abs{\Reals} = \abs{\PowerSet(\Naturals)}\), and then by Cantor's Theorem,
\(\abs{\Reals} > \abs{\Naturals}\).

The cardinality of the reals \(\abs{\Reals}\) is also denoted \(\mathfrak c\) (a
lowercase fraktur script c) and \(\beth_1\) (Hebrew letter ``beth'' with
subscript one). If the continuum hypothesis is true, then \(\beth_1 = \aleph_1\)
(this is in fact basically what CH states). However CH is known to be
independent of ZFC, so you can't prove or disprove it with the ``standard''
axioms of set theory.

% TODO: diagonalisation argument

It is also useful to know that generally, the cardinality of a non-empty
interval of the reals is the same as the cardinality of the reals. Similarly,
the cardinality of a non-empty interval of the rationals is the same as the
cardinality of the rationals.

% TODO: plots
The case for reals is easier to show. First, note that any non-empty interval
\(\intoo{a, b}\) (that is, where \(a < b\)) is equipotent to \(\intoo{0, 1}\),
by using the bijection
\begin{align*}
 f \colon \intoo{0, 1} &\to \intoo{a, b} \\
          x &\mapsto (b - a)x + a
\end{align*}

Now we just need to show that any interval has the same cardinality as the
reals. Take \(\intoo{-1, 1}\), and consider the function
\begin{align*}
 g \colon \intoo{-1, 1} &\to \Reals \\
          x &\mapsto \frac x{1 - x^2}
\end{align*}
Noting that
\begin{equation*}
 g'(x) = \frac{1 - x^2 + 2x^2}{(1 - x^2)^2}
       = \frac{1 + x^2}{(1 - x^2)^2}
\end{equation*}
it is clear that \(g'(x) > 0\) for all \(x \in \Reals\). Further, as there are
no discontinuities for \(x \in \intoo{-1, 1}\) and \(g(x) \to -\infty\) as
\(x \to^+ -1\) and \(g(x) \to \infty\) as \(x \to^- 1\), the range of \(g(x)\)
must be \(\Reals\) and \(g\) must be bijective as it is monotonically
increasing.

I picked this function because it has some generally nice properties (such as
being a rational function, implying rational input have rational outputs), and
is nice to work with. I could have used some corruption of \(\tan x\) or
\(\arctanh x\).

It seems reasonable that half-closed or entirely closed intervals should also
have the same cardinality, as they only have one more element. It is in fact
true that the union of any countable set with a set with cardinality \(\beth_1\)
has cardinality \(\beth_1\) but the proof thereof is overkill for this case.

We can use the same kind of linear map as our \(f\) to show that any closed
interval has the same cardinality as \(\intcc{0, 1}\). Note then that
\begin{equation*}
 \abs[\Big]{\intcc{0, 1}} = \abs[\bigg]{\intcc[\Big]{\frac 14, \frac 34}} =
  \abs[\Big]{\intcc{-1, 2}}
\end{equation*}
But \(\intcc{\frac 14, \frac 34} \subset \intoo{0, 1}\) so
\(\abs{\intcc{\frac 14, \frac 34}} \le \abs{\intoo{0, 1}}\). Similarly,
\(\abs{\intcc{-1, 2}} \ge \abs{\intoo{0, 1}}\). So then
\(\abs{\intcc{0, 1}} \le \abs{\intoo{0, 1}}\) and
\(\abs{\intcc{0, 1}} \ge \abs{\intoo{0, 1}}\), implying that
\(\abs{\intcc{0, 1}} = \abs{\intoo{0, 1}}\), and hence any totally open
non-empty interval is equipotent to any totally closed non-empty interval.

The two remaining cases are half-closed or half-open intervals. Again, we need
only consider \(\intoc{0, 1}\) and \(\intco{0, 1}\). The argument is in fact
identical for each. Let \(S\) denote the range in question. Note that
\(S \subset \intcc{0, 1}\) and \(S \supset \intoo{0, 1}\), so
\(\abs S \ge \abs \Reals\) and \(\abs S \le \abs \Reals\), implying
\(\abs S = \abs \Reals\).

Now let's consider rational intervals. If some rational interval
\(\intoo{a, b} \cup \Rationals\) has boundaries \(a, b \in \Rationals\), then
the argument works exactly the same as for the real intervals above
(particularly due to the choice of \(g\) as a rational function).
The subtlety lies in our \(f\) function. The multiplication by \(b - a\) and the
addition to \(a\) won't necessarily keep our elements in \(\Rationals\) if
\(a, b \notin \Rationals\). It will be necessary to show the density of the
rationals in the reals to get around this. This is basically the fact that
between any to real numbers, there will be two rational numbers. It seems fairly
obvious.

A way to demonstrate this fact is by the pigeon-hole principle. First, we would
like to pick some rational \(r \in \Rationals\) less than \(b - a \in \Reals\).
This could be done, for example, by taking
\begin{equation*}
 r = 2^{\floor{\log_2(b - a)} - 1}
\end{equation*}
which is the second highest power of 2 (allowing negative exponents) under
\(b - a\). We take the second highest just to be safe around edge cases where
\(b - a\) happens to be a power of 2, because we aren't here to think things
through. Here \(\floor{\cdot}\) is the floor function, basically ``rounding
down'' to the nearest integer.

Now consider the rationals which are \(\frac r4\) multiplied by an integer.
Because \(r < b - a\), there must be at least \(2\) of these multiples of
\(\frac r4\) in \(\intoo{a, b} \cup \Rationals\), in the worst possible case
where two of them lie on a boundary. This argument works much the same for all
the other types of boundary.

Now we have determined a rational interval with rational boundaries that is a
subset of our general rational interval in the reals. We have shown that this
has the same cardinality as the rationals, so our rational interval has
cardinality greater than or equal to that of the rationals. But the rational
interval is itself a subset of the rationals, so must have cardinality less than
or equal to that of the rationals. So it must be equipotent to the rationals.

The cardinality of the set of all infinite sequences of naturals
(\(\Naturals^\Naturals\)) is the same as that of the powerset of the naturals,
\(\beth_1\). This is for example also the cardinality of the real numbers. We
can show this by encoding an infinite sequence of naturals as a (possible
infinite) subset of the naturals. This can be done, once again, with some binary
trickery. We need a somewhat sophisticated encoding though.

First determine the binary expansion of each natural in the sequence. Now, we
can encode a \texttt 0 as \texttt{00} and a \texttt 1 as \texttt{11}. We can
then use \texttt{01} as a delimiting sequence. This means that we encode each
natural with this method, and then write it all in one long sequence, where
\texttt{01} is used to indicate the boundaries between elements. This new unique
``longer'' binary sequence also corresponds to a unique subset of the naturals,
as we discussed earlier (by including each natural which corresponds to an index
at which there is a \texttt 1 in the sequence). So we have an injection from the
set of infinite natural sequences to the powerset of the naturals.

An injection the other way is fairly straightforward - you could for example map
a subset of the naturals to a list of the naturals in that set in order of
increasing size. If the subset is finite, you could just ``pad'' the rest of it
with any number of your choosing. I recommend 5. Wham, bam, thank you
Cantor-Schr\"oder-Bernstein.

The Cartesian product of the set of reals with itself, \(\Reals^2\),
corresponding to the two-dimensional plane, the set of complex numbers
\(\mathbb C\) or generally any state space that can be characterised by two real
numbers, is equipotent to the reals. This is sometimes hand-waved into fact by
waffling about space-filling curves but we can do better. A particularly nice
approach is to use the previous paragraph to deduce that we can encode any real
as an infinite sequence of integers, and any infinite sequence of integers as a
real. We then need only take the two real numbers in question, take their
infinite sequences of integers and interlace them to form a new infinite
sequence, reading that back as a single real number. This in fact will work for
any finite ``power'' of \(\Reals\), \(\Reals^n\).

A similar result for \(\Rationals\) is even easier. We can just use the G\"odel
codings, for example to form the injection
\begin{alignat*}2
 f \colon &&\Rationals^2 &\to \Rationals \\
          &&\parens[\Big]{\frac ab, \frac cd} &\mapsto 2^a 3^b 5^c 7^d
\end{alignat*}
An injection the other way is fairly straightforward:
\begin{align*}
 g \colon \Rationals &\to \Rationals^2 \\
          r &\to \parens[\Big]{r, \frac 01}
\end{align*}
It is again possible to use this approach for any finite power,
\(\Rationals^n\).

\subsection{Functions}

% add involutions

\subsubsection{Jections}

%FIXME mnemonic or diagram

\begin{itemize}
 \item An injection maps each element of its domain to a unique element of
       its codomain.
 \item A surjection maps an element of its domain to each element of its
       codomain.
 \item A bijection is an injection and a surjection.
\end{itemize}
