\section{Set Theory}

\subsection{Set Operations}

\subsection{Common sets}

%FIXME: sets of congruence classes
%       sets of polynomial rings: apply set setstyle A

\begin{longtable}{M p{0.7\textwidth}}
 \toprule
 \text{\bfseries Set} & \bfseries Description \\
 \midrule
 \endhead
 \bottomrule
 \endfoot
 \endlastfoot
 \emptyset, \varnothing & The empty set \(\set{}\). \\
 \Naturals & The set of natural numbers, \(\set{1, 2, 3, \dotsc}\).
             May or may not include 0, so sometimes the next few presented
             alternatives are preferable. \\
 \Integers & The set of integers
             \(\set{\dotsc, -2, 1, 0, 1, 2, \dotsc}\) \\
 \Integers^+, \Integers_{> 0} & The set of strictly positive integers
             \(\set{1, 2, 3, \dotsc}\). \\
 \Integers^+_0, \Integers_{\ge 0} &
             The set of strictly nonnegative integers
             \(\set{0, 1, 2, \dotsc}\). \\
 \Integers^-, \Integers_{< 0} & The set of strictly negative integers
             \(\set{-1, -2, -3, \dotsc}\). \\
 \Integers^-_0, \Integers_{\le 0} &
             The set of strictly nonpositive integers
             \(\set{0, -1, -2, \dotsc}\). \\
 \Rationals & The set of rational numbers
             \(\set{\frac ab \mid a, b \in \Integers \land b \neq 0}\).\\
 \setstyle A & The set of algebraic numbers, ie numbers that are roots of
             polynomials in \(\Integers[x]\). \\
 \Reals & The set of real numbers, which may be constructed as in
         \ref{sec_dedekind_cut}. \\
 \Complex & The set of complex numbers
             \(\set{a + bi \mid a, b \in \Reals}\),
             where \(i^2 = 1\).\\
 \intoo{a, b} & The open interval
                 \(\set{x \in \Reals \mid a < x < b}.\)\\
 \intcc{a, b} & The closed interval
                 \(\set{x \in \Reals \mid a \le x \le b}\).\\
 \intco{a, b} & The half-open interval
                 \(\set{x \in \Reals \mid a \le x < b}\).\\
 \intoc{a, b} & The half-open interval
                 \(\set{x \in \Reals \mid a < x \le b}\).\\
 \bottomrule
 \caption{Common sets}
\end{longtable}

\subsection{Closed, Open, Clopen}

\subsection{Axiom of Choice}

\subsection{ZFC}

\subsection{Construction of Sets}

\subsubsection{Dedekind Cuts} \label{sec_dedekind_cut}

\subsection{Cardinality}

% powerset of integers, diagonal argument, schroeder-bernstein, interval
% cardinality

Cardinality is a way to think about the ``size'' of sets. Cardinality is really
a kind of equivalence relation on the class of sets, where two sets have the
same cardinality iff there exists a bijection between them - ie there is a way
to produce a one-to-one mapping between the two sets. If two sets have the same
cardinality they are said to be equipotent.

Any two finite sets obviously have the same cardinality iff they have the same
number of elements. The cardinality of a finite set is usually just given as the
number of elements it has. For example, \(\abs{\emptyset} = 0\) and
\(\abs{\set{1, 3, 2}} = 3\), etc.

We also denote the cardinality of the natural numbers
\(\abs{\Naturals} = \aleph_0\).

It is fairly straightforward to show, for example, that we also have
\(\abs{\Integers} = \aleph_0\), with for example the function \(f\) such that we
have
\begin{equation*}
 \begin{array}{c|r|r|r|r|r|r|r|r|r|r}
  n & 1 & 2 & 3 & 4 & 5 & 6 & 7 & 8 & 9 & \cdots \\
  \hline
  f(n) & 0 & 1 & -1 & 2 & -2 & 3 & -3 & 4 & -4 & \cdots \\
 \end{array}
\end{equation*}
which is to say
\begin{align*}
 f(n) &=
  \frac 12 \begin{cases}
   n & \text{\(n\) is even} \\
   1 - n & \text{\(n\) is odd}
  \end{cases} \\
 f^{-1}(n) &=
  \begin{cases}
   2n & n \ge 1 \\
   -2n + 1 & n \le 0
  \end{cases}
\end{align*}
Similarly, the cardinality of for example the odd numbers is the same as that of
the naturals.

It can also be shown that there exists a bijection between the natural numbers
and the rationals. Explicitly, you can enumerate the rationals in a large grid,
and then simply traverse the grid by spiralling outwards. The bijection here is
that \(n \in \Naturals\) corresponds the \(n\)th rational you reach.

% TODO: pretty pictures
Another approach is to leverage the Cantor-Schr\"oder-Bernstein Theorem, which
states that there exists a bijection between the sets \(A\) and \(B\) iff there
exists an injection from \(A\) to \(B\) and there exists an injection from \(B\)
to \(A\).

% TODO how are you actually supposed to write this with like fancy arrows
For example, we can take
\begin{equation*}
 f(n) = \frac n1
\end{equation*}
and
\begin{equation*}
 g\parens[\Big]{\frac pq} = 2^p 3^q
\end{equation*}
which is guaranteed by the Fundamental Theorem of Arithmetic to be injective for
distinct \(p, q\).

The nice thing about this approach is that it is quite neat, and it easily
generalises to sets of objects that can be identified by finite sequences of
natural numbers (or integers). This is (or at least resembles) a technique
called G\"odel coding, which can be applied to statements in formal systems. See
G\"odels incompleteness theorems.

\begin{theorem}[Cantor's Theorem]
 For any set \(S\), there will be no surjection from \(S\) to its powerset
 \(\PowerSet(S)\), which, to be clear, is defined as
 \begin{equation*}
  \PowerSet(S) \defeq \set{X \mid X \subseteq S}
 \end{equation*}
 Clearly then there is also no bijection.
\end{theorem}
\begin{proof}
 Suppose to the contrary that for some set \(S\) there does exist some
 surjective function \(f\) from \(S\) to \(\PowerSet(s)\). Consider the set
 \(R = \set{x \in S \mid x \not\in f(x)}\), that is all elements of \(S\) that
 are not in their own image under \(f\). As \(f\) is surjective onto
 \(\PowerSet(S)\) and \(R\) must be a subset of \(S\), we can now consider the
 preimage of \(R\) under \(f\) - there must exist some \(\zeta\) such that
 \(f(\zeta) = R\). But then \(\zeta \in R \iff \zeta \not\in R\), due to the
 definition of \(R\). This is a contradiction.

 % TODO: is this the law of excluded middle?
 Elaborating, we first of all know that either \(\zeta \in R\) or
 \(\zeta \not\in R\), as they are logically opposite statements that must be
 either true or false.

 If \(\zeta \in R\), then by the definition of \(R\),
 \(\zeta \not\in R\) because \(R = f(\zeta)\), so
 \(\zeta \in R \iff \zeta \in f(\zeta)\).

 If on the other hand, \(\zeta \not\in R\), then \(\zeta \in R\) by the
 construction of R, following the same argument. So
 \(\zeta \in R \implies \zeta \not\in R\) and
 \(\zeta \not\in R \implies \zeta \in R\).
\end{proof}

This theorem can be leveraged to give a nice proof that the reals have strictly
greater cardinality than the naturals, by showing that they are equipotent with
the powerset of the naturals.

Firstly, we construct an injection \(f\) from \(\Reals\) to
\(\PowerSet(\Rationals)\). For some real number \(x\), we define \(f(x)\) as the
set of all rationals less than or equal to \(x\).
\begin{equation*}
 f(x) = \set{q \in \Rationals \mid q \le x}
\end{equation*}
This is the Dedekind cut corresponding to \(x\), and uniquely identifies each
\(x\). Recalling that the rationals have the same cardinality as the natural
numbers, clearly there must then exist an injection from \(\Reals\) to
\(\PowerSet(\Naturals)\).

Now we construct an injection \(g\) from \(\PowerSet(\Naturals)\) to \(\Reals\).
Note \(\PowerSet(\Naturals)\) is isomorphic to the set of all infinite sequences
of binary digits. Particularly, for some subset
\(S \in \PowerSet(\Naturals) \iff S \subseteq \Naturals\), the \(n\)th digit of
the binary sequence corresponding to \(S\) may be given by \(1\) if \(n \in S\),
and \(0\) otherwise.

Now for some \(S \in \PowerSet(\Naturals)\), generate the binary sequence, then
read it back as a ternary expansion of a real number. That is to say, if the
digits are given by \(U_i, i \in \Naturals\), then we extract the real number
\begin{equation*}
 x = \sum_{r \in \Naturals} \frac{U_r}{3^r}
\end{equation*}
This sum will always converge, as it is bounded by
\begin{equation*}
 x = \sum_{r \in \Naturals} \frac 1{3^r} = \frac 12
\end{equation*}
Furthermore, none of these sequences conflict with one another, as we have
interpreted them in ternary, which avoids any infinite sequences of \(2\)s,
which could be replaced by a single zero. For example, had we read them in
binary, then the sequences \(0.011111\ldots\) and \(0.1000000\ldots\) would both
correspond to the real number \(\frac 12\).

So we have our injection, and then by the Cantor-Schr\"oder-Bernstein Theorem,
\(\abs{\Reals} = \abs{\PowerSet(\Naturals)}\), and then by Cantor's Theorem,
\(\abs{\Reals} > \abs{\Naturals}\).

% TODO: diagonalisation argument

\subsection{Functions}

% add involutions

\subsubsection{Jections}

%FIXME mnemonic or diagram

\begin{itemize}
 \item An injection maps each element of its domain to a unique element of
       its codomain.
 \item A surjection maps an element of its domain to each element of its
       codomain.
 \item A bijection is an injection and a surjection.
\end{itemize}
