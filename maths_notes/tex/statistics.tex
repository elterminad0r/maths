\section{Statistics}

\subsection{Discrete Random Variables}

\subsection{Probability Generating Functions}

\subsection{Common Discrete Distributions}

\subsubsection{Binomial Distribution}

%FIXME add diagram

The binomial distribution \(\Binomial(\nu, \pi)\) measures the probability
of having a certain number of outcomes in a set of trials,  where there are
\(\nu\) trials each with probability \(\pi\) of having that outcome.
The validity of this distribution is subject to these constraints:
\begin{enumerate}
\item Each event must be independent.
\item Each event must either have that outcome or not have it.
\item The probability of having this outcome must be identically the same in
      each trial.
\end{enumerate}
This means that the binomial distribution can be used to model the expected
number of occurrences of some attribute in a random sample of a population
with replacement. However, in a sample without replacement, each trial is
not strictly independent, so the binomial distribution can't be used. For
large population sizes, however, it remains a good approximation.

In this case, ``large'' can be taken to mean more than about 30-100.

The PMF of \(X \sim \Binomial(\nu, \pi)\) is given by
\begin{equation*}
f_X(k) = \Prob(X = k) = \binom \nu k \pi^k (1 - \pi)^{\nu - k}
    \quad \text{where \(k \in \set{0, 1, \dotsc, \nu}\)}
\end{equation*}
where
\begin{equation*}
\binom \nu k = \nCr \nu k = \frac{\nu!}{k!\cdot (\nu - k)!}
\end{equation*}

\begin{theorem}[Binomial properties]
Where \(X \sim \Binomial(\nu, \pi)\),
\begin{align*}
\Expect(X) &= \nu \pi \\
\Var(X) &= \nu \pi (1 - \pi)
\end{align*}
\end{theorem}
\begin{proof}
\end{proof}

A binomial distribution can be approximated by a normal distribution.

\subsubsection{Poisson Distribution}

%FIXME derivation as limit of binomial, see generating functions

\subsection{Continuous Random Variables}

\subsection{Moment Generating Functions}

\subsection{Common Continuous Distributions}

\subsubsection{Normal Distribution}

%FIXME add diagram

The PDF of \(X \sim \Normal(\mu, \sigma^2)\) is given by
\begin{equation*}
f_X(x) = \frac 1{\sqrt{2\pi\sigma^2}} \cdot
    \exp(\frac{(x - \mu)^2}{2\sigma^2})
\end{equation*}

\subsubsection{Exponential Distribution}

\subsubsection[Student's \texorpdfstring{\(t\)}{t}-distribution]
              {Student's \boldmath\(t\)-distribution}

\subsection{Hypothesis Testing}

%FIXME Types of Error
