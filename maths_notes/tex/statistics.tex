\section{Statistics}

\subsection{Discrete Random Variables}

\subsection{Probability Generating Functions}

The PGF of some discrete random variable \(G_X(\eta)\) is, essentially, a power
series in \(\eta\) with as coefficients the probability associated with that
power of \(\eta\):
\begin{equation*}
G_X(\eta) \defeq \sum_{r = 0}^\infty \Prob(X = r)\eta^r
    = \sum_{r = 0}^\infty p_X(r)\eta^r
    = \Expect(\eta^X)
\end{equation*}
This does imply the restriction that \(X\) only takes
non-negative integer values, but this is not usually a huge hindrance.

It is easy to see that \(G(\eta)\) has a radius of convergence and is therefore
defined, as each coefficient is necessarily less than or equal to \(1\), so
\(G(\eta)\) is bounded by
\begin{equation*}
G_X(\eta) < 1 + x + x^2 + \dotsb = \frac 1{1 - x}
\end{equation*}
which is defined for \(\eta \in \intco{0, 1}\). Moreover,
\begin{equation*}
G(1) = \sum_{r = 0}^\infty \Prob(X = r) = 1
\end{equation*}
Some particularly useful properties arise when considering the derivatives of
\(G\) wrt \(\eta\).

Note that we can extract the coefficients of the power series by taking
derivatives,
\begin{equation*}
\Prob(X = k) = \frac{G_X^{(k)}(0)}{k!}
\end{equation*}
Therefore a probability distribution is uniquely identified by its associated
generating function. Obverve also that
\begin{alignat*}6
G_X(\nu) &= \sum_{r = 0}^\infty p_X(r)\eta^r
    &&={}& p_X(0) &+{}& p_X(1)\eta
    &+{}& p_X(2)\eta^2 &+{}& p_X(3) \eta^3 + \dotsb \\
G_X'(\nu) &= \sum_{r = 0}^\infty rp_X(r)\eta^{(r - 1)}
    &&={}& 1 p_X(1) &+{}& 2p_X(2)\eta
    &+{}& 3p_X(3) \eta^2 &+{}& 4p_X(4) \eta^3 + \dotsb \\
G_X''(\nu) &= \sum_{r = 0}^\infty r(r - 1)p_X(r)\eta^{(r - 2)}
    &&={}& 2 p_X(2) &+{}& 6p_X(3)\eta
    &+{}& 12p_X(4) \eta^2 &+{}& 20p_X(5) \eta^3 + \dotsb
\end{alignat*}
So that particularly,
\begin{align*}
G_X(1) &= \Expect(1) \\
G_X'(1) &= \Expect(X) \\
G_X''(1) &= \Expect(X(X - 1))
\end{align*}
This is really just spelling out that by considering the expectation definition
of \(G_X\), in addition to the fact that an expectation is really just a sum, we
can just differentiate the inside when differentiating. This generalises to the
fact that the \(n\)th derivative of \(G_X\) at \(\eta = 1\) gives the \(n\)th
\emph{factorial moment} of \(X\).
\begin{equation*}
    G_X^{(n)}(1) = \Expect\bracks[\bigg]{\frac{X!}{(X - n)!}}
\end{equation*}

Regardless, we now have a useful tool. If we know \(G_X\), we can work out
\(\Expect(X)\) by differentiating once, and to obtain the variance we can use
\begin{align*}
\Var(x) &= \Expect(X^2) - (\Expect(X))^2
    = \Expect(X^2 - X) + \Expect(X) - (\Expect(X))^2 \\
    &= G_X''(1) + G_X'(1) - (G_X'(1))^2
\end{align*}
The generating function of the sum of two independent random variables \(X + Y\)
(that need not necessarily be identically distributed) has a particularly useful
form. Consider
\begin{align*}
G_{X+Y}(\eta) &= \Expect(\eta^{X + Y}) = \Expect(\eta^X \eta^Y) \\
    &= \Expect(\eta^X) \Expect(\eta^Y) \impliedby X \independent Y \\
    &= G_X(\eta) G_Y(\eta)
\end{align*}
In fact, this generalises to the sum of \(n\) independent random variables, each
multiplied by some constant coefficient:
\begin{alignat*} 3
&& S &= \sum_{r = 1}^n \alpha_r X_r
    &&= \alpha_1 X_1 + \alpha_2 X_2 + \dotsb + \alpha_n X_n \\
    &\implies& G_S(\eta) &= \Expect\parens[\Big]{\eta^{\sum_{r = 1}^n \alpha_r X_r}}
    &&= \Expect(\eta^{\alpha_1 X_1 + \alpha_2 X_2 + \dotsb + \alpha_n X_n}) \\
    && &= \Expect\parens[\Big]{\prod_{r = 1}^n \eta^{\alpha_r X_r}}
    &&= \Expect(\eta^{\alpha_1 X_1} \eta^{\alpha_2 X_2} \dotsm
              \eta^{\alpha_N X_n}) \\
&& &= \prod_{r = 1}^\infty\Expect(\eta^{\alpha_r X_r}) \\
% a bit of witchcraft so that the width of this cell is ignored
&& &= \mathrlap{
        \Expect(\eta^{\alpha_n X_n}) \Expect(\eta^{\alpha_2 X_2}) \dotsm
        \Expect(\eta^{\alpha_n X_n})
            \impliedby X_1 \independent X_2 \independent
                             \dotsb \independent X_n} \\
&& &= \prod_{r = 1}^\infty G_{X_r}(\eta^{\alpha_r})
    &&= G_{X_1}(\eta^{\alpha_1}) G_{X_2}(\eta^{\alpha_2}) \dotsm
        G_{X_n}(\eta^{\alpha_n})
\end{alignat*}

\subsection{Common Discrete Distributions}

\subsubsection{Binomial Distribution}

%FIXME add diagram

The binomial distribution \(\Binomial(\nu, \pi)\) measures the probability
of having a certain number of outcomes in a set of trials,  where there are
\(\nu\) trials each with probability \(\pi\) of having that outcome.
The validity of this distribution is subject to these constraints:
\begin{enumerate}
\item Each event must be independent.
\item Each event must either have that outcome or not have it.
\item The probability of having this outcome must be identically the same in
      each trial.
\end{enumerate}
This means that the binomial distribution can be used to model the expected
number of occurrences of some attribute in a random sample of a population
with replacement. However, in a sample without replacement, each trial is
not strictly independent, so the binomial distribution can't be used. For
large population sizes, however, it remains a good approximation.

In this case, ``large'' can be taken to mean more than about 30-100.

\begin{theorem}[Binomial properties]
Where \(X \sim \Binomial(\nu, \pi)\),
\begin{align*}
p_X(k) &= \Prob(X = k) = \binom \nu k \pi^k (1 - \pi)^{\nu - k}
    \quad \text{where \(k \in \set{0, 1, \dotsc, \nu}\)} \\
\Expect(X) &= \nu \pi \\
\Var(X) &= \nu \pi (1 - \pi) \\
\intertext{where}
\binom \nu k &= \nCr \nu k = \frac{\nu!}{k!\cdot (\nu - k)!}
\end{align*}
\end{theorem}
\begin{proof}
From Lemma \ref{lem_binom_pgf}, we have that if \(X \sim \Binomial(\nu, \pi)\)
\begin{alignat*} 2
&& G_X(\eta) &= (1 - \pi + \pi \eta)^\nu \\
&&           &= \sum_{k = 0}^\nu
                \binom \nu k (1 - \pi)^{\nu - k} \pi^k \eta^k
                    \quad \text{by the Binomial Theorem} \\
\intertext{
So clearly, where \(k \in \set{0, 1, \dotsc, \nu}\), the coefficient in
\(\eta^k\) (which is \(\Prob(X = k)\)) is
\(\binom \nu k (1 - \pi)^{\nu - k} \pi^k\).}
&\implies& G_X'(\eta) &= \nu \pi (1 - \pi + \pi \eta)^{\nu - 1} \\
&& G_X'(1) &= \nu \pi (1 - pi + pi)^{\nu - 1} = \nu \pi \\
&\implies& G_X''(\eta) &= \nu(\nu - 1) \pi^2 (1 - \pi + \pi \eta)^{\nu - 2} \\
&& G_X''(1) &= \nu(\nu - 1)\pi^2(1 - \pi + \pi)^{\nu - 2} = \nu(\nu - 1)\pi^2 \\
&\implies& \Expect(X) &= G_X'(1) = \nu \pi \\
&& \Var(X) &= G_X''(1) + G_X'(1) - (G_X''(1))^2
    = \nu(\nu - 1) \pi^2 + \nu \pi - (\nu \pi)^2 \\
&& &= \nu \pi - \nu \pi^2 = \nu \pi (1 - \pi) \qedhere
\end{alignat*}
\end{proof}
\begin{proof}[Alternative for \(p_X(k)\)]
We can also argue for \(\Prob(X = k)\) combinatorially. We can consider the
possible outcomes where the trials have an ordering, ie possible sequences
of successes and failures. If the outcome is a total of \(k\) successes, then
there are precisely \(\binom \nu k\) sequences achieving this, as we must choose
\(k\) of the \(\nu\) trials to be successful. Each of these sequences has
probability \(\pi^k (1 - \pi)^{\nu - k}\) of occurring, as \(k\) of the trials
must be successful and \(\nu - k\) of the trials must be unsuccessful. As they
are all mutually exclusive, the probability of getting any one such sequence is
\(\binom \nu k \pi^k (1 - \pi)^{\nu - k}\).
\end{proof}
\begin{lemma}[Binomial PGF] \label{lem_binom_pgf}
If \(X \sim \Binomial(\nu, \pi)\), then
\begin{equation*}
G_X(\eta) = (1 - \pi + \pi \eta)^\nu
\end{equation*}
\end{lemma}
\begin{proof}
We can use the sum of indendent random variables property of PGFs. If we let the
random variable \(X_k'\) denote the number of successes in the \(k\)th trial, we
have by definition that either \(X_k'\) is 0 or 1, with probability \(1 - \pi\)
and \(\pi\) respectively, and that each different \(X_k'\) will be mutually
independent. Each \(X_k'\) is identically distributed, so they have identical
PGFs:
\begin{equation*}
\Forall k \in \set{1, 2, \dotsc, \eta}: G_{X_k'}(\eta) = 1 - \pi + \pi \eta
\end{equation*}
Now, because \(X = X_1' + X_2' + \dotsb + X_n' = \sum_{r = 1}^\nu X_r'\),
\begin{equation*}
G_X(\eta) = \prod_{r = 1}^\nu G_{X_r'}(\eta)
    = (1 - \pi + \pi \eta)^\nu \qedhere
\end{equation*}
\end{proof}

A binomial distribution can be approximated by a normal distribution.

\subsubsection{Poisson Distribution}

The Poisson distribution \(\Poisson(\lambda)\) models the rate of occurrences of
some event that has a known, constant, average rate of occurrence \(\lambda\).
It's phrased like that because it can be a rate of occurrence per unit distance,
or per unit time, etc.

\begin{theorem}[Poisson properties]
Where \(X \sim \Poisson(\lambda)\),
\begin{align*}
\Prob(X = k) &= p_X(k) = \frac{\lambda^k e^{-\lambda}}{k!}
    \quad \text{for \(k \in \Integers^+_0\)} \\
\Expect(X) &= \lambda \\
\Var(X) &= \lambda
\end{align*}
\end{theorem}
\begin{proof}
If we let \(X \sim \Poisson(\lambda)\), and \(Y \sim \Binomial(\nu, \pi)\),
we can take \(X\) to be the limit of Y if we set \(\Expect(Y) = \lambda\)
as \(\nu \to \infty\). This means particularly that we can use
\begin{alignat*}2
&& \Expect(X) = \nu \pi &= \lambda \\
&\implies& \pi &= \frac \lambda \nu
\end{alignat*}
From Lemma \ref{lem_binom_pgf} we know the PGF of \(Y\):
\begin{alignat*}2
&& G_Y(\eta) &= (1 - \pi + \pi \eta)^\nu \\
    &&  &= \parens[\Big]{1 - \frac \lambda \nu + \frac \lambda \nu \eta}^\nu \\
&\implies& G_X(\eta) = \lim_{\nu \to \infty}G_Y(\eta)
    &= \lim_{\nu \to \infty} \parens[\Big]{1 - \frac \lambda \nu
                                 + \frac \lambda \nu \eta}^\nu \\
    &&  &= \lim_{\nu \to \infty} \parens[\Big]{1 + \frac{\lambda(\eta - 1)}\nu}^\nu \\
&&  &= e^{\lambda(\eta - 1)} = e^{-\lambda}e^{\lambda \eta}
    \quad \text{see section \ref{sec_e}} \\
&&  &= e^{-\lambda} \sum_{k = 0}^\infty \frac{(\lambda \eta)^k}{k!} \\
&&  &= \sum_{k = 0}^\infty \frac{\lambda ^k e^{-\lambda}}{k!} \eta^k \\
\intertext{
So clearly the coefficient of the term in \(\eta^k\) (ie \(\Prob(X = k)\)) is
\(\dfrac{\lambda^k e^{-\lambda}}{k!}\). Also,}
&\implies& G_X'(\eta) &= \lambda e^{\lambda(\eta - 1)} \\
&& G_X'(1) &= \lambda \\
&\implies& G_X''(\eta) &= \lambda^2 e^{\lambda(\eta - 1)} \\
&& G_X''(1) &= \lambda^2 \\
&& \Var(X) &= G_X''(1) + G_X'(1) - (G_X''(1))^2
    = \lambda^2 + \lambda - \lambda^2 = \lambda \qedhere
\end{alignat*}
\end{proof}

\subsection{Continuous Random Variables}

\subsection{Moment Generating Functions}

\subsection{Common Continuous Distributions}

\subsubsection{Normal Distribution}

%FIXME add diagram

The PDF of \(X \sim \Normal(\mu, \sigma^2)\) is given by
\begin{equation*}
f_X(x) = \frac 1{\sqrt{2\pi\sigma^2}} \cdot
    e^{\frac{(x - \mu)^2}{2\sigma^2}}
\end{equation*}

\subsubsection{Exponential Distribution}

\subsubsection[Student's \texorpdfstring{\(t\)}{t}-distribution]
              {Student's \boldmath\(t\)-distribution}

\subsection{Hypothesis Testing}

%FIXME Types of Error
