\section{Sums, Sequences, Series}

\subsection{Binomial Theorem}

\begin{theorem}[Binomial theorem] \label{thm_binomial_thm}
 Where \(n \in \Naturals\),
 \begin{equation*}
  (a + b)^n \equiv \sum_{r = 0}^n \binom nr a^r b^{n - r}
 \end{equation*}
 where the binomial coefficients are given by
 \begin{equation*}
  \binom nr = \nCr nr \defeq \frac{n!}{r!\cdot(n - r)!}
 \end{equation*}
\end{theorem}
\begin{proof}
 We can prove that this is the case by induction. Let \(P(n)\) denote the
 binomial theorem for exponent \(n, \Forall n \in \Integers^+\).
 \begin{enumerate}[label=\Roman*.]
  \item \label{basec_thm_binomial} Consider \(P(1)\):
        \begin{equation*}
        (a + b)^1 \equiv a + b
            \equiv \frac{0!}{0!\cdot 1!} a^0b^1 +
                \frac{1!}{1!\cdot 0!} a^1b^0
        \end{equation*}
        so \(P(1)\) is true.
  \item \label{induct_thm_binomial} We now consider \(P(n + 1)\), supposing
        that \(P(n)\) is true.
        \begin{align*}
        (a + b)^{n + 1} &\equiv (a + b)(a + b)^n \\
            &\equiv (a + b)\sum_{r = 0}^n \binom nr a^r b^{n - r}
                \impliedby P(n) \\
            &\equiv a\sum_{r = 0}^n \binom nr a^r b^{n - r}
                + b\sum_{r = 0}^n \binom nr a^r b^{n - r} \\
            &\equiv \sum_{r = 0}^n \binom nr a^{r + 1} b^{n - r}
                + \sum_{r = 0}^n \binom nr a^r b^{n - r + 1} \\
            &\equiv \sum_{r = 1}^{n + 1} \binom n{r - 1} a^r b^{n - r + 1}
                + \sum_{r = 0}^n \binom nr a^r b^{n - r + 1} \\
            &\equiv a^{n + 1} + b^{n + 1}
                + \sum_{r = 1}^n
                      \bracks[\bigg]{\binom n{r - 1} + \binom nr}
                    a^r b^{n - r + 1} \\
            &\equiv a^{n + 1} + b^{n + 1}
                + \sum_{r = 1}^n \binom{n + 1}r a^r b^{n - r + 1}
                    \quad \text{due to Lemma \ref{lem_bin_coef_sum}} \\
            &\equiv \sum_{r = 0}^{n + 1}
                  \binom{n + 1}r a^r b^{(n + 1) - r}
        \end{align*}
        So \(P(n + 1)\) is true if \(P(n)\) is true.
  \item By the principle of mathematical induction
        \ref{basec_thm_binomial} and \ref{induct_thm_binomial} imply
        that \(P(n)\) is true for \(\Forall n \in \Integers^+\). \qedhere
 \end{enumerate}
\end{proof}

\begin{lemma}[Sum of adjacent binomial coefficients] \label{lem_bin_coef_sum}
 \begin{equation*}
  \binom n{r - 1} + \binom nr \equiv \binom{n + 1} r
 \end{equation*}
\end{lemma}
\begin{proof}
 \begin{align*}
  \binom n{r - 1} + \binom nr &\equiv \frac{n!}{(r - 1)! \cdot (n - r + 1)!}
                                   + \frac{n!}{r! \cdot (n - r)!}
                                      \quad \text{by definition} \\
      &\equiv \frac{n!\cdot(r + (n - r + 1))}{r! \cdot (n - r + 1)!} \\
      &\equiv \frac{n! \cdot (n + 1)}{r! \cdot ((n + 1) - r)!} \\
      &\equiv \frac{(n + 1)!}{r! \cdot ((n + 1) - r)!} \\
      &\equiv \binom{n + 1} r \quad \text{by definition} \qedhere
 \end{align*}
\end{proof}

\subsection{Arithmetic Progressions} \label{sec_seq_AP}

An arithmetic progression is a sequence \(U_i\) where each progressive term is
given by the previous term plus some constant \emph{common difference}, denotes
\(d\). The first term is denoted \(a\), such that
\begin{gather*}
 U_1 = a, \qquad U_2 = a + d, \qquad U_3 = a + 2d, \qquad \dots\\
 U_n = a + \underbrace{d  + d  + \dotsb + d}_{\text{\(n - 1\) times}}
     = a + (n - 1)d
\end{gather*}
We can deduce that the sum of the first \(n\) terms of some AP is given by
\begin{align*}
 S_n &= \sum_{k = 1}^n U_nj = \sum_{k = 1}^n (a + (k - 1)d) \\
     &= an + d\sum_{k = 1}^n (k - 1) \\
     &= an + \frac{d(n - 1)n} 2 \\
     &= \tfrac 12 n (2a + (n - 1)d)
\end{align*}
An alternative method to deduce this is to see that
\begin{align*}
 U_k + U_{n + 1 - k} &= a + (k - 1)d + a + (n - k)d \\
     &= 2a + (n - 1)d
\end{align*}
and that therefore, grouping the sum,
\begin{align*}
 S_n &= (U_1 + U_n) + (U_2 + U_{n - 1}) + \dotsb
      + (U_{n / 2} + U_{(n / 2) + 1}) \\
     &= \tfrac 12 n(2a + (n - 1)d) \quad \text{if n is even} \\
 S_n &= (U_1 + U_n) + (U_2 + U_{n - 1}) + \dotsb
      + (U_{(n - 1) / 2} + U_{(n + 3) / 2}) + U_{(n + 1) / 2} \\
     &= \tfrac 12 (n - 1)(2a + (n - 1)d) + a + \tfrac 12 (n - 1) d \\
     &= \tfrac 12 n(2a + (n - 1)d) \quad \text{if n is odd}
\end{align*}
This feels a little more tedious to do fully to me, though, although it may
prove easier to remember.


\subsection{Geometric Progressions} \label{sec_seq_GP}

A geometric progression is a sequence \(U_i\) where each progressive term is
given by the previous term multiplied by some constant \emph{common ratio},
denoted \(r\). The first term is denoted \(a\), such that
\begin{gather*}
 U_1 = a, \qquad U_2 = ar, \qquad U_3 = ar^2, \qquad \dots\\
 U_n = a\underbrace{r \cdot r \cdot \dotsb \cdot r}_{\text{\(n - 1\) times}}
     = ar^{n - 1}
\end{gather*}

\subsection{Fibonacci Sequences}

% TODO: why is this sometimes indeterminate
\begin{theorem}[Fibonacci \(n\)th term]
 Where some sequence \((U_i)\) has given its two starting values
 \(U_1, U_2 \in \Complex\), and obeys the recurrence
 \begin{equation*}
  U_{n + 2} = \alpha U_n + \beta U_{n + 1}
 \end{equation*}
 for some constant \(\alpha, \beta \in \Complex\), the \(n\)th term is given by
 \begin{equation*}
  \frac 1{\lambda_1 - \lambda_2}(\lambda_1^{n - 1} (U_2 - U_1 \lambda_2)
                               - \lambda_2^{n - 1} (U_2 - U_1 \lambda_1))
 \end{equation*}
 where \(\lambda_1, \lambda_2\) are explicitly
 \begin{align*}
  \lambda_1 &= \frac {\beta + \sqrt{\beta^2 + 4\alpha}}2 \\
  \lambda_1 &= \frac {\beta - \sqrt{\beta^2 + 4\alpha}}2
 \end{align*}
 Quadratic equation fans may realise that they could be characterised as the two
 solutions of \(\lambda^2 - \beta \lambda - \alpha = 0\), or equivalently as the
 two unique numbers in \(\Complex\) satisfying \(\lambda_1 \lambda_2 = -\alpha\)
 and \(\lambda_1 + \lambda_2 = \beta\).
\end{theorem}

\begin{proof}
 Consider the power series
 \begin{equation*}
  f(x) \defeq \sum_{k=1}^\infty U_k x^k
            = U_1 x + U_2 x^2 + (\alpha U_1 + \beta U_2) x^3
            + ((\alpha + \beta^2)U_2 + \alpha \beta U_1)3x^4 + \dotsb
 \end{equation*}
 Note that this series has a radius of convergence, by Lemma
 \ref{lem_fibo_convergence}.

 Consider now:
 \begin{alignat*}7
  f(x) &= \sum_{k = 1}^\infty U_k x^k
   &&={}& U_1 x &+{}& U_2 x^2 &+{}& (\alpha U_1 + \beta U_2) x^3
       &+{}& ((\alpha + \beta^2)U_2 + \alpha \beta U_1)3x^4
       &+{}& \dotsb \\
  x f(x) &= \sum_{k = 1}^\infty U_k x^k
   &&={}& && U_1 x^2 &+{}& U_2 x^3 &+{}& (\alpha U_1 + \beta U_2) x^4
       &+{}& \dotsb \\
  x^2 f(x) &= \sum_{k = 1}^\infty U_k x^k
   &&={}& &&&& U_1 x^2 &+{}& U_2 x^3 &+{}& \dotsb
 \end{alignat*}
 Now we can manipulate these a little, to get a nice closed form for \(f(x)\).
 \begin{alignat*}2
  && (1 - \beta x - \alpha x^2) f(x) &= (U_2 - \beta U_1)x^2 + U_1 x \\
  \implies{}&& f(x) &= \frac{(U_2 - \beta U_1)x^2 + U_1 x}
                            {1 - \beta x - \alpha x^2}
 \end{alignat*}
 Now we perform a partial fraction decomposition on \(f(x)\). We do this in a
 somewhat tricky way in order to make our lives easier. Note that
 \begin{equation*}
  1 - \beta x - \alpha x^2 = x^2 \parens[\Big]
        {\frac 1x^2 - \frac \beta x - \alpha}
      = x^2\parens[\Big]{\frac 1x - \lambda_1}
           \parens[\Big]{\frac 1x - \lambda_2}
      = (1 - \lambda_1 x)(1 - \lambda_2 x)
 \end{equation*}
 Having a denominator with constant term 1 is particularly useful for Maclaurin
 series expansion, which will come in later. Now,
 \begin{alignat*}2
  && f(x) &= \frac{(U_2 - \beta U_1)x^2 + U_1 x}
                 {(1 - \lambda_1 x)(1 - \lambda_2 x)} \\
  && &\equiv - \frac{U_2 - \beta U_1}\alpha +
                \frac A{1 - \lambda_1 x} + \frac B{1 - \lambda_2 x} \\
  \implies{}&& (U_2 - \beta U_1)x^2 + U_1 x &\equiv
               -\frac{(U_2 - \beta U_1)x^2 + U_1 x}\alpha
                     (1 - \lambda_1 x)(1 - \lambda_2 x) \\
  && &\phantom{\equiv} + A(1 - \lambda_2 x) + B(1 - \lambda_1 x) \\
  (x = \lambda_1^{-1}) \implies{}&&
   A\parens[\Big]{1 - \frac{\lambda_2}{\lambda_1}}
   &= \frac{U_2 - \beta U_1 + U_1 \lambda_1}{\lambda_1^2} \\
  \implies{}&& A &= \frac{U_2 - \beta U_1 + U_1 \lambda_1}
                         {\lambda_1(\lambda_1 - \lambda_2)} \\
  (x = \lambda_2^{-1}) \implies{}&&
   B\parens[\Big]{1 - \frac{\lambda_1}{\lambda_2}}
   &= \frac{U_2 - \beta U_1 + U_1 \lambda_2}{\lambda_2^2} \\
  \implies{}&& B &= \frac{U_2 - \beta U_1 + U_1 \lambda_2}
                         {\lambda_2(\lambda_2 - \lambda_1)}
 \end{alignat*}
 Now we can use a binomial series expansion to obtain an equivalent series.
 \begin{align*}
  f(x) &= -\frac{U_2 - \beta U_1}\alpha +
           \frac{U_2 - \beta U_1 + U_1 \lambda_1}
                {\lambda_1(\lambda_1 - \lambda_2)(1 - \lambda_1 x)} +
           \frac{U_2 - \beta U_1 + U_1 \lambda_2}
                {\lambda_2(\lambda_2 - \lambda_1)(1 - \lambda_2 x)} \\
       &= -\frac{U_2 - \beta U_1}\alpha +
           \frac{U_2 - \beta U_1 + U_1 \lambda_1}
                {\lambda_1(\lambda_1 - \lambda_2)}
                \parens[\Big]{\sum_{r = 0}^\infty \lambda_1^r x^r} +
           \frac{U_2 - \beta U_1 + U_1 \lambda_2}
                {\lambda_2(\lambda_2 - \lambda_1)}
                \parens[\Big]{\sum_{r = 0}^\infty \lambda_2^r x^r} \\
       &= -\frac{U_2 - \beta U_1}\alpha +
          \sum_{r = 0}^\infty \bracks[\Bigg]{
           \frac{U_2 + U_1(\lambda_1 - \beta)}
                {\lambda_1(\lambda_1 - \lambda_2)}
            (\lambda_1^r x^r) +
           \frac{U_2 + U_1(\lambda_2 - \beta)}
                {\lambda_2(\lambda_2 - \lambda_1)}
            (\lambda_2^r x^r)
          } \\
       &= -\frac{U_2 - \beta U_1}\alpha +
          \sum_{r = 0}^\infty \bracks[\Bigg]{
           \frac{U_2 - U_1 \lambda_1}
                {\lambda_1(\lambda_1 - \lambda_2)}
            (\lambda_1^r x^r) +
           \frac{U_2 - U_1 \lambda_2}
                {\lambda_2(\lambda_2 - \lambda_1)}
            (\lambda_2^r x^r)
          } \\
       &= -\frac{U_2 - \beta U_1}\alpha +
          \frac 1{\lambda_1 - \lambda_2} \sum_{r = 0}^\infty \bracks[\Bigg]{
           (U_2 - U_1 \lambda_1) \lambda_1^{r - 1} x^r -
           (U_2 - U_1 \lambda_2) \lambda_2^{r - 1} x^r
          }
 \end{align*}
 Now, recalling the definition of \(f(x)\) as
 \begin{equation*}
  f(x) \defeq \sum_{k=1}^\infty U_k x^k
            = U_1 x + U_2 x^2 + (\alpha U_1 + \beta U_2) x^3
            + ((\alpha + \beta^2)U_2 + \alpha \beta U_1)3x^4 + \dotsb
 \end{equation*}
 and the well-definedness of \(f(x)\) we can equate the coefficients to get our
 result:
 \begin{equation*}
  U_n = \frac 1{\lambda_1 - \lambda_2}(\lambda_1^{n - 1}(U_2 - U_1\lambda_2)
                                     - \lambda_2^{n - 1}(U_2 - U_1\lambda_2))
                                    \qedhere
 \end{equation*}
\end{proof}
\begin{proof}[Proof by matrices]
 Noting that the given recurrence \(U_n = U_{n - 1} + U_{n - 2}\) is a linear
 transition, we can define the vector
 \begin{equation*}
  \vec u_n =
   \begin{pmatrix*}[l]
    U_n \\
    U_{n + 1}
   \end{pmatrix*}
 \end{equation*}
 and take advantage of the linear transition to write
 \begin{equation*}
  \vec u_{n + 1} = \mat T \vec u_n \quad \text{where} \quad
  \mat T =
  \begin{pmatrix}
   0 & 1 \\
   \alpha & \beta
  \end{pmatrix}
 \end{equation*}
 so that in fact,
 \begin{equation*}
  \vec u_n = \mat T^{n - 1}\vec u_1
 \end{equation*}
 Now we proceed to diagonalise \(\mat T\) by first solving its
 characteristic equation for its eigenvalues.
 \begin{alignat*}4
  && \abs{\mat T - \lambda \mat I} &= 0 \\
  \iff{}&&
   \begin{vmatrix}
    -\lambda & 1 \\
    \alpha & \beta - \lambda
   \end{vmatrix} &= 0
  \iff{}&& \lambda^2 - \beta \lambda - \alpha &= 0 \\
  \iff{}&& \lambda \in \set{\lambda_1, \lambda_2}
 \end{alignat*}
 which we defined as such earlier.

 \begin{alignat*}5
  & \lambda_1 \colon&
  &&
   \begin{pmatrix}
    -\lambda_1 & 1 \\
    \alpha & \beta - \lambda_1
   \end{pmatrix} \vec e_1 &= \vec 0 \\
  && \iff{}&&
   \begin{pmatrix}
    -\lambda_1 & 1 \\
    -\lambda_1 \lambda_2 & \lambda_2
   \end{pmatrix} \vec e_1 &= \vec 0 \impliedby \vec e_1 &{}=
   \begin{pmatrix}
    1 \\
    \lambda_1
   \end{pmatrix} \\
  & \lambda_2 \colon&
  &&
   \begin{pmatrix}
    -\lambda_2 & 1 \\
    \alpha & \beta - \lambda_2
   \end{pmatrix} \vec e_2 &= \vec 0 \\
  && \iff{}&&
   \begin{pmatrix}
    -\lambda_2 & 1 \\
    -\lambda_1 \lambda_2 & \lambda_1
   \end{pmatrix} \vec e_2 &= \vec 0 \impliedby \vec e_2 &{}=
   \begin{pmatrix}
    1 \\
    \lambda_2
   \end{pmatrix}
 \end{alignat*}
 So that
 \begin{align*}
  \mat T^n &=
  \parens[\bigg]{
   \begin{array}{c|c}
    \vec e_1 & \vec e_2
   \end{array}
  }
  \begin{pmatrix}
   \lambda_1^n & 0 \\
   0 & \lambda_2^n
  \end{pmatrix}
  \parens[\bigg]{
   \begin{array}{c|c}
    \vec e_1 & \vec e_2
   \end{array}
  }^{-1} \\
  &=
  \begin{pmatrix}
   1 & 1 \\
   \lambda_1 & \lambda_2
  \end{pmatrix}
  \begin{pmatrix}
   \lambda_1^n & 0 \\
   0 & \lambda_2^n
  \end{pmatrix}
  \begin{pmatrix}
   1 & 1 \\
   \lambda_1 & \lambda_2
  \end{pmatrix}^{-1} \\
  &=
  \begin{pmatrix}
   1 & 1 \\
   \lambda_1 & \lambda_2
  \end{pmatrix}
  \begin{pmatrix}
   \lambda_1^n & 0 \\
   0 & \lambda_2^n
  \end{pmatrix}
  \frac 1{\lambda_2 - \lambda_1}
  \begin{pmatrix}
   \lambda_2 & -1 \\
   -\lambda_1 & 1
  \end{pmatrix} \\
  &=
  \begin{pmatrix}
   1 & 1 \\
   \lambda_1 & \lambda_2
  \end{pmatrix}
  \begin{pmatrix}
   \lambda_1^n & 0 \\
   0 & \lambda_2^n
  \end{pmatrix}
  \frac 1{\lambda_2 - \lambda_1}
  \begin{pmatrix}
   \lambda_2 & -1 \\
   -\lambda_1 & 1
  \end{pmatrix} \\
  &=
  -\frac 1{\lambda_1 - \lambda_2}
  \begin{pmatrix}
   \lambda_1^n & \lambda_2^n \\
   \lambda_1^{n + 1} & \lambda_2^{n + 1} \\
  \end{pmatrix}
  \begin{pmatrix}
   \lambda_2 & -1 \\
   -\lambda_1 & 1
  \end{pmatrix} \\
  &=
  -\frac 1{\lambda_1 - \lambda_2}
  \begin{pmatrix}
   \lambda_2 \lambda_1^n -\lambda_1 \lambda_2^n & \lambda_2^n - \lambda_1^n \\
   \lambda_2 \lambda_1^{n + 1} - \lambda_1 \lambda_2^{n + 1} &
       \lambda_2^{n + 1} - \lambda_1^{n + 1}
  \end{pmatrix}
 \end{align*}
 and
 \begin{alignat*}4
  &&\vec u_n &= \mat T^{n - 1} \vec u_1 \\
  \iff{}&&
   \begin{pmatrix}
    U_n \\
    U_{n + 1}
   \end{pmatrix} &=
   -\frac 1{\lambda_1 - \lambda_2}
   \begin{pmatrix}
    \lambda_2 \lambda_1^{n - 1} -\lambda_1 \lambda_2^{n - 1}n
     & \lambda_2^{n - 1} - \lambda_1^{n - 1} \\
    \lambda_2 \lambda_1^n - \lambda_1 \lambda_2^n & \lambda_2^n - \lambda_1^n
   \end{pmatrix}
   \begin{pmatrix}
    U_1 \\
    U_2
   \end{pmatrix} \\
  && &=
  -\frac 1{\lambda_1 - \lambda_2}
   \begin{pmatrix}
    U_1(\lambda_2 \lambda_1^{n - 1} - \lambda_1 \lambda_2^{n - 1})
  + U_2(\lambda_2^{n - 1} - \lambda_1^{n - 1}) \\
    U_1(\lambda_2 \lambda_1^n - \lambda_1 \lambda_2^n)
  + U_2(\lambda_2^n - \lambda_1^n) \\
   \end{pmatrix} \\
   \implies{}&& U_n &=
    -\frac 1{\lambda_1 - \lambda_2} (
       U_1(\lambda_2 \lambda_1^{n - 1} - \lambda_1 \lambda_2^{n - 1})
     + U_2(\lambda_2^{n - 1} - \lambda_1^{n - 1})) \\
   && &= 
    \frac 1{\lambda_1 - \lambda_2} (
       \lambda_1^{n - 1} (U_2 - U_1 \lambda_2)
     - \lambda_2^{n - 1} (U_2 - U_1 \lambda_1)) \qedhere
 \end{alignat*}
\end{proof}
\begin{proof}[Proof by Induction]
 This can also be proven by induction. This proof is indirect, so it provides
 less insight, and is generally not as useful. However it remains an
 interesting excercise.

 Let \(P(n)\) denote the theorem for \(U_n\).
 \begin{enumerate}[label=\Roman*.]
  \item \label{basec_thm_fibo} We verify the cases \(n = 1\) and \(n = 2\):
        \begin{align*}
         n = 1&\colon \frac 1{\lambda_1 - \lambda_2}
                 ((U_2 - U_1 \lambda_2) - (U_2 - U_1 \lambda_1))
                = \frac {U_1(\lambda_1 - \lambda_2)}{\lambda_1 - \lambda_2}
                = U_1 \\
         n = 2&\colon \frac 1{\lambda_1 - \lambda_2}
                 (\lambda_1(U_2 - U_1 \lambda_2)
                - \lambda_2(U_2 - U_1 \lambda_1))
                = \frac {U_2(\lambda_1 - \lambda_2)}{\lambda_1 - \lambda_2}
                = U_2
        \end{align*}
  \item \label{induct_thm_fibo} Now we suppose that \(P(n)\) and \(P(n + 1)\)
        are true, and consider \(P(n + 2)\):
        \begin{alignat*}2
         U_{n + 2} &= \mathrlap{\alpha U_n + \beta U_{n + 1}
                      \quad \text{by definition}} \\
                   &= \frac 1{\lambda_1 - \lambda_2}(
                     &&\alpha\lambda_1^{n - 1}(U_2 - U_1 \lambda_2)
                     - \alpha\lambda_2^{n - 1}(U_2 - U_1 \lambda_1) \\
              &&{}+{}& \beta\lambda_1^n(U_2 - U_1 \lambda_2)
                     - \beta\lambda_2^n(U_2 - U_1 \lambda_1))
                       \impliedby P_n, P_{n + 1} \\
                   &= \frac 1{\lambda_1 - \lambda_2}(
                     &&\lambda_1^{n - 1}(
                        \alpha U_2 - \alpha U_1 \lambda_2 +
                        \beta U_2 \lambda_1 - \beta U_1 \lambda_1 \lambda_2) \\
              &&{}-{}& \lambda_2^{n - 1}(
                        \alpha U_2 - \alpha U_1 \lambda_1 +
                        \beta U_2 \lambda_2 - \beta U_1 \lambda_2 \lambda_1)
                      ) \\
                   &= \frac 1{\lambda_1 - \lambda_2}(
                     &&\lambda_1^{n - 1}(
                        \lambda_1 \lambda_2 U_2 -
                        \lambda_1 \lambda_2^2 U_1 +
                        (\lambda_1 + \lambda_2) U_2 \lambda_1 -
                        (\lambda_1 + \lambda_2) U_1 \lambda_1 \lambda_2) \\
              &&{}-{}& \lambda_2^{n - 1}(
                        \lambda_1 \lambda_2 U_2 -
                        \lambda_1^2 \lambda_2 U_1 +
                        (\lambda_1 + \lambda_2) U_2 \lambda_2 -
                        (\lambda_1 + \lambda_2) U_1 \lambda_2 \lambda_1)
                      ) \\
                   &= \mathrlap{
                       \frac 1{\lambda_1 - \lambda_2}(
                        \lambda_1^{n - 1}(
                         U_2 \lambda_1^2 - U_1 \lambda_1^2 \lambda_2)
                      - \lambda_2^{n - 1}(
                         U_2 \lambda_2^2 - U_1 \lambda_1 \lambda_2^2)
                      )} \\
                   &= \mathrlap{
                       \frac 1{\lambda_1 - \lambda_2}(
                        \lambda_1^{n + 1}(
                         U_2 - U_1 \lambda_2)
                      - \lambda_2^{n + 1}(
                         U_2 - U_1 \lambda_1)
                      )}
        \end{alignat*}
  \item Now, by the principle of mathematical induction, \ref{basec_thm_fibo}
        and \ref{induct_thm_fibo} imply that \(P(n)\) must be true for all \(n
        \in \Integers^+\). \qedhere
 \end{enumerate}
\end{proof}
\begin{lemma}[Fibonacci power series convergence]
\label{lem_fibo_convergence}
 Where \(U_1, U_2, \alpha, \beta \in \Complex\), the power series
 \begin{equation*}
  f(x) \defeq \sum_{k=1}^\infty U_k x^k
            = U_1 x + U_2 x^2 + (\alpha U_1 + \beta U_2) x^3
            + ((\alpha + \beta^2)U_2 + \alpha \beta U_1)3x^4 + \dotsb
 \end{equation*}
 has a radius of convergence and can therefore be justifiably manipulated.
\end{lemma}
% TODO
\begin{proof}
 Hang tight!
\end{proof}
% \begin{proof}
%  First we define
%  \begin{equation*}
%   g(x) \defeq \sum_{k=1}^\infty 2^k x^k
%  \end{equation*}
%  Now we aim to show that \(f(x) < g(x)\) for \(x > 0\). This is because
%  \(U_n < 2^n\). This can be proven by induction. Let \(P(n)\) denote
%  ``\(U_n < 2^n\)'' for all \(n \in \Integers^+\).
%  \begin{enumerate}[label=\Roman*.]
%   \item \label{basec_lem_fibo}
%         Note that \(U_1 = 1 < 2^1 = 2\) and \(U_2 = 1 < 2^2 = 4\).
%   \item \label{induct_lem_fibo}
%         Suppose \(P(n)\) and \(P(n + 1)\) are true. Consider \(P(n + 2)\):
%         \begin{equation*}
%         U_{n + 2} = U_{n + 1} + U_n < 2^{n + 1} + 2^n < 2^{n + 1} + 2^{n + 1}
%             = 2^{n + 2}
%         \end{equation*}
%   \item By the principle of mathematical induction, \ref{basec_lem_fibo} and
%         \ref{induct_lem_fibo} imply that \(P(n)\) is true for all
%         \(n \in \Integers^+\).
%  \end{enumerate}
%  Then for \(\abs x < \frac 12\), \(f(x)\)  must be convergent, as \(g(x)\) is
%  convergent for \(\abs x < \frac 12\).
% \end{proof}
Combined with Exponentation by Squaring (\ref{sec_exp_by_squaring}), and some
simple surd arithmetic, this theorem provides a fairly fast way to calculate
\(U_n\), as compared to utterly na\"ive recursion, or somewhat faster iteration
or memoized/linearised recursion. However it is often in fact faster to
simply use the matrix form given in the second proof, to calculate
\(\vec u_n\) as \(\mat T^{n - 1}\vec u_1\), again using exponentiation by
squaring. This usually requires significantly fewer arithmetic operations, as if
you're doing exact computation you don't need to worry about surds.

A straightforward corollary is to give the \(n\)th term of the ``real''
Fibonacci sequence:
\begin{corollary}[Real Fibonacci \(n\)th term]
 Where the Fibonacci sequence \((F_n)\) is defined by \(F_1 = F_2 = 1\) and
 \(F_{n + 2} = F_n + F_{n + 1}\),
 \begin{equation*}
  F_n = \frac{\varphi^n - (-\varphi)^{-n}}{\varphi + \varphi^{-1}}
      = \frac{\varphi^n - (-\varphi)^{-n}}{\sqrt 5}
 \end{equation*}
 where
 \begin{equation*}
  \varphi = \frac{1 + \sqrt 5}2
 \end{equation*}
\end{corollary}
This can be derived by simply substituting \(U_1 = U_2 = \alpha = \beta = 1\)
into the previous result, or by specifying any of the proofs to these
parameters. Generally the proof then becomes easier to follow.

\subsection{Taylor Series}

The Maclaurin series is the Taylor series around \(0\).
\begin{equation*}
f(x) = f(0) + f'(0) x + \frac{f''(0)} 2 x^2 + \frac{f'''(0)}6 x^3 +\dotsb
  = \sum_{k=0}^\infty \frac{f^{(k)}(0)}{k!}x^k
\end{equation*}

\subsection{Common series}
