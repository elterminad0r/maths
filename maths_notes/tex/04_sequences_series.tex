\section{Sums, Sequences, Series}

\subsection{Binomial Theorem}

\begin{theorem}[Binomial theorem] \label{thm_binomial_thm}
 Where \(n \in \Naturals\),
 \begin{equation*}
  (a + b)^n \equiv \sum_{r = 0}^n \binom nr a^r b^{n - r}
 \end{equation*}
 where the binomial coefficients are given by
 \begin{equation*}
  \binom nr = \nCr nr \defeq \frac{n!}{r!\cdot(n - r)!}
 \end{equation*}
\end{theorem}
\begin{proof}
 We can prove that this is the case by induction. Let \(P(n)\) denote the
 binomial theorem for exponent \(n, \Forall n \in \Integers^+\).
 \begin{enumerate}[I.]
  \item \label{basec_thm_binomial} Consider \(P(1)\):
        \begin{equation*}
        (a + b)^1 \equiv a + b
            \equiv \frac{0!}{0!\cdot 1!} a^0b^1 +
                \frac{1!}{1!\cdot 0!} a^1b^0
        \end{equation*}
        so \(P(1)\) is true.
  \item \label{induct_thm_binomial} We now consider \(P(n + 1)\), supposing
        that \(P(n)\) is true.
        \begin{align*}
        (a + b)^{n + 1} &\equiv (a + b)(a + b)^n \\
            &\equiv (a + b)\sum_{r = 0}^n \binom nr a^r b^{n - r}
                \impliedby P(n) \\
            &\equiv a\sum_{r = 0}^n \binom nr a^r b^{n - r}
                + b\sum_{r = 0}^n \binom nr a^r b^{n - r} \\
            &\equiv \sum_{r = 0}^n \binom nr a^{r + 1} b^{n - r}
                + \sum_{r = 0}^n \binom nr a^r b^{n - r + 1} \\
            &\equiv \sum_{r = 1}^{n + 1} \binom n{r - 1} a^r b^{n - r + 1}
                + \sum_{r = 0}^n \binom nr a^r b^{n - r + 1} \\
            &\equiv a^{n + 1} + b^{n + 1}
                + \sum_{r = 1}^n
                      \bracks[\bigg]{\binom n{r - 1} + \binom nr}
                    a^r b^{n - r + 1} \\
            &\equiv a^{n + 1} + b^{n + 1}
                + \sum_{r = 1}^n \binom{n + 1}r a^r b^{n - r + 1}
                    \quad \text{due to Lemma \ref{lem_bin_coef_sum}} \\
            &\equiv \sum_{r = 0}^{n + 1}
                  \binom{n + 1}r a^r b^{(n + 1) - r}
        \end{align*}
        So \(P(n + 1)\) is true if \(P(n)\) is true.
  \item By the principle of mathematical induction
        \ref{basec_thm_binomial} and \ref{induct_thm_binomial} imply
        that \(P(n)\) is true for \(\Forall n \in \Integers^+\). \qedhere
 \end{enumerate}
\end{proof}

\begin{lemma}[Sum of adjacent binomial coefficients] \label{lem_bin_coef_sum}
 \begin{equation*}
  \binom n{r - 1} + \binom nr \equiv \binom{n + 1} r
 \end{equation*}
\end{lemma}
\begin{proof}
 \begin{align*}
  \binom n{r - 1} + \binom nr &\equiv \frac{n!}{(r - 1)! \cdot (n - r + 1)!}
                                   + \frac{n!}{r! \cdot (n - r)!}
                                      \quad \text{by definition} \\
      &\equiv \frac{n!\cdot(r + (n - r + 1))}{r! \cdot (n - r + 1)!} \\
      &\equiv \frac{n! \cdot (n + 1)}{r! \cdot ((n + 1) - r)!} \\
      &\equiv \frac{(n + 1)!}{r! \cdot ((n + 1) - r)!} \\
      &\equiv \binom{n + 1} r \quad \text{by definition} \qedhere
 \end{align*}
\end{proof}

\subsection{Arithmetic Progressions} \label{sec_seq_AP}

An arithmetic progression is a sequence \(U_i\) where each progressive term is
given by the previous term plus some constant \emph{common difference}, denotes
\(d\). The first term is denoted \(a\), such that
\begin{gather*}
 U_1 = a, \qquad U_2 = a + d, \qquad U_3 = a + 2d, \qquad \dots\\
 U_n = a + \underbrace{d  + d  + \dotsb + d}_{\text{\(n - 1\) times}}
     = a + (n - 1)d
\end{gather*}
We can deduce that the sum of the first \(n\) terms of some AP is given by
\begin{align*}
 S_n &= \sum_{k = 1}^n U_nj = \sum_{k = 1}^n (a + (k - 1)d) \\
     &= an + d\sum_{k = 1}^n (k - 1) \\
     &= an + \frac{d(n - 1)n} 2 \\
     &= \tfrac 12 n (2a + (n - 1)d)
\end{align*}
An alternative method to deduce this is to see that
\begin{align*}
 U_k + U_{n + 1 - k} &= a + (k - 1)d + a + (n - k)d \\
     &= 2a + (n - 1)d
\end{align*}
and that therefore, grouping the sum,
\begin{align*}
 S_n &= (U_1 + U_n) + (U_2 + U_{n - 1}) + \dotsb
      + (U_{n / 2} + U_{(n / 2) + 1}) \\
     &= \tfrac 12 n(2a + (n - 1)d) \quad \text{if n is even} \\
 S_n &= (U_1 + U_n) + (U_2 + U_{n - 1}) + \dotsb
      + (U_{(n - 1) / 2} + U_{(n + 3) / 2}) + U_{(n + 1) / 2} \\
     &= \tfrac 12 (n - 1)(2a + (n - 1)d) + a + \tfrac 12 (n - 1) d \\
     &= \tfrac 12 n(2a + (n - 1)d) \quad \text{if n is odd}
\end{align*}
This feels a little more tedious to do fully to me, though, although it may
prove easier to remember.


\subsection{Geometric Progressions} \label{sec_seq_GP}

A geometric progression is a sequence \(U_i\) where each progressive term is
given by the previous term multiplied by some constant \emph{common ratio},
denoted \(r\). The first term is denoted \(a\), such that
\begin{gather*}
 U_1 = a, \qquad U_2 = ar, \qquad U_3 = ar^2, \qquad \dots\\
 U_n = a\underbrace{r \cdot r \cdot \dotsb \cdot r}_{\text{\(n - 1\) times}}
     = ar^{n - 1}
\end{gather*}

\subsection{Fibonacci Sequence}

\begin{theorem}[Fibonacci \(n\)th term]
 The Fibonacci numbers \(F_n: n \in \Naturals\) are such that
 \(F_1 = F_2 = 1\) and \(F_n = F_{n - 1} + F_{n - 2}\).  \(F_n\) can be given
 by the closed form
 \begin{equation*}
  F_n = \frac{\varphi^n - \psi^n}{\sqrt 5}
  \quad \text{where}\quad \varphi = \frac{1 + \sqrt 5} 2
  \quad \text{and}\quad \psi = \frac{1 - \sqrt 5} 2
 \end{equation*}
\end{theorem}
\begin{proof}
 Consider the power series
 \begin{equation*}
  f(x) \defeq \sum_{k=1}^\infty F_k x^k = x + x^2 + 2x^3 + 3x^4 + 5x^5
      + \dotsb
 \end{equation*}
 Note that this series has a radius of convergence, by Lemma
 \ref{lem_fibo_convergence}.

 Consider now:
 \begin{alignat*}9
  && f(x) &= \sum_{k = 1}^\infty F_k x^k
      &&={}& x &+{}& x^2 &+{}& 2x^3 &+{}&
          3x^4 &+{}& 5x^5 &+{}& 8x^6 + \dotsb \\
  && x f(x) &= \sum_{k = 2}^\infty F_{k - 1} x^k
      &&={}& &&x^2 &+{}& x^3 &+{}& 2x^4 &+{}&
          3x^5 &+{}& 5x^6 + \dotsb \\
  && x^2 f(x) &= \sum_{k = 3}^\infty F_{k - 2}x^k
      &&={}& &&&& x^3 &+{}& x^4 &+{}& 2x^5 &+{}& 3x^6 + \dotsb \\
  &\implies{}& (1 - x - x^2) f(x) &= x \\
  &\implies{}& f(x) &= \frac x{1 - x - x^2}
 \end{alignat*}
 Now we perform a partial fraction decomposition on \(f(x)\). We do this in a
 somewhat tricky way in order to make our lives easier. Note that
 \begin{equation*}
  1 - x - x^2 = x^2 \parens[\Big]{\frac 1x^2 - \frac 1x - 1}
      = x^2\parens[\Big]{\frac 1x - \varphi}\parens[\Big]{\frac 1x - \psi}
      = (1 - \varphi x)(1 - \psi x)
 \end{equation*}
 where \(\varphi, \psi = \frac 12(1 \pm \sqrt 5)\), ie the roots of
 \(x^2 - x - 1\), ie the roots of the golden ratio. Now,
 \begin{alignat*}2
  && f(x) = \frac x{(1 - \varphi x)(1 - \psi x)}
      &\equiv \frac A{1 - \varphi x} + \frac B{1 - \psi x} \\
  &\implies{}& A(1 - \psi x) + B(1 - \varphi x) &\equiv x \\
  &\implies{}& A + B &= 0 \\
  && -\psi A - \varphi B &= 1 \\
  &\implies{}& B &= -A  \\
  && A(\psi - \varphi) &= 1 \\
  &\implies{}& A(\sqrt 5) &= 1 \\
  &\implies{}& A &= \frac 1{\sqrt 5}, B = -\frac 1{\sqrt 5}
 \end{alignat*}
 Now we can use a binomial series expansion to obtain an equivalent series.
 \begin{align*}
  f(x) &= \frac 1{\sqrt 5}\parens[\Big]{\frac 1{1 - \varphi x}
                              - \frac 1{1 - \psi x}} \\
  &= \frac 1{\sqrt 5}\parens[\Big]{\sum_{k = 0}^\infty (\varphi x)^k
                      - \sum_{k = 0}^\infty (\psi x)^k} \\
  &= \sum_{k = 0}^\infty \frac 1{\sqrt 5}(\varphi^k - \psi^k)x^k
  = \sum_{k = 1}^\infty F_k x^k
 \end{align*}
 So we have \(F_n = \frac 1{\sqrt 5}(\varphi^n - \psi^n)\). Note that the
 constant term is indeed zero, as \(\varphi^0 - \psi^0 = 0\).
\end{proof}
\begin{proof}[Proof by matrices]
 Noting that the given recurrence \(F_n = F_{n - 1} + F_{n - 2}\) is a linear
 transition (and a fairly simple one at that), we can define the vector
 \begin{equation*}
  \vec f_n =
  \begin{pmatrix*}[l]
   F_n \\
   F_{n + 1}
  \end{pmatrix*}
 \end{equation*}
 and take advantage of the linear transition to write
 \begin{equation*}
  \vec f_{n + 1} = \mat T \vec f_n \quad \text{where} \quad
  \mat T =
  \begin{pmatrix}
   0 & 1 \\
   1 & 1
  \end{pmatrix}
 \end{equation*}
 so that in fact,
 \begin{equation*}
  \vec f_n = \mat T^n\vec f_0
 \end{equation*}
 where we define \(F_0 = 0\). This avoids messily subtracting one from
 things. Now we proceed to diagonalise \(\mat T\) by first solving its
 characteristic equation for its eigenvalues.
 \begin{alignat*}4
  && \abs{\mat T - \lambda \mat I} &= 0 \\
  &\iff{}&
   \begin{vmatrix}
    -\lambda & 1 \\
    1 & 1 - \lambda
   \end{vmatrix} &= 0
  &\iff{}& \lambda (\lambda - 1) - 1 &= 0 \\
  &\iff{}& \lambda^2 - \lambda - 1 &= 0 \\
  &\iff{}& \lambda &\in \set{\varphi, \psi}
 \end{alignat*}
 as we saw earlier.

 Now we can find the eigenvectors \footnote{
     Note that as \(\varphi\) satisfies \(\varphi^2 - \varphi - 1 = 0\),
     we can rearrange to get \(\varphi - 1 = \frac 1\varphi\).}
 \begin{alignat*}6
  \lambda_1 &= \varphi:&
  &&
   \begin{pmatrix}
    -\varphi & 1 \\
    1 & 1 - \varphi
   \end{pmatrix} \vec e_1 &= \vec 0 \\
  && &\iff{}&
   \begin{pmatrix}
    -\varphi & 1 \\
    1 & -\varphi^{-1}
   \end{pmatrix} \vec e_1 &= \vec 0 \impliedby \vec e_1 &{}=
   \begin{pmatrix}
    1 \\
    \varphi
   \end{pmatrix} \\
  \lambda_2 &= \psi:&
  &&
   \begin{pmatrix}
    -\psi & 1 \\
    1 & 1 - \psi
   \end{pmatrix} \vec e_2 &= \vec 0 \\
  && &\iff{}&
   \begin{pmatrix}
    -\psi & 1 \\
    1 & -\psi^{-1}
   \end{pmatrix} \vec e_2 &= \vec 0 \impliedby \vec e_2 &{}=
   \begin{pmatrix}
    1 \\
    \psi
   \end{pmatrix}
 \end{alignat*}
 So that
 \begin{align*}
  \mat T^n &=
  \parens[\bigg]{
   \begin{array}{c|c}
    \vec e_1 & \vec e_2
   \end{array}
  }
  \begin{pmatrix}
   \lambda_1^n & 0 \\
   0 & \lambda_2^n
  \end{pmatrix}
  \parens[\bigg]{
   \begin{array}{c|c}
    \vec e_1 & \vec e_2
   \end{array}
  }^{-1} \\
  &=
  \begin{pmatrix}
   1 & 1 \\
   \varphi & \psi
  \end{pmatrix}
  \begin{pmatrix}
   \varphi^n & 0 \\
   0 & \psi^n
  \end{pmatrix}
  \begin{pmatrix}
   1 & 1 \\
   \varphi & \psi
  \end{pmatrix}^{-1} \\
  &=
  \begin{pmatrix}
   1 & 1 \\
   \varphi & \psi
  \end{pmatrix}
  \begin{pmatrix}
   \varphi^n & 0 \\
   0 & \psi^n
  \end{pmatrix}
  \frac 1 {\psi - \varphi}
  \begin{pmatrix}
   \psi & -1 \\
   -\varphi & 1
  \end{pmatrix} \\
  &=
  \begin{pmatrix}
   1 & 1 \\
   \varphi & \psi
  \end{pmatrix}
  \begin{pmatrix}
   \varphi^n & 0 \\
   0 & \psi^n
  \end{pmatrix}
  \frac 1 {\psi - \varphi}
  \begin{pmatrix}
   \psi & -1 \\
   -\varphi & 1
  \end{pmatrix} \\
  &=
  -\frac 1 {\sqrt 5}
  \begin{pmatrix}
   \varphi^n & \psi^n \\
   \varphi^{n + 1} & \psi^{n + 1} \\
  \end{pmatrix}
  \begin{pmatrix}
   \psi & -1 \\
   -\varphi & 1
  \end{pmatrix} \\
  &=
  -\frac 1 {\sqrt 5}
  \begin{pmatrix}
   \psi \varphi^n -\varphi \psi^n & \psi^n - \varphi^n \\
   \psi \varphi^{n + 1} - \varphi \psi^{n + 1} &
       \psi^{n + 1} - \varphi^{n + 1}
  \end{pmatrix}
 \end{align*}
 and
 \begin{alignat*}4
  &&\vec f_n &= \mat T_n \vec f_0 \\
  &\iff{}&
   \begin{pmatrix}
    F_n \\
    F_{n + 1}
   \end{pmatrix} &=
   -\frac 1 {\sqrt 5}
   \begin{pmatrix}
    \psi \varphi^n -\varphi \psi^n & \psi^n - \varphi^n \\
    \psi \varphi^{n + 1} - \varphi \psi^{n + 1} &
        \psi^{n + 1} - \varphi^{n + 1}
   \end{pmatrix}
   \begin{pmatrix}
    0 \\
    1
   \end{pmatrix} \\
  && &=
  \frac 1 {\sqrt 5}
   \begin{pmatrix}
    \varphi^n - \psi^n \\
    \varphi^{n + 1} - \psi^{n + 1}
   \end{pmatrix} \\
   &\implies{}& F_n &= \frac{\varphi^n - \psi^n}{\sqrt 5} \qedhere
 \end{alignat*}
\end{proof}
\begin{proof}[Proof by Induction]
 This can also be proven by induction. This proof is indirect, so it provides
 less insight, and is generally not as useful. However it remains an
 interesting excercise.

 Let \(P(n)\) denote the theorem for \(F_n\).
 \begin{enumerate}[I.]
  \item \label{basec_thm_fibo} We verify the cases \(F_1\) and \(F_2\):
        \begin{align*}
            n = 1&: \frac{\varphi^1 - \psi^1}{\sqrt 5}
                = \frac{\sqrt 5}{\sqrt 5} = 1 = F_1 \\
            n = 2&: \frac{\varphi^2 - \psi^2}{\sqrt 5}
                = \frac{(\varphi + \psi)(\varphi - \psi)}{\sqrt 5}
                = \frac{(1)(\sqrt 5)}{\sqrt 5}= 1 = F_1
        \end{align*}
  \item \label{induct_thm_fibo} Now we suppose that \(P(n)\) and \(P(n + 1)\)
        are true, and consider \(P(n + 2)\):
        \begin{align*}
            F_{n + 2} &= F_n + F_{n + 1} \\
                      &= \frac{\varphi^n - \psi^n +
                               \varphi^{n + 1} - \psi^{n + 1}}
                              {\sqrt 5} \impliedby P(n), P(n + 1) \\
                      &= \frac{(\varphi + 1)\varphi^n - (\psi + 1)\psi^n}
                               {\sqrt 5} \\
                      &= \frac{\varphi^2\varphi^n - \psi^2\psi^n}{\sqrt 5} \\
                      &= \frac{\varphi^{n + 2} - \psi^{n + 1}}{\sqrt 5}
        \end{align*}
        This uses the fact that \(\varphi\) and \(\psi\) satisfy
        \(x^2 - x - 1 = 0\), which we rearrange to find \(x + 1 = x^2\).
  \item Now, by the principle of mathematical induction, \ref{basec_thm_fibo}
        and \ref{induct_thm_fibo} imply that \(P(n)\) must be true for all \(n
        \in \Integers^+\). \qedhere
 \end{enumerate}
\end{proof}
\begin{lemma}[Fibonacci power series convergence]
\label{lem_fibo_convergence}
 The power series
 \begin{equation*}
  f(x) \defeq \sum_{k=1}^\infty F_k x^k = x + x^2 + 2x^3 + 3x^4 + 5x^5
      + \dotsb
 \end{equation*}
 has a radius of convergence and can therefore be manipulated.
\end{lemma}
\begin{proof}
 First we define
 \begin{equation*}
  g(x) \defeq \sum_{k=1}^\infty 2^k x^k
 \end{equation*}
 Now we aim to show that \(f(x) < g(x)\) for \(x > 0\). This is because
 \(F_n < 2^n\). This can be proven by induction. Let \(P(n)\) denote
 ``\(F_n < 2^n\)'' for all \(n \in \Integers^+\).
 \begin{enumerate}[I.]
  \item \label{basec_lem_fibo}
        Note that \(F_1 = 1 < 2^1 = 2\) and \(F_2 = 1 < 2^2 = 4\).
  \item \label{induct_lem_fibo}
        Suppose \(P(n)\) and \(P(n + 1)\) are true. Consider \(P(n + 2)\):
        \begin{equation*}
        F_{n + 2} = F_{n + 1} + F_n < 2^{n + 1} + 2^n < 2^{n + 1} + 2^{n + 1}
            = 2^{n + 2}
        \end{equation*}
  \item By the principle of mathematical induction, \ref{basec_lem_fibo} and
        \ref{induct_lem_fibo} imply that \(P(n)\) is true for all
        \(n \in \Integers^+\).
 \end{enumerate}
 Then for \(\abs x < \frac 12\), \(f(x)\)  must be convergent, as \(g(x)\) is
 convergent for \(\abs x < \frac 12\).
\end{proof}
Combined with Exponentation by Squaring (\ref{sec_exp_by_squaring}), and some
simple surd arithmetic, this theorem provides a fairly fast way to calculate
\(F_n\), as compared to utterly na\"ive recursion, or somewhat faster iteration
or memoized recursion. However it is often in fact faster to simply use the
matrix form given in the second proof, to calculate \(\vec f_n\) as \(\mat
M^n\vec f_0\), again using exponentiation by squaring. This requires fewer
arithmetic operations.

\subsection{Taylor Series}

The Maclaurin series is the Taylor series around \(0\).
\begin{equation}
f(x) = f(0) + f'(0) x + \frac{f''(0)} 2 x^2 + \frac{f'''(0)}6 x^3 +\dotsb
  = \sum_{k=0}^\infty \frac{f^{(k)}(0)}{k!}x^k
\end{equation}

\subsection{Common series}
