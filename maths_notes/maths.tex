% Compiling with
% latexmk -halt-on-error -shell-escape -synctex=1 -pdf maths.tex
% (Recommend using a latexmkrc file so as just to run latexmk -pvc, for example)
% Probably you can achieve the same with an inordinate number of invocations of
% pdflatex -halt-on-error -shell-escape -synctex=1 maths.tex

\RequirePackage{silence}
\WarningFilter{remreset}{The remreset package}

% fleqn aligns equations to the left, a4 paper size, 11pt font, article class
\documentclass[fleqn,a4paper,11pt]{article}
\title{Maths}
\author{Izaak van Dongen}

\usepackage{mymaths}
\usepackage{mystyle}

\begin{document}
 \maketitle\thispagestyle{empty} % no page number under title
 \tableofcontents
 \listoftables
 \listoffigures
 \listoflistings
 \listoftheorems

 \section{Intropreface}

 This is a haphazardly glued together set of notes, consisting mostly of things
 I'm intending to do, or wish that I was able to do. It serves as a way for me
 to mess around with \LaTeX{} to my heart's content, and for me to store bits of
 maths I find interesting in a way I find comprehensible/useful. Some of it acts
 almost more as a mental aid to me than a proper explanation of anything. The
 set theory is particularly guilty of this.

 \section{Set Theory}

\subsection{Set Operations}

\subsection{Common sets}

%FIXME: sets of congruence classes
%       sets of polynomial rings: apply set setstyle A

\begin{longtable}{M p{0.7\textwidth}}
 \toprule
 \text{\bfseries Set} & \bfseries Description \\
 \midrule
 \endhead
 \bottomrule
 \endfoot
 \endlastfoot
 \emptyset, \varnothing & The empty set \(\set{}\). \\
 \Naturals & The set of natural numbers, \(\set{1, 2, 3, \dotsc}\).
             May or may not include 0, so sometimes the next few presented
             alternatives are preferable. \\
 \Integers & The set of integers
             \(\set{\dotsc, -2, 1, 0, 1, 2, \dotsc}\) \\
 \Integers^+, \Integers_{> 0} & The set of strictly positive integers
             \(\set{1, 2, 3, \dotsc}\). \\
 \Integers^+_0, \Integers_{\ge 0} &
             The set of strictly nonnegative integers
             \(\set{0, 1, 2, \dotsc}\). \\
 \Integers^-, \Integers_{< 0} & The set of strictly negative integers
             \(\set{-1, -2, -3, \dotsc}\). \\
 \Integers^-_0, \Integers_{\le 0} &
             The set of strictly nonpositive integers
             \(\set{0, -1, -2, \dotsc}\). \\
 \Rationals & The set of rational numbers
             \(\set{\frac ab \mid a, b \in \Integers \land b \neq 0}\).\\
 \setstyle A & The set of algebraic numbers, ie numbers that are roots of
             polynomials in \(\Integers[x]\). \\
 \Reals & The set of real numbers, which may be constructed as in
         \ref{sec_dedekind_cut}. \\
 \Complex & The set of complex numbers
             \(\set{a + bi \mid a, b \in \Reals}\),
             where \(i^2 = 1\).\\
 \intoo{a, b} & The open interval
                 \(\set{x \in \Reals \mid a < x < b}.\)\\
 \intcc{a, b} & The closed interval
                 \(\set{x \in \Reals \mid a \le x \le b}\).\\
 \intco{a, b} & The half-open interval
                 \(\set{x \in \Reals \mid a \le x < b}\).\\
 \intoc{a, b} & The half-open interval
                 \(\set{x \in \Reals \mid a < x \le b}\).\\
 \bottomrule
 \caption{Common sets}
\end{longtable}

\subsection{Closed, Open, Clopen}

\subsection{Axiom of Choice}

\subsection{ZFC}

\subsection{Construction of Sets}

\subsubsection{Dedekind Cuts} \label{sec_dedekind_cut}

\subsection{Cardinality}

% powerset of integers, diagonal argument, schroeder-bernstein, interval
% cardinality

Cardinality is a way to think about the ``size'' of sets. Cardinality is really
a kind of equivalence relation on the class of sets, where two sets have the
same cardinality iff there exists a bijection between them - ie there is a way
to produce a one-to-one mapping between the two sets. If two sets have the same
cardinality they are said to be equipotent.

Any two finite sets obviously have the same cardinality iff they have the same
number of elements. The cardinality of a finite set is usually just given as the
number of elements it has. For example, \(\abs{\emptyset} = 0\) and
\(\abs{\set{1, 3, 2}} = 3\), etc.

We also denote the cardinality of the natural numbers
\(\abs{\Naturals} = \aleph_0\). This is the Hebrew letter ``aleph'' or ``alef'',
with subscript naught.

It is fairly straightforward to show, for example, that we also have
\(\abs{\Integers} = \aleph_0\), with for example the function \(f\) such that we
have
\begin{equation*}
 \begin{array}{c|r|r|r|r|r|r|r|r|r|r}
  n & 1 & 2 & 3 & 4 & 5 & 6 & 7 & 8 & 9 & \cdots \\
  \hline
  f(n) & 0 & 1 & -1 & 2 & -2 & 3 & -3 & 4 & -4 & \cdots \\
 \end{array}
\end{equation*}
which is to say
\begin{align*}
 f(n) &=
  \frac 12 \left\{
  \begin{array}{cl}
   n & \text{\(n\) is even} \\
   1 - n & \text{\(n\) is odd}
  \end{array}
  \right. \\
 f^{-1}(n) &=
  \begin{cases}
   \hfil 2n & n \ge 1 \\
   \hfil -2n + 1 & n \le 0
  \end{cases}
\end{align*}
Similarly, the cardinality of for example the odd numbers is the same as that of
the naturals.

It can also be shown that there exists a bijection between the natural numbers
and the rationals. Explicitly, you can enumerate the rationals in a large grid,
and then simply traverse the grid by spiralling outwards. The bijection here is
that \(n \in \Naturals\) corresponds the \(n\)th rational you reach.

% TODO: pretty pictures
Another approach is to leverage the Cantor-Schr\"oder-Bernstein Theorem, which
states that there exists a bijection between the sets \(A\) and \(B\) iff there
exists an injection from \(A\) to \(B\) and there exists an injection from \(B\)
to \(A\).

For example, we can take
\begin{align*}
 f \colon \Integers &\to \Rationals \\
          n &\mapsto \frac n1
\end{align*}
and
\begin{align*}
 g \colon \Rationals &\to \Integers \\
          \frac pq &\mapsto 2^p 3^q
\end{align*}
which is guaranteed by the Fundamental Theorem of Arithmetic to be injective for
distinct \(p, q\).

The nice thing about this approach is that it is quite neat, and it easily
generalises to sets of objects that can be identified by finite sequences of
natural numbers (or integers). This is (or at least resembles) a technique
called G\"odel coding, which can be applied to statements in formal systems. See
G\"odels incompleteness theorems.

\begin{theorem}[Cantor's Theorem]
 For any set \(S\), there will be no surjection from \(S\) to its powerset
 \(\PowerSet(S)\), which, to be clear, is defined as
 \begin{equation*}
  \PowerSet(S) \defeq \set{X \mid X \subseteq S}
 \end{equation*}
 Clearly then there is also no bijection.
\end{theorem}
\begin{proof}
 Suppose to the contrary that for some set \(S\) there does exist some
 surjective function \(f \colon S \to \PowerSet(s)\). Consider the set
 \(R = \set{x \in S \mid x \notin f(x)}\), that is all elements of \(S\) that
 are not in their own image under \(f\). As \(f\) is surjective onto
 \(\PowerSet(S)\) and \(R\) must be a subset of \(S\), we can now consider the
 preimage of \(R\) under \(f\) - there must exist some \(\zeta\) such that
 \(f(\zeta) = R\). But then \(\zeta \in R \iff \zeta \notin R\), due to the
 definition of \(R\). This is a contradiction.

 % TODO: is this the law of excluded middle?
 Elaborating, we first of all know that either \(\zeta \in R\) or
 \(\zeta \notin R\), as they are logically opposite statements that must be
 either true or false.

 If \(\zeta \in R\), then by the definition of \(R\),
 \(\zeta \notin R\) because \(R = f(\zeta)\), so
 \(\zeta \in R \iff \zeta \in f(\zeta)\).

 If on the other hand, \(\zeta \notin R\), then \(\zeta \in R\) by the
 construction of R, following the same argument. So
 \(\zeta \in R \implies \zeta \notin R\) and
 \(\zeta \notin R \implies \zeta \in R\).
\end{proof}

This theorem can be leveraged to give a nice proof that the reals have strictly
greater cardinality than the naturals, by showing that they are equipotent with
the powerset of the naturals.

Firstly, we construct an injection \(f\) from \(\Reals\) to
\(\PowerSet(\Rationals)\). For some real number \(x\), we define \(f(x)\) as the
set of all rationals less than or equal to \(x\).
\begin{align*}
 f \colon \Reals &\to \PowerSet(\Rationals) \\
          x &\mapsto \set{r \in \Rationals \mid r \le x}
\end{align*}
This is the Dedekind cut corresponding to \(x\), and uniquely identifies each
\(x\). Recalling that the rationals have the same cardinality as the natural
numbers, clearly there must then exist an injection from \(\Reals\) to
\(\PowerSet(\Naturals)\), by the composition of some bijection
\(\Rationals \to \Naturals\) applied to each element of a subset of
\(\Naturals\) with \(f\).

Now we construct an injection \(g\) from \(\PowerSet(\Naturals)\) to \(\Reals\).
Note \(\PowerSet(\Naturals)\) is isomorphic to the set of all infinite sequences
of binary digits. Particularly, for some subset
\(S \in \PowerSet(\Naturals) \iff S \subseteq \Naturals\), the \(n\)th digit of
the binary sequence corresponding to \(S\) may be given by \(1\) if \(n \in S\),
and \(0\) otherwise.

Now for some \(S \in \PowerSet(\Naturals)\), generate the binary sequence, then
read it back as a ternary expansion of a real number. That is to say, if the
digits are given by \(U_i, i \in \Naturals\), then \(g\) will map \(s\) to the
real number
\begin{equation*}
 x = \sum_{r \in \Naturals} \frac{U_r}{3^r}
\end{equation*}
This sum will always converge, as it is bounded by
\begin{equation*}
 x = \sum_{r \in \Naturals} \frac 1{3^r} = \frac 12
\end{equation*}
Furthermore, none of these sequences conflict with one another, as we have
interpreted them in ternary, which avoids any infinite sequences of \(2\)s,
which could be replaced by a single zero. For example, had we read them in
binary, then the sequences \(0.011111\ldots\) and \(0.1000000\ldots\) would both
correspond to the real number \(\frac 12\).

So we have our injection, and then by the Cantor-Schr\"oder-Bernstein Theorem,
\(\abs{\Reals} = \abs{\PowerSet(\Naturals)}\), and then by Cantor's Theorem,
\(\abs{\Reals} > \abs{\Naturals}\).

The cardinality of the reals \(\abs{\Reals}\) is also denoted \(\mathfrak c\) (a
lowercase fraktur script c) and \(\beth_1\) (Hebrew letter ``beth'' with
subscript one). If the continuum hypothesis is true, then \(\beth_1 = \aleph_1\)
(this is in fact basically what CH states). However CH is known to be
independent of ZFC, so you can't prove or disprove it with the ``standard''
axioms of set theory.

% TODO: diagonalisation argument

\subsection{Functions}

% add involutions

\subsubsection{Jections}

%FIXME mnemonic or diagram

\begin{itemize}
 \item An injection maps each element of its domain to a unique element of
       its codomain.
 \item A surjection maps an element of its domain to each element of its
       codomain.
 \item A bijection is an injection and a surjection.
\end{itemize}

 \section{Number Theory}

\subsection{Fundamental Theorem of Arithmetic}

\subsection{Prime Number Theorem}

\subsection{Totient Function} \label{sec_totient}

\subsection{Infinitude of Primes}

\subsection[Irrationality of \texorpdfstring{\(\sqrt 2\)}{the Square Root of 2}]
           {Irrationality of \boldmath\(\sqrt 2\)}

Assume \(\sqrt 2 = \frac ab : a, b \in \Integers \land \gcd(a, b) = 1\), ie
\(a\) and \(b\) are coprime.

\subsection{Square Triangular Numbers}

%FIXME: add link annotations, explanations, oeis
%FIXME: format table
%fixme: improve this disgusting code

The sequence of perfect squares \(\mathrm{ST}_n\) such that
\(\mathrm{ST}_n= a_n^2 = \frac 12 b_n(b_n + 1)\), where
\(a, b \in \Integers_0^+\). The first few are:

\begin{longtable}{rrrr}
 \toprule
 \boldmath\(n\) & \boldmath\(\text{\bfseries ST}_n\) & \boldmath\(a_n\) &
     \boldmath\(b_n\) \\
 \midrule
 \endhead
 \bottomrule
 \endfoot
 \endlastfoot
 0 & 0 & 0 & 0 \\
 1 & 1 & 1 & 1 \\
 2 & 36 & 6 & 8 \\
 3 & 1225 & 35 & 49 \\
 4 & 41616 & 204 & 288 \\
 5 & 1413721 & 1189 & 1681 \\
 6 & 48024900 & 6930 & 9800 \\
 7 & 1631432881 & 40391 & 57121 \\
 8 & 55420693056 & 235416 & 332928 \\
 9 & 1882672131025 & 1372105 & 1940449 \\
 10 & 63955431761796 & 7997214 & 11309768 \\
 11 & 2172602007770041 & 46611179 & 65918161 \\
 \multicolumn 4c{\(\cdots\)} \\
 \bottomrule
 \caption{Square triangular numbers}
\end{longtable}

Generated by Listing \ref{lst_st_gen}.

\begin{longlisting}
\begin{minted}{python}
def get_sqr_tris(n):
nums = [0, 1]
for _ in range(n):
    nums.append(34 * nums[-1] - nums[-2] + 2)
return nums

def itrirt(n):
return (isqrt(1 + 8 * n) - 1) // 2

def isqrt(n):
if n < 2:
    return n
else:
    small = isqrt(n >> 2) << 1
    big = small + 1
    if big ** 2 > n:
        return small
    else:
        return big

for n, i in enumerate(get_sqr_tris(10)):
print(" & ".join(map(str, [n, i, isqrt(i), itrirt(i)])) + " \\\\")
\end{minted}
\caption{Generating ST numbers \label{lst_st_gen}}
\end{longlisting}

%FIXME Pell's equation here

Where \(p / q\) is the \(n\)th convergent of \(\sqrt 2\),
\begin{equation}
 \mathrm{ST}_n = p^2 q^2
\end{equation}

\(\mathrm{ST}_n, a_n, b_n\) also satisfy the recurrence relations where
\(\mathrm{ST}_0 = 0\) and \(\mathrm{ST}_1 = 1\)
\begin{align}
 \mathrm{ST}_n &= 34\mathrm{ST}_{n - 1} - \mathrm{ST}_{n - 2} + 2\\
 a_n &= 6a_{n - 1} - a_{n - 2} \\
 b_n &= 6b_{n - 1} - b_{n - 2} + 2
\end{align}

See \cite{WikiSTNumbers,WolframSTNumbers} for more information.

 \section{Algebra} \label{sec_algebra}

\subsection{Fundamental Theorem of Algebra}

\subsection{Difference of two Squares}

A difference of squares can be factorised
\begin{equation*}
 a^2 - b^2 \equiv (a + b)(a - b)
\end{equation*}
This can be verified with fairly simple
algebra.

\subsubsection{Difference of Higher Powers}

Similar results can be found in higher powers of \(a\) and \(b\):
\begin{align*}
 a^3 - b^3 &\equiv (a - b)(a^2 + ab + b^2) \\
 a^4 - b^4 &\equiv (a - b)(a^3 + a^2b + b^2a + b^3) \\
 \dots
\end{align*}
In general, you take out a factor of \((a - b)\) and then start with the
term \(a^{n - 1}\), and for each subsequent term decrease the power in \(a\)
and increase the power in \(b\):
\begin{equation*}
 a^n - b^n \equiv (a - b)(a^{n - 1} + a^{n - 2}b + a^{n - 3}b^2 + \dotsb +
                         ab^{n - 2} + b^{n - 1})
\end{equation*}
This can in fact be derived from the partial sum of a geometric progression
(\ref{sec_seq_GP}). The sum \(a^n + a^{n - 1}b + \dotsb + ab^{n - 1} + b^n\) is
a geometric progression of \(n + 1\) terms with first term \(a^n\) and
common ratio \(b/a\). Therefore,
\begin{alignat*}2
 &&a^n + a^{n - 1}b + \dotsb + ab^{n - 1} + b^n &=
         a^n \cdot \frac{(b / a)^{n + 1} - 1}{(b / a) - 1} \\
 &&    &= \frac{a^{n + 1} - b^{n + 1}}{a - b} \\
 &\implies{}& (a - b)(a^n + a^{n - 1}b + \dotsb + ab^{n - 1} + b^n) &=
         a^{n + 1} - b^{n + 1}
\end{alignat*}

Note that for odd \(n\), \(a^n + b^n\) can also be factorised as
\(a^n + b^n \equiv a^n - (-b)^n\), so you get the same expansion but with
alternating positive and negative terms.

% FIXME: arguments from totient function and with different powers? number
% of common factors?

\subsubsection{Difference of Composite Powers}

In fact, in the last section, \(a^4 - b^4\) may be more fully factorised by
using:
\begin{equation*}
 a^4 - b^4 \equiv (a^2)^2 - (b^2)^2 \equiv (a^2 - b^2)(a^2 + b^2) \equiv
     (a - b)(a + b)(a^2 + b^2)
\end{equation*}
This happens as \(4\) is composite. Any composite \(n\) will in fact result
in the cototient (\ref{sec_totient}) of \(n = n - \phi(n)\) factors (at
least as far as I can see, but probably can't prove). This
also holds for prime \(n\), but for prime \(n\),
\(\phi(n) = n - 2 \implies n - \phi(n) = 2\), as the only divisors of
a prime \(n\) are \(1\) and \(n\).

We may derive this from the fact that any
\(n = pq \colon p, q \in \setstyle P\),
\(a^n - b^n\) will factorise as
\begin{equation*}
 a^{pq} - b^{pq} \equiv
     (a^p - b^p)(a^{p(q - 1)} + a^{p(q - 2)}b^{p(1)} + \dotsb +
                 a^{p(1)}b^{p(q - 2)} + b^{p(q - 1)}
\end{equation*}

\subsection{Quadratic formula} \label{sec_quad_formula}

There is a formula to give the roots of a general quadratic.
\begin{theorem}[Quadratic formula]
 \label{thm_quad_form}
 \begin{align*}
  ax^2 + bx + c &= 0\ \text{where}\ a \neq 0 \\
  \iff x &= \frac{-b \pm \sqrt{b^2 - 4ac}}{2a}
 \end{align*}
\end{theorem}
\begin{proof}
 We can complete the square to solve a general quadratic equation.
 \begin{alignat*}2
  &&ax^2 + bx + c &= 0 \\
  &\iff{}& x^2 + \frac ba x + \frac ca &= 0 \\
  &\iff{}& \parens[\Big]{x + \frac b{2a}}^2
      &= \parens[\Big]{\frac b{2a}}^2 - \frac ca
      = \frac{b^2 - 4ac}{4a^2} \\
  &\iff{}& x + \frac b{2a} &= \pm \sqrt{\frac{b^2 - 4ac}{4a^2}}
      = \pm \frac{\sqrt{b^2 - 4ac}}{2a} \\
  &\iff{}& x &= \frac{-b \pm \sqrt{b^2 - 4ac}}{2a} \qedhere
 \end{alignat*}
\end{proof}
Notational note: the notation \(\bullet_0 = \bullet_1 \pm \bullet_2\) is really
a shorthand for
\(\bullet_0 \in \set{\bullet_1 + \bullet_2, \bullet_1 - \bullet_2}\), where
\(\bullet_i\) are placeholders for expressions. This means it becomes fairly
important to keep a handle on the direction that your implications are going.
Generally, the \(\pm\) ``operator'' behaves similarly to addition -
multiplication and division distribute over it, for example.

The symbol can occur multiple times in an expression, which usually means you
should just convert them all to \(+\) once, and to \(-\) once - that is, it does
not exhibit branching behaviour. If that's what you want to show, probably
better off just explicitly writing that out. Occasionally
\(\bullet_0 \mp \bullet_1\) is used as a shorthand for
\(\bullet_0 \pm -\bullet_1\)
(eg \(x = 1 \pm 2 \mp 3 \iff x = 1 \pm -1\)).

Confusingly, it is also used inside a set to indicate an element in addition to
its negation, sometimes:
\(\set{\pm \bullet, \cdots} \defeq \set{\bullet, -\bullet, \cdots}\).

Back to business. A corollary of Theorem \ref{thm_quad_form} is that the number
of real roots of a quadratic is
determined by the discriminant \({\Delta = b^2 - 4ac}\).
\begin{corollary}[Quadratic real roots]
 Where \(P(x) \defeq ax^2 + bx + c\) and \(\Delta \defeq b^2 - 4ac\),
 \begin{align*}
  \Delta &= 0 \iff \text{``\(P(x) = 0\) has one repeated real root''} \\
  \Delta &< 0 \iff \text{``\(P(x) = 0\) has no real roots''} \\
  \Delta &> 0 \iff \text{``\(P(x) = 0\) has two real roots''}
 \end{align*}
\end{corollary}

\subsection{Partial Fraction Decomposition}

When you have a large rational function, it can often be useful to decompose it
into a sum of smaller rational functions.

\subsection[Cauchy-Shwarz inequality for
            \texorpdfstring{\(\Reals^n\)}{sequences of real numbers}]
           {Cauchy-Shwarz inequality for \boldmath\(\Reals^n\)}

\begin{theorem}[Cauchy-Shwarz inequality]
 For two sequences of length \(n\), \(u_i, v_i \in \Reals\), indexed by
 \(i \in I\) where \(I = \set{1, 2, \dotsc, n}\),
 \begin{equation*}
  \parens[\Big]{\sum u_i v_i}^2 \le
      \parens[\Big]{\sum u_i^2} \parens[\Big]{\sum v_i^2}
 \end{equation*}
 with equality iff
 \(\Exists k \in \Reals \colon \Forall i \in I \colon u_i = k v_i\)
 (ie one sequence is a multiple of the other).
\end{theorem}
\begin{proof}
    We consider the polynomial
    \begin{equation*}
    P(x) = \sum (u_i x + v_i)^2 = 0
    \end{equation*}
    If there is any \(u_i\) term which is nonzero, this is a quadratic in \(x\)
    (if this condition is not met, the inequality becomes obviously true with
    equality).
    \begin{equation*}
     \parens[\Big]{\sum u_i^2} x^2 +
     \parens[\Big]{\sum 2 u_i v_i} x + \sum v_i^2 = 0
    \end{equation*}
    As it is a sum of squares of real terms, it must be nonnegative. In fact, it
    can only be zero if each contributing term has precisely the same zero,
    which happens only if all \(-v_i/u_i\) are equal, which leads to the
    condition for equality.

    As it has no zeroes or one zero (in the case of the condition for equality)
    \begin{alignat*}2
     &&\Delta = b^2 - 4ac &\le 0 \\
     &\iff{}&
     \parens[\Big]{\sum 2 u_i v_i}^2 -
                4\parens[\Big]{\sum u_i^2}\parens[\Big]{\sum v_i^2} &\le 0 \\
     &\iff{}& \parens[\Big]{\sum u_i v_i}^2
         &\le \parens[\Big]{\sum u_i^2} \parens[\Big]{\sum v_i^2} \qedhere
    \end{alignat*}
\end{proof}

\subsection{AM-GM inequality}

\begin{theorem}[AM-GM inequality]
 For a sequence \(u_i \in \Reals\), for \(1 \le i \le n\) (ie, the
 sequence is of length \(n\)) and \(u_i \ge 0\),
 \begin{equation*}
  \sqrt[n]{u_1 u_2 \dotsm u_n} \le \frac{u_1 + u_2 + \dotsb + u_n}n
 \end{equation*}
 with equality iff all \(u_i\) are equal.

 Equivalently, using sigma and pi notation:
 \begin{equation*}
  \parens[\Big]{\prod u_i}^\frac 1n \le \frac 1n\sum u_i
 \end{equation*}
\end{theorem}
\begin{proof}
 We can use a kind of wonky induction. First, we verify the base
 case, \(n = 2\):
 \begin{alignat*}2
  &&\sqrt{ab} &\le \frac{a + b}2 \\
  &\iff{}& 4ab &\le a^2 + 2ab + b^2 \\
  &\iff{}& 0 &\le a^2 - 2ab + b^2 \\
  &\iff{}& 0 &\le (a - b)^2\quad \text{with equality iff \(a = b\)}
 \end{alignat*}
 Then, supposing AM-GM holds for \(n\) and 2, we show that it holds for
 \(2n\). Taking
 \begin{alignat*}3
  &&a &= \sqrt[n]{u_1 u_2 \dotsm u_n} \\
  &&b &= \sqrt[n]{u_{n+1} u_{n+2} \dotsm u_{2n}} \\
  &\implies{}& a &\le \frac{u_1 + u_2 + \dotsb + u_n}n
          &&\quad \text{with equality iff \(u_1 = \dotsb = u_n\)}\\
  &&b &\le \frac{u_{n + 1} + u_{n + 2} + \dotsb + u_{2n}}n
          &&\quad \text{with equality iff \(u_{n+1} = \dotsb = u_{2n}\)}\\
  &&\text{and}\ \sqrt{ab} &\le \frac{a + b}2
      &&\quad \text{with equality iff \(a = b\)}\\
  &\implies{}& \sqrt[2n]{u_1 u_2 \dotsm u_{2n}} &\le
          \frac{u_1 + u_2 + \dotsb u_{2n}}{2n}
          &&\quad \text{with equality iff \(u_1 = \dotsb = u_{2n}\)}
 \end{alignat*}
 Now, supposing AM-GM holds for \(n\), we show that it holds for \(n - 1\).
 Taking
 \begin{alignat*}2
  &&u_n &= \sqrt[n - 1]{u_1 u_2 \dotsm u_{n - 1}} \\
  &\implies{}& (u_1 u_2 \dotsm u_n)^{1 / n}
          &= (u_1 u_2 \dotsm u_{n - 1})^{1 / (n - 1)} \\
  &\implies{}& (u_1 u_2 \dotsm u_{n - 1})^{1 / (n - 1)} &\le
          \frac 1n (u_1 + u_2 + \dotsb u_{n - 1}) +
          \frac 1n (u_1 u_2 \dotsm u_{n - 1})^{1 / (n - 1)} \\
  &\implies{}& \parens[\Big]{1 - \frac 1n}
      (u_1 u_2 \dotsm u_{n - 1})^{1 / (n - 1)} &\le
      \frac 1n (u_1 + u_2 + \dotsb + u_{n - 1}) \\
  &\implies{}& \frac {n - 1}n
          (u_1 u_2 \dotsb u_{n - 1})^{1 / (n - 1)} &\le
          \frac 1n (u_1 + u_2 + \dotsb + u_{n - 1}) \\
  &\implies{}& (u_1 u_2 \dotsm u_{n - 1})^{1 / (n - 1)} &\le
          \frac 1{n - 1} (u_1 + u_2 + \dotsb + u_{n - 1})
 \end{alignat*}
 As equality for \(n\) was iff all \(u_i\) were the same, this is still true.
 Now, for any \(n \in \Integers^+\), we can induct up to a power of 2 above
 \(n\), and then descend from there, so we know \(P(n)\) is true for all
 \(n \in \Integers^+\).
\end{proof}

\subsubsection{Generalized Power Means}

\subsection{de Moivre's Theorem}

\begin{theorem}[de Moivre's Theorem]
 \begin{equation*}
  (\cos \theta + i \sin \theta)^n \equiv \cos n\theta + i \sin n\theta
 \end{equation*}
\end{theorem}
\begin{proof}
 This can be proven for the integers by induction. Let \(P(n)\) denote de
 Moivre's theorem for \(\Forall n \in \Integers_0^+\).
 \begin{enumerate}[I.]
  \item \label{basec_thm_demoivre} Consider \(P(0)\):
        \begin{equation*}
         (\cos \theta + i \sin \theta)^0 \equiv 1 \equiv \cos 0 + i \sin 0
        \end{equation*}
  \item \label{induct_thm_demoivre} Now suppose \(P(n)\) is true and consider
        \(P(n + 1)\):
        \begin{align*}
         (\cos \theta + i \sin \theta)^{n + 1} &\equiv
          (\cos \theta + i \sin \theta)(\cos \theta + i \sin \theta)^n \\
          &\equiv (\cos \theta + i \sin \theta)
                  (\cos n\theta + i \sin n\theta) \impliedby P(n) \\
          &\equiv \cos \theta \cos n\theta - \sin \theta \sin n\theta
                + i(\cos \theta \sin n\theta + \sin \theta \cos n\theta) \\
          &\equiv \cos(\theta + n\theta) + i \sin(\theta + n\theta) \\
          &\equiv \cos (n + 1)\theta + i \sin (n + 1)\theta
        \end{align*}
        So \(P(n + 1)\) is true if \(P(n)\) is true.
  \item By the principle of mathematical induction, \ref{basec_thm_demoivre} and
        \ref{induct_thm_demoivre} imply that \(P(n)\) is true for
        \(\Forall n \in \Integers_0^+\).
 \end{enumerate}
\end{proof}

\subsection{Euler's Formula}

\begin{theorem}[Euler's Formula]
 \begin{equation*}
  e^{i\theta} \equiv \cos \theta + i \sin \theta
 \end{equation*}
\end{theorem}
\begin{proof}
 The standard, though fairly non-intuitive proof of this fact comes from a
 consideration of power series. Using the Maclaurin series of \(e^x\), we get
 \begin{alignat*}2
  e^{i\theta} &= \sum_{r = 0}^\infty \frac{(i\theta)^r}{r!}
            &{}={}& \frac{(i\theta)^0}{0!} + \frac{(i\theta)^1}{1!}
               + \frac{(i\theta)^2}{2!} + \frac{(i\theta)^3}{3!} + \dotsb \\
              &= \sum_{r = 0}^\infty \frac{(i\theta)^{2r}}{(2r)!}
               + \sum_{r = 0}^\infty \frac{(i\theta)^{2r + 1}}{(2r + 1)!}
            &{}={}& \parens[\Bigg]{
                  \frac{(i\theta)^0}{0!} + \frac{(i\theta)^2}{2!} + \dotsb}
               + \parens[\Bigg]{
                  \frac{(i\theta)^1}{1!} + \frac{(i\theta)^3}{3!} + \dotsb} \\
              &= \sum_{r = 0}^\infty \frac{(-1)^r\theta^{2r}}{(2r)!}
               + i\sum_{r = 0}^\infty \frac{(-1)^r\theta^{2r + 1}}{(2r + 1)!}
            &{}={}& \parens[\Bigg]{
                  1 - \frac{\theta^2}{2!}
                + \frac{\theta^4}{4!} - \dotsb}
               + i\parens[\Bigg]{
                  \theta - \frac{\theta^3}{3!}
                + \frac{\theta^5}{5!} - \dotsb} \\
              &= \cos \theta + i \sin \theta
 \end{alignat*}
 Here the reordering of the series \emph{is} justified as each series is
 absolutely convergent. % TODO
\end{proof}
\begin{proof}
 A way I find more intuitive to understand is by considering the function
 \(e^x\) as the unique solution to the differential equation
 \begin{equation*}
  \dv<y>{x} = y
 \end{equation*}
 with \(y(0) = 1\). Using the chain rule, it can be seen that
 \(f(\theta) = e^{i\theta}\) must satisfy
 \begin{equation*}
  f'(\theta) = if(\theta)
 \end{equation*}
 % TODO expansion and diagrams
 This means that the rate at which \(f(\theta)\) is changing is \(f(\theta)\)
 multiplied by \(i\). In the complex plane, multiplication by \(i\) corresponds
 to rotation by \(\SI{90}{\degree}\). Thinking of the derivative of \(f\) as the
 velocity of the point described in the plane by \(f(t)\), we see that its
 tangential velocity is always perpendicular to its position vector with respect
 to the origin. Furthermore, by differentiating again we can show that
 \(f''(\theta) = -f(\theta)\), ie its acceleration is always centripetal. This,
 to me, explains why \(f(\theta)\) should be a circle. I first encountered this
 explanation in \cite{3B1BEToTheIPi}.

 A more rigorous (wordy) procedure from here might be as follows. If we suppose
 that \(f\) is complex-valued - ie for any \(\theta\),
 \(f(\theta) \in \Complex\), then we can decompose \(f(\theta)\) into its real
 and imaginary parts. This assumption will be justified by the fact that we
 later get a solution, but it might also be justified by considering again the
 Maclaurin series of \(e^{i\theta}\), noting that it is a sum of complex and
 real numbers, and so must be complex valued. A less rigorous approach is to
 note that \(f(0) = 1 \in \Reals\), and \(f'(0) = i \in \Complex\). Then the
 first ``step'' the function takes is a complex-valued one, but then so must be
 any subsequent step, as the commmplex numbers are closed under multiplication.
  Anyway,
 \begin{equation*}
  f(\theta) = g(\theta) + i h(\theta)
   \quad \text{where} \quad
    g(\theta) = \Re(f(\theta)),\quad h(\theta) = \Im(f(\theta))
 \end{equation*}
 But
 \begin{alignat*}2
  && f'(\theta) &= if(\theta) = -h(\theta) + ig(\theta)
                 = g'(\theta) + ih'(\theta) \\
  &\implies{}&
   % TODO: centre asterisks
      g'(\theta) &= -h(\theta) \tag{\(\ast\)} \label{eqn_thm_euler_gh} \\
  &&  h'(\theta) &= -g(\theta) \tag{\(\ast\ast\)} \label{eqn_thm_euler_hg}
 \end{alignat*}
    By differentiating \ref{eqn_thm_euler_gh}, and then substituting in
    \ref{eqn_thm_euler_hg}, we get
    \begin{equation*}
     g''(\theta) = -h'(\theta) = -g(\theta)
    \end{equation*}
    Symmetrically,
    \begin{equation*}
     h''(\theta) = -g'(\theta) = -h(\theta)
    \end{equation*}
    Noting that \(f(0) = 1\), we have the initial conditions
    \begin{equation*}
     g(0) = 1, \quad h(0) = 0
    \end{equation*}
    This is sufficient to determine that \(g(\theta) = \cos \theta\), and
    subsequently \(h(\theta) = -\dv{\theta}(\cos \theta) = \sin \theta\).
\end{proof}
\begin{proof}[Proof from the exponential limit definition]
 We can use the definition
 \begin{equation*}
  e^x = \lim_{n \to \infty} \parens[\Big]{1 + \frac xn}^n
 \end{equation*}
 to form a geometric construction in the complex plane of \(e^{i \theta}\).

 Using some properties of complex numbers written in polar form under
 exponentiation, we can deduce the modulus and argument of the result in order
 to deduce its value.

 Consider first the argument.
 \begin{align*}
  \arg e^{i \theta} &= \arg \lim_{n \to \infty} \parens[\Big]
                             {1 + \frac{i \theta}n}^n \\
                    &= \lim_{n \to \infty} \arg \parens[\Big]
                             {1 + \frac{i \theta}n}^n \\
                    &= \lim_{n \to \infty} n\parens[\Big]{\arg \parens[\Big]
                             {1 + \frac{i \theta}n}} \\
                    &= \lim_{n \to \infty} n \arctan \frac \theta n \\
                    &= \lim_{n \to \infty} n \cdot \frac \theta n \\
                    &= \lim_{n \to \infty} \theta \\
                    &= \theta
 \end{align*}
 Consider now the modulus.
 \begin{align*}
  \abs{e^{i \theta}}^2 &= \abs[\Big]{\lim_{n \to \infty} \parens[\Big]
                            {1 + \frac{i \theta}n}^n}^2 \\
                       &= \lim_{n \to \infty} \abs[\Big]{\parens[\Big]
                            {1 + \frac{i \theta}n}^n}^2 \\
                       &= \lim_{n \to \infty} \abs[\Big]{\parens[\Big]
                            {1 + \frac{i \theta}n}}^{2n} \\
                       &= \lim_{n \to \infty} \parens[\Big]
                          {1 + \parens[\Big]{\frac \theta n}^2}^n \\
                       &= \lim_{n \to \infty} \parens[\Big]
                          {\frac{n^2 + \theta^2}{n^2}}^n \\
                       &= \lim_{n \to \infty} \parens[\Big]
                          {\frac
                           {\sum_{r = 0}^n \binom nr n^{2(n - r)} \theta^{2r}}
                           {n^{2n}}} \\
                       &= \lim_{n \to \infty} \parens[\Big]
                          {\frac
                           {n^{2n} +
                            \sum_{r = 1}^n \binom nr n^{2(n - r)} \theta^{2r}}
                           {n^{2n}}} \\
                       &= \lim_{n \to \infty} \parens[\Big]
                          {1 + \frac
                           {\sum_{r = 1}^n \binom nr n^{2(n - r)} \theta^{2r}}
                           {n^{2n}}} \\
                       &= 1 \quad \text{(see Lemma \ref{lem_nn_exp_limit})}
 \end{align*}
 So indeed, \(e^{i \theta} = \cos \theta + i \sin \theta\).
 % TODO: diagram
\end{proof}
\begin{proof}[Proof by polar coordinates]
 I found this proof first in \cite{WikiEulersFormula}.

 If we suppose that \(e^{i \theta} \in \Complex\), then it can be expressed in
 polar form:
 \begin{equation*}
  e^{i \theta} = \rho (\cos \phi + i \sin \phi)
 \end{equation*}
 Where \(\rho\) and \(\phi\) are (currently) undetermined functions of
 \(\theta\). We differentiate on each side with respect to \(\theta\):
 \begin{equation*}
  i e^{i \theta} = \dv<\rho>{\theta} (\cos \phi + i \sin \phi)
                 + \dv<\phi>{\theta} \rho (-\sin \phi + i \cos \phi)
 \end{equation*}
 We can now use make use of the substitution
 \(e^{i \theta} = \rho (\cos \phi + i \sin \phi)\)
 \begin{alignat*}4
  && i \rho (\cos \phi + i \sin \phi) &=
                   \dv<\rho>{\theta} (\cos \phi + i \sin \phi)
                 + \dv<\phi>{\theta} \rho (-\sin \phi + i \cos \phi) \\
  &\implies{}& i \rho \cos \phi - \rho \sin \phi &=
                   \dv<\rho>{\theta} (\cos \phi + i \sin \phi)
                 + \dv<\phi>{\theta} \rho (-\sin \phi + i \cos \phi)
 \end{alignat*}
 Equating real and imaginary parts, we get
 \begin{alignat*}2
  && -\rho \sin \phi &= \dv<\rho>{\theta} \cos \phi
                      - \rho \dv<\phi>{\theta} \sin \phi \\
  && \rho \cos \phi &= \dv<\rho>{\theta} \sin \phi
                     + \rho \dv<\phi>{\theta} \cos \phi \\
  &\implies{}& \rho \sin^2 \phi \parens[\Big]{\dv<\phi>{\theta} - 1} &=
               \rho \cos^2 \phi \parens[\Big]{1 - \dv<\phi>{\theta}} \\
  && \cos \phi \parens[\Big]{\dv<\rho>{\theta} \cos \phi + \rho \sin \phi} &=
     \sin \phi \parens[\Big]{\rho \cos \phi - \dv<\rho>{\theta} \sin \phi} \\
  &\implies{}& \rho \parens[\Big]{\dv<\phi>{\theta} - 1} &= 0
   % TODO: centre asterisks
   \tag{\(\ast\)} \label{eq_euler_polar_1} \\
  && \dv<\rho>{\theta} &= 0
   \tag{\(\ast\ast\)} \label{eq_euler_polar_2}
 \end{alignat*}
 From \ref{eq_euler_polar_1}, we determine that \(\dv<\phi>{\theta} = 1\), as
 there exist values of \(\theta\) for which \(\rho \ne 0\)
 (eg \(\theta = 0 \implies \rho = \abs{e^{i \cdot 0}} = 1\)). So \(\phi\) is a
 function of the form \(\theta + C\).
 \(\abs{e^{i \cdot 0}} = 1\). Noting that \(\arg e^{i \cdot 0} = 0\), it is
 clear that \(\phi = \theta\).

 Even more immediately, from \ref{eq_euler_polar_2} we see that \(\rho\) is a
 constant function of \(\theta\). As we showed earlier, for \(\theta = 0\)
 \(\rho = 1\) so \(\rho = 1\) for all \(\theta\), and therefore
 \begin{equation*}
  e^{i \theta} = 1 (\cos \theta + i \sin \theta) \qedhere
 \end{equation*}
\end{proof}

\begin{lemma}[Limit of \(n^{an} n^{-bn}\) as \(n \to \infty\), where \(a < b\)]
 \label{lem_nn_exp_limit}
 This corresponds to
 \begin{align*}
  \lim_{n \to \infty} n^{n(a - b)}
   &= \lim_{n \to \infty} e^{(a - b)n \ln n} \\
   &= 0
 \end{align*}
 because \(a < b \implies a - b < 0 \implies e^{a - b} < 1\), and
 \(n \ln n \to \infty\) as \(n \to \infty\).
\end{lemma}

\begin{corollary}[Euler's Identity]
 By simply taking \(\theta = \pi\), we get the famous ``Euler's identity'':
 \begin{equation*}
  e^{i \pi} = \cos \pi + i \sin \pi = -1
 \end{equation*}
\end{corollary}

\subsection[The \texorpdfstring{\(\Gamma\)}{Gamma} function]
           {The \boldmath\(\Gamma\) function}

The ``gamma'' or \(\Gamma\) function is defined for
\(z \in \Complex, \Re(z) > 0\) as
\begin{equation*}
 \Gamma(z) = \integ[0]<\infty>{x^{z - 1}e^{-x}}{x}
\end{equation*}
By integrating by parts, we show the following:
\begin{align*}
 \Gamma(z + 1) &= \integ[0]<\infty>{x^z e^{-x}}{x} \\
           &= -x^z e^{-x}\eval_0^\infty
              + \integ[0]<\infty>{zx^{z - 1}e^{-x}}{x} \\
           &= z\Gamma(z)
\end{align*}
Noting also that
\begin{align*}
 \Gamma(1) &= \integ[0]<\infty>{e^{-x}}{x} \\
       &= -e^{-x}\eval_0^\infty \\
       &= 1
\end{align*}
it can be seen from this recurrence that where \(n \in \Integers^+\),
\begin{equation*}
 \Gamma(n) = (n - 1)!
\end{equation*}
In fact, the gamma function is used as an extension of the idea of
factorials to the real and complex numbers other than the negative integers.

An interesting value that the gamma function takes is for \(z = \frac 12\).
\begin{equation*}
 \Gamma(\tfrac 12) = \integ[0]<\infty>{x^{-\frac 12} e^{-x}}{x}
\end{equation*}
Letting \(x = u^2\)
\begin{alignat*}2
 &\implies{}& \dv<x>{u} &= 2u \\
 &\implies{}& \Gamma(\tfrac 12) &=
     \integ[0]<\infty>{\frac{2u}u e^{-u^2}}{u} \\
 &&  &= 2\integ[0]<\infty>{e^{-u^2}}{u} \\
 &&  &= \sqrt \pi
\end{alignat*}
from Theorem \ref{thm_gauss_integral}. From this we can derive the value of
any \(\Gamma(n + \frac 12)\) where \(n \in \Integers\) from the recurrence
relation on \(\Gamma\), and in some very informal sense, we can find that
the ``factorials'' of the half-integers are rational multiples of the square
root of pi. The first few are shown in Table \ref{tab_gamma_halves}.
\begin{longtable}{*2M}
 \toprule
 \text{\boldmath\(z\)}
     & \text{\bfseries\boldmath\(\Gamma(z)\), AKA ``\boldmath\((z - 1)!\)''} \\
 \midrule
 \endhead
 \bottomrule
 \endfoot
 \endlastfoot
 \rule{0pt}{4ex}
 \frac 12 & \sqrt{\pi} \\[3ex]
 \frac 32 & \frac{\sqrt{\pi}}2 \\[3ex]
 \frac 52 & \frac{3 \sqrt{\pi}}4 \\[3ex]
 \frac 72 & \frac{15 \sqrt{\pi}}8 \\[3ex]
 \frac 92 & \frac{105 \sqrt{\pi}}{16} \\[3ex]
 \multicolumn 2c{\(\cdots\)} \\
 \bottomrule
 \caption{Half-integer values of the gamma function
 \label{tab_gamma_halves}}
\end{longtable}

 \section{Sums, Sequences, Series}

\subsection{Binomial Theorem}

\begin{theorem}[Binomial theorem] \label{thm_binomial_thm}
 Where \(n \in \Naturals\),
 \begin{equation*}
  (a + b)^n \equiv \sum_{r = 0}^n \binom nr a^r b^{n - r}
 \end{equation*}
 where the binomial coefficients are given by
 \begin{equation*}
  \binom nr = \nCr nr \defeq \frac{n!}{r!\cdot(n - r)!}
 \end{equation*}
\end{theorem}
\begin{proof}
 We can prove that this is the case by induction. Let \(P(n)\) denote the
 binomial theorem for exponent \(n, \Forall n \in \Integers^+\).
 \begin{enumerate}[I.]
  \item \label{basec_thm_binomial} Consider \(P(1)\):
        \begin{equation*}
        (a + b)^1 \equiv a + b
            \equiv \frac{0!}{0!\cdot 1!} a^0b^1 +
                \frac{1!}{1!\cdot 0!} a^1b^0
        \end{equation*}
        so \(P(1)\) is true.
  \item \label{induct_thm_binomial} We now consider \(P(n + 1)\), supposing
        that \(P(n)\) is true.
        \begin{align*}
        (a + b)^{n + 1} &\equiv (a + b)(a + b)^n \\
            &\equiv (a + b)\sum_{r = 0}^n \binom nr a^r b^{n - r}
                \impliedby P(n) \\
            &\equiv a\sum_{r = 0}^n \binom nr a^r b^{n - r}
                + b\sum_{r = 0}^n \binom nr a^r b^{n - r} \\
            &\equiv \sum_{r = 0}^n \binom nr a^{r + 1} b^{n - r}
                + \sum_{r = 0}^n \binom nr a^r b^{n - r + 1} \\
            &\equiv \sum_{r = 1}^{n + 1} \binom n{r - 1} a^r b^{n - r + 1}
                + \sum_{r = 0}^n \binom nr a^r b^{n - r + 1} \\
            &\equiv a^{n + 1} + b^{n + 1}
                + \sum_{r = 1}^n
                      \bracks[\bigg]{\binom n{r - 1} + \binom nr}
                    a^r b^{n - r + 1} \\
            &\equiv a^{n + 1} + b^{n + 1}
                + \sum_{r = 1}^n \binom{n + 1}r a^r b^{n - r + 1}
                    \quad \text{due to Lemma \ref{lem_bin_coef_sum}} \\
            &\equiv \sum_{r = 0}^{n + 1}
                  \binom{n + 1}r a^r b^{(n + 1) - r}
        \end{align*}
        So \(P(n + 1)\) is true if \(P(n)\) is true.
  \item By the principle of mathematical induction
        \ref{basec_thm_binomial} and \ref{induct_thm_binomial} imply
        that \(P(n)\) is true for \(\Forall n \in \Integers^+\). \qedhere
 \end{enumerate}
\end{proof}

\begin{lemma}[Sum of adjacent binomial coefficients] \label{lem_bin_coef_sum}
 \begin{equation*}
  \binom n{r - 1} + \binom nr \equiv \binom{n + 1} r
 \end{equation*}
\end{lemma}
\begin{proof}
 \begin{align*}
  \binom n{r - 1} + \binom nr &\equiv \frac{n!}{(r - 1)! \cdot (n - r + 1)!}
                                   + \frac{n!}{r! \cdot (n - r)!}
                                      \quad \text{by definition} \\
      &\equiv \frac{n!\cdot(r + (n - r + 1))}{r! \cdot (n - r + 1)!} \\
      &\equiv \frac{n! \cdot (n + 1)}{r! \cdot ((n + 1) - r)!} \\
      &\equiv \frac{(n + 1)!}{r! \cdot ((n + 1) - r)!} \\
      &\equiv \binom{n + 1} r \quad \text{by definition} \qedhere
 \end{align*}
\end{proof}

\subsection{Arithmetic Progressions} \label{sec_seq_AP}

An arithmetic progression is a sequence \(U_i\) where each progressive term is
given by the previous term plus some constant \emph{common difference}, denotes
\(d\). The first term is denoted \(a\), such that
\begin{gather*}
 U_1 = a, \qquad U_2 = a + d, \qquad U_3 = a + 2d, \qquad \dots\\
 U_n = a + \underbrace{d  + d  + \dotsb + d}_{\text{\(n - 1\) times}}
     = a + (n - 1)d
\end{gather*}
We can deduce that the sum of the first \(n\) terms of some AP is given by
\begin{align*}
 S_n &= \sum_{k = 1}^n U_nj = \sum_{k = 1}^n (a + (k - 1)d) \\
     &= an + d\sum_{k = 1}^n (k - 1) \\
     &= an + \frac{d(n - 1)n} 2 \\
     &= \tfrac 12 n (2a + (n - 1)d)
\end{align*}
An alternative method to deduce this is to see that
\begin{align*}
 U_k + U_{n + 1 - k} &= a + (k - 1)d + a + (n - k)d \\
     &= 2a + (n - 1)d
\end{align*}
and that therefore, grouping the sum,
\begin{align*}
 S_n &= (U_1 + U_n) + (U_2 + U_{n - 1}) + \dotsb
      + (U_{n / 2} + U_{(n / 2) + 1}) \\
     &= \tfrac 12 n(2a + (n - 1)d) \quad \text{if n is even} \\
 S_n &= (U_1 + U_n) + (U_2 + U_{n - 1}) + \dotsb
      + (U_{(n - 1) / 2} + U_{(n + 3) / 2}) + U_{(n + 1) / 2} \\
     &= \tfrac 12 (n - 1)(2a + (n - 1)d) + a + \tfrac 12 (n - 1) d \\
     &= \tfrac 12 n(2a + (n - 1)d) \quad \text{if n is odd}
\end{align*}
This feels a little more tedious to do fully to me, though, although it may
prove easier to remember.


\subsection{Geometric Progressions} \label{sec_seq_GP}

A geometric progression is a sequence \(U_i\) where each progressive term is
given by the previous term multiplied by some constant \emph{common ratio},
denoted \(r\). The first term is denoted \(a\), such that
\begin{gather*}
 U_1 = a, \qquad U_2 = ar, \qquad U_3 = ar^2, \qquad \dots\\
 U_n = a\underbrace{r \cdot r \cdot \dotsb \cdot r}_{\text{\(n - 1\) times}}
     = ar^{n - 1}
\end{gather*}

\subsection{Fibonacci Sequence}

\begin{theorem}[Fibonacci \(n\)th term]
 The Fibonacci numbers \(F_n \colon n \in \Naturals\) are such that
 \(F_1 = F_2 = 1\) and \(F_n = F_{n - 1} + F_{n - 2}\).  \(F_n\) can be given
 by the closed form
 \begin{equation*}
  F_n = \frac{\varphi^n - (-\varphi^{-n})}{\sqrt 5}
  \quad \text{where}\quad \varphi = \frac{1 + \sqrt 5} 2
 \end{equation*}
\end{theorem}
Some people like to write
\begin{equation*}
 (-\varphi)^{-1} = \psi = \frac{1 - \sqrt 5}2
\end{equation*}
here. This can be nicer because then \(\varphi\) and \(\psi\) are the two
roots of \(x^2 - x - 1 = 0\). However I prefer to use \((-\varphi)^{-1}\) as
it makes some algebraic manipulations more immediately obvious.
\begin{proof}
 Consider the power series
 \begin{equation*}
  f(x) \defeq \sum_{k=1}^\infty F_k x^k = x + x^2 + 2x^3 + 3x^4 + 5x^5
      + \dotsb
 \end{equation*}
 Note that this series has a radius of convergence, by Lemma
 \ref{lem_fibo_convergence}.

 Consider now:
 \begin{alignat*}9
  && f(x) &= \sum_{k = 1}^\infty F_k x^k
      &&={}& x &+{}& x^2 &+{}& 2x^3 &+{}&
          3x^4 &+{}& 5x^5 &+{}& 8x^6 + \dotsb \\
  && x f(x) &= \sum_{k = 2}^\infty F_{k - 1} x^k
      &&={}& &&x^2 &+{}& x^3 &+{}& 2x^4 &+{}&
          3x^5 &+{}& 5x^6 + \dotsb \\
  && x^2 f(x) &= \sum_{k = 3}^\infty F_{k - 2}x^k
      &&={}& &&&& x^3 &+{}& x^4 &+{}& 2x^5 &+{}& 3x^6 + \dotsb \\
  &\implies{}& (1 - x - x^2) f(x) &= x \\
  &\implies{}& f(x) &= \frac x{1 - x - x^2}
 \end{alignat*}
 Now we perform a partial fraction decomposition on \(f(x)\). We do this in a
 somewhat tricky way in order to make our lives easier. Note that
 \begin{equation*}
  1 - x - x^2 = x^2 \parens[\Big]{\frac 1x^2 - \frac 1x - 1}
      = x^2\parens[\Big]{\frac 1x - \varphi}\parens[\Big]{\frac 1x - \psi}
      = (1 - \varphi x)(1 - \psi x)
 \end{equation*}
 where \(\varphi, \psi = \frac 12(1 \pm \sqrt 5)\), ie the roots of
 \(x^2 - x - 1\), ie the roots of the golden ratio. Now,
 \begin{alignat*}2
  && f(x) = \frac x{(1 - \varphi x)(1 - \psi x)}
      &\equiv \frac A{1 - \varphi x} + \frac B{1 - \psi x} \\
  &\implies{}& A(1 - \psi x) + B(1 - \varphi x) &\equiv x \\
  &\implies{}& A + B &= 0 \\
  && -\psi A - \varphi B &= 1 \\
  &\implies{}& B &= -A  \\
  && A(\varphi - \psi) &= 1 \\
  &\implies{}& A(\sqrt 5) &= 1 \\
  &\implies{}& A &= \frac 1{\sqrt 5}, B = -\frac 1{\sqrt 5}
 \end{alignat*}
 Now we can use a binomial series expansion to obtain an equivalent series.
 \begin{align*}
  f(x) &= \frac 1{\sqrt 5}\parens[\Big]{\frac 1{1 - \varphi x}
                              - \frac 1{1 - \psi x}} \\
  &= \frac 1{\sqrt 5}\parens[\Big]{\sum_{k = 0}^\infty (\varphi x)^k
                      - \sum_{k = 0}^\infty (\psi x)^k} \\
  &= \sum_{k = 0}^\infty \frac 1{\sqrt 5}(\varphi^k - \psi^k)x^k
  = \sum_{k = 1}^\infty F_k x^k
 \end{align*}
 So we have \(F_n = \frac 1{\sqrt 5}(\varphi^n - \psi^n)\). Note that the
 constant term is indeed zero, as \(\varphi^0 - \psi^0 = 0\).
\end{proof}
\begin{proof}[Proof by matrices]
 Noting that the given recurrence \(F_n = F_{n - 1} + F_{n - 2}\) is a linear
 transition (and a fairly simple one at that), we can define the vector
 \begin{equation*}
  \vec f_n =
  \begin{pmatrix*}[l]
   F_n \\
   F_{n + 1}
  \end{pmatrix*}
 \end{equation*}
 and take advantage of the linear transition to write
 \begin{equation*}
  \vec f_{n + 1} = \mat T \vec f_n \quad \text{where} \quad
  \mat T =
  \begin{pmatrix}
   0 & 1 \\
   1 & 1
  \end{pmatrix}
 \end{equation*}
 so that in fact,
 \begin{equation*}
  \vec f_n = \mat T^n\vec f_0
 \end{equation*}
 where we define \(F_0 = 0\). This avoids messily subtracting one from
 things. Now we proceed to diagonalise \(\mat T\) by first solving its
 characteristic equation for its eigenvalues.
 \begin{alignat*}4
  && \abs{\mat T - \lambda \mat I} &= 0 \\
  &\iff{}&
   \begin{vmatrix}
    -\lambda & 1 \\
    1 & 1 - \lambda
   \end{vmatrix} &= 0
  &\iff{}& \lambda (\lambda - 1) - 1 &= 0 \\
  &\iff{}& \lambda^2 - \lambda - 1 &= 0 \\
  &\iff{}& \lambda &\in \set{\varphi, \psi}
 \end{alignat*}
 as we saw earlier.

 Now we can find the eigenvectors \footnote{
     Note that as \(\varphi\) satisfies \(\varphi^2 - \varphi - 1 = 0\),
     we can rearrange to get \(\varphi - 1 = \frac 1\varphi\).}
 \begin{alignat*}6
  \lambda_1 &= \varphi\colon&
  &&
   \begin{pmatrix}
    -\varphi & 1 \\
    1 & 1 - \varphi
   \end{pmatrix} \vec e_1 &= \vec 0 \\
  && &\iff{}&
   \begin{pmatrix}
    -\varphi & 1 \\
    1 & -\varphi^{-1}
   \end{pmatrix} \vec e_1 &= \vec 0 \impliedby \vec e_1 &{}=
   \begin{pmatrix}
    1 \\
    \varphi
   \end{pmatrix} \\
  \lambda_2 &= \psi\colon&
  &&
   \begin{pmatrix}
    -\psi & 1 \\
    1 & 1 - \psi
   \end{pmatrix} \vec e_2 &= \vec 0 \\
  && &\iff{}&
   \begin{pmatrix}
    -\psi & 1 \\
    1 & -\psi^{-1}
   \end{pmatrix} \vec e_2 &= \vec 0 \impliedby \vec e_2 &{}=
   \begin{pmatrix}
    1 \\
    \psi
   \end{pmatrix}
 \end{alignat*}
 So that
 \begin{align*}
  \mat T^n &=
  \parens[\bigg]{
   \begin{array}{c|c}
    \vec e_1 & \vec e_2
   \end{array}
  }
  \begin{pmatrix}
   \lambda_1^n & 0 \\
   0 & \lambda_2^n
  \end{pmatrix}
  \parens[\bigg]{
   \begin{array}{c|c}
    \vec e_1 & \vec e_2
   \end{array}
  }^{-1} \\
  &=
  \begin{pmatrix}
   1 & 1 \\
   \varphi & \psi
  \end{pmatrix}
  \begin{pmatrix}
   \varphi^n & 0 \\
   0 & \psi^n
  \end{pmatrix}
  \begin{pmatrix}
   1 & 1 \\
   \varphi & \psi
  \end{pmatrix}^{-1} \\
  &=
  \begin{pmatrix}
   1 & 1 \\
   \varphi & \psi
  \end{pmatrix}
  \begin{pmatrix}
   \varphi^n & 0 \\
   0 & \psi^n
  \end{pmatrix}
  \frac 1 {\psi - \varphi}
  \begin{pmatrix}
   \psi & -1 \\
   -\varphi & 1
  \end{pmatrix} \\
  &=
  \begin{pmatrix}
   1 & 1 \\
   \varphi & \psi
  \end{pmatrix}
  \begin{pmatrix}
   \varphi^n & 0 \\
   0 & \psi^n
  \end{pmatrix}
  \frac 1 {\psi - \varphi}
  \begin{pmatrix}
   \psi & -1 \\
   -\varphi & 1
  \end{pmatrix} \\
  &=
  -\frac 1 {\sqrt 5}
  \begin{pmatrix}
   \varphi^n & \psi^n \\
   \varphi^{n + 1} & \psi^{n + 1} \\
  \end{pmatrix}
  \begin{pmatrix}
   \psi & -1 \\
   -\varphi & 1
  \end{pmatrix} \\
  &=
  -\frac 1 {\sqrt 5}
  \begin{pmatrix}
   \psi \varphi^n -\varphi \psi^n & \psi^n - \varphi^n \\
   \psi \varphi^{n + 1} - \varphi \psi^{n + 1} &
       \psi^{n + 1} - \varphi^{n + 1}
  \end{pmatrix}
 \end{align*}
 and
 \begin{alignat*}4
  &&\vec f_n &= \mat T_n \vec f_0 \\
  &\iff{}&
   \begin{pmatrix}
    F_n \\
    F_{n + 1}
   \end{pmatrix} &=
   -\frac 1 {\sqrt 5}
   \begin{pmatrix}
    \psi \varphi^n -\varphi \psi^n & \psi^n - \varphi^n \\
    \psi \varphi^{n + 1} - \varphi \psi^{n + 1} &
        \psi^{n + 1} - \varphi^{n + 1}
   \end{pmatrix}
   \begin{pmatrix}
    0 \\
    1
   \end{pmatrix} \\
  && &=
  \frac 1 {\sqrt 5}
   \begin{pmatrix}
    \varphi^n - \psi^n \\
    \varphi^{n + 1} - \psi^{n + 1}
   \end{pmatrix} \\
   &\implies{}& F_n &= \frac{\varphi^n - \psi^n}{\sqrt 5} \qedhere
 \end{alignat*}
\end{proof}
\begin{proof}[Proof by Induction]
 This can also be proven by induction. This proof is indirect, so it provides
 less insight, and is generally not as useful. However it remains an
 interesting excercise.

 Let \(P(n)\) denote the theorem for \(F_n\).
 \begin{enumerate}[I.]
  \item \label{basec_thm_fibo} We verify the cases \(F_1\) and \(F_2\):
        \begin{align*}
            n = 1&\colon \frac{\varphi^1 - \psi^1}{\sqrt 5}
                = \frac{\sqrt 5}{\sqrt 5} = 1 = F_1 \\
            n = 2&\colon \frac{\varphi^2 - \psi^2}{\sqrt 5}
                = \frac{(\varphi + \psi)(\varphi - \psi)}{\sqrt 5}
                = \frac{(1)(\sqrt 5)}{\sqrt 5}= 1 = F_1
        \end{align*}
  \item \label{induct_thm_fibo} Now we suppose that \(P(n)\) and \(P(n + 1)\)
        are true, and consider \(P(n + 2)\):
        \begin{align*}
            F_{n + 2} &= F_n + F_{n + 1} \\
                      &= \frac{\varphi^n - \psi^n +
                               \varphi^{n + 1} - \psi^{n + 1}}
                              {\sqrt 5} \impliedby P(n), P(n + 1) \\
                      &= \frac{(\varphi + 1)\varphi^n - (\psi + 1)\psi^n}
                               {\sqrt 5} \\
                      &= \frac{\varphi^2\varphi^n - \psi^2\psi^n}{\sqrt 5} \\
                      &= \frac{\varphi^{n + 2} - \psi^{n + 1}}{\sqrt 5}
        \end{align*}
        This uses the fact that \(\varphi\) and \(\psi\) satisfy
        \(x^2 - x - 1 = 0\), which we rearrange to find \(x + 1 = x^2\).
  \item Now, by the principle of mathematical induction, \ref{basec_thm_fibo}
        and \ref{induct_thm_fibo} imply that \(P(n)\) must be true for all \(n
        \in \Integers^+\). \qedhere
 \end{enumerate}
\end{proof}
\begin{lemma}[Fibonacci power series convergence]
\label{lem_fibo_convergence}
 The power series
 \begin{equation*}
  f(x) \defeq \sum_{k=1}^\infty F_k x^k = x + x^2 + 2x^3 + 3x^4 + 5x^5
      + \dotsb
 \end{equation*}
 has a radius of convergence and can therefore be manipulated.
\end{lemma}
\begin{proof}
 First we define
 \begin{equation*}
  g(x) \defeq \sum_{k=1}^\infty 2^k x^k
 \end{equation*}
 Now we aim to show that \(f(x) < g(x)\) for \(x > 0\). This is because
 \(F_n < 2^n\). This can be proven by induction. Let \(P(n)\) denote
 ``\(F_n < 2^n\)'' for all \(n \in \Integers^+\).
 \begin{enumerate}[I.]
  \item \label{basec_lem_fibo}
        Note that \(F_1 = 1 < 2^1 = 2\) and \(F_2 = 1 < 2^2 = 4\).
  \item \label{induct_lem_fibo}
        Suppose \(P(n)\) and \(P(n + 1)\) are true. Consider \(P(n + 2)\):
        \begin{equation*}
        F_{n + 2} = F_{n + 1} + F_n < 2^{n + 1} + 2^n < 2^{n + 1} + 2^{n + 1}
            = 2^{n + 2}
        \end{equation*}
  \item By the principle of mathematical induction, \ref{basec_lem_fibo} and
        \ref{induct_lem_fibo} imply that \(P(n)\) is true for all
        \(n \in \Integers^+\).
 \end{enumerate}
 Then for \(\abs x < \frac 12\), \(f(x)\)  must be convergent, as \(g(x)\) is
 convergent for \(\abs x < \frac 12\).
\end{proof}
Combined with Exponentation by Squaring (\ref{sec_exp_by_squaring}), and some
simple surd arithmetic, this theorem provides a fairly fast way to calculate
\(F_n\), as compared to utterly na\"ive recursion, or somewhat faster iteration
or memoized/linearised recursion. However it is often in fact faster to
simply use the matrix form given in the second proof, to calculate
\(\vec f_n\) as \(\mat M^n\vec f_0\), again using exponentiation by squaring.
This requires fewer arithmetic operations.

\subsection{Taylor Series}

The Maclaurin series is the Taylor series around \(0\).
\begin{equation*}
f(x) = f(0) + f'(0) x + \frac{f''(0)} 2 x^2 + \frac{f'''(0)}6 x^3 +\dotsb
  = \sum_{k=0}^\infty \frac{f^{(k)}(0)}{k!}x^k
\end{equation*}

\subsection{Common series}

 \section{Linear Algebra}

\subsection{Matrices}

An \(m \times n\) matrix is a rectangular array of numbers, with \(m\) rows and
\(n\) columns:
\begin{equation*}
 \mat M \in \fld K^{m \times n} \iff \mat M =
 \begin{pmatrix}
  a_{11} & a_{12} & a_{13} & \cdots & a_{1n} \\
  a_{21} & a_{22} & a_{23} & \cdots & a_{2n} \\
  a_{31} & a_{32} & a_{33} & \cdots & a_{3n} \\
  \vdots & \vdots & \vdots & \ddots & \vdots \\
  a_{m1} & a_{m2} & a_{m3} & \cdots & a_{mn}
 \end{pmatrix}
 \quad \text{where \(a_{ij} \in \fld K\)}
\end{equation*}



 \section{Statistics}

\subsection{Discrete Random Variables}

\subsection{Probability Generating Functions}

The PGF of some discrete random variable \(G_X(\eta)\) is, essentially, a power
series in \(\eta\) with as coefficients the probability associated with that
power of \(\eta\):
\begin{equation*}
 G_X(\eta) \defeq \sum_{r = 0}^\infty \Prob(X = r)\eta^r
     = \sum_{r = 0}^\infty p_X(r)\eta^r
     = \Expect(\eta^X)
\end{equation*}
This does imply the restriction that \(X\) only takes
non-negative integer values, but this is not usually a huge hindrance.

It is easy to see that \(G(\eta)\) has a radius of convergence and is therefore
defined, as each coefficient is necessarily less than or equal to \(1\), so
\(G(\eta)\) is bounded by
\begin{equation*}
 G_X(\eta) < 1 + x + x^2 + \dotsb = \frac 1{1 - x}
\end{equation*}
which is defined for \(\eta \in \intco{0, 1}\). Moreover,
\begin{equation*}
 G(1) = \sum_{r = 0}^\infty \Prob(X = r) = 1
\end{equation*}
Some particularly useful properties arise when considering the derivatives of
\(G\) wrt \(\eta\).

Note that we can extract the coefficients of the power series by taking
derivatives,
\begin{equation*}
 \Prob(X = k) = \frac{G_X^{(k)}(0)}{k!}
\end{equation*}
Therefore a probability distribution is uniquely identified by its associated
generating function. Obverve also that
\begin{alignat*}6
 G_X(\nu) &= \sum_{r = 0}^\infty p_X(r)\eta^r
     &&={}& p_X(0) &+{}& p_X(1)\eta
     &+{}& p_X(2)\eta^2 &+{}& p_X(3) \eta^3 + \dotsb \\
 G_X'(\nu) &= \sum_{r = 0}^\infty rp_X(r)\eta^{(r - 1)}
     &&={}& 1 p_X(1) &+{}& 2p_X(2)\eta
     &+{}& 3p_X(3) \eta^2 &+{}& 4p_X(4) \eta^3 + \dotsb \\
 G_X''(\nu) &= \sum_{r = 0}^\infty r(r - 1)p_X(r)\eta^{(r - 2)}
     &&={}& 2 p_X(2) &+{}& 6p_X(3)\eta
     &+{}& 12p_X(4) \eta^2 &+{}& 20p_X(5) \eta^3 + \dotsb
\end{alignat*}
So that particularly,
\begin{align*}
 G_X(1) &= \Expect(1) \\
 G_X'(1) &= \Expect(X) \\
 G_X''(1) &= \Expect(X(X - 1))
\end{align*}
This is really just spelling out that by considering the expectation definition
of \(G_X\), in addition to the fact that an expectation is really just a sum, we
can just differentiate the inside when differentiating. This generalises to the
fact that the \(n\)th derivative of \(G_X\) at \(\eta = 1\) gives the \(n\)th
\emph{factorial moment} of \(X\).
\begin{equation*}
 G_X^{(n)}(1) = \Expect\bracks[\bigg]{\frac{X!}{(X - n)!}}
\end{equation*}

Regardless, we now have a useful tool. If we know \(G_X\), we can work out
\(\Expect(X)\) by differentiating once, and to obtain the variance we can use
\begin{align*}
 \Var(x) &= \Expect(X^2) - (\Expect(X))^2
     = \Expect(X^2 - X) + \Expect(X) - (\Expect(X))^2 \\
     &= G_X''(1) + G_X'(1) - (G_X'(1))^2
\end{align*}
The generating function of the sum of two independent random variables \(X + Y\)
(that need not necessarily be identically distributed) has a particularly useful
form. Consider
\begin{align*}
 G_{X+Y}(\eta) &= \Expect(\eta^{X + Y}) = \Expect(\eta^X \eta^Y) \\
     &= \Expect(\eta^X) \Expect(\eta^Y) \impliedby X \independent Y \\
     &= G_X(\eta) G_Y(\eta)
\end{align*}
In fact, this generalises to the sum of \(n\) independent random variables, each
multiplied by some constant coefficient:
\begin{alignat*} 3
 && S &= \sum_{r = 1}^n \alpha_r X_r
 &&= \alpha_1 X_1 + \alpha_2 X_2 + \dotsb + \alpha_n X_n \\
 &\implies{}& G_S(\eta)
     &= \Expect\parens[\Big]{\eta^{\sum_{r = 1}^n \alpha_r X_r}}
 &&= \Expect(\eta^{\alpha_1 X_1 + \alpha_2 X_2 + \dotsb + \alpha_n X_n}) \\
 && &= \Expect\parens[\Big]{\prod_{r = 1}^n \eta^{\alpha_r X_r}}
 &&= \Expect(\eta^{\alpha_1 X_1} \eta^{\alpha_2 X_2} \dotsm
           \eta^{\alpha_N X_n}) \\
 && &= \prod_{r = 1}^\infty\Expect(\eta^{\alpha_r X_r})
  \mathrlap{
   {}\impliedby X_1 \independent X_2 \independent
                          \dotsb \independent X_n} \\
 % a bit of witchcraft so that the width of this cell is ignored
 && &= \mathrlap{
     \Expect(\eta^{\alpha_1 X_1}) \Expect(\eta^{\alpha_2 X_2}) \dotsm
     \Expect(\eta^{\alpha_n X_n})} \\
 && &= \prod_{r = 1}^\infty G_{X_r}(\eta^{\alpha_r})
 &&= G_{X_1}(\eta^{\alpha_1}) G_{X_2}(\eta^{\alpha_2}) \dotsm
     G_{X_n}(\eta^{\alpha_n})
\end{alignat*}

\subsection{Common Discrete Distributions}

\subsubsection{Binomial Distribution}

%FIXME add diagram

The binomial distribution \(\Binomial(\nu, \pi)\) measures the probability
of having a certain number of outcomes in a set of trials,  where there are
\(\nu\) trials each with probability \(\pi\) of having that outcome.
The validity of this distribution is subject to these constraints:
\begin{enumerate}
 \item Each event must be independent.
 \item Each event must either have that outcome or not have it.
 \item The probability of having this outcome must be identically the same in
       each trial.
\end{enumerate}
This means that the binomial distribution can be used to model the expected
number of occurrences of some attribute in a random sample of a population
with replacement. However, in a sample without replacement, each trial is
not strictly independent, so the binomial distribution can't be used. For
large population sizes, however, it remains a good approximation.

In this case, ``large'' can be taken to mean more than about 30-100.

\begin{theorem}[Binomial properties]
 Where \(X \sim \Binomial(\nu, \pi)\),
 \begin{align*}
  p_X(k) &= \Prob(X = k) = \binom \nu k \pi^k (1 - \pi)^{\nu - k}
      \quad \text{where \(k \in \set{0, 1, \dotsc, \nu}\)} \\
  \Expect(X) &= \nu \pi \\
  \Var(X) &= \nu \pi (1 - \pi) \\
  \intertext{where}
  \binom \nu k &= \nCr \nu k = \frac{\nu!}{k!\cdot (\nu - k)!}
 \end{align*}
\end{theorem}
\begin{proof}
 From Lemma \ref{lem_binom_pgf}, we have that if \(X \sim \Binomial(\nu, \pi)\)
 \begin{alignat*} 2
  && G_X(\eta) &= (1 - \pi + \pi \eta)^\nu \\
  &&           &= \sum_{k = 0}^\nu
                  \binom \nu k (1 - \pi)^{\nu - k} \pi^k \eta^k
                      \quad \text{by the Binomial Theorem} \\
 \end{alignat*}
  So clearly, where \(k \in \set{0, 1, \dotsc, \nu}\), the coefficient in
  \(\eta^k\) (which is \(\Prob(X = k)\)) is
  \(\binom \nu k (1 - \pi)^{\nu - k} \pi^k\).
  \begin{alignat*}2
  &\implies{}& G_X'(\eta) &= \nu \pi (1 - \pi + \pi \eta)^{\nu - 1} \\
  && G_X'(1) &= \nu \pi (1 - pi + pi)^{\nu - 1} = \nu \pi \\
  &\implies{}& G_X''(\eta)
      &= \nu(\nu - 1) \pi^2 (1 - \pi + \pi \eta)^{\nu - 2} \\
  && G_X''(1)
      &= \nu(\nu - 1)\pi^2(1 - \pi + \pi)^{\nu - 2} = \nu(\nu - 1)\pi^2 \\
  &\implies{}& \Expect(X) &= G_X'(1) = \nu \pi \\
  && \Var(X) &= G_X''(1) + G_X'(1) - (G_X''(1))^2
      = \nu(\nu - 1) \pi^2 + \nu \pi - (\nu \pi)^2 \\
  && &= \nu \pi - \nu \pi^2 = \nu \pi (1 - \pi) \qedhere
 \end{alignat*}
\end{proof}
\begin{proof}[Alternative for \(p_X(k)\)]
 We can also argue for \(\Prob(X = k)\) combinatorially. We can consider the
 possible outcomes where the trials have an ordering, ie possible sequences of
 successes and failures. If the outcome is a total of \(k\) successes, then
 there are precisely \(\binom \nu k\) sequences achieving this, as we must
 choose \(k\) of the \(\nu\) trials to be successful. Each of these sequences
 has probability \(\pi^k (1 - \pi)^{\nu - k}\) of occurring, as \(k\) of the
 trials must be successful and \(\nu - k\) of the trials must be unsuccessful.
 As they are all mutually exclusive, the probability of getting any one such
 sequence is \(\binom \nu k \pi^k (1 - \pi)^{\nu - k}\).
\end{proof}
\begin{lemma}[Binomial PGF] \label{lem_binom_pgf}
 If \(X \sim \Binomial(\nu, \pi)\), then
 \begin{equation*}
 G_X(\eta) = (1 - \pi + \pi \eta)^\nu
 \end{equation*}
\end{lemma}
\begin{proof}
 We can use the sum of indendent random variables property of PGFs. If we let the
 random variable \(X_k'\) denote the number of successes in the \(k\)th trial, we
 have by definition that either \(X_k'\) is 0 or 1, with probability \(1 - \pi\)
 and \(\pi\) respectively, and that each different \(X_k'\) will be mutually
 independent. Each \(X_k'\) is identically distributed, so they have identical
 PGFs:
 \begin{equation*}
  \Forall k \in \set{1, 2, \dotsc, \eta} \colon
   G_{X_k'}(\eta) = 1 - \pi + \pi \eta
 \end{equation*}
 Now, because \(X = X_1' + X_2' + \dotsb + X_n' = \sum_{r = 1}^\nu X_r'\),
 \begin{equation*}
  G_X(\eta) = \prod_{r = 1}^\nu G_{X_r'}(\eta)
      = (1 - \pi + \pi \eta)^\nu \qedhere
 \end{equation*}
\end{proof}

A binomial distribution can be approximated by a normal distribution.

\subsubsection{Poisson Distribution}

The Poisson distribution \(\Poisson(\lambda)\) models the rate of occurrences of
some event that has a known, constant, average rate of occurrence \(\lambda\).
It's phrased like that because it can be a rate of occurrence per unit distance,
or per unit time, etc.

\begin{theorem}[Poisson properties]
 Where \(X \sim \Poisson(\lambda)\),
 \begin{align*}
  \Prob(X = k) &= p_X(k) = \frac{\lambda^k e^{-\lambda}}{k!}
      \quad \text{for \(k \in \Integers^+_0\)} \\
  \Expect(X) &= \lambda \\
  \Var(X) &= \lambda
 \end{align*}
\end{theorem}
\begin{proof}
 If we let \(X \sim \Poisson(\lambda)\), and \(Y \sim \Binomial(\nu, \pi)\),
 we can take \(X\) to be the limit of Y if we set \(\Expect(Y) = \lambda\)
 as \(\nu \to \infty\). This means particularly that we can use
 \begin{alignat*}2
  && \Expect(X) = \nu \pi &= \lambda \\
  &\implies{}& \pi &= \frac \lambda \nu
 \end{alignat*}
 From Lemma \ref{lem_binom_pgf} we know the PGF of \(Y\):
 \begin{alignat*}2
  && G_Y(\eta) &= (1 - \pi + \pi \eta)^\nu \\
  &&  &= \parens[\Big]{1 - \frac \lambda \nu + \frac \lambda \nu \eta}^\nu \\
  &\implies{}& G_X(\eta) = \lim_{\nu \to \infty}G_Y(\eta)
      &= \lim_{\nu \to \infty} \parens[\Big]{1 - \frac \lambda \nu
                                  + \frac \lambda \nu \eta}^\nu \\
  &&  &= \lim_{\nu \to \infty} \parens[\Big]
      {1 + \frac{\lambda(\eta - 1)}\nu}^\nu \\
  &&  &= e^{\lambda(\eta - 1)} = e^{-\lambda}e^{\lambda \eta}
      \quad \text{see section \ref{sec_e}} \\
  &&  &= e^{-\lambda} \sum_{k = 0}^\infty \frac{(\lambda \eta)^k}{k!} \\
  &&  &= \sum_{k = 0}^\infty \frac{\lambda ^k e^{-\lambda}}{k!} \eta^k \\
 \end{alignat*}
  So clearly the coefficient of the term in \(\eta^k\) (ie \(\Prob(X = k)\))
  is \(\frac{\lambda^k e^{-\lambda}}{k!}\). Also,
  \begin{alignat*}2
  &\implies{}& G_X'(\eta) &= \lambda e^{\lambda(\eta - 1)} \\
  && G_X'(1) &= \lambda \\
  &\implies{}& G_X''(\eta) &= \lambda^2 e^{\lambda(\eta - 1)} \\
  && G_X''(1) &= \lambda^2 \\
  && \Var(X) &= G_X''(1) + G_X'(1) - (G_X''(1))^2
      = \lambda^2 + \lambda - \lambda^2 = \lambda \qedhere
 \end{alignat*}
\end{proof}

\subsection{Continuous Random Variables}

\subsection{Moment Generating Functions}

\subsection{Common Continuous Distributions}

\subsubsection{Normal Distribution}

%FIXME add diagram

The PDF of \(X \sim \Normal(\mu, \sigma^2)\) is given by
\begin{equation*}
 f_X(x) = \frac 1{\sqrt{2\pi\sigma^2}} \cdot
     e^{\frac{(x - \mu)^2}{2\sigma^2}}
\end{equation*}

\subsubsection{Exponential Distribution}

\subsubsection[Student's \texorpdfstring{\(t\)}{t}-distribution]
              {Student's \boldmath\(t\)-distribution}

\subsection{Hypothesis Testing}

%FIXME Types of Error

 \section{Geometry}

\subsection{Pythagoras' Theorem} \label{sec_pythagoras}

%FIXME add diagram

If a triangle has sides \(a\), \(b\), \(c\) opposed by angles \(A\), \(B\),
\(C\), then
\begin{equation*}
 C = \frac 12 \pi \iff a^2 + b^2 = c^2
\end{equation*}

\subsection{Angles in a Polygon} \label{sec_geom_polygon_angles}

\subsection{Area of a Circle}

%FIXME add diagram

\begin{theorem}[Area of a circle]
 The area of a circle of radius \(r\) is given by
 \begin{equation*}
  A = \pi r^2
 \end{equation*}
\end{theorem}
\begin{proof}
 From the definition of a circle as the set of points equidistant from a
 centre point, we form the equation of a circle using \ref{sec_pythagoras}:
 \(x^2 + y^2 = r^2\), where the centre is \((0, 0)\) and the radius is \(r\).

 We can find half the area enclosed by the curve by integrating:
 \begin{equation*}
  \frac 12 A = \integ[-r]<r>{\sqrt{r^2 - x^2}}{x}
 \end{equation*}
 We use the substitution (\ref{sec_calc_trig_substitution})
 \(x = r \cos \theta \implies \dv<x>{\theta} = -r \sin \theta\) so that
 \begin{equation*}
  \frac 12 A = \integ[\arccos -1]<\arccos 1>
  {r\sqrt{1 - \cos^2 \theta} \cdot -r \sin \theta}{\theta}
  = -\integ[\pi]<0>{r^2 \sin^2 \theta}{\theta}
 \end{equation*}
 by \ref{sec_trig_pythag}. We use \ref{sec_trig_double_angle} to derive
 \begin{align*}
  \frac 12 A &= -\integ[\pi]<0>
  {r^2 \frac 12 (1 - \cos 2 \theta)}{\theta} \implies
  A = r^2 \integ[0]<\pi>{(1 - \cos 2 \theta)}{\theta} =
  r^2\parens[\Big]{\theta - \frac 12 \sin 2 \theta}\eval_0^\pi \\
  &= r^2 \parens[\Big]{\pi - \frac 12 \sin 2 \pi -
  \parens[\Big]{0 - \frac 12 \sin 0}} = \pi r^2 \qedhere
 \end{align*}
\end{proof}
\begin{proof}[Alternative proof]
 One might alternatively use the parametric form of a circle,
 \(y = r \sin \theta\) and \(x = r \cos \theta\)
 (note that \(x^2 + y^2 = r^2(\sin^2 \theta + \cos^2 \theta) = r^2\)
 (\ref{sec_trig_pythag})), and calculate the area using
 \ref{sec_calc_parametric_area}:
 \begin{align*}
  A &= \integ[0]<2\pi>
  {r \sin \theta \cdot -r \sin \theta}{\theta} =
  \frac 12 r^2 \integ[0]<2\pi>{(1 - \cos 2 \theta)}{\theta} =
  \frac 12 r^2 \parens[\Big]{\theta - \frac 12 \sin 2 \theta}\eval_0^{2\pi} \\
  &= \frac 12 r^2 \parens[\Big]{2\pi - \frac 12 \sin 4 \pi -
  (0 - \sin 0)} = \pi r^2 \qedhere
 \end{align*}
\end{proof}

\subsection{Volume of a sphere}

\subsection{Circle Theorems}

 \section{Trigonometry}

\subsection{Definitions} \label{sec_trig_definitions}

%FIXME add diagram

\begin{align}
 \sin \theta &= \frac OH \\
 \cos \theta &= \frac AH \\
 \tan \theta &= \frac OA = \frac{\sin \theta}{\cos \theta}
\end{align}

\subsubsection{Periodicity} \label{sec_trig_periodic}

%FIXME add more identities, diagrams (or refer to previous diagrams).

Because they are in the same right triangle, and the sum of angles in a
triangle is \(\pi\) (\(\ang{180}\)) (\ref{sec_geom_polygon_angles}), the
angle other than \(\theta\) must be \(\frac 12 \pi - \theta\). Therefore,
\begin{align}
 \sin \theta &\equiv \cos(\frac 12 \pi - \theta) \\
 \cos \theta &\equiv \sin(\frac 12 \pi - \theta)
\end{align}
From their parity \ref{sec_trig_parity} we may deduce
\begin{align}
 \sin(\theta - \frac 12 \pi) &\equiv -\cos \theta \\
 \cos(\theta - \frac 12 \pi) &\equiv \sin \theta
\end{align}
implying
\begin{align}
 \sin(\theta + \pi) &\equiv -\sin \theta \\
 \cos(\theta + \pi) &\equiv -\cos \theta
\end{align}
From this, we see that
\begin{alignat}2
 \sin(\theta + 2\pi) &\equiv \sin(\theta + \pi + \pi) &&\equiv \sin \theta \\
 \cos(\theta + 2\pi) &\equiv \cos(\theta + \pi + \pi) &&\equiv \cos \theta \\
 \tan(\theta + \pi) &\equiv \frac{\sin(\theta + \pi)}{\cos(\theta + \pi)}
     &&\equiv \frac{-\sin \theta}{-\cos \theta} \equiv \tan \theta
\end{alignat}
and in fact,
\begin{alignat}2
 \sin(\theta + 2\pi n) &\equiv \sin \theta &&\iff n \in \Integers \\
 \cos(\theta + 2\pi n) &\equiv \cos \theta &&\iff n \in \Integers \\
 \tan(\theta + \pi n) &\equiv \tan \theta &&\iff n \in \Integers
\end{alignat}
Therefore, the \emph{period} of \(\sin \theta\) and \(\cos \theta\) is
\(2\pi\), whereas the period of \(\tan \theta\) is \(\pi\).

\subsubsection{Parity} \label{sec_trig_parity}

From \ref{sec_trig_definitions}, we see that \(\sin \theta\) and
\(\tan \theta\) are \emph{odd} functions:
\(\sin -x = -\sin x\) and \(\tan -x = -\tan x\),
whereas \(\cos \theta\) is an \emph{even} function:
\(\cos -x = \cos x\).

\subsubsection{Reciprocal functions} \label{sec_trig_reciprocal}

%fixme some graphs

The reciprocals of sine, cosine and tangent are cosecant, secant and
cotangent respectively. These are written
\begin{align}
 \sec \theta &= \frac 1{\cos \theta} \\
 \operatorname{cosec} \theta = \csc \theta
     &= \frac 1{\sin \theta} \\
 \cot \theta &= \frac 1{\tan \theta}
     = \frac{\cos \theta}{\sin \theta}
\end{align}

\subsection{Special Angles}

%FIXME add diagrams, derive other angles

From geometric construction of triangles in addition to Pythagoras' Theorem
(\ref{sec_pythagoras}), we can determine that the following values of
trigonometric functions at certain angles hold.

\begin{table}[H]
 \centering
 \begin{tabular}{*4M}
  \toprule
  \text{\boldmath\(\theta\)} & \text{\boldmath\(\sin \theta\)}
  & \text{\boldmath\(\cos \theta\)} & \text{\boldmath\(\tan \theta\)} \\
  \midrule
  0 & 0 & 1 & 0 \\[1ex]
  \frac \pi 6 & \frac 12 & \frac{\sqrt 3} 2 & \frac {\sqrt 3} 3 \\[3ex]
  \frac \pi 4 & \frac {\sqrt 2} 2 & \frac {\sqrt 2} 2 & 1 \\[3ex]
  \frac \pi 3 & \frac{\sqrt 3} 2 & \frac 12 & \sqrt 3 \\[3ex]
  \frac \pi 2 & 1 & 0 & \infty\\[2ex]
  \bottomrule
 \end{tabular}
 \caption{The basic trigonometric constants}
\end{table}

Strictly, \(\tan \frac 12 \pi\) is undefined.

The symmetry of sine and cosine going opposite ways across this table is due
to the fact that \(\sin \theta \equiv \cos(\frac 12 \pi - \theta)\).

We need only discuss special angles in \(\intcc{0, \frac 12 \pi}\) because
the corresponding angles in the other quadrants can be found from symmetries
and periodicities. This is convenient as it means all values of trig
functions that we discuss here are positive, so we can take positive square
roots without remorse.

We can always calculate half-angles in the first quadrant using the
half-angle formulae (Theorem \ref{thm_trig_half_angle}). We can also use
compound angle formulae to build some angles, such as
\(\frac \pi {12} = \frac \pi 3 - \frac \pi 4\).

A small number of more esoteric angles are presented below. Many more can be
found \cite{WikiTrigConstants}.

\begin{table}[H]
 \centering
 \begin{tabular}{*4M}
  \toprule
  \text{\boldmath\(\theta\)} & \text{\boldmath\(\sin \theta\)}
  & \text{\boldmath\(\cos \theta\)} & \text{\boldmath\(\tan \theta\)} \\
  \midrule
  \frac \pi {12} & \frac{\sqrt 6 - \sqrt 2} 4 & \frac{\sqrt 6 + \sqrt 2} 4
  & 2 - \sqrt 3 \\[3ex]
  \frac \pi 8 & \frac 12 \sqrt{2 - \sqrt 2} & \frac 12 \sqrt{2 + \sqrt 2}
  & \sqrt 2 - 1 \\[3ex]
  \frac \pi 5 & \frac 14 \sqrt{10 - 2\sqrt 5} & \frac{\sqrt 5 + 1} 4
  & \sqrt{5 - 2\sqrt 5} \\[3ex]
  \frac{3\pi}{10} & \frac{\sqrt 5 + 1} 4 & \frac 14 \sqrt{10 - 2\sqrt 5}
  & \frac 15 \sqrt{25 + 10\sqrt 5} \\[3ex]
  \frac{3\pi} 8 & \frac 12 \sqrt{2 + \sqrt 2} & \frac 12 \sqrt{2 - \sqrt 2}
  & \sqrt 2 + 1 \\[3ex]
  \frac{5\pi}{12} & \frac{\sqrt 6 + \sqrt 2} 4 & \frac{\sqrt 6 - \sqrt 2} 4
  & 2 + \sqrt 3  \\[2ex]
  \bottomrule
 \end{tabular}
 \caption{More advanced trigonometric constants}
\end{table}

The second half of the table can be derived from the first half.

The result for \(\frac 1{12} \pi\) can be derived by using
\(\frac 1{12} \pi = \frac 13 \pi - \frac 14 \pi\).

The result for \(\frac 18 \pi\) can be derived by using
\(\frac 18 \pi = \frac 12 \cdot \frac 14 \pi\).

To get the result for \(\frac 15 \pi\) we can let
\(\varphi = \frac 15 \pi\).  Noting that
\(\sin 2\varphi = \sin 3\varphi\), we have
\begin{alignat*}2
 && 2\sin \varphi \cos \varphi
     &= \sin \varphi (2\cos^2 \varphi - 1)
     + \cos \varphi \cdot 2\sin \varphi \cos \varphi \\
 &\iff{}& 0 &= \sin \varphi (2\cos^2 \varphi - 1
                         + 2\cos^2 \varphi - 2\cos \varphi) \\
 &\iff{}& 0 &= 4\cos^2 \varphi - 2\cos \varphi - 1
     \quad \text{as \(\sin \varphi \ne 0\)} \\
 &\iff{}& \cos \varphi &= \frac{1 \pm \sqrt 5} 4 \\
 &\iff{}& \cos \varphi &= \frac{1 + \sqrt 5} 4
     \quad \text{as \(\cos \varphi > 0\)}
\end{alignat*}
We can then proceed to derive sine, and therefore also tangent, by applying
the Pythagorean identity and simplifying.

\subsection{Cosine Rule}

\subsection{Sine Rule}

\subsection{Sine Area Rule}

\subsection{Pythagorean Identities} \label{sec_trig_pythag}

It follows from Pythagoras' Theorem (\ref{sec_pythagoras})
and subsequently from the definition of
\(\tan\), \(\cot\), \(\sec\), \(\csc\) (\ref{sec_trig_reciprocal}) that
\begin{align}
 \cos^2 \theta + \sin^2 \theta &\equiv 1 \\
 \implies \tan^2 \theta + 1 &\equiv \sec^2 \theta \\
 \cot^2 \theta + 1 &\equiv \csc^2 \theta
\end{align}

\subsection{Compound Angle Identities} \label{sec_comp_angle}

%FIXME add diagram

\begin{theorem}[Addition formulae for trig functions]
 Trigonometric functions of a sum or difference of angles can be expanded into
 trigonometric functions of each angle.
 \begin{align*}
  \sin(\alpha + \beta) &\equiv
     \sin \alpha \cos \beta +  \sin \beta \cos \alpha \\
  \cos(\alpha + \beta) &\equiv
     \cos \alpha \cos \beta - \sin \alpha \sin \beta \\
  \tan(\alpha + \beta) &\equiv
      \frac{\tan \alpha + \tan \beta}{1 - \tan \alpha \tan \beta}
 \end{align*}
\end{theorem}
\begin{proof}
 The formula for sine can be shown geometrically.

 The formula for cosine could also be shown with a similar geometrical argument,
 but we can simply use section \ref{sec_trig_periodic} to show
 \begin{align*}
  \cos(\alpha + \beta) &\equiv
      \sin(\frac 12 \pi - \alpha - \beta) \equiv
      \sin(\frac 12 \pi - \alpha)\cos \beta +
          \sin \beta \cos(\frac 12 \pi - \alpha) \\
  &\equiv
      \cos \alpha \cos \beta + \sin \beta \sin \alpha \qedhere
 \end{align*}
\end{proof}
\begin{proof}
 \begin{align*}
  \tan(\alpha + \beta) &\equiv
      \frac{\sin(\alpha + \beta)}{\cos(\alpha + \beta)} \equiv
      \frac{\sin \alpha \cos \beta + \sin \beta \cos \alpha}
           {\cos \alpha \cos \beta + \sin \alpha \sin \beta} \\
      &\equiv \frac{
            \frac{\sin \alpha \cos \beta}{\cos \alpha \cos \beta} +
            \frac{\sin \alpha \sin \beta}{\cos \alpha \cos \beta}}
           {\frac{\cos \alpha \cos \beta}{\cos \alpha \cos \beta} +
            \frac{\sin \alpha \sin \beta}{\cos \alpha \cos \beta}}
           \equiv
      \frac{\tan \alpha + \tan \beta}{1 - \tan \alpha \tan \beta} \qedhere
 \end{align*}
\end{proof}

\begin{theorem}[Compound angle formulae] \label{thm_trig_compound}
 Trigonometric functions of a sum or difference of angles can be expanded into
 trigonometric functions of each angle.
 \begin{align*}
  \sin \alpha \pm \beta &\equiv
     \sin \alpha \cos \beta \pm \sin \beta \cos \alpha \\
  \cos \alpha \pm \beta &\equiv
     \cos \alpha \cos \beta \mp \sin \alpha \sin \beta \\
  \tan \alpha \pm \beta &\equiv
      \frac{\tan \alpha \pm \tan \beta}{1 \mp \tan \alpha \tan \beta}
 \end{align*}
\end{theorem}
\begin{proof}
 Results for a function of a difference can be derived from the parity of
 each function (\ref{sec_trig_parity}), and rewriting \(\alpha - \beta\) as
 \(\alpha + (-\beta)\).
\end{proof}
One might also just assume it's probably OK to flip signs.

\subsubsection{Double Angle Formulae} \label{sec_trig_double_angle}
\begin{theorem}[Double angle formulae] \label{thm_trig_double_angle}
 Trigonometric functions of double angles can be expressed in terms of
 trigonometric functions of the angle.
 \begin{align*}
  \sin 2\theta &\equiv
     2\sin \theta \cos \theta \\
  \cos 2\theta &\equiv
     \cos^2 \theta - \sin^2 \theta \equiv
     2\cos^2 \theta - 1 \equiv 1 - 2\sin^2 \theta
     \ \text{(by \ref{sec_trig_pythag})} \\
  \tan 2\theta &\equiv
      \frac{2\tan \theta}{1 - \tan^2 \theta}
 \end{align*}
\end{theorem}
\begin{proof}
 These follow from \(2\theta \equiv \theta + \theta\) in combination with
 Theorem \ref{thm_trig_compound}.
\end{proof}
It follows, by rearrangement of identities in Theorem
\ref{thm_trig_double_angle}, that
\begin{theorem}[Squared half-angle identities]
 Squared trigonometric functions can be rewritten in terms of trigonometric
 functions of the double angle.
 \begin{alignat*} 3
  \sin^2 \theta &\equiv
      \frac{1 - \cos 2 \theta} 2 \\
  \cos^2 \theta &\equiv
      \frac{1 + \cos 2 \theta} 2
  \intertext{and}
  \tan^2 \theta &\equiv \frac{\sin^2 \theta}{\cos^2 \theta}
      \equiv \frac{1 - \cos 2 \theta}{1 + \cos 2 \theta}
      &&\equiv \frac{1 - \cos^2 2 \theta}{(1 + \cos 2 \theta)^2}
      \equiv \parens[\Big]{\frac{\sin 2 \theta}{1 + \cos 2 \theta}}^2 \\
  &   &&\equiv \frac{(1 - \cos 2 \theta)^2}{1 - \cos^2 2 \theta}
      \equiv \parens[\Big]{\frac{1 - \cos 2 \theta}{\sin 2 \theta}}^2
 \end{alignat*}
\end{theorem}
\begin{theorem}[Half angle identities] \label{thm_trig_half_angle}
 Trigonometric functions of half-angles can be expressed in terms of
 trigonometric functions of the full angle.
 \begin{alignat*}2
  &&\abs*{\sin \theta} &\equiv
      \sqrt{\frac{1 - \cos 2 \theta} 2} \\
  &\iff{}& \abs*{\sin \tfrac 12 \theta} &\equiv
      \sqrt{\frac{1 - \cos \theta} 2} \\
  &&\abs*{\cos \theta} &\equiv
      \sqrt{\frac{1 + \cos 2 \theta} 2} \\
  &\iff{}& \abs*{\cos \tfrac 12 \theta} &\equiv
      \sqrt{\frac{1 + \cos \theta} 2} \\
  &&\tan \theta &\equiv
      \frac{\sin 2 \theta}{1 + \cos 2 \theta}
      \equiv \frac{1 - \cos 2 \theta}{\sin 2 \theta} \\
  &\iff{}& \tan \tfrac 12 \theta &\equiv
      \frac{\sin \theta}{1 + \cos \theta}
      \equiv \frac{1 - \cos \theta}{\sin \theta}
 \end{alignat*}
 The modulus signs here are due to the sign changes between quadrants. They can
 often by eliminated by consideration of the domain of \(\theta\).
\end{theorem}

\subsubsection{Triple Angles and Beyond}

\subsection{Sum-Product Identities} \label{sec_trig_sum_product}

\begin{theorem}[Sum-Product Identities]
 The product of trigonometric functions can be rewritten as a sum:
 \begin{align*}
  \sin \alpha \cos \beta &\equiv
      \frac{\sin(\alpha + \beta) + \sin(\alpha - \beta)} 2 \\
  \cos \alpha \cos \beta &\equiv
   \frac{\cos(\alpha + \beta) + \cos(\alpha - \beta)} 2 \\
  \sin \alpha \sin \beta &\equiv
   \frac{\cos(\alpha - \beta) - \cos(\alpha + \beta)} 2
 \end{align*}
\end{theorem}
\begin{proof}
 In each case we can derive by applying compound angle formulae to the RHS:
 \begin{align*}
  \frac{\sin(\alpha + \beta) + \sin(\alpha - \beta)} 2 &\equiv
   \frac{\sin \alpha \cos \beta + \sin \beta \cos \alpha +
         \sin \alpha \cos \beta - \sin \beta \cos \alpha} 2 \\
  &\equiv \frac{2 \sin \alpha \cos \beta}
   2 \equiv \sin \alpha \cos \beta \\[3ex]
  \frac{\cos(\alpha + \beta) + \cos(\alpha - \beta)} 2 &\equiv
   \frac{\cos \alpha \cos \beta - \sin \alpha \sin \beta +
         \cos \alpha \cos \beta + \sin \alpha \sin \beta} 2 \\
  &\equiv \frac{2 \cos \alpha \cos \beta}
   2 \equiv \cos \alpha \cos \beta \\[3ex]
  \frac{\cos(\alpha - \beta) - \cos(\alpha + \beta)} 2 &\equiv
   \frac{\cos \alpha \cos \beta + \sin \alpha \sin \beta -
         \cos \alpha \cos \beta + \sin \alpha \sin \beta} 2 \\
  &\equiv \frac{2 \sin \alpha \sin \beta} 2 \equiv \sin \alpha \sin \beta
      \qedhere
 \end{align*}
\end{proof}

\subsubsection{Product-Sum Identities} \label{sec_trig_product_sum}

%FIXME: annotate sections used

\begin{theorem}[Sum-Product Identities]
 Similarly, the sums and differences of trigonometric functions can be rewritten
 as a product:
 \begin{align*}
  \sin \alpha + \sin \beta &\equiv
      2 \sin \frac{\alpha + \beta}2 \cos \frac{\alpha - \beta}2 \\
  \sin \alpha - \sin \beta &\equiv
      2 \sin \frac{\alpha - \beta} 2 \cos\frac{\alpha + \beta} 2 \\
  \cos \alpha + \cos \beta &\equiv
      2 \cos\frac{\alpha + \beta} 2 \cos \frac{\alpha - \beta} 2 \\
  \cos \alpha - \cos \beta &\equiv
   -2 \sin \frac{\alpha + \beta} 2 \sin \frac{\alpha - \beta} 2
 \end{align*}
 \end{theorem}
\begin{proof}
 We can derive the sums by considering \(\alpha' = \frac 12 (\alpha + \beta)\)
 and \(\beta' = \frac 12 (\alpha - \beta)\). Then,
 \begin{align*}
  \sin \alpha + \sin \beta &\equiv
   \sin(\alpha' + \beta') + \sin(\alpha' - \beta') \equiv
   2\sin \alpha' \cos \beta'\quad \text{(from \ref{sec_trig_sum_product})}
  \\&\equiv 2 \sin \frac{\alpha + \beta}2 \cos \frac{\alpha - \beta} 2 \\[3ex]
   \cos \alpha + \cos \beta &\equiv
   \cos(\alpha' + \beta') + \cos(\alpha' - \beta') \equiv
   2\cos \alpha' \cos \beta'\quad \text{(from \ref{sec_trig_sum_product})}
   \\&\equiv 2 \cos\frac{\alpha + \beta} 2 \cos \frac{\alpha - \beta} 2
 \end{align*}
 The difference of sines is quite straightforward to show using
 (\ref{sec_trig_parity}).
 \begin{equation*}
  \sin \alpha - \sin \beta =
      \sin \alpha + \sin -\beta =
      2 \sin \frac{\alpha - \beta} 2 \cos\frac{\alpha + \beta} 2
 \end{equation*}
 For the difference of cosines, can exploit the periodicity of cosine
 (\ref{sec_trig_periodic}).
 \begin{align*}
  \cos \alpha - \cos \beta &\equiv
  \cos \alpha + \cos (\beta + \pi)  \equiv
  2 \cos \frac{\alpha + \beta + \pi} 2 \cos \frac {\alpha - \beta - \pi} 2
  \\&\equiv
  2 \cos \parens[\Big]{\frac \pi 2 - \parens[\Big]{-\frac{\alpha + \beta} 2}}
    \cos \parens[\Big]{\frac \pi 2 - \frac{\alpha - \beta} 2} \equiv
  2 \sin -\frac{\alpha + \beta} 2 \sin \frac{\alpha - \beta} 2 \\&\equiv
  -2 \sin \frac{\alpha + \beta} 2 \sin \frac{\alpha - \beta} 2 \qedhere
 \end{align*}
\end{proof}

\subsection[The Weierstrass substitution
            \texorpdfstring{(\(\tan(\theta / 2)\))}{(tangent half-angle)}]
   {The Weierstrass substitution \boldmath\(\parens[\Big]{\tan(\tfrac 12 \theta)}\)}

\begin{theorem}[The Weierstrass substitution]
 If we let \(t = \tan \tfrac 12 \theta\), such that \(t\) is defined (ie
 \(\NExists n \in \Integers: \theta = (2n + 1)\pi\)) then we have:
 \begin{align*}
  \sin \theta &\equiv \frac{2t}{1 + t^2} \\
  \cos \theta &\equiv \frac{1 - t^2}{1 + t^2} \\
  \tan \theta &\equiv \frac{2t}{1 - t^2} \\
  \dv<\theta>{t} &\equiv \frac 2{1 + t^2}
 \end{align*}
\end{theorem}
\begin{proof}
 The identity for the tangent is an immediate consequence of the double
 angle formula for the tangent (\ref{sec_trig_double_angle}), where we're
 considering \(\tan{(2 \cdot \frac 12 \theta)}\).

 The identity for the sine can be derived by considering the double angle
 formula for sine, and using a smattering of identities from sections
 \ref{sec_trig_definitions}, \ref{sec_trig_pythag}:
 \begin{align*}
  \sin \theta &\equiv 2\sin \tfrac 12 \theta \cos\tfrac 12 \theta
      \equiv 2\tan \tfrac 12 \theta \cos^2 \tfrac 12 \theta \\
      &\equiv \frac{2\tan \tfrac 12 \theta}{\sec^2 \tfrac 12 \theta}
      \equiv \frac{2t}{1 + t^2}
 \end{align*}
 Similarly for the cosine:
 \begin{align*}
  \cos \theta
      &\equiv \cos^2 \tfrac 12 \theta - \sin^2 \tfrac 12 \theta
      \equiv \cos^2 \tfrac 12 \theta
          (1 - \tan^2 \tfrac 12 \theta) \\
      &\equiv \frac{1 - \tan^2 \tfrac 12 \theta}
                   {\sec^2 \tfrac 12 \theta}
      \equiv \frac{1 - t^2}{1 + t^2} \qedhere
 \end{align*}
\end{proof}

This fact is useful for a number of reasons. Firstly, when integrating a
rational function of trig functions, we can apply this trick to transform
the integral into a rational function of \(t\), as all trig functions and
also \(\diff \theta\) become rational functions of \(t\).

It can also be used as an alternative parametrisation of a circle: instead
of
\begin{equation*}
 (x, y) = (\cos\theta, \sin\theta): \theta \in \intco{0, 2\pi}
\end{equation*}
we can use
\begin{equation*}
 (x, y) = \parens[\Big]{\frac{1 - t^2}{1 + t^2}, \frac{2t}{1 + t^2}}: t \in \Reals
\end{equation*}
This is useful as it is a rational function, so for rational \(t\) it gives
a rational point on the unit circle, which can be used, for example, to find
Pythagorean triples. The only point it misses is \((0, 1)\).

\subsubsection{Heuristic for sine and cosine}

%FIXME add graphic.

The formula involving the tangent can be easily derived from the double
angle formula. Having this, we can draw a right triangle with angle
\(\theta\), opposite side \(2t\) and adjacent side \(1 - t^2\). We can
deduce from Pythagoras' theorem (\ref{sec_pythagoras}) that the length of
the hypotenuse is
\begin{equation*}
 \sqrt{(2t)^2 + (1 - t^2)^2} = \sqrt{1 + 2t^2 + t^4} = 1 + t^2
\end{equation*}
and then deduce the magnitudes of sine and cosine from their definitions.
Fortuitously, it turns out we can immediately drop any absolute value
signs, making this trick quite useful.

 \section{Limits}

\subsection{Definition of a Limit}

\subsection[Euler's number \texorpdfstring{\(e\)}{e}]
           {Euler's number \boldmath\(e\)} \label{sec_e}

\(e\) is often defined as

\subsection{L'H\^opital's Rule}

 \section{Differentiation}

%FIXME: derivations from first principles

\subsection{Definition of the derivative}

The derivative of \(f(x)\) gives the gradient of the curve \(y = f(x)\) at
\(x\). It is defined as
\begin{equation}
 f'(x) = \lim_{h \to 0} \frac{f(x + h) - f(x)}h
\end{equation}
Other notations include, if \(y = f(x)\), then
\begin{equation}
 f'(x) = \dv{x}(f(x)) = \dv<y>{x} = y'
\end{equation}

There is also various notation for repeated differentiation. If
\(g(x) = f'(x)\), then \(g'(x) = f''(x)\).
\begin{equation*}
 f''(x) = \dv[2]<y>{x} = \dv[2]{x}(f(x))
\end{equation*}
In general, differentiating \(y = f(x)\) \(n\) times, we have the \(n\)th
derivative
\begin{equation*}
 f^{\overbrace{\prime\prime\prime\dotsb}^{\text{\(n\) times}}}
     = f^{(n)}(x) = \dv[n]<y>{x} = \dv[n]{x}(f(x))
\end{equation*}

\subsection{Basic properties of the derivative}
\label{sec_calc_derivative_properties}

Some basic properties of limits result in corresponding basic properties of
the derivative.
\begin{equation*}
 \dv{x}(a f(x)) = af'(x) \quad \text{where \(a\) is a constant}
\end{equation*}
\begin{equation*}
 \dv{x}(f(x) + g(x)) = f'(x) + g'(x)
\end{equation*}

\subsection{Chain Rule} \label{sec_calc_chain}

\subsection{Product Rule} \label{sec_calc_product}

\begin{theorem}[Product rule]
 For two differentiable functions \(f, g\),
 \begin{equation*}
  \dv{x} (f(x)g(x)) = f(x)g'(x) + f'(x)g(x)
 \end{equation*}
\end{theorem}
\begin{proof}[Proof from first principles]
 This can be derived from the definition of the derivative. Let
 \begin{equation*}
  h(x) = f(x)g(x)
 \end{equation*}
 Then, by definition,
 \begin{align*}
  h'(x) &= \lim_{t \to 0} \frac{h(x + t) - h(x)} t \\
        &= \lim_{t \to 0} \frac{f(x + t)g(x + t) - f(x)g(x)} t \\
        &= \lim_{t \to 0} \frac{f(x + t)g(x + t) - f(x + t)g(x)
                              + f(x + t)g(x) - f(x)g(x)} t \\
        &= \lim_{t \to 0} \bracks[\bigg]{
            f(x + t) \frac{g(x + t) - g(x)} t
          + g(x) \frac{f(x + t) - f(x)} t} \\
        &= \lim_{t \to 0} \bracks[\bigg]{
            f(x + t)g'(x)
          + g(x) f'(x)} \quad \text{by definition} \\
        &= f(x)g'(x) + g(x)f'(x)
  \qedhere
 \end{align*}
\end{proof}
\begin{proof}[Proof from the chain rule]
 We can alternatively apply the chain rule after completing the square of
 \(f(x)g(x)\) by writing
 \begin{alignat*}{2}
  && f(x)g(x) &= \frac 12 \bracks[\Big]
                  {(f(x))^2 + (g(x))^2 - (f(x) - g(x))^2} \\
  &\implies{}& \dv{x} (f(x)g(x))
              &= \frac 12 \dv{x} \bracks[\Big]
                  {(f(x))^2 + (g(x))^2 - (f(x) - g(x))^2} \\
  &&          &= \frac 12 \bracks[\Big]
                  {2f(x)f'(x) + 2g(x)g'(x)
                 - 2(f(x) - g(x))(f'(x) - g'(x))} \\
  &&          &= f(x)f'(x) + g(x)g'(x)
               - (f(x)f'(x) + g(x)g'(x) - f'(x)g(x) - f(x)g'(x)) \\
  &&          &= f'(x)g(x) + f(x)g'(x) \qedhere
 \end{alignat*}
\end{proof}
\begin{proof}[Proof by exponentials]
 Yet another way to show this is to rewrite \(f(x)g(x)\) in terms of
 exponentials, eliminating the product so we can again apply the chain rule.
 \begin{alignat*}{2}
  && f(x)g(x) &= e^{\ln f(x) + \ln g(x)} \\
  &\implies{}& \dv{x} (f(x)g(x))
              &= \dv{x} e^{\ln f(x) + \ln g(x)} \\
  &&          &= \parens[\bigg]{\frac{f'(x)}{f(x)} + \frac{g'(x)}{g(x)}}
                   e^{\ln f(x) + \ln g(x)} \\
  &&          &= \parens[\bigg]{\frac{f'(x)}{f(x)} + \frac{g'(x)}{g(x)}}
                   f(x)g(x) \\
  &&          &= f'(x)g(x) + f(x)g'(x)
 \end{alignat*}
 Now a little more care must be taken, as this does assume that \(\ln f(x)\)
 and \(\ln g(x)\) exist. This will be true if both \(f\) and \(g\) are
 positive at \(x\). If either or both is negative, we can factor out a
 \(-1\), and instead continue with the proof using \(-f(x)\) or \(-g(x)\).
 This minus will then eliminate. Suppose wlog that \(f(x) < 0\):
 \begin{alignat*}{2}
  && f(x)g(x) &= -e^{\ln(-f(x)) + \ln g(x)} \\
  &\implies{}& \dv{x} (f(x)g(x))
              &= -\dv{x} e^{\ln(-f(x)) + \ln g(x)} \\
  &&          &= -\parens[\bigg]{\frac{-f'(x)}{-f(x)}
                               + \frac{g'(x)}{g(x)}}
                   e^{\ln(-f(x)) + \ln g(x)} \\
  &&          &= -\parens[\bigg]{\frac{f'(x)}{f(x)} + \frac{g'(x)}{g(x)}}
                   \parens[\big]{-f(x)g(x)} \\
  &&          &= \parens[\bigg]{\frac{f'(x)}{f(x)} + \frac{g'(x)}{g(x)}}
                   (f(x)g(x)) \\
  &&          &= f'(x)g(x) + f(x)g'(x) \qedhere
 \end{alignat*}
\end{proof}

\subsubsection{Quotient Rule} \label{sec_calc_quotient}

\begin{theorem}[Quotient rule]
 For two differentiable functions \(f, g\),
 \begin{equation*}
  \dv{x}\parens[\bigg]{\frac{f(x)}{g(x)}} = \frac{g(x)f'(x) - f(x)g'(x)}{(g(x))^2}
 \end{equation*}
\end{theorem}
There are a few ways to prove this. We could reappropriate some of the arguments
made in the proof of the product rule (\ref{sec_calc_product}), but it is much
more straightforward to use the product rule as a stepping stone.
\begin{proof}[Proof by explicit differentiation]
 Let
 \begin{alignat*}{2}
  && h(x) &= \frac{f(x)}{g(x)} \\
  &&      &= f(x) \cdot \frac 1{g(x)} \\
  &\implies{}& h'(x)
          &= f'(x) \cdot \frac 1{g(x)}
              + f(x) \cdot \parens[\Bigg]{\frac{-g'(x)}{(g(x))^2}}
              \quad \text{from the product and chain rules} \\
  &&      &= \frac{f'(x)}{g(x)} - \frac{f(x)g'(x)}{(g(x))^2} \\
  &&      &= \frac{f'(x)g(x) - f(x)g'(x)}{(g(x))^2} \qedhere
 \end{alignat*}
\end{proof}
\begin{proof}[Proof by implicit differentiation]
 Let
 \begin{alignat*}{2}
  && h(x) &= \frac{f(x)}{g(x)} \\
  &\implies{}& h(x)g(x) &= f(x) \\
  &\implies{}& h(x)g'(x) + h'(x)g(x)
          &= f'(x) \quad \text{from the product and chain rules} \\
  &\implies{}& h'(x)
          &= \frac{f'(x) - h(x)g'(x)}{g(x)} \\
  &&      &= \frac{f'(x) - \frac{f(x)}{g(x)}g'(x)}{g(x)}
              \quad \text{by definition of \(h\)} \\
  &&      &= \frac{f'(x)g(x) - f(x)g'(x)}{(g(x))^2} \qedhere
 \end{alignat*}
\end{proof}

\subsection{Common derivatives} \label{calc_common}

Differentiation is, informally speaking, an easy problem. Any elementary
function, which is to say any of the functions in table
\ref{tab_calc_derivatives}, or any composition of elementary functions, is
differentiable to an elementary function, and basically the algorithm to do
so just consists of identifying the form of a function as some composition
of other subexpressions, finding the derivative in terms of these
subexpressions and their derivatives, and continuing to recurse down
subexpressions until they are exhausted.

Some of the more common rules for taking the derivative of a composition of
unspecified functions are shown in table
\ref{tab_calc_functions_derivatives}. The most essential ones are marked
with \(\note\). All the other ones can be derived using only these
rules, and the derivatives of some functions in table
\ref{tab_calc_derivatives}.

\begin{longtable}{*4M c}
 \toprule
 \text{\boldmath\(f(x)\)} & \text{\boldmath\(f'(x)\)}
     & \text{\bfseries Alternatives/Notes} & \note & \bfseries ref \\
 \midrule
 \endhead
 \bottomrule
 \endfoot
 \endlastfoot
 g(x) + h(x) & g'(x) + h'(x) & & \note
  & \ref{sec_calc_derivative_properties} \\[1ex]
 a g(x) & a g'(x) & & \note
  & \ref{sec_calc_derivative_properties} \\[1ex]
 g(x) h(x) & g'(x) h(x) + g(x) h'(x)
  & & \note & \ref{sec_calc_product} \\[1ex]
 g(h(x)) & h'(x) g'(h(x)) & & \note & \ref{sec_calc_chain} \\[1ex]
 \frac{g(x)}{h(x)} & \frac{h(x) g'(x) - g(x) h'(x)}{(h(x))^2}
  &&& \ref{sec_calc_quotient} \\[3ex]
 g(ax + b) & a g'(ax + b) \\[3ex]
 g(x)^{h(x)}
  & \parens[\bigg]{\frac{h(x)g'(x)}{g(x)} + h'(x)\ln(g(x))}g(x)^{h(x)}
  & g(x) > 0 && \ref{sec_calc_powers} \\[3ex]
 \sqrt{g(x)} & \frac{g'(x)}{2\sqrt{g(x)}} & g(x) \ge 0 \\[3ex]
 \ln(g(x)) & \frac{g'(x)}{g(x)} & g(x) > 0 \\[3ex]
 \log_a(g(x)) & \frac{g'(x)}{g(x) \ln a} & g(x) > 0 \\[3ex]
 \log_{g(x)} a & \frac{-\ln a}{g(x) (\ln(g(x)))^2}
  & \text{const \(a > 0\), \(g(x) > 0\)} \\[3ex]
  \log_{g(x)} h(x) & \frac 1{(\ln(g(x)))^2}\parens[\bigg]{\frac{\ln(g(x))}{h(x)}
                                           - \frac{\ln(h(x))}{g(x)}}
  & g(x), h(x) > 0 \\[3ex]
 \bottomrule
 \caption{General derivatives of compositions of functions
  \label{tab_calc_functions_derivatives}}
\end{longtable}

In table \ref{tab_calc_derivatives} is a set of common derivatives of
functions. Note that by FTC (\ref{sec_calc_FTC}), these also give a number
of common integrations, although these are reiterated in table
\ref{tab_calc_int_funcs}.

A large number of these can be readily derived from preceding entries. The
most essential ones are marked with a \(\note\). The further down the table
you get, the more obscure the functions get. For most, it can save time to
know the derivatives, but people obviously have their priorities and often
it's more worthwile to understand how to derive the derivative, anyway.

\begin{longtable}{*4M c}
 \toprule
 \text{\boldmath\(f(x)\)} & \text{\boldmath\(f'(x)\)}
     & \text{\bfseries Alternatives/Notes} & \note & \bfseries ref \\
 \midrule
 \endhead
 \bottomrule
 \endfoot
 \endlastfoot
 a & 0 & \text{\(a\) is constant} & \note \\[1ex]
 x^n & nx^{n-1} & \text{\(n\) is constant} & \note
  & \ref{sec_calc_powers} \\[1ex]
 e^x & e^x & \text{\(e\) is Euler's number} & \note & \ref{sec_e} \\[1ex]
 a^x & \ln a \cdot a^x & \text{\(a\) is constant}
  && \ref{sec_calc_powers}\\[1ex]
 \ln x & \frac 1{x} && \note \\[3ex]
 \log_a x & \frac 1{x\ln a} & \text{\(a\) is a constant} \\[3ex]
 \sin x & \cos x && \note \\[1ex]
 \cos x & -\sin x && \note & \ref{sec_calc_trig_basic} \\[1ex]
 \tan x & \sec^2 x & \tan^2 x + 1 && \ref{sec_calc_trig_basic} \\[1ex]
 \sec x & \sec x \tan x &&& \ref{sec_calc_trig_basic} \\[1ex]
 \csc x & -\csc x \cot x &&& \ref{sec_calc_trig_basic} \\[1ex]
 \cot x & -\csc^2 x & -(\cot^2 x + 1) && \ref{sec_calc_trig_basic} \\[1ex]
 \arcsin x & \frac 1{\sqrt{1 - x^2}} && \note & \ref{sec_calc_trig_inv} \\[3ex]
 \arccos x & \frac{-1}{\sqrt{1 - x^2}} && \note & \ref{sec_calc_trig_inv} \\[3ex]
 \arctan x & \frac 1{x^2 + 1} && \note & \ref{sec_calc_trig_inv} \\[3ex]
 \arcsec x & \frac 1{\abs x \sqrt{x^2 - 1}}
  & \frac 1{x^2 \sqrt{1 - x^{-2}}} && \ref{sec_calc_trig_inv} \\[3ex]
 \arccsc x & \frac{-1}{\abs x \sqrt{x^2 - 1}}
  & \frac{-1}{x^2 \sqrt{1 - x^{-2}}} && \ref{sec_calc_trig_inv} \\[3ex]
 \arccot x & -\frac 1{x^2 + 1} &&& \ref{sec_calc_trig_inv} \\[3ex]
 \sinh x & \cosh x \\[1ex]
 \cosh x & \sinh x \\[1ex]
 \tanh x & \sech^2 x & -\tanh^2 x + 1 \\[1ex]
 \sech x & -\tanh x \sech x \\[1ex]
 \csch x & -\coth x \csch x \\[1ex]
 \coth x & -\csch^2 x & -(\coth^2 x + 1) \\[1ex]
 \arcsinh x & \frac 1{\sqrt{x^2 + 1}} \\[3ex]
 \arccosh x & \frac 1{\sqrt{x^2 - 1}} \\[3ex]
 \arctanh x & \frac 1{1 - x^2} \\[3ex]
 \arcsech x & \frac{-1}{x\sqrt{1 - x^2}} \\[3ex]
 \arccsch x & \frac{-1}{\abs x\sqrt{1 + x^2}} \\[3ex]
 \arccoth x & \frac 1{1 - x^2} \\[3ex]
 \bottomrule
 \caption{Common derivatives} \label{tab_calc_derivatives}
\end{longtable}

\subsubsection{Derivatives with powers} \label{sec_calc_powers}

Often when being introduced to differentiation, derivatives like that of
\(x^2\) are demonstrated from first principles, as follows:
\begin{align*}
 \dv{x}(x^2) &= \lim_{h \to 0} \frac{(x + h)^2 - x^2} h \\
             &= \lim_{h \to 0} \frac{2hx + h^2} h \\
             &= \lim_{h \to 0} (2x + h) \\
             &= 2x
\end{align*}
We can do this for a general positive integer power of \(x\),
\(n \in \Integers^+\) by using the binomial theorem:
\begin{align*}
 \dv{x}(x^n) &= \lim_{h \to 0} \frac{(x + h)^n - x^n} h \\
             &= \lim_{h \to 0} \bracks[\bigg]{\frac 1h \parens[\Big]{-x^n
                     + \sum_{r = 0}^n \binom nr x^r h^{n - r}}} \\
             &= \lim_{h \to 0} \bracks[\bigg]{\frac 1h \parens[\Big]{
                       -x^n + \binom n0 x^n + \binom n1 hx^{n - 1}
                       + \sum_{r = 0}^{n - 2}
               \binom nr x^r h^{n - r}}} \\
             &= \lim_{h \to 0} \bracks[\bigg]{\frac 1h \parens[\Big]{-x^n + x^n
                         + hnx^{n - 1} + h^2 \sum_{r = 0}^{n - 2}
               \binom nr x^r h^{n - (r + 2)}}} \\
             &= \lim_{h \to 0} \bracks[\bigg]{nx^{n - 1} + h \sum_{r = 0}^{n - 2}
               \binom nr x^r h^{n - (r + 2)}} \\
             &= nx^{n-1}\quad
               \text{as the smallest power of \(h\) inside the sum is 0}
\end{align*}
Now if we have some negative integer exponent \(-n\) where
\(n \in \Integers^+\), we can use implicit differentiation to show the rule
still holds. Letting \(y = x^{-n}\),
\begin{alignat*}{2}
 &\implies{}& yx^n &= 1 \\
 &\implies{}& nx^{n - 1}y + x^n \dv<y>{x} &= 0 \\
 &\implies{}& \dv<y>{x} &= -nx^{n - 1}yx^{-n}
     = -nx^{-1}x^{-n} = -nx^{-n - 1}
\end{alignat*}
We can then use a little implicit differentiation to carry this onto any
rational exponent \(p / q \in \Rationals\). We let \(y = x^{p / q}\)
\begin{alignat*}{2}
 &\implies{}& y^q &= x^p \\
 &\implies{}& qy^{q - 1}\dv<y>{x} &= px^{p - 1} \\
 &\implies{}& \dv<y>{x} &= \frac pq x^{p - 1}y^{1 - q}
     = \frac pq x^{p - 1}x^{(p / q)(1 - q)} \\
 &&  &= \frac pq x^{p - 1 + (p / q) - p}
     = \frac pq x^{(p / q) - 1}
\end{alignat*}

We can show that this power rule generalizes to any real exponent \(k \in
\Reals\), for \(x > 0\), using the chain rule and the derivative of \(e^x\).
\begin{equation*}
 \dv{x}(x^k) = \dv{x}(e^{k\ln x}) = \frac kx e^{k\ln x}
             = \frac kx x^k = k x^{k - 1}
\end{equation*}
Unfortunately there's no straightforward way to even define exponentiation as a
differentiable function with a negative base for a general real exponent.

To differentiate an exponential function of a positive constant
\(a \in \Reals^+\), we can rewrite it as a power of \(e\), and use the chain
rule.
\begin{equation*}
 \dv{x}(a^x) = \dv{x}(e^{(\ln a)x}) = \ln a \cdot e^{(\ln a)x}
             = \ln a \cdot a^x
\end{equation*}
Lastly, if we have a positive function to the power of another function,
\(g(x)^{h(x)}\) where \(g(x) > 0\), we do a similar rewrite in terms of
\(e\).
\begin{align*}
 \dv{x}\parens[\Big]{g(x)^{h(x)}}
                       &= \dv{x}\parens[\Big]{e^{(\ln(g(x)))h(x)}} \\
                       &= \parens[\Big]{\frac{h(x)g'(x)}{g(x)}
                              + h'(x)\ln(g(x))}e^{(\ln(g(x)))h(x)} \\
                       &= \parens[\Big]{\frac{h(x)g'(x)}{g(x)}
                              + h'(x)\ln(g(x))}g(x)^{h(x)}
\end{align*}

\subsubsection{Basic trigonometric functions} \label{sec_calc_trig_basic}

%FIXME decide how to get around this

For now, we assume that \(\dv{x}(\sin x) = \cos x\). This can
be shown in a number of ways. There are geometric arguments, but we could
also appeal to the definition of sine and cosine in terms of complex
exponentials.

We can deduce from the chain rule that
\begin{equation*}
 \dv{x}(\cos x) = \dv{x}(\sin(\tfrac 12 \pi - x))
                = -\cos(\tfrac 12 \pi - x) = -\sin(x)
\end{equation*}
Now we can use the quotient rule for the tangent.
\begin{equation*}
 \dv{x}(\tan x) = \dv{x}\parens[\Big]{\frac{\sin x}{\cos x}}
                = \frac{\cos x \cos x + \sin x \sin x}{\cos^2 x}
                = \frac 1{\cos^2 x} = \sec^2 x
\end{equation*}
We can use the chain rule for the reciprocal functions.
\begin{equation*}
 \dv{x}(\sec x) = \dv{x}\parens[\Big]{\frac 1{\cos x}}
                = -\frac{-\sin x}{\cos^2 x} = \tan x \sec x
\end{equation*}
\begin{equation*}
 \dv{x}(\csc x) = \dv{x}\parens[\Big]{\frac 1{\sin x}}
                = -\frac{\cos x}{\sin^2 x} = -\cot x \csc x
\end{equation*}
The quotient rule can again be applied for the cotangent.
\begin{equation*}
 \dv{x}(\cot x) = \dv{x}\parens[\Big]{\frac{\cos x}{\sin x}}
                = \frac{-\sin x \sin x - \cos x \cos x}{\sin^2 x}
                = -\frac 1{\sin^2 x} = -\csc^2 x
\end{equation*}

\subsubsection{Inverse trigonometric functions} \label{sec_calc_trig_inv}

We can use implicit differentiation and some identities to tackle
derivatives of inverse trig functions. If we let \(y = \arcsin x\), then
\begin{equation*}
 \sin y = x \implies \dv<y>{x} \cos y = 1
     \implies \dv{x}(\arcsin x) = \frac 1{\cos y} = \frac 1{\sqrt{1 - x^2}}
\end{equation*}
This is allowed because the domain of \(\arcsin x\) is restricted to
\(\intcc{-1, 1}\), meaning that the range of \(y\) is
\(\intcc{-\frac 12 \pi, \frac 12 \pi}\) and \(\cos y > 0\)
(so \(\cos y = \abs{\cos y} = \sqrt{1 - \sin^2 y}\)).

Now, for the inverse cosine, rather than repeating this whole song and
dance, we can use the fact that
\(\arccos x \equiv \frac 12 \pi - \arcsin x\).
\begin{equation*}
 \dv{x}(\arccos x) = \dv{x}(\frac \pi 2 - \arcsin x)
     = -\frac 1{\sqrt{1 - x^2}}
\end{equation*}
For the inverse tangent, we again let \(y = \arctan x\), so that
\begin{equation*}
 \tan y = x \implies \dv<y>{x}(\tan^2 y + 1) = 1
     \implies \dv{x}(\arctan x) = \frac 1{1 + \tan^2 y}
         = \frac 1{1 + x^2}
\end{equation*}
For the inverse secant, we let \(y = \arcsec x\), so that
\begin{equation*}
 \sec y = x \implies \dv<y>{x} \sec y \tan y = 1
     \implies \dv{x}(\arcsec x) = \frac 1{x \tan y}
\end{equation*}
Unfortunately, we can't just take \(\tan y = \sqrt{x^2 - 1}\) as
\(x \in \Reals \setminus \intoo{-1, 1}\), and
\(y \in \intcc{0, \pi} \setminus \set{\frac 12 \pi}\). This means that
\(\tan y > 0\) for
\(y \in \intco{0, \frac 12 \pi} \iff x \in \intco{1, \infty}\), but also
that \(\tan y < 0\) for
\(y \in \intoc{\frac 12 \pi, \pi} \iff x \in \intoc{-\infty, -1}\).
Therefore, \(\tan y\) has the same sign as \(x\), so we can rewrite
\begin{equation*}
 \dv{x}(\arcsec x) = \frac 1{x \tan y} = \frac 1{\abs{x \tan y}}
     = \frac 1{\abs x\abs{\tan y}} = \frac 1{\abs x\sqrt{x^2 - 1}}
\end{equation*}
This is perhaps a little tricky, and it is in some cases more obvious to
just go about it piecewise:
\begin{equation*}
 \dv{x}(\arcsec x) =
 \begin{dcases}
  \frac 1{x\sqrt{x^2 - 1}} & x \ge 0 \\[3ex]
  \frac {-1}{x\sqrt{x^2 - 1}} & x < 0
 \end{dcases}
\end{equation*}
The other equivalent can be found by taking out a factor of \(\abs x^2\)
from the square root, to cancel the absolute sign by squaring it. This can
be useful sometimes.
\begin{equation*}
 \dv{x}(\arcsec x) = \frac 1{\abs x\sqrt{x^2 - 1}}
     = \frac 1{\abs x\sqrt{x^2}\sqrt{1 - x^{-2}}}
     = \frac 1{x^2\sqrt{1 - x^{-2}}}
\end{equation*}
To attack the inverse cosecant, it will be order of magnitude easier to just
make use of the fact that \(\arccsc x = \frac 12 \pi - \arcsec x\),
so that
\begin{equation*}
 \dv{x}(\arccsc x) = \dv{x}\parens[\Big]{\frac \pi 2 - \arcsec x}
                   = -\dv{x}(\arcsec x)
\end{equation*}
and then taking your pick of your favourite inverse secant derivative.

Similarly, \(\arccot x = \frac 12 \pi - \arctan x\), so
\begin{equation*}
 \dv{x}(\arccot x) = \dv{x}\parens[\Big]{\frac \pi 2 - \arctan x}
                   = -\frac 1{1 + x^2}
\end{equation*}

\subsection{Implicit differentiation}

% fix this

%FIXME finish this

 \section{Integration}

\subsection{Definition of a Riemann integral}

\subsection{Fundamental Theorem of Calculus} \label{sec_calc_FTC}

\subsection{Integration by Substitution} \label{sec_calc_substitution}

Integration by substitution is effectively the chain rule
(\ref{sec_calc_chain}) in reverse. If we consider the function
\begin{equation*}
 f(x) = g(h(x)) \implies f'(x) = h'(x) g'(h(x))
\end{equation*}
then the integral
\begin{equation*}
 \integ{h'(x) g'(h(x))}{x} = \integ{f'(x)}{x} = f(x) + C
\end{equation*}

\subsubsection{Trig Substitution} \label{sec_calc_trig_substitution}

\subsection{Integration by Parts}

\begin{theorem}[Integration by parts]
 \begin{equation*}
  \integ{g'(x)h(x)}{x} = g(x)h(x) - \integ{g(x)h'(x)}{x} + C
 \end{equation*}
\end{theorem}
\begin{proof}
 We can derive this from the product rule.
 (\ref{sec_calc_product}).  Consider the function
 \begin{equation*}
  f(x) = g(x) h(x) \implies f'(x) = g'(x) h(x) + g(x) h'(x)
 \end{equation*}
 then, by integrating both sides, and applying the distributivity of
 integration over addition,
 \begin{alignat*} 2
  && \integ{f'(x)}{x} &= \integ{(g'(x)h(x) + g(x)h'(x))}{x} + C \\
  &\implies{}& f(x) &= \integ{g'(x)h(x)}{x} + \integ{g(x)h'(x)}{x} + C \\
  &\implies{}& f(x) &= \integ{g'(x)h(x)}{x} + \integ{g(x)h'(x)}{x} + C \\
  &\implies{}& g(x)h(x)
      &= \integ{g'(x)h(x)}{x} + \integ{g(x)h'(x)}{x} + C \\
  &\implies{}& \integ{g'(x)h(x)}{x}
      &= g(x)h(x) - \integ{g(x)h'(x)}{x} + C \qedhere \\
 \end{alignat*}
\end{proof}

\subsection{Common integrals}

Similarly to the derivatives section, there are a number of slightly differing
tables in this section, and a number of subsection providing working. The first,
Table \ref{tab_calc_int_funcs}, gives antiderivatives of forms of composed
functions.  These are mostly obtained by reversing table
\ref{tab_calc_functions_derivatives}, using FTC.

Table \ref{tab_calc_int_common} gives the antiderivatives of some specific
functions, sometimes using FTC, sometimes by other techniques.

Table \ref{tab_calc_int_uncommon} gives further antiderivatives of oddly
specific functions.

Table \ref{tab_calc_int_uncommon_def} goes even further, giving examples of
interesting integrals that are only of interest when evaluated over a definite
interval.

In each table, again, particularly important rows are denoted \(\note\).

\begin{longtable}{*4M c}
 \toprule
 \text{\boldmath\(f(x)\)}
     & \text{\boldmath\(\integ{f(x)}{x}\ ({}+{} C)\)}
     & \text{\bfseries Alternatives/Notes} & \note & \bfseries Reference\\
 \midrule
 \endhead
 \bottomrule
 \endfoot
 \endlastfoot
 g'(x) & g(x) \\[1ex]
     g(x) + h(x) & \integ{g(x)}{x} + \integ{h(x)}{x} \\[2ex]
     a g(x) & a\integ{g(x)}{x} & \text{\(a\) is a constant} \\[2ex]
     g(ax + b) & \frac 1a \integ{g(ax + b)}{x}
     & \text{\(a, b\) are constant} \\[2ex]
 g'(h(x)) h'(x) & g(h(x)) \\[1ex]
     g'(x) h(x) & g(x) h(x) - \integ{g(x) h'(x)}{x} \\[2ex]
 \frac{g'(x)}{g(x)} & \ln \abs{g(x)} \\[3ex]
 \frac{g'(x)}{\sqrt{g(x)}} & 2\sqrt{g(x)} \\[3ex]
 \bottomrule
 \caption{Antiderivatives of general forms of functions
  \label{tab_calc_int_funcs}}
\end{longtable}

\begin{longtable}{*4M c}
 \toprule
 \text{\boldmath\(f(x)\)}
     & \text{\boldmath\(\integ{f(x)}{x}\ (+ C)\)}
     & \text{\bfseries Alternatives/Notes} & \note & \bfseries Reference\\
 \midrule
 \endhead
 \bottomrule
 \endfoot
 \endlastfoot
 x^n & \frac 1{n + 1}x^{n + 1} & n \ne -1 \\[3ex]
 \frac 1x & \ln \abs x \\[3ex]
 e^x & e^x & \text{\(e\) is Euler's constant} \\[1ex]
 a^x & \frac 1{\ln a}a^x & \text{\(a\) is a constant} \\[3ex]
 \sin x & -\cos x \\[1ex]
 \cos x & \sin x \\[1ex]
 \tan x & -\ln \abs{\cos x} & \ln \abs{\sec x} \\[1ex]
 \sec x & \ln \abs{\sec x + \tan x}
     & \ln \abs*{\tan{(\tfrac 12 x + \tfrac 14 \pi)}} \\[2ex]
 \csc x & \ln \abs{\csc x - \cot x} & \ln \abs*{\tan{(\tfrac 12 x)}} \\[2ex]
 \cot x & \ln \abs{\sin x} \\[1ex]
 \sin^2 x & \tfrac 12 x - \tfrac 14 \sin 2x \\[1ex]
 \cos^2 x & \tfrac 12 x + \tfrac 14 \sin 2x \\[1ex]
 \tan^2 x & \tan x - x \\[1ex]
 \sec^2 x & \tan x \\[1ex]
 \csc^2 x & -\cot x \\[1ex]
 \cot^2 x & -\cot x - x \\[1ex]
 \bottomrule
 \caption{Common antiderivatives
  \label{tab_calc_int_common}}
\end{longtable}

\begin{longtable}{*4M c}
 \toprule
 \text{\boldmath\(f(x)\)}
     & \text{\boldmath\(\integ{f(x)}{x}\ (+ C)\)}
     & \text{\bfseries Alternatives/Notes} & \note & \bfseries Reference\\
 \midrule
 \endhead
 \bottomrule
 \endfoot
 \endlastfoot
 \frac{\cos x}{\sin x + \cos x}
     & \tfrac 12 x + \tfrac 12 \ln \abs{\sin x + \cos x} \\[3ex]
 \frac{\sin x}{\sin x + \cos x}
     & \tfrac 12 x - \tfrac 12 \ln \abs{\sin x + \cos x} \\[3ex]
 \bottomrule
 \caption{Niche but interesting antiderivatives
  \label{tab_calc_int_uncommon}}
\end{longtable}

\begin{longtable}{*6M c}
 \toprule
 \text{\boldmath\(f(x)\)}
     & \text{\boldmath\(x_1\)} & \text{\boldmath\(x_2\)}
     & \text{\boldmath\(\integ[x_1]<x_2>{f(x)}{x}\)}
     & \text{\bfseries Notes} & \note & \bfseries Reference\\
 \midrule
 \endhead
 \bottomrule
 \endfoot
 \endlastfoot
 e^{-x^2} & -\infty & \infty & \sqrt \pi && \note \\[1ex]
 \frac{f(ax) - f(bx)} x & 0 & \infty & (f(\infty) - f(0)) \ln \frac ab
     & \text{\(a, b\) const.} && \ref{sec_calc_frullani} \\[3ex]
 \frac{f(x)}{f(x) + f(a + b - x)} & a & b & \frac{b - a} 2
     & \text{\(a, b\) const.} \\[3ex]
 \frac{f(x)}{x(f(x) + f(ab / x))} & a & b & \frac 12 \ln \frac ba
     & \text{\(a, b\) const.} \\[3ex]
 \bottomrule
 \caption{Very niche but interesting definite integrals
  \label{tab_calc_int_uncommon_def}}
\end{longtable}

% TODO: TBC
\begin{longtable}{*2M c}
 \toprule
 \text{\boldmath\(I_n\)} & \text{\bfseries Formula for \boldmath\(I_n\)}
  & \bfseries Reference \\
 \midrule
 \endhead
 % TODO: why does the caption spacing break if I put it inside endlastfoot?
 \bottomrule
 \endfoot
 \endlastfoot
 \integ{x^n e^x}{x} & \frac{x^{n + 1} e^x}{n + 1} - nI_{n - 1} \\[3ex]
 \integ{x^n \sin x}{x} & \frac{x^{n + 1} \sin x}{n + 1} +
                         x^n \cos x - n(n - 1)I_{n - 2} \\[3ex]
 \integ{x^n \cos x}{x} & \frac{x^{n + 1} \cos x}{n + 1} -
                         x^n \sin x - n(n - 1)I_{n - 2} \\[3ex]
 \integ{\sin^n x}{x} & \frac{-\cos x \sin^{n - 1} x
                             + (n - 1) I_{n - 2}} n \\[3ex]
 \integ{\cos^n x}{x} & \frac{\sin x \cos^{n - 1} x
                             + (n - 1) I_{n - 2}} n \\[3ex]
 \integ{\sinh^n x}{x} & \frac{\cosh x \sinh^{n - 1}x
                              - (n - 1)I_{n - 2}} n \\[3ex]
 \integ{\cosh^n x}{x} & \frac{\sinh x \cosh^{n - 1}x
                              + (n - 1)I_{n - 2}} n \\[3ex]
 \integ{\frac{x^n}{\sqrt{1 - x}}}{x} &
  \frac{2x^n \sqrt{1 - x} + 2n I_{n - 1}}{2n + 1} \\[3ex]
 \integ{\tan^n x}{x} & \frac 1{n - 1} \tan^{n - 1} x - I_{n - 2} \\[3ex]
 \integ{x^n \sqrt{1 - x^2}}{x} &
  \frac{-x^{n - 1}(1 - x^2)^{\frac 32} + (n - 1) I_{n - 2}}{n + 2} \\[3ex]
 \integ{\frac{\sin nx}{\sin x}}{x} &
  \frac{2 \sin (n - 1) x}{n - 1} + I_{n - 2} \\[3ex]
 \integ{(1 + x^2)^n}{x} & \frac{2n I_{n - 1} + x(1 + x^2)^n}{2n + 1} \\[3ex]
 \bottomrule
 \caption{Reduction formulae \label{tab_calc_reduction_formulae}}
\end{longtable}

\subsubsection{Frullani Integral} \label{sec_calc_frullani}

\begin{theorem}[Frullani's Integral]
 Where the limit of \(f(x)\) exists at both 0 and \(\infty\), and \(f'(x)\)
 exists and is continuous on \(\intoo{0, \infty}\),
 \begin{equation*}
  \integ[0]<\infty>{\frac{f(ax) - f(bx)} x}{x}
  = (f(\infty) - f(0))\ln \frac ab
 \end{equation*}
\end{theorem}
\begin{proof}
 This can be shown by using differentiation under the integral sign. We note
 that the integrand is itself the definite integral of \(f'(tx)\) with
 respect to \(t\), from \(b\) to \(a\). We can then exchange the integrals.
 %FIXME justify
 \begin{align*}
  \integ[0]<\infty>{\frac{f(ax) - f(bx)} x}{x} &=
  \integ[0]<\infty>{\integ[b]<a>{f'(tx)}{t}}{x} \\
  &= \integ[b]<a>{\integ[0]<\infty>{f'(tx)}{x}}{t} \\
  &= \integ[b]<a>{\frac{f(tx)}{t}\eval_{x=0}^\infty}{t} \\
  &= \integ[b]<a>{\frac 1t(f(\infty) - f(0))}{t} \\
  &= (f(\infty) - f(0))\integ[b]<a>{\frac 1t}{t} \\
  &= (f(\infty) - f(0))\ln \frac ab \qedhere
 \end{align*}
\end{proof}

\subsubsection[Archyperbolic \texorpdfstring{\(t\)}{t} substitutions]
              {Archyperbolic \boldmath\(t\)-substitutions}

\begin{theorem}[Archyperbolic t substitution]
 \begin{equation*}
  \integ{f(x + \sqrt{x^2 + 1})}{x}
  = \frac 12 \integ{\parens[\Big]{1 + \frac 1{t^2}}f(t)}{t}
 \end{equation*}
 where \(t = x + \sqrt{x^2 + 1}\).
 \begin{equation*}
  \integ{f(x + \sqrt{x^2 - 1})}{x}
  = \frac 12 \integ{\parens[\Big]{1 - \frac 1{t^2}}f(t)}{t}
 \end{equation*}
 where \(t = x + \sqrt{x^2 - 1}\).
\end{theorem}
\begin{proof}
 To find the integral of some function \(f(x + \sqrt{x^2 + 1})\), we can make
 the substitution \(t = x + \sqrt{x^2 + 1}\). Noting that
 \(t^{-1} = \sqrt{x^2 + 1} - x\), we have \(\frac 12(t - t^{-1}) = x\)
 and also \(\frac 12(t + t^{-1}) = \sqrt{x^2 + 1}\), so in fact the form of
 \(f\) can involve both \(x\) and \(\sqrt{x^2 + 1}\) as these can be
 written in terms of \(t\). From the former equation, we have also
 \begin{alignat*}{2}
  && \dv<x>{t} &= \frac 12\parens[\Big]{1 + \frac 1{t^2}} \\
  &\implies{}& \integ{f(x + \sqrt{x^2 + 1})}{x}
  &= \frac 12 \integ{\parens[\Big]{1 + \frac 1{t^2}}f(t)}{t}
 \end{alignat*}
 Similarly for the second one, we let \(t = x + \sqrt{x^2 - 1}\) so that
 \(t^{-1} = x - \sqrt{x^2 - 1}\), and we have \(\frac 12(t + t^{-1}) = x\)
 and also \(\frac 12(t - t^{-1}) = \sqrt{x^2 - 1}\). Therefore,
 \begin{alignat*}{2}
  && \dv<x>{t} &= \frac 12\parens[\Big]{1 - \frac 1{t^2}} \\
  &\implies{}& \integ{f(x + \sqrt{x^2 - 1})}{x}
  &= \frac 12 \integ{\parens[\Big]{1 - \frac 1{t^2}}f(t)}{t} \qedhere
 \end{alignat*}
\end{proof}

\subsubsection{Negated involution substitution}

\begin{theorem}[Additive involution substitution]
 \begin{equation*}
  \integ[a]<b>{\frac{f(x)}{f(x) + f(a + b - x)}}{x} = \frac{b - a} 2
 \end{equation*}
\end{theorem}
\begin{proof}
 Let \(I\) denote the integral, and make the substitution \(u = a + b - x\),
 \begin{alignat*}{2}
  &\implies{}& \dv<u>{x} &= -1, \quad x = a + b - u \\
  &\implies{}& I &= \integ[a + b - a]<a + b - b>{
                      \frac{-f(a + b - u)}{f(a + b - u) + f(u)}}{u} \\
  &&  &= \integ[a]<b>{\frac{f(a + b - u)}{f(a + b - u) + f(u)}}{u} \\
  &\implies{}& I + I &= \integ[a]<b>{
                      \frac{f(a + b - x) + f(x)}
                          {f(a + b - x) + f(x)}}{x} \\
  &&  &= \integ[a]<b>{1}{x} = b - a \\
  &\implies{}& I &= \frac{b - a} 2 \qedhere
 \end{alignat*}
\end{proof}
\begin{theorem}[Divisive involution substitution]
 \begin{equation*}
  \integ[a]<b>{\frac{f(x)}{x(f(x) + f(ab / x))}}{x} = \frac 12 \ln \frac ba
 \end{equation*}
\end{theorem}
\begin{proof}
 Again let \(I\) denote the integral. Make the substitution \(u = ab / x\),
 \begin{alignat*}{2}
  &\implies{}& \dv<u>{x} &= -\frac{ab}{x^2} = -\frac ux, \quad x = \frac{ab}u \\
  &\implies{}& I &= \integ[ab/a]<ab/b>{\frac{-f(ab / u)}{u(f(ab / u) + f(u))}
  }{u} \\
  && &= \integ[a]<b>{\frac{f(ab / u)}{u(f(ab / u) + f(u))}}{u} \\
  &\implies{}& I + I &= \integ[a]<b>{\frac{f(ab / x) + f(x)}
                                  {x(f(ab / x) + f(x))}}{x} \\
  && &= \integ[a]<b>{\frac 1x}{x} = \ln \frac ba \\
  &\implies{}& I &= \frac 12 \ln \frac ba \qedhere
 \end{alignat*}
\end{proof}

\subsubsection{Gaussian Integral}

\begin{theorem}[Gaussian Integral] \label{thm_gauss_integral}
 \begin{equation*}
  \integ[-\infty]<\infty>{e^{-x^2}}{x} = \sqrt \pi
 \end{equation*}
\end{theorem}
\begin{proof}
 We can consider first the square of the integral.
 \begin{align*}
  \parens[\Big]{\integ[-\infty]<\infty>{e^{-x^2}}{x}}^2
  &= \parens[\Big]{\integ[-\infty]<\infty>{e^{-x^2}}{x}}
     \parens[\Big]{\integ[-\infty]<\infty>{e^{-y^2}}{y}} \\
  &= \integ[-\infty]<\infty>{
                 \parens[\Big]{\integ[-\infty]<\infty>{e^{-y^2}}{y}}
             e^{-x^2}}{x} \\
  &= \integ[-\infty]<\infty>{
         \parens[\Big]{\integ[-\infty]<\infty>{e^{-y^2} e^{-x^2}}{y}}}{x} \\
  &= \integ[-\infty]<\infty>{
         \integ[-\infty]<\infty>{e^{-x^2-y^2}}{y}}{x}
 \end{align*}
 %FIXME diagram of this
 This is effectively integrating some function of \((x, y)\) over
 \(\Reals^2\), giving the volume between an infinite solid and the \((x, y)\)
 plane. By noting that the function only depends on the distance of
 \((x, y)\) from \((0, 0)\) (ie \(x^2 + y^2\)), we see that this volume is
 equivalent to the volume obtained by fully rotating the curve in the
 \(y = e^{-x^2}\) where
 \(x \in \intco{0, \infty} \implies y \in \intoc{0, 1}\) around the y axis.
 This means
 \begin{alignat*}{2}
  && \parens[\Big]{\integ[-\infty]<\infty>{e^{-x^2}}{x}}^2
      &= \integ[0]<1>{\pi x^2}{y} \\
  &&  &= -\pi \integ[0]<1>{\ln y}{y} \\
  &&  &= -\pi (y\ln y - y)\eval_0^1 \\
  &&  &= -\pi (0 - 1 - 0 + 0) \\
  &&  &= \pi \\
  &\implies{}& \integ[-\infty]<\infty>{e^{-x^2}}{x}
      &= \sqrt \pi \qedhere
 \end{alignat*}
\end{proof}
This proof is a bit of a ``trick''. The standard way to approach an integral
like this is by contour integration or something.

This integral is used as the normalising constant of the normal
distribution. It also pops up in the evaluation of \(\Gamma(\frac 12)\).

It's a quite famous example of an integral of an elementary integrand that
has no elementary antiderivative, but when taken as a definite integral can
be solved analytically.

\subsubsection{Reduction Formulae}

\begin{theorem}
 Where
 \begin{equation*}
  I_n \defeq \integ{x^n e^x}{x}
 \end{equation*}
 we can use the reduction formula
 \begin{equation*}
  I_n = \frac{x^{n + 1} e^x}{n + 1} - nI_{n - 1}
 \end{equation*}
\end{theorem}
\begin{proof}
 Integrating by parts, we have
 \begin{align*}
  I_n &= \integ{x^n e^x}{x} \\
      &= \frac{x^{n + 1}e^x}{n + 1} - \integ{n x^{n - 1} e^x}{x} \\
      &= \frac{x^{n + 1}e^x}{n + 1} - n I_{n - 1} \qedhere
 \end{align*}
\end{proof}
\begin{theorem}
 Where
 \begin{equation*}
  I_n \defeq \integ{x^n \sin x}{x}
 \end{equation*}
 we can use the reduction formula
 \begin{equation*}
  I_n = \frac{x^{n + 1} \sin x}{n + 1} + x^n \cos x - n(n - 1)I_{n - 2}
 \end{equation*}
\end{theorem}
\begin{proof}
 Integrating by parts twice, we have
 \begin{align*}
  I_n &= \integ{x^n \sin x}{x} \\
      &= \frac{x^{n + 1} \sin x}{n + 1} + \integ{n x^{n - 1} \cos x}{x} \\
      &= \frac{x^{n + 1} \sin x}{n + 1} + n\parens[\Big]{
       \frac{x^n \cos x}{n} - \integ{(n - 1)x^{n - 2} \sin x}{x}} \\
      &= \frac{x^{n + 1} \sin x}{n + 1} + x^n \cos x - n(n - 1)I_{n - 2}
       \qedhere
 \end{align*}
\end{proof}
\begin{theorem}
 Where
 \begin{equation*}
  I_n \defeq \integ{x^n \cos x}{x}
 \end{equation*}
 we can use the reduction formula
 \begin{equation*}
  I_n = \frac{x^{n + 1} \cos x}{n + 1} - x^n \sin x - n(n - 1)I_{n - 2}
 \end{equation*}
\end{theorem}
\begin{proof}
 Integrating by parts twice, we have
 \begin{align*}
  I_n &= \integ{x^n \cos x}{x} \\
      &= \frac{x^{n + 1} \cos x}{n + 1} - \integ{n x^{n - 1} \sin x}{x} \\
      &= \frac{x^{n + 1} \cos x}{n + 1} - n\parens[\Big]{
       \frac{x^n \sin x}{n} + \integ{(n - 1)x^{n - 2} \cos x}{x}} \\
      &= \frac{x^{n + 1} \cos x}{n + 1} - x^n \sin x - n(n - 1)I_{n - 2}
       \qedhere
 \end{align*}
\end{proof}
\begin{theorem}
 Where
 \begin{equation*}
  I_n \defeq \integ{\sin^n x}{x}
 \end{equation*}
 we can use the reduction formula
 \begin{equation*}
  I_n = \frac{-\cos x \sin^{n - 1} x + (n - 1) I_{n - 2}} n
 \end{equation*}
\end{theorem}
\begin{proof}
 Integrating by parts, we have
 \begin{alignat*}2
  && I_n &= \integ{\sin^n x}{x} \\
  &&     &= \integ{\sin x \sin^{n - 1}x}{x} \\
  &&     &= -\cos x \sin^{n - 1} x +
             \integ{\cos^2 x (n - 1) \sin^{n - 2} x}{x} \\
  &&     &= -\cos x \sin^{n - 1} x + (n - 1)(I_{n - 2} - I_n) \\
  &\implies{}& n I_n &= -\cos x \sin^{n - 1} x + (n - 1)I_{n - 2} \qedhere
 \end{alignat*}
\end{proof}
\begin{theorem}
 Where
 \begin{equation*}
  I_n \defeq \integ{\cos^n x}{x}
 \end{equation*}
 we can use the reduction formula
 \begin{equation*}
  I_n = \frac{\sin x \cos^{n - 1} x + (n - 1) I_{n - 2}} n
 \end{equation*}
\end{theorem}
\begin{proof}
 Integrating by parts, we have
 \begin{alignat*}2
  && I_n &= \integ{\cos^n x}{x} \\
  &&     &= \integ{\cos x \cos^{n - 1}x}{x} \\
  &&     &= \sin x \cos^{n - 1} x +
             \integ{\sin^2 x (n - 1) \cos^{n - 2} x}{x} \\
  &&     &= \sin x \cos^{n - 1} x + (n - 1)(I_{n - 2} - I_n) \\
  &\implies{}& n I_n &= \sin x \cos^{n - 1} x + (n - 1)I_{n - 2} \qedhere
 \end{alignat*}
\end{proof}
\begin{theorem}
 Where
 \begin{equation*}
  I_n \defeq \integ{\sinh^n x}{x}
 \end{equation*}
 we can use the reduction formula
 \begin{equation*}
  I_n = \frac{\cosh x \sinh^{n - 1}x - (n - 1)I_{n - 2}} n
 \end{equation*}
\end{theorem}
\begin{theorem}
 Where
 \begin{equation*}
  I_n \defeq \integ{\cosh^n x}{x}
 \end{equation*}
 we can use the reduction formula
 \begin{equation*}
  I_n = \frac{\sinh x \cosh^{n - 1}x + (n - 1)I_{n - 2}} n
 \end{equation*}
\end{theorem}
\begin{theorem}
 Where
 \begin{equation*}
  I_n \defeq \integ{\frac{x^n}{\sqrt{1 - x}}}{x}
 \end{equation*}
 we can use the reduction formula
 \begin{equation*}
  I_n = \frac{-2x^n \sqrt{1 - x} + 2n I_{n - 1}}{2n + 1}
 \end{equation*}
\end{theorem}
\begin{proof}
 Integrating by parts,
 \begin{alignat*}2
  && I_n &= \integ{\frac{x^n}{\sqrt{1 - x}}}{x} \\
  &&     &= -2x^n \sqrt{1 - x} + \integ{2n x^{n - 1} \sqrt{1 - x}}{x} \\
  &&     &= -2x^n \sqrt{1 - x} + \integ{\frac{2n x^{n - 1}(1 - x)}
                                             {\sqrt{1 - x}}}{x} \\
  &&     &= -2x^n \sqrt{1 - x} + 2n (I_{n - 1} - I_n) \\
  &\implies{}& (2n + 1)I_n &= -2x^n \sqrt{1 - x} + 2n I_{n - 1} \qedhere
 \end{alignat*}
\end{proof}
\begin{theorem}
 Where
 \begin{equation*}
  I_n \defeq \integ{\tan^n x}{x}
 \end{equation*}
 we can use the reduction formula
 \begin{equation*}
  I_n = \frac{\tan^{n - 1} x}{n - 1} - I_{n - 2}
 \end{equation*}
\end{theorem}
\begin{proof}
 Applying a Pythagorean identity, and then integrating by subtitution,
 \begin{align*}
  I_n &= \integ{\tan^n x}{x} \\
      &= \integ{(\sec^2 x - 1) \tan^{n - 2} x}{x} \\
      &= \integ{\sec^2 x \tan^{n - 2} x}{x} - I_{n - 2} \\
      &= \frac{\tan^{n - 1} x}{n - 1} - I_{n - 2} \qedhere
 \end{align*}
\end{proof}
\begin{theorem}
 Where
 \begin{equation*}
  I_n \defeq \integ{x^n \sqrt{1 - x^2}}{x}
 \end{equation*}
 we can use the reduction formula
 \begin{equation*}
  I_n = \frac{-x^{n - 1}(1 - x^2)^{\frac 32} + (n - 1) I_{n - 2}}{n + 2}
 \end{equation*}
\end{theorem}
\begin{proof}
 Integrating by parts, we have
 \begin{alignat*}2
  && I_n &= \integ{x^n \sqrt{1 - x^2}}{x} \\
  &&     &= \integ{x^{n - 1} x \sqrt{1 - x^2}}{x} \\
  &&     &= -\frac 13 (1 - x^2)^{\frac 32} x^{n - 1} +
            \integ{\frac 13 (n - 1)(1 - x^2)^{\frac 32}x^{n - 2}}{x} \\
  &\implies{}& 3 I_n
         &= -(1 - x^2)^{\frac 32} x^{n - 1} +
            (n - 1)(I_{n - 2} - I_n) \\
  &\implies{}& (n + 2)I_n
         &= -(1 - x^2)^{\frac 32} x^{n - 1} +
            (n - 1)I_{n - 2} \qedhere
 \end{alignat*}
\end{proof}
\begin{theorem}
 Where
 \begin{equation*}
  I_n \defeq \integ{\frac{\sin nx}{\sin x}}{x}
 \end{equation*}
 we can use the reduction formula
 \begin{equation*}
  I_n = \frac{2 \sin (n - 1) x}{n - 1} + I_{n - 2}
 \end{equation*}
\end{theorem}
\begin{proof}
 By gently applying some trig identities, we get
 \begin{align*}
  I_n &= \integ{\frac{\sin nx}{\sin x}}{x} \\
      &= \integ{\frac{\sin 2x \cos (n - 2)x +
                      \cos 2x \sin (n - 2) x}{\sin x}}{x} \\
      &= \integ{\frac{2\sin x \cos x \cos (n - 2)x +
                      (1 - 2 \sin^2 x) \sin (n - 2) x}{\sin x}}{x} \\
      &= \integ{(2\cos x \cos (n - 2)x - 2\sin x \sin (n - 2) x)}{x}
         + I_{n - 2} \\
      &= \integ{2 \cos (n - 1) x}{x} + I_{n - 2} \\
      &= \frac{2 \sin (n - 1) x}{n - 1} + I_{n - 2} \qedhere
 \end{align*}
\end{proof}
\begin{theorem}
 Where
 \begin{equation*}
  I_n \defeq \integ{(1 + x^2)^n}{x}
 \end{equation*}
 we can use the reduction formula
 \begin{equation*}
  I_n = \frac{2n I_{n - 1} + x(1 + x^2)^n}{2n + 1}
 \end{equation*}
\end{theorem}
\begin{proof}
 Integrating by parts, we have
 \begin{alignat*}2
  && I_n &= \integ{(1 + x^2)^n}{x} \\
  &&     &= I_{n - 1} + \integ{x^2 (1 + x^2)^{n - 1}}{x} \\
  &&     &= I_{n - 1} + \frac{x(x^2 + 1)^n}{2n} -
                        \integ{\frac{(x^2 + 1)^n}{2n}}{x} \\
  &&     &= I_{n - 1} + \frac{x(x^2 + 1)^n}{2n} - \frac 1{2n} I_n \\
  &\implies{}& (2n + 1) I_n
         &= I_{n - 1} + x(x^2 + 1)^n \qedhere
 \end{alignat*}
\end{proof}

\subsection{Volume of Revolution}

\subsection{Area under Parametric Curve} \label{sec_calc_parametric_area}

To find the area between a curve and the \(x\)-axis from \(x_1\) to \(x_2\)
when the curve is given in parametric form, \((x, y) = (f(t), g(t))\),
we rewrite the integral, effectively using a substitution
(\ref{sec_calc_substitution}) \(x = f(t)\) in order to express the integrand
in \(t\).
\begin{equation*}
 \integ[x_1]<x_2>{y}{x} =
 \integ[t_1]<t_2>{y \dv<x>{t}}{t} =
 \integ[t_1]<t_2>{g(t)f'(t)}{t}
\end{equation*}
where \(f(t_1) = x_1\) and \(f(t_2) = x_2\).

Similarly, the area between the curve and the y-axis from \(y_1\) to \(y_2\)
is:
\begin{equation*}
 \integ[y_1]<y_2>{x}{y} =
 \integ[t_1]<t_2>{x \dv<y>{t}}{t} =
 \integ[t_1]<t_2>{f(t)g'(t)}{t}
\end{equation*}
where \(g(t_1) = y_1\) and \(g(t_2) = y_2\).

\subsection{Arc Length}

\subsection{Improper Integrals}

%FIXME add types of improper integral

 \section{Curiosities}

\subsection[\texorpdfstring{\(e^\pi\) vs \(\pi^e\)}{e to the pi vs pi to the e}]
           {\boldmath\(e^\pi\) vs \(\pi^e\)}

This is a classic problem. Is \(e^\pi\) or \(\pi^e\) greater? A nice way to
tackle it is by noticing that you can rewrite it as an equivalent inequality, by
raising to the power \(1 / (\pi e)\), as both sides are positive.
\begin{alignat*}2
 && e^\pi &> \pi^e \\
 &\iff{}& e^{\frac 1e} &> \pi^{\frac 1\pi}
\end{alignat*}
This is easier to tackle, as by separating the two constants to either side like
this, we have basically turned the question into one about the shape of the
graph \(y = x^{\frac 1x}\). Note that
\begin{equation*}
 \dv<y>{x} = \parens[\Big]{\frac 1{x^2} - \frac 1{x^2}\ln x} x^{\frac 1x}
           = \frac 1{x^2} (1 - \ln x) x^{\frac 1x}
\end{equation*}
so in fact, for \(x < e\), \(\dv<y>{x} > 0\), and for
\(x > e\), \(\dv<y>{x} < 0\), so the graph has a local maximum at \(x = 1\) and
is decreasing from there on out. We conclude that
% TODO draw the actual graph
\begin{equation*}
 e^{\frac 1e} < \pi^{\frac 1\pi}
\end{equation*}
and therefore,
\begin{equation*}
 e^\pi > \pi^e
\end{equation*}

 \input{tex/13_numerical_methods}
 \input{tex/14_mechanics}
 \section{Computer Science}

\subsection{Exponentiation by Squaring} \label{sec_exp_by_squaring}

%FIXME add fast integer square root algorithm

For \(y \in \Integers_0^+\), \(x^y\) is given by
\begin{equation*}
 x^y =
  \begin{dcases}
   1 & y = 0 \\
   x & y = 1 \\
   (x^2)^{\frac 12 y} & \text{\(y\) is even} \\
   x(x^2)^{\frac 12 (y - 1)} & \text{\(y\) is odd} \\
  \end{dcases}
\end{equation*}

This follows from the fact that
\(x^{2y'} = (x^2)^{y'}\) and \(x^{2y' + 1} = x(x^2)^{y'}\).

 \section{Maths Jokes}

\begin{itemize}
 \item
  Q: What do you call something that's purple and equipped with a commutative
  operation? \\
  A: An Abelian grape!
 \item
  Q: What sound does a drowning number theoretician make? \\
  A: \(\log \log \log \log \log \log \log\)
 \item
  A countably infinite number of mathematicians walk into a bar.
  {The first asks for one pint.
  \footnotesize The second asks for half a pint.
  \scriptsize The third asks for a quarter of a pint.
  \tiny The fourth asks for an eighth of a pint.}
  The bartender pours two pints and proclaims, ``Know your limits!''
 \item
  Q: What did \(i \pi\) say to the exponential function? \\
  A: You make me feel mighty real.
 \item
  Q: What is the result of evaluating the indefinite integral
  \begin{equation*}
   \integ{\frac 1{\mathrm{cabin}}}{\,\mathrm{cabin}} \ \text?
  \end{equation*}
  A (hopefully): *exasperated sigh* god you're an idiot, it's a
  \(\log \mathrm{cabin}\).

  Jubilant rebuttal: Actually, it's a houseboat! You forgot to add the C!
 \item
  An engineer, a physicist, and a mathematician are shown a pasture with a herd
  of sheep, and told to put them inside the smallest possible amount of fence.
  \\
  The engineer is first. He herds the sheep into a circle and then puts the
  fence around them, declaring, ``A circle will use the least fence for a given
  area, so this is the best solution.'' \\
  The physicist is next. He creates a circular fence of infinite radius around
  the sheep, and then draws the fence tight around the herd, declaring, ``This
  will give the smallest circular fence around the herd.'' \\
  The mathematician is last. After giving the problem a little thought, he puts
  a small fence around himself and then declares, ``I define myself to be on the
  outside.''
 \item
  An engineer, a physicist, and a mathematician were on a train heading north,
  and had just crossed the border into Scotland. \\
  The engineer looked out of the window and said ``Look! Scottish sheep are
  black!'' \\
  The physicist said, ``No, no. Some Scottish sheep are black.'' \\
  The mathematician looked irritated. ``There is at least one field, containing
  at least one sheep, of which at least one side is black.'' \\
 \item
  An engineer and a mathematician were shown into a kitchen, given an empty pan,
  and told to boil a pint of water. They both filled the pan with water, put it
  on the stove, and boiled it. \\
  The next day they were shown into the kitchen again, given a pan full of
  water, and told to boil a pint of water. \\
  The engineer took the pan, put it on the stove, and boiled it. \\
  The mathematician took the pan and emptied it, thereby reducing it to a
  previously solved problem.
 \item
  A mathematician, physicist, and engineer are taking a math test. One question
  asks ``Are all odd numbers prime?'' \\
  The mathematician thinks, ``3 is prime, 5 is prime, 7 is prime, 9 is not prime
  -- nope, not all odd numbers are prime.'' \\
  The physicist thinks, `` 3 is prime, 5 is prime, 7 is prime, 9 is not prime --
  that could be experimental error -- 11 is prime, 13 is prime, yes, they're all
  prime.'' \\
  The engineer thinks, `` 3 is prime, 5 is prime, 7 is prime, 9 is prime, 11 is prime, ...''
 \item
  A physicist, a biologist and a mathematician sit in a sidewalk cafe, looking
  at the building across the road. Two people go into the building, then three
  people come out. \\
  Physicist: ``This must be a measuring error!'' \\
  Biologist: ``This is proof of procreation!'' \\
  Mathematician: ``If one more person goes into the building, it will be
  empty!''
\end{itemize}

 \section{Sandbox}

\ref{sec_algebra}

\lettrine{\color{RoyalBlue4}I}{zaak van Dongen}.
 Lorem ipsum dolor sit amet, consectetur adipiscing elit. Proin accumsan
 pellentesque elit sit amet luctus. Aenean sollicitudin nibh nec dui
 pharetra, vitae mattis odio consectetur. Mauris et tincidunt turpis. Sed non
 lectus feugiat, pulvinar enim ac, iaculis libero. Mauris aliquam justo vitae
 dapibus sodales. Pellentesque feugiat metus at accumsan ullamcorper. Donec
 in sem odio. In vestibulum non mauris a tincidunt.

\begin{longtable}{*3c}
\(
\begin{array}{*8c}
\alpha&\Alpha&\beta&\Beta&\gamma&\Gamma&\delta&\Delta\\
\epsilon&\Epsilon&\zeta&\Zeta&\eta&\Eta&\theta&\Theta\\
\iota&\Iota&\kappa&\Kappa&\lambda&\Lambda&\mu&\Mu\\
\nu&\Nu&\omicron&\Omicron&\pi&\Pi&\rho&\Rho\\
\sigma&\Sigma&\tau&\Tau&\upsilon&\Upsilon&\phi&\Phi\\
\chi&\Chi&\psi&\Psi&\omega&\Omega&\xi&\Xi\\
\vartheta&\varpi&\varphi&\varrho&\varepsilon&\varsigma&\uptau&\ell\\
\end{array}\) &
\(\begin{array}{*4c}
\pm&\cap&\diamond&\oplus\\
\mp&\cup&\bigtriangleup&\ominus\\
\times&\uplus&\bigtriangledown&\otimes\\
\div&\sqcap&\triangleleft&\oslash\\
\ast&\sqcup&\triangleright&\odot\\
\star&\vee \land&\lhd&\bigcirc\\
\circ&\wedge \lor&\rhd&\dagger\\
\bullet&\setminus&\unlhd&\ddagger\\
\cdot&\wr&\unrhd&\amalg\\
+&-\\
\end{array}\) &
\(\begin{array}{*4c}
\leq&\geq&\equiv&\models\\
\prec&\succ&\sim&\perp\\
\preceq&\succeq&\simeq&\mid\\
\ll&\gg&\asymp&\parallel\\
\subset&\supset&\approx&\bowtie\\
\subseteq&\supseteq&\cong&\Join\\
\sqsubset&\sqsupset&\neq&\smile\\
\sqsubseteq&\sqsupseteq&\doteq&\frown\\
\in&\ni&\propto&=\\
\vdash&\dashv&<&>\\
\end{array}\) \\
\(\begin{array}{*5c}
\arccos&\cos&\csc&\exp\\
\ker&\limsup&\min&\sinh\\
\arcsin&\cosh&\deg&\gcd\\
\lg&\ln&\Pr&\sup\\
\arctan&\cot&\det&\hom\\
\lim&\log&\sec&\tan\\
\arg&\coth&\dim&\inf\\
\liminf&\max&\sin&\tanh\\
\Re&\Im&\operatorname{Adj} A\\
\end{array}\) &
\(\begin{array}{*3c}
\leftarrow&\longleftarrow&\uparrow\\
\Leftarrow&\Longleftarrow&\Uparrow\\
\rightarrow&\longrightarrow&\downarrow\\
\Rightarrow&\Longrightarrow&\Downarrow\\
\leftrightarrow&\longleftrightarrow&\updownarrow\\
\Leftrightarrow&\Longleftrightarrow&\Updownarrow\\
\mapsto&\longmapsto&\nearrow\\
\hookleftarrow&\hookrightarrow&\searrow\\
\leftharpoonup&\rightharpoonup&\swarrow\\
\leftharpoondown&\rightharpoondown&\nwarrow\\
\rightleftharpoons&\leadsto\\
\end{array}\) &
\(\begin{array}{*4c}
\ldots&\cdots&\vdots&\ddots\\
\aleph&\prime&\forall&\infty\\
\hbar&\emptyset&\exists&\Box\\
\imath&\nabla&\neg&\Diamond\\
\jmath&\surd&\flat&\triangle\\
\ell&\top&\natural&\clubsuit\\
\wp&\bot&\sharp&\diamondsuit\\
\Re&\|&\backslash&\heartsuit\\
\Im&\angle&\partial&\spadesuit\\
\end{array}\) \\
\(\begin{array}{*2c}
DIFferent          & % symbols
\mathit{DIFferent} \\ % italic
\mathrm{DIFferent} & % roman
\mathsf{DIFferent} \\ % sans-serif
\mathtt{DIFferent} & % typewriter text
\mathbf{DIFferent} \\ % bold font
\operatorname{DIFferen}(t) & % operator
\mathcal{DIFFERENT} \\ % calligraphic
\mathbb{DIFFERENT} & % blackboard bold
\mathfrak{DIFferent} \\ % fraktur
\textfrak{DIFferent} & % also fraktur
\textswab{DIFferent} \\ % swabacher
\textgoth{DIFferent} & % gothic font
\mathscr{DIFFERENT} \\ % script font
\multicolumn 2c{
(\vec d \times \vec i)
 \cdot \uvec f - \abs{\veca{FE}} \norm{\veca{RE}} +
 \veca{NT}} \\
\end{array}\) &
\(\begin{array}{*3c}
\sum&\bigcap&\bigodot\\
\prod&\bigcup&\bigotimes\\
\coprod&\bigsqcup&\bigoplus\\
\int&\bigvee&\biguplus\\
\oint&\bigwedge\\
\floor*{} & \ceil*{} \\
\end{array}\) &
\(\begin{array}{*3c}
\widetilde{abc}&\widehat{abc}&\overleftarrow{abc}\\
\overrightarrow{abc}&\overline{abc}&\underline{abc}\\
\overbrace{abc}&\underbrace{abc}&\sqrt{abc}\\
\sqrt[n]{abc}&\hat a&\acute a\\
\bar a&\dot a&\breve a\\
\check a&\grave a&\vec a\\
\ddot a&\tilde a\\
\end{array}\) \\
\multicolumn 3c{
\begin{tabular}{*{21}c}
{\color{AntiqueWhite1} t}\textcolor{AntiqueWhite1} t &
{\color{AntiqueWhite2} t}\textcolor{AntiqueWhite2} t &
{\color{AntiqueWhite3} t}\textcolor{AntiqueWhite3} t &
{\color{AntiqueWhite4} t}\textcolor{AntiqueWhite4} t &
{\color{Aquamarine1} t}\textcolor{Aquamarine1} t &
{\color{Aquamarine2} t}\textcolor{Aquamarine2} t &
{\color{Aquamarine3} t}\textcolor{Aquamarine3} t &
{\color{Aquamarine4} t}\textcolor{Aquamarine4} t &
{\color{Azure1} t}\textcolor{Azure1} t &
{\color{Azure2} t}\textcolor{Azure2} t &
{\color{Azure3} t}\textcolor{Azure3} t &
{\color{Azure4} t}\textcolor{Azure4} t &
{\color{Bisque1} t}\textcolor{Bisque1} t &
{\color{Bisque2} t}\textcolor{Bisque2} t &
{\color{Bisque3} t}\textcolor{Bisque3} t &
{\color{Bisque4} t}\textcolor{Bisque4} t &
{\color{Blue1} t}\textcolor{Blue1} t &
{\color{Blue2} t}\textcolor{Blue2} t &
{\color{Blue3} t}\textcolor{Blue3} t &
{\color{Blue4} t}\textcolor{Blue4} t &
{\color{Brown1} t}\textcolor{Brown1} t \\
{\color{Brown2} t}\textcolor{Brown2} t &
{\color{Brown3} t}\textcolor{Brown3} t &
{\color{Brown4} t}\textcolor{Brown4} t &
{\color{Burlywood1} t}\textcolor{Burlywood1} t &
{\color{Burlywood2} t}\textcolor{Burlywood2} t &
{\color{Burlywood3} t}\textcolor{Burlywood3} t &
{\color{Burlywood4} t}\textcolor{Burlywood4} t &
{\color{CadetBlue1} t}\textcolor{CadetBlue1} t &
{\color{CadetBlue2} t}\textcolor{CadetBlue2} t &
{\color{CadetBlue3} t}\textcolor{CadetBlue3} t &
{\color{CadetBlue4} t}\textcolor{CadetBlue4} t &
{\color{Chartreuse1} t}\textcolor{Chartreuse1} t &
{\color{Chartreuse2} t}\textcolor{Chartreuse2} t &
{\color{Chartreuse3} t}\textcolor{Chartreuse3} t &
{\color{Chartreuse4} t}\textcolor{Chartreuse4} t &
{\color{Chocolate1} t}\textcolor{Chocolate1} t &
{\color{Chocolate2} t}\textcolor{Chocolate2} t &
{\color{Chocolate3} t}\textcolor{Chocolate3} t &
{\color{Chocolate4} t}\textcolor{Chocolate4} t &
{\color{Coral1} t}\textcolor{Coral1} t &
{\color{Coral2} t}\textcolor{Coral2} t \\
{\color{Coral3} t}\textcolor{Coral3} t &
{\color{Coral4} t}\textcolor{Coral4} t &
{\color{Cornsilk1} t}\textcolor{Cornsilk1} t &
{\color{Cornsilk2} t}\textcolor{Cornsilk2} t &
{\color{Cornsilk3} t}\textcolor{Cornsilk3} t &
{\color{Cornsilk4} t}\textcolor{Cornsilk4} t &
{\color{Cyan1} t}\textcolor{Cyan1} t &
{\color{Cyan2} t}\textcolor{Cyan2} t &
{\color{Cyan3} t}\textcolor{Cyan3} t &
{\color{Cyan4} t}\textcolor{Cyan4} t &
{\color{DarkGoldenrod1} t}\textcolor{DarkGoldenrod1} t &
{\color{DarkGoldenrod2} t}\textcolor{DarkGoldenrod2} t &
{\color{DarkGoldenrod3} t}\textcolor{DarkGoldenrod3} t &
{\color{DarkGoldenrod4} t}\textcolor{DarkGoldenrod4} t &
{\color{DarkOliveGreen1} t}\textcolor{DarkOliveGreen1} t &
{\color{DarkOliveGreen2} t}\textcolor{DarkOliveGreen2} t &
{\color{DarkOliveGreen3} t}\textcolor{DarkOliveGreen3} t &
{\color{DarkOliveGreen4} t}\textcolor{DarkOliveGreen4} t &
{\color{DarkOrange1} t}\textcolor{DarkOrange1} t &
{\color{DarkOrange2} t}\textcolor{DarkOrange2} t &
{\color{DarkOrange3} t}\textcolor{DarkOrange3} t \\
{\color{DarkOrange4} t}\textcolor{DarkOrange4} t &
{\color{DarkOrchid1} t}\textcolor{DarkOrchid1} t &
{\color{DarkOrchid2} t}\textcolor{DarkOrchid2} t &
{\color{DarkOrchid3} t}\textcolor{DarkOrchid3} t &
{\color{DarkOrchid4} t}\textcolor{DarkOrchid4} t &
{\color{DarkSeaGreen1} t}\textcolor{DarkSeaGreen1} t &
{\color{DarkSeaGreen2} t}\textcolor{DarkSeaGreen2} t &
{\color{DarkSeaGreen3} t}\textcolor{DarkSeaGreen3} t &
{\color{DarkSeaGreen4} t}\textcolor{DarkSeaGreen4} t &
{\color{DarkSlateGray1} t}\textcolor{DarkSlateGray1} t &
{\color{DarkSlateGray2} t}\textcolor{DarkSlateGray2} t &
{\color{DarkSlateGray3} t}\textcolor{DarkSlateGray3} t &
{\color{DarkSlateGray4} t}\textcolor{DarkSlateGray4} t &
{\color{DeepPink1} t}\textcolor{DeepPink1} t &
{\color{DeepPink2} t}\textcolor{DeepPink2} t &
{\color{DeepPink3} t}\textcolor{DeepPink3} t &
{\color{DeepPink4} t}\textcolor{DeepPink4} t &
{\color{DeepSkyBlue1} t}\textcolor{DeepSkyBlue1} t &
{\color{DeepSkyBlue2} t}\textcolor{DeepSkyBlue2} t &
{\color{DeepSkyBlue3} t}\textcolor{DeepSkyBlue3} t &
{\color{DeepSkyBlue4} t}\textcolor{DeepSkyBlue4} t \\
{\color{DodgerBlue1} t}\textcolor{DodgerBlue1} t &
{\color{DodgerBlue2} t}\textcolor{DodgerBlue2} t &
{\color{DodgerBlue3} t}\textcolor{DodgerBlue3} t &
{\color{DodgerBlue4} t}\textcolor{DodgerBlue4} t &
{\color{Firebrick1} t}\textcolor{Firebrick1} t &
{\color{Firebrick2} t}\textcolor{Firebrick2} t &
{\color{Firebrick3} t}\textcolor{Firebrick3} t &
{\color{Firebrick4} t}\textcolor{Firebrick4} t &
{\color{Gold1} t}\textcolor{Gold1} t &
{\color{Gold2} t}\textcolor{Gold2} t &
{\color{Gold3} t}\textcolor{Gold3} t &
{\color{Gold4} t}\textcolor{Gold4} t &
{\color{Goldenrod1} t}\textcolor{Goldenrod1} t &
{\color{Goldenrod2} t}\textcolor{Goldenrod2} t &
{\color{Goldenrod3} t}\textcolor{Goldenrod3} t &
{\color{Goldenrod4} t}\textcolor{Goldenrod4} t &
{\color{Green1} t}\textcolor{Green1} t &
{\color{Green2} t}\textcolor{Green2} t &
{\color{Green3} t}\textcolor{Green3} t &
{\color{Green4} t}\textcolor{Green4} t &
{\color{Honeydew1} t}\textcolor{Honeydew1} t \\
{\color{Honeydew2} t}\textcolor{Honeydew2} t &
{\color{Honeydew3} t}\textcolor{Honeydew3} t &
{\color{Honeydew4} t}\textcolor{Honeydew4} t &
{\color{HotPink1} t}\textcolor{HotPink1} t &
{\color{HotPink2} t}\textcolor{HotPink2} t &
{\color{HotPink3} t}\textcolor{HotPink3} t &
{\color{HotPink4} t}\textcolor{HotPink4} t &
{\color{IndianRed1} t}\textcolor{IndianRed1} t &
{\color{IndianRed2} t}\textcolor{IndianRed2} t &
{\color{IndianRed3} t}\textcolor{IndianRed3} t &
{\color{IndianRed4} t}\textcolor{IndianRed4} t &
{\color{Ivory1} t}\textcolor{Ivory1} t &
{\color{Ivory2} t}\textcolor{Ivory2} t &
{\color{Ivory3} t}\textcolor{Ivory3} t &
{\color{Ivory4} t}\textcolor{Ivory4} t &
{\color{Khaki1} t}\textcolor{Khaki1} t &
{\color{Khaki2} t}\textcolor{Khaki2} t &
{\color{Khaki3} t}\textcolor{Khaki3} t &
{\color{Khaki4} t}\textcolor{Khaki4} t &
{\color{LavenderBlush1} t}\textcolor{LavenderBlush1} t &
{\color{LavenderBlush2} t}\textcolor{LavenderBlush2} t \\
{\color{LavenderBlush3} t}\textcolor{LavenderBlush3} t &
{\color{LavenderBlush4} t}\textcolor{LavenderBlush4} t &
{\color{LemonChiffon1} t}\textcolor{LemonChiffon1} t &
{\color{LemonChiffon2} t}\textcolor{LemonChiffon2} t &
{\color{LemonChiffon3} t}\textcolor{LemonChiffon3} t &
{\color{LemonChiffon4} t}\textcolor{LemonChiffon4} t &
{\color{LightBlue1} t}\textcolor{LightBlue1} t &
{\color{LightBlue2} t}\textcolor{LightBlue2} t &
{\color{LightBlue3} t}\textcolor{LightBlue3} t &
{\color{LightBlue4} t}\textcolor{LightBlue4} t &
{\color{LightCyan1} t}\textcolor{LightCyan1} t &
{\color{LightCyan2} t}\textcolor{LightCyan2} t &
{\color{LightCyan3} t}\textcolor{LightCyan3} t &
{\color{LightCyan4} t}\textcolor{LightCyan4} t &
{\color{LightGoldenrod1} t}\textcolor{LightGoldenrod1} t &
{\color{LightGoldenrod2} t}\textcolor{LightGoldenrod2} t &
{\color{LightGoldenrod3} t}\textcolor{LightGoldenrod3} t &
{\color{LightGoldenrod4} t}\textcolor{LightGoldenrod4} t &
{\color{LightPink1} t}\textcolor{LightPink1} t &
{\color{LightPink2} t}\textcolor{LightPink2} t &
{\color{LightPink3} t}\textcolor{LightPink3} t \\
{\color{LightPink4} t}\textcolor{LightPink4} t &
{\color{LightSalmon1} t}\textcolor{LightSalmon1} t &
{\color{LightSalmon2} t}\textcolor{LightSalmon2} t &
{\color{LightSalmon3} t}\textcolor{LightSalmon3} t &
{\color{LightSalmon4} t}\textcolor{LightSalmon4} t &
{\color{LightSkyBlue1} t}\textcolor{LightSkyBlue1} t &
{\color{LightSkyBlue2} t}\textcolor{LightSkyBlue2} t &
{\color{LightSkyBlue3} t}\textcolor{LightSkyBlue3} t &
{\color{LightSkyBlue4} t}\textcolor{LightSkyBlue4} t &
{\color{LightSteelBlue1} t}\textcolor{LightSteelBlue1} t &
{\color{LightSteelBlue2} t}\textcolor{LightSteelBlue2} t &
{\color{LightSteelBlue3} t}\textcolor{LightSteelBlue3} t &
{\color{LightSteelBlue4} t}\textcolor{LightSteelBlue4} t &
{\color{LightYellow1} t}\textcolor{LightYellow1} t &
{\color{LightYellow2} t}\textcolor{LightYellow2} t &
{\color{LightYellow3} t}\textcolor{LightYellow3} t &
{\color{LightYellow4} t}\textcolor{LightYellow4} t &
{\color{Magenta1} t}\textcolor{Magenta1} t &
{\color{Magenta2} t}\textcolor{Magenta2} t &
{\color{Magenta3} t}\textcolor{Magenta3} t &
{\color{Magenta4} t}\textcolor{Magenta4} t \\
{\color{Maroon1} t}\textcolor{Maroon1} t &
{\color{Maroon2} t}\textcolor{Maroon2} t &
{\color{Maroon3} t}\textcolor{Maroon3} t &
{\color{Maroon4} t}\textcolor{Maroon4} t &
{\color{MediumOrchid1} t}\textcolor{MediumOrchid1} t &
{\color{MediumOrchid2} t}\textcolor{MediumOrchid2} t &
{\color{MediumOrchid3} t}\textcolor{MediumOrchid3} t &
{\color{MediumOrchid4} t}\textcolor{MediumOrchid4} t &
{\color{MediumPurple1} t}\textcolor{MediumPurple1} t &
{\color{MediumPurple2} t}\textcolor{MediumPurple2} t &
{\color{MediumPurple3} t}\textcolor{MediumPurple3} t &
{\color{MediumPurple4} t}\textcolor{MediumPurple4} t &
{\color{MistyRose1} t}\textcolor{MistyRose1} t &
{\color{MistyRose2} t}\textcolor{MistyRose2} t &
{\color{MistyRose3} t}\textcolor{MistyRose3} t &
{\color{MistyRose4} t}\textcolor{MistyRose4} t &
{\color{NavajoWhite1} t}\textcolor{NavajoWhite1} t &
{\color{NavajoWhite2} t}\textcolor{NavajoWhite2} t &
{\color{NavajoWhite3} t}\textcolor{NavajoWhite3} t &
{\color{NavajoWhite4} t}\textcolor{NavajoWhite4} t &
{\color{OliveDrab1} t}\textcolor{OliveDrab1} t \\
{\color{OliveDrab2} t}\textcolor{OliveDrab2} t &
{\color{OliveDrab3} t}\textcolor{OliveDrab3} t &
{\color{OliveDrab4} t}\textcolor{OliveDrab4} t &
{\color{Orange1} t}\textcolor{Orange1} t &
{\color{Orange2} t}\textcolor{Orange2} t &
{\color{Orange3} t}\textcolor{Orange3} t &
{\color{Orange4} t}\textcolor{Orange4} t &
{\color{OrangeRed1} t}\textcolor{OrangeRed1} t &
{\color{OrangeRed2} t}\textcolor{OrangeRed2} t &
{\color{OrangeRed3} t}\textcolor{OrangeRed3} t &
{\color{OrangeRed4} t}\textcolor{OrangeRed4} t &
{\color{Orchid1} t}\textcolor{Orchid1} t &
{\color{Orchid2} t}\textcolor{Orchid2} t &
{\color{Orchid3} t}\textcolor{Orchid3} t &
{\color{Orchid4} t}\textcolor{Orchid4} t &
{\color{PaleGreen1} t}\textcolor{PaleGreen1} t &
{\color{PaleGreen2} t}\textcolor{PaleGreen2} t &
{\color{PaleGreen3} t}\textcolor{PaleGreen3} t &
{\color{PaleGreen4} t}\textcolor{PaleGreen4} t &
{\color{PaleTurquoise1} t}\textcolor{PaleTurquoise1} t &
{\color{PaleTurquoise2} t}\textcolor{PaleTurquoise2} t \\
{\color{PaleTurquoise3} t}\textcolor{PaleTurquoise3} t &
{\color{PaleTurquoise4} t}\textcolor{PaleTurquoise4} t &
{\color{PaleVioletRed1} t}\textcolor{PaleVioletRed1} t &
{\color{PaleVioletRed2} t}\textcolor{PaleVioletRed2} t &
{\color{PaleVioletRed3} t}\textcolor{PaleVioletRed3} t &
{\color{PaleVioletRed4} t}\textcolor{PaleVioletRed4} t &
{\color{PeachPuff1} t}\textcolor{PeachPuff1} t &
{\color{PeachPuff2} t}\textcolor{PeachPuff2} t &
{\color{PeachPuff3} t}\textcolor{PeachPuff3} t &
{\color{PeachPuff4} t}\textcolor{PeachPuff4} t &
{\color{Pink1} t}\textcolor{Pink1} t &
{\color{Pink2} t}\textcolor{Pink2} t &
{\color{Pink3} t}\textcolor{Pink3} t &
{\color{Pink4} t}\textcolor{Pink4} t &
{\color{Plum1} t}\textcolor{Plum1} t &
{\color{Plum2} t}\textcolor{Plum2} t &
{\color{Plum3} t}\textcolor{Plum3} t &
{\color{Plum4} t}\textcolor{Plum4} t &
{\color{Purple1} t}\textcolor{Purple1} t &
{\color{Purple2} t}\textcolor{Purple2} t &
{\color{Purple3} t}\textcolor{Purple3} t \\
{\color{Purple4} t}\textcolor{Purple4} t &
{\color{Red1} t}\textcolor{Red1} t &
{\color{Red2} t}\textcolor{Red2} t &
{\color{Red3} t}\textcolor{Red3} t &
{\color{Red4} t}\textcolor{Red4} t &
{\color{RosyBrown1} t}\textcolor{RosyBrown1} t &
{\color{RosyBrown2} t}\textcolor{RosyBrown2} t &
{\color{RosyBrown3} t}\textcolor{RosyBrown3} t &
{\color{RosyBrown4} t}\textcolor{RosyBrown4} t &
{\color{RoyalBlue1} t}\textcolor{RoyalBlue1} t &
{\color{RoyalBlue2} t}\textcolor{RoyalBlue2} t &
{\color{RoyalBlue3} t}\textcolor{RoyalBlue3} t &
{\color{RoyalBlue4} t}\textcolor{RoyalBlue4} t &
{\color{Salmon1} t}\textcolor{Salmon1} t &
{\color{Salmon2} t}\textcolor{Salmon2} t &
{\color{Salmon3} t}\textcolor{Salmon3} t &
{\color{Salmon4} t}\textcolor{Salmon4} t &
{\color{SeaGreen1} t}\textcolor{SeaGreen1} t &
{\color{SeaGreen2} t}\textcolor{SeaGreen2} t &
{\color{SeaGreen3} t}\textcolor{SeaGreen3} t &
{\color{SeaGreen4} t}\textcolor{SeaGreen4} t \\
{\color{Seashell1} t}\textcolor{Seashell1} t &
{\color{Seashell2} t}\textcolor{Seashell2} t &
{\color{Seashell3} t}\textcolor{Seashell3} t &
{\color{Seashell4} t}\textcolor{Seashell4} t &
{\color{Sienna1} t}\textcolor{Sienna1} t &
{\color{Sienna2} t}\textcolor{Sienna2} t &
{\color{Sienna3} t}\textcolor{Sienna3} t &
{\color{Sienna4} t}\textcolor{Sienna4} t &
{\color{SkyBlue1} t}\textcolor{SkyBlue1} t &
{\color{SkyBlue2} t}\textcolor{SkyBlue2} t &
{\color{SkyBlue3} t}\textcolor{SkyBlue3} t &
{\color{SkyBlue4} t}\textcolor{SkyBlue4} t &
{\color{SlateBlue1} t}\textcolor{SlateBlue1} t &
{\color{SlateBlue2} t}\textcolor{SlateBlue2} t &
{\color{SlateBlue3} t}\textcolor{SlateBlue3} t &
{\color{SlateBlue4} t}\textcolor{SlateBlue4} t &
{\color{SlateGray1} t}\textcolor{SlateGray1} t &
{\color{SlateGray2} t}\textcolor{SlateGray2} t &
{\color{SlateGray3} t}\textcolor{SlateGray3} t &
{\color{SlateGray4} t}\textcolor{SlateGray4} t &
{\color{Snow1} t}\textcolor{Snow1} t \\
{\color{Snow2} t}\textcolor{Snow2} t &
{\color{Snow3} t}\textcolor{Snow3} t &
{\color{Snow4} t}\textcolor{Snow4} t &
{\color{SpringGreen1} t}\textcolor{SpringGreen1} t &
{\color{SpringGreen2} t}\textcolor{SpringGreen2} t &
{\color{SpringGreen3} t}\textcolor{SpringGreen3} t &
{\color{SpringGreen4} t}\textcolor{SpringGreen4} t &
{\color{SteelBlue1} t}\textcolor{SteelBlue1} t &
{\color{SteelBlue2} t}\textcolor{SteelBlue2} t &
{\color{SteelBlue3} t}\textcolor{SteelBlue3} t &
{\color{SteelBlue4} t}\textcolor{SteelBlue4} t &
{\color{Tan1} t}\textcolor{Tan1} t &
{\color{Tan2} t}\textcolor{Tan2} t &
{\color{Tan3} t}\textcolor{Tan3} t &
{\color{Tan4} t}\textcolor{Tan4} t &
{\color{Thistle1} t}\textcolor{Thistle1} t &
{\color{Thistle2} t}\textcolor{Thistle2} t &
{\color{Thistle3} t}\textcolor{Thistle3} t &
{\color{Thistle4} t}\textcolor{Thistle4} t &
{\color{Tomato1} t}\textcolor{Tomato1} t &
{\color{Tomato2} t}\textcolor{Tomato2} t \\
{\color{Tomato3} t}\textcolor{Tomato3} t &
{\color{Tomato4} t}\textcolor{Tomato4} t &
{\color{Turquoise1} t}\textcolor{Turquoise1} t &
{\color{Turquoise2} t}\textcolor{Turquoise2} t &
{\color{Turquoise3} t}\textcolor{Turquoise3} t &
{\color{Turquoise4} t}\textcolor{Turquoise4} t &
{\color{VioletRed1} t}\textcolor{VioletRed1} t &
{\color{VioletRed2} t}\textcolor{VioletRed2} t &
{\color{VioletRed3} t}\textcolor{VioletRed3} t &
{\color{VioletRed4} t}\textcolor{VioletRed4} t &
{\color{Wheat1} t}\textcolor{Wheat1} t &
{\color{Wheat2} t}\textcolor{Wheat2} t &
{\color{Wheat3} t}\textcolor{Wheat3} t &
{\color{Wheat4} t}\textcolor{Wheat4} t &
{\color{Yellow1} t}\textcolor{Yellow1} t &
{\color{Yellow2} t}\textcolor{Yellow2} t &
{\color{Yellow3} t}\textcolor{Yellow3} t &
{\color{Yellow4} t}\textcolor{Yellow4} t &
{\color{Gray0} t}\textcolor{Gray0} t &
{\color{Green0} t}\textcolor{Green0} t &
{\color{Grey0} t}\textcolor{Grey0} t \\
{\color{Maroon0} t}\textcolor{Maroon0} t &
{\color{Purple0} t}\textcolor{Purple0} t &
{\color{black} t}\textcolor{black} t &
{\color{blue} t}\textcolor{blue} t &
{\color{brown} t}\textcolor{brown} t &
{\color{cyan} t}\textcolor{cyan} t &
{\color{darkgray} t}\textcolor{darkgray} t &
{\color{gray} t}\textcolor{gray} t &
{\color{green} t}\textcolor{green} t &
{\color{lightgray} t}\textcolor{lightgray} t &
{\color{lime} t}\textcolor{lime} t &
{\color{magenta} t}\textcolor{magenta} t &
{\color{olive} t}\textcolor{olive} t &
{\color{orange} t}\textcolor{orange} t &
{\color{pink} t}\textcolor{pink} t &
{\color{purple} t}\textcolor{purple} t &
{\color{red} t}\textcolor{red} t &
{\color{teal} t}\textcolor{teal} t &
{\color{violet} t}\textcolor{violet} t &
{\color{white} t}\textcolor{white} t &
{\color{yellow} t}\textcolor{yellow} t \\
\end{tabular}} \\
\begin{tabular}{*2c}
\textmd{text} {\mdseries text} & % = medium weight
\textrm{text} {\rmfamily text} \\ % = roman font
\textup{text} {\upshape text} & % = upright
\textbf{text} {\bfseries text} \\ % = bold font
\textsf{text} {\sffamily text} & % = sans-serif
\texttt{text} {\ttfamily text} \\ % = typewriter font
\textit{text} {\itshape text} & % = italic
\textsl{text} {\slshape text} \\ % = slanted/oblique
\multicolumn 2c{\emph{some \emph{really} emphasised}} \\ % = emphasised
\multicolumn 2c{\em some {\em really} emphasised} \\ % = emphasised
\textsc{I am} & {\scshape Death} \\ % = small caps
\end{tabular} &
\begin{tabular}{*2c}
\begin{tiny}text\end{tiny} &
{\tiny text} \\ % smallest size
\begin{scriptsize}text\end{scriptsize} &
{\scriptsize text} \\ % size of (first level) super and supscripts
\begin{footnotesize}text\end{footnotesize} &
{\footnotesize text} \\ % size of footnotes
\begin{small}text\end{small} &
{\small text} \\
\begin{normalsize}text\end{normalsize} &
{\normalsize text} \\ % size of regular text
\begin{large}text\end{large} &
{\large text} \\
\begin{Large}text\end{Large} &
{\Large text} \\
\begin{LARGE}text\end{LARGE} &
{\LARGE text} \\
\begin{huge}text\end{huge} &
{\huge text} \\
\begin{Huge}text\end{Huge} &
{\Huge text} \\
\end{tabular} \\
\end{longtable}

re: L, \(\log_2 255 \ne 8\)

\(\uveci + \uvecj + \uveck\)

\begin{lemma}[Four by two]
Where \(4 \defeq 1 + 1 + 1 + 1\) and \(2 \defeq 1 + 1\)
\begin{equation*}
\Forall k \in \Reals \colon \frac 42 = 2
\end{equation*}
\end{lemma}

\begin{proof}
Suppose that
\begin{equation*}
\frac 42 = x = \frac{1 + x}{1 + x^{-1}}
\end{equation*}
for some \(x > 0\) (hence \(x^{-1} \neq -1\) and the expression is defined).

Now,
\begin{alignat*} 2
 &&\frac 42 &= \frac{1 + x}{1 + x^{-1}} \\
 &\iff{}& 4(1 + x^{-1}) &= 2(1 + x) \\
 &\iff{}& 2x^2 + 2x - 4x - 4 &= 0 \\
 &\iff{}& 2(x - 2)(x + 1) &= 0 \\
 &\iff{}& x &\in \set{2, -1} 
\end{alignat*}
But \(x > 0\) and therefore \(\frac 42 = x = 2\).
\end{proof}

% Some right proper weird stuff with scalebox
% https://tex.stackexchange.com/questions/164040/scalebox-doesnt-compile
\scalebox{1.5}{\parbox{\linewidth}{%
 \begin{equation*}
  \left\{ \begin{array}{*5c}
  \underbrace{\dv<\text{\textcolor{Orange4} \Cat}>{t}}_{\text{cat interest}}
   &=& \underbrace{
    \vphantom{\dv<\text{\textcolor{Orange4} \Cat}>{t}}
    \pi \text{\textcolor{Blue4} \PencilLeftDown}}_{
      \text{productivity term}}
   &-& \underbrace{
    \vphantom{\dv<\text{\textcolor{Orange4} \Cat}>{t}}
    \mu \text{\textcolor{Orange4} \Cat}}_{
    \text{mew term}} \\[6ex]
  \underbrace{\dv<\text{\textcolor{Blue4} \PencilLeftDown}>{t}}_{
   \text{productivity}}
   &=& \underbrace{
    \vphantom{\dv<\text{\textcolor{Blue4} \PencilLeftDown}>{t}}
    -\kappa \text{\textcolor{Orange4} \Cat}}_{
      \text{cat term}}
   &+& \underbrace{
    \vphantom{\dv<\text{\textcolor{Blue4} \PencilLeftDown}>{t}}
    \int_C \vec F \cdot \diff \vec s}_{\text{work term}}
  \end{array} \right.
 \end{equation*}
}}

\nocite{*}

\bibliographystyle{agsm}
\bibliography{sources}
\end{document}
