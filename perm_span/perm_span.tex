\documentclass[a4paper,12pt]{article}
\author{Izaak van Dongen (\texttt{imv26})}
\title{On the span of the permutation matrices in \(\mathcal M_{n \times n}\)}

\usepackage{mysty}
\usepackage{mymaths}

% Embed source files into PDF in case of loss. You can view or extract the
% source files by doing `pdfdetach -list <file.pdf>` or
% `pdfdetach -saveall <file.pdf>`, using pdfdetach from poppler, or some other
% suitable method.
\usepackage[main]{embedall}
\embedfile{mymaths.sty}
\embedfile{mysty.sty}

\begin{document}
\maketitle

We concern ourselves with the set \(\mathscr P_n = \Span S_n\), where
\(S_n\) is thought of as those matrices in \(\mathcal M_{n \times n}\) each of
whose columns and rows has all zeroes, except for exactly one \(1\). Most of the
following works for any ground field, so long as it has more than one element,
since all we really appeal to is the existence of distinct \(0\) and \(1\)
elements.

Note that every row and column of such a matrix has components summing to \(1\).
Particularly, each row and column sums to the same number.

This property is preserved under linear combinations, but for \(n\) reasonably
large, there are matrices not having this property. Hence, despite the fact that
\(n!\) is much bigger than \(n^2\), \(\mathscr P_n \ne \mathcal M_{n \times n}\)
for \(n > 1\).

This means that \(\mathscr P_n\) is a subspace of the (non-diagonal) magic
square space of \(n \times n\) matrices, which we will call \(\mathfrak M_n\).

\begin{theorem}
 \(\dim \mathfrak M_n = n^2 - 2n + 2\).
\end{theorem}
\begin{proof}
 We are fresh-faced linear algebraists, so we do some yucky calculation.

 Firstly, for \(1 \le i, j \le n - 1\), define a block matrix
 \begin{equation*}
  \mat B_{ij} =
  \begin{pmatrix}
   \mat E_{ij} & \vec v_i \\[1ex]
   \tran{\vec v_j} & 3 - n
  \end{pmatrix} \in \mathfrak M_n
 \end{equation*}
 where \(\mat E_{ij}\) is the usual \((n - 1) \times (n - 1)\) ``basis matrix''
 with entries \((\mat E_{ij})_{k\ell} = \delta_{ik} \delta_{j\ell}\), and
 \(\vec v_i\) is the unusual basis vector for \(\R^{n - 1}\) with
 \([\vec v_i]_j = 1 - \delta_{ij}\).

 Then since the \(\mat E_{ij}\) are all linearly independent, all the
 \(\mat B_{ij}\) are independent. Furthermore, \(\mat B_{ij} \in \mathfrak M_n\)
 for all \(i, j \le n - 1\), since each row and each column sums to \(1\).

 Also define \(\mat B\) with \([\mat B]_{ij} = 1\). \(\mat B\) is then also in
 \(\mathfrak M_n\), and \(\mat B\) is linearly independent of the previous
 \(\mat B_{ij}\)s, since if it was to be in their span, it would have to be
 precisely the sum of each \(\mat B_{ij}\), but then we would have
 \(1 = (\mat B)_{nn} = (3 - n)(n - 1)^2\), implying that
 \(\abs{3 - n} = \abs{n - 1} = 1\), implying \(n = 2\). But when \(n = 2\), we
 only have the matrix \(\mat B_{11}\), which is not equal to \(\mat B\).

 This proves that \(\dim \mathfrak M_n \ge (n - 1)^2 + 1 = n^2 - 2n + 2\), as we
 have a linearly independent set of size \((n - 1)^2 + 1\).

 Now suppose \(\mat A \in \mathfrak M_n\). Say \(\mat A = (a_{ij})\). Note that
 if we are given \(a_{ij}\) for \(1 \le i, j \le n - 1\), and we are also given
 \(a_{1n}\), then we can deduce that the ``magic constant'' of \(\mat A\) must
 be \(C_{\mat A} = \sum_{j = 1}^n a_{1j}\), and from there, we can deduce that
 for any \(i, j \le n - 1\),
 \begin{align*}
  a_{in} = C_{\mat A} - \sum_{k = 1}^{n - 1} a_{ik}, &&
  a_{nj} = C_{\mat A} - \sum_{k = 1}^{n - 1} a_{kj}
 \end{align*}
 and that \(a_{nn} = C_{\mat A} - \sum_{k = 1}^{n - 1} a_{nk}\).

 To prove that this does give a matrix \(\mat A\) in \(\mathfrak M_n\), the only
 thing to check is that \(\sum_{k = 1}^n a_{kn} = C_\mat A\). This isn't so
 hard, since
 \begin{align*}
  \sum_{k = 1}^n a_{kn}
   &= \sum_{k = 1}^n \bracks[\Big]{
       C_{\mat A} - \sum_{\ell = 1}^{n - 1} a_{k\ell}} \\
   &= n \cdot C_{\mat A} -
       \sum_{k = 1}^n \sum_{\ell = 1}^{n - 1} a_{k\ell} \\
   &= n \cdot C_{\mat A} -
       \sum_{\ell = 1}^{n - 1} \sum_{k = 1}^n a_{k\ell} \\
   &= n \cdot C_{\mat A} -
       \sum_{\ell = 1}^{n - 1} C_{\mat A} \\
   &= C_{\mat A}
 \end{align*}
 It follows that by setting one of the pre-supposed \(a_{ij}\)s-and-\(a_{1n}\)
 to \(1\) and the rest to \(0\), and applying this procedure, we can obtain a
 set of \((n - 1)^2 + 1\) vectors that span \(\mathfrak M_n\), because each
 remaining component has been expressed as a linear combination of the original
 given components.
\end{proof}
\begin{corollary}
 The \(\mat B_{ij}\)-and-\(\mat B\) we found are in fact a basis for
 \(\mathfrak M_n\).
\end{corollary}
\begin{theorem}
 \(\mathscr P_n = \mathfrak M_n\).
\end{theorem}
\begin{proof}
 We only need to show \(\mathfrak M_n \subspace \mathscr P_n = \Span S_n\).

 We do this as yuckily as we possibly can.

 It suffices to write each basis element \(\mat B_{ij}\) of \(\mathfrak M_n\),
 from earlier, as a linear combination of permutation matrices.

 We do \(\mat B\) first. This one is nice, because by symmetry,
 \(\mat B = ((n - 1)!)^{-1} \sum_{\mat P \in S_n} \mat P\).

 Now we do \(\mat B_{11}\).

 For \(1 \le i \le n - 1\), define the permutation matrix \(\mat P_i\) by its
 action on the standard basis \((\vec e_i)\):
 \begin{itemize}
  \item
   For \(1 \le j \le n - 1\),
   \(\mat P_i \vec e_j = \vec e_{i + j \bmod n - 1}\), taking the result of the
   modulo operation to be between \(1\) and \(n - 1\).
  \item
   \(\mat P_i \vec e_n = \vec e_n\)
 \end{itemize}
 That is to say \(\mat P_i\) fixes \(\vec e_n\) and cycles
 \(\vec e_1, \dotsc, \vec e_{n - 1}\) forwards by \(i\) steps. Note that
 \(\mat P_{n - 1} = \mat I_n\).

 Then let \(\mat M = \sum_{i = 1}^{n - 2} \mat P_i = (m_{ij})\).
 Clearly, \(m_{nn} = n - 2\).

 Furthermore, if \(1 \le i, j \le n - 1\), then
 \begin{itemize}
  \item
   If \(i \ne j\), then \(m_{ij} = 1\), since there is exactly one \(\mat P_k\)
   in the sum with ``\(ij\)''-component equal to \(1\)
   (that is, \(\mat P_{i - j \bmod n - 1}\)).
  \item
   \(m_{ii} = 0\), as no \(\mat P_k\) in the sum fixes any of the first
   \(n - 1\) basis vectors.
  \item
   \(m_{in} = 0\), as each \(\mat P_k\) in the sum has ``\(in\)''-component
   equal to \(0\).
  \item
   Similarly, \(m_{ni} = 0\).
 \end{itemize}
 So let \(\mat N = \mat B - \mat M - \mat B_{11} = (n_{ij})\).

 Since \(\mat B\) and \(\mat M\) are in \(\mathscr P_n\), it suffices to prove
 that \(\mat N \in \mathscr P_n\). But, observe:
 \begin{itemize}
  \item
   If \(2 \le i, j \le n - 1\), then
   \begin{itemize}
    \item
     \(n_{ii} = b_{ii} - m_{ii} - [\mat B_{11}]_{ii} = 1 - 0 - 0 = 1\).
    \item
     If \(i \ne j\), then
     \(n_{ij} = b_{ij} - m_{ij} - [\mat B_{11}]_{ij} = 1 - 1 - 0 = 0\).
    \item
     \(n_{1j} = b_{1j} - m_{1j} - [\mat B_{11}]_{1j} = 1 - 1 - 0 = 0\).
    \item
     \(n_{nj} = b_{nj} - m_{nj} - [\mat B_{11}]_{nj} = 1 - 0 - 1 = 0\).
    \item
     \(n_{i1} = b_{i1} - m_{i1} - [\mat B_{11}]_{i1} = 1 - 1 - 0 = 0\).
    \item
     \(n_{in} = b_{in} - m_{in} - [\mat B_{11}]_{in} = 1 - 0 - 1 = 0\).
   \end{itemize}
  \item
   \(n_{11} = b_{11} - m_{11} - [\mat B_{11}]_{11} = 1 - 0 - 1 = 0\).
  \item
   \(n_{1n} = b_{1n} - m_{1n} - [\mat B_{11}]_{1n} = 1 - 0 - 0 = 1\).
  \item
   \(n_{nn}
     = b_{nn} - m_{nn} - [\mat B_{11}]_{nn} = 1 - (3 - n) - (n - 2) = 0\).
  \item
   \(n_{n1} = b_{n1} - m_{n1} - [\mat B_{11}]_{n1} = 1 - 0 - 0 = 1\).
 \end{itemize}
 So in fact, \(\mat N\) is the permutation matrix that transposes \(\vec e_1\)
 and \(\vec e_n\), and certainly \(\mat N \in \mathscr P_n\).

 Now, noting that \(S_n\) acts transitively on
 \(\set{\mat B_{ij} \mid 1 \le i, j \le n - 1}\) by multiplication\footnote{%
  Exercise :D}, fix some other basis matrix
 \(\mat B_{ij}\), and fix a permutation matrix \(\mat P\) such that
 \(\mat B_{ij} = \mat P \mat B_{11}\). Write
 \(\mat B_{11} = \sum_{\mat Q \in S_n} \lambda_{\mat Q} \mat Q\).

 Then
 \(\mat B_{ij}
   = \sum_{\mat Q \in S_N} \lambda_{\mat Q} (\mat P \mat Q)
   \in \mathscr P_n\).
 %TODO
\end{proof}
\begin{corollary}
 \(\dim \mathscr P_n = n^2 - 2n + 2\).
\end{corollary}


\end{document}
