\documentclass[a4paper,11pt]{article}
\title{Maths}
\author{Izaak van Dongen}

% so the title can be accessed by fancyhdr (and is automatically correctly
% spelled etc)
\makeatletter
\let\thetitle\@title
\makeatother

%% fonts
%\usepackage[p,osf]{cochineal}
%\usepackage[scale=.95,type1]{cabin}
%\usepackage[cochineal,bigdelims,cmintegrals,vvarbb]{newtxmath}
%% fixed width font with 80 chars per listing line
%\usepackage[scaled=.94]{newtxtt}
%\usepackage[cal=boondoxo]{mathalfa}

% provides mathbb
\usepackage{amsfonts}

% make the document take up more of the page
\usepackage[margin=1in,headheight=13.6pt]{geometry}

% no paragraph indent
\usepackage[parfill]{parskip}

% custom document header/footer
\usepackage{fancyhdr}
\usepackage{lastpage}

\pagestyle{fancy}
\fancyhf{}
\lhead{\thetitle}
\rhead{Izaak van Dongen}
\rfoot{Page \thepage\ of \pageref{LastPage}}

% pretty table rules and multirow entries. Also page-breaking tables
\usepackage{booktabs}
\usepackage{multirow}
\usepackage{longtable}

% plotting mathematical functions (needs version request)
\usepackage{pgfplots}
\pgfplotsset{compat=1.15}

% \url function and clickable table of contents. no ugly red boxes though
\usepackage[hidelinks]{hyperref}

% maths symbols and other stuff (supersedes the ams* packages)
\usepackage{mathtools}

% provides typesetting of derivatives
\usepackage{commath}

% typesetting units
\usepackage{siunitx}

% For framing definitions
\usepackage[framemethod=tikz]{mdframed}
\usepackage[most]{tcolorbox}

\newtcolorbox{definition}{
freelance,
before=\par\vspace{2\bigskipamount}\noindent,
after=\par\bigskip,
frame code={
  \node[
  anchor=south west,
  inner xsep=8pt,
  xshift=8pt,
  rounded corners=5pt,
  font=\bfseries\color{white},
  fill=gray] at (frame.north west) (tit) {\strut Definition:};
  \draw[
  line width=3pt,
  rounded corners=5pt,gray
  ] (tit.west) -| (frame.south west) -- ([xshift=15pt]frame.south west);
},
interior code={},
top=2pt
}

% for better table of contents stuff, providing the \listof* commands and not
% listing the tables in the table of contents
\usepackage[nottoc,notlof,notlot]{tocbibind}

% more advanced handling of utf8 and fonts or something. apparently good to have
\usepackage[utf8]{inputenc}
\usepackage[T1]{fontenc}

% bibliography management with square braces for citations
\usepackage[square,numbers]{natbib}

% graphics, like eps files and stuff (supersedes graphics)
\usepackage{graphicx}

% used to horizontally align floats
\usepackage{subfig}

% used for figures
\usepackage{float}

% needed for colouring and stuff (xcolor supersedes color)
\usepackage{xcolor}

\definecolor{codegreen}{rgb}{ 0,0.6,0}

% listings of code
\usepackage{minted}
\setminted{breaklines,
           breakbytokenanywhere,
           linenos
}
\usemintedstyle{friendly}
% bigger line numbers
\renewcommand\theFancyVerbLine{\footnotesize\arabic{FancyVerbLine}}

% that can break across pages while being captioned figures
\usepackage{caption}
\newenvironment{longlisting}
{\addvspace{\baselineskip}\captionsetup{type=listing}}
{\addvspace{\baselineskip}}

% allow maths to break across pages
\allowdisplaybreaks

\begin{document}
    \maketitle\thispagestyle{empty} % no page number under title
    \tableofcontents

    %FIXME generally: add links to everything.

    \section{Miscellaneous}

    \subsection{Numerical solutions to equations}

    \subsection{Fundamental Theorem of Arithmetic}

    \subsection{Fundamental Theorem of Algebra}

    \subsection{Dimensional Analysis}

    \subsection{Difference of two Squares}

    \subsection{Infinitude of Primes}

    \subsection[Irrationality of $\sqrt 2$]{Irrationality of \boldmath$\sqrt 2$}

    Assume $\sqrt 2 = \frac ab : a, b \in \mathbb{Z}$.

    \subsection{Pythagoras' Theorem} \label{sec:pythagoras}

    %FIXME add diagram

    If a triangle has sides $a$, $b$, $c$ opposed by angles $A$, $B$, $C$,
    then $C = \frac 12 \pi$ iff $a^2 + b^2 = c^2$.

    \subsection{Quadratic formula} \label{sec:quad_formula}

    Roots of $P(x) = ax^2 + bx + c$ are
    $x = \frac{-b \pm \sqrt{b^2 - 4ac}}{2a}$

    $P(x) = ax^2 + bx + c = 0$, so $x^2 + \frac{b}{a}x + \frac{c}{a} = 0$
    By completing the square, we have
    \\$(x + \frac{b}{2a}x)^2 = (\frac{b}{2a})^2 - \frac{c}{a}
    = \frac{b^2 - 4ac}{4a^2}$. It follows that
    \\$x + \frac{b}{2a} = \pm \sqrt{\frac{b^2 - 4ac}{4a^2}}
    = \pm \frac{\sqrt{b^2 - 4ac}}{2a}$ so
    \\$x = \frac{-b \pm \sqrt{b^2 - 4ac}}{2a}$ as required.

    Corollaries are that the number of real roots is determined by the
    discriminant \\$\Delta = b^2 - 4ac$.
    \\$\Delta = 0 \Rightarrow P(x)$ has one real root with multiplicity.
    \\$\Delta < 0 \Rightarrow P(x)$ has no real roots.
    \\$\Delta > 0 \Rightarrow P(x)$ has two real roots.

    \subsection{Cauchy-Shwarz inequality for $\mathbb{R}^n$}

    For sequences $u_i, v_i \in \mathbb{R}$,
    $(\sum u_i v_i)^2 \le (\sum u_i^2) (\sum v_i^2)$.

    Consider
    $P(x) = \sum (u_i x + v_i)^2 = 0$. Note that $P(x) \ge 0$ and is a quadratic
    \\$(\sum u_i^2) x^2 + (\sum 2 u_i v_i) x + \sum v_i^2 = 0$, so has
    at most one real root. Therefore, the discriminant
    \\$\Delta = b^2 - 4ac \le 0$ (by \ref{sec:quad_formula}), so
    $(\sum 2 u_i v_i)^2 - 4(\sum u_i^2)(\sum v_i^2) \le 0 \Rightarrow
     (\sum u_i v_i)^2 \le (\sum u_i^2) (\sum v_i^2)$, as required.

    \subsection{AM-GM inequality}

    For a sequence $u_i \in \mathbb{R}$, for $1 \le i \le k$ and $u_i \ge 0$,
    $(\prod u_i)^\frac{1}{k} \le \frac{1}{k}\sum u_i$.

    \subsection{Square Triangular numbers}

    %FIXME: add link annotations

    The sequence of perfect squares $\text{ST}_n = a_n^2$
    such that $a_n^2 = \frac 12 b_n(b_n + 1)$, where
    $a, b \in \mathbb{Z}_0^+$.

    Where $\frac{b}{c}$ is the $n$th convergent of $\sqrt 2$,
    $\text{ST}_n = b^2 c^2$.

    $\text{ST}_n$ also satisfies the recurrence relations where
    $\text{ST}_0 = 0$ and $\text{ST}_1 = 1$
    \\$\text{ST}_n = 34\text{ST}_{n - 1} - \text{ST}_{n - 2} + 2$ and
    \\$a_n = 6a_{n - 1} - a_{n - 2}$.

    \subsection{Fibonacci Sequence}

    The Fibonacci numbers $F_n: n \in \mathbb{N}$ are such that $F_1 = F_2 = 1$
    and $F_n = F_{n - 1} + F_{n - 2}$.

    $F_n$ can be given by the closed form
    $\displaystyle F_n = \frac{\varphi^n - \psi^n}{\sqrt 5}$, where
    $\displaystyle \varphi = \frac{1 + \sqrt 5} 2$ and
    $\displaystyle \psi = \frac{1 - \sqrt 5} 2$.

    Combined with Exponentation by Squaring (\ref{sec:exp_by_squaring}), and
    some simple surd arithmetic, this should be a particularly fast way to
    calculate $F_n$, as compared to utterly na\"ive recursion, or somewhat
    faster iteration or memoized recursion.

    \subsection{de Moivre's Theorem}

    \section{Discrete Maths (Computer Science)}

    \subsection{Exponentiation by Squaring} \label{sec:exp_by_squaring}

    For $y \in \mathbb{Z}_0^+$, $x^y$ is given by
    $x^y =
        \begin{cases}
        1 & y = 0 \\
        x & y = 1 \\
        (x ^ 2)^{\frac 12 y} & y \equiv 0 \pmod 2\\
        x(x ^ 2)^{\frac 12 (y-1)} & y \equiv 1 \pmod 2\\
        \end{cases}$

    This follows from the fact that
    $x^{2y'} = (x^2)^{y'}$ and $x^{2y' + 1} = x(x^2)^{y'}$.

    \section{Set Theory}

    \section{Linear Algebra}

    \section{Geometry}

    \subsection{Area of a Circle}

    %FIXME add diagram

    From the definition of a circle as the set of points equidistant from a
    centre point, we form the equation of a circle using \ref{sec:pythagoras}:
    $x^2 + y^2 = r^2$, where the centre is $(0, 0)$ and the radius is $r$.

    We can find half the area enclosed by the curve by integrating:

    $\displaystyle \frac 12 A = \int_{-r}^r \sqrt{r^2 - x^2} \dif x$. We use the
    substitution (\ref{sec:calc_trig_substitution})
    $x = r \cos \theta \iff \dod x\theta = -r \sin \theta$ so that

    $\displaystyle \frac 12 A = \int_{\arccos -1}^{\arccos 1}
        r\sqrt{1 - \cos^2 \theta} \cdot -r \sin \theta \dif \theta
      = -\int_\pi^0 r^2 \sin^2 \theta \dif \theta$, by \ref{sec:trig_pythag}. We
      use \ref{sec:trig_double_angle} to derive that

    $\displaystyle \frac 12 A = -\int_\pi^0
        r^2 \frac 12 (1 - \cos 2 \theta) \dif \theta \Rightarrow
     A = r^2 \int_0^\pi (1 - \cos 2 \theta) \dif \theta =
     r^2\left[\theta - \frac 12 \sin 2 \theta \right]_0^\pi
     \\= r^2 \left(\pi - \frac 12 \sin 2 \pi -
                   \left(0 - \frac 12 \sin 0\right)\right) \boxed{= \pi r^2}$

    One might alternatively use the parametric form of a circle,
    $y = r \sin \theta$ and $x = r \cos \theta$
    (note that $x^2 + y^2 = r^2(\sin^2 \theta + \cos^2 \theta) = r^2$
    (\ref{sec:trig_pythag})), and calculate the area using
    \ref{sec:calc_parametric_area}:

    $\displaystyle A = \int_0^{2\pi}
        r \sin \theta \cdot -r \sin \theta \dif \theta =
     \frac 12 r^2 \int_0^{2\pi} (1 - \cos 2 \theta) \dif \theta =
     \frac 12 r^2 \left[\theta - \frac 12 \sin 2 \theta\right]_0^{2\pi}
     \\= \frac 12 r^2 \left(2\pi - \frac 12 \sin 4 \pi -
                      \left(0 - \sin 0\right)\right) = \pi r^2$

    \subsection{Volume of a sphere}

    \subsection{Circle Theorems}

    \section{Trigonometry}

    \subsection{Definitions} \label{sec:trig_definitions}

    %FIXME add diagram

    $\displaystyle\sin \theta = \frac OH$

    $\displaystyle\cos \theta = \frac AH$

    $\displaystyle\tan \theta = \frac OA = \frac{\sin \theta}{\cos \theta}$

    \subsubsection{Angular relationships} \label{sec:trig_periodic}

    %FIXME add more identities, diagrams (or refer to previous diagrams).

    Because they are in the same right triangle, and the sum of angles in a
    triangle is $\pi$ ($\ang{180}$) (\ref{sec:polygon_angles}), the angle other
    than $\theta$ must be $\frac 12 \pi - \theta$. Therefore,
    \\$\sin \theta \equiv \cos(\frac 12 \pi - \theta)$ and
    $\cos \theta \equiv \sin(\frac 12 \pi - \theta)$.

    We also have that
    \\$\sin(\theta + \pi) \equiv \cos \theta$, and
    $\cos(\theta + \pi) \equiv \sin \theta$.

    From this, one might see that therefore
    \\$\sin(\theta + 2\pi) \equiv \sin \theta$,
    $\cos(\theta + 2\pi) \equiv \cos \theta$, and in fact,
    \\$\sin(\theta + 2\pi n) \equiv \sin \theta \iff n \in \mathbb{Z}$,
    and similarly
    \\$\cos(\theta + 2\pi n) \equiv \cos \theta \iff n \in \mathbb{Z}$.

    However, $\tan(\theta + \pi) \equiv \tan \theta$, so
    \\$\tan(\theta + \pi n) \equiv \tan \theta \iff n \in \mathbb{Z}$

    Therefore, the \textit{period} of $\sin \theta$ and $\cos \theta$ is $2\pi$,
    whereas the period of $\tan \theta$ is $\pi$.

    \subsubsection{Parity} \label{sec:trig_parity}

    From \ref{sec:trig_definitions}, we see that $\sin \theta$ and $\tan \theta$
    are \textit{odd} functions:
    \\$\sin -x = -\sin x$ and $\tan -x = -\tan x$,
    whereas $\cos \theta$ is an \textit{even} function:
    \\$\cos -x = \cos x$.

    \subsubsection{Reciprocal functions} \label{sec:trig_reciprocal}

    $\displaystyle\sec \theta = \frac{1}{\cos \theta}$

    $\displaystyle\csc \theta = \operatorname{cosec} \theta
        = \frac{1}{\sin \theta}$

    $\displaystyle\cot \theta = \frac{1}{\tan \theta}
        = \frac{\cos \theta}{\sin \theta}$

    \subsection{Special Angles}

    %FIXME: diagrams
    %FIXME: talk about deriving other special angles using formulae

    \subsection{Cosine Rule}

    \subsection{Sine Rule}

    \subsection{Sine Area Rule}

    \subsection{Pythagorean Identities} \label{sec:trig_pythag}

    It follows from the triangle definition of the trig functions, and
    Pythagoras' Theorem (\ref{sec:pythagoras}) that
    \\$\cos^2 \theta + \sin^2 \theta \equiv 1$, and then from the definition of
    $\tan$, $\cot$, $\sec$, $\csc$ (\ref{sec:trig_reciprocal}) that
    \\$\tan^2 \theta + 1 \equiv \sec^2 \theta$ and
    \\$\cot^2 \theta + 1 \equiv \csc^2 \theta$.

    Equivalent formulations that are sometimes useful include:

    %FIXME finish this

    \subsection{Compound Angle Identities} \label{sec:comp_angle}

    %FIXME add diagram

    It can be shown geometrically that
    \\$\sin(\alpha + \beta) \equiv
       \sin \alpha \cos \beta +  \sin \beta \cos \alpha$

    From this, it follows that
    \\$\cos(\alpha + \beta) \equiv
       \cos \alpha \cos \beta - \sin \alpha \sin \beta$
    \\as $\cos(\alpha + \beta) \equiv \sin(\frac 12 \pi - \alpha - \beta) \equiv
        \sin(\frac 12 \pi - \alpha)\cos \beta +
        \sin \beta \cos(\frac 12 \pi - \alpha) \equiv
        \cos \alpha \cos \beta + \sin \beta \sin \alpha$
    (using section \ref{sec:trig_periodic}). This may also be shown
    with a similar geometrical argument.

    We can now deduce that

    $\displaystyle \tan(\alpha + \beta) \equiv
        \frac{\tan \alpha + \tan \beta}{1 - \tan \alpha \tan \beta}$,

    as $\displaystyle\tan(\alpha + \beta) \equiv
        \frac{\sin(\alpha + \beta)}{\cos(\alpha + \beta)} \equiv
        \frac{\sin \alpha \cos \beta + \sin \beta \cos \alpha}
             {\cos \alpha \cos \beta + \sin \alpha \sin \beta} \equiv
        \frac{\frac{\sin \alpha}{\cos \alpha} + \frac{\sin \beta}{\cos \beta}}
             {\frac 11 + \frac{\sin \alpha \sin \beta}{\cos \alpha \cos \beta}}
             \equiv
        \frac{\tan \alpha + \tan \beta}{1 - \tan \alpha \tan \beta}$.

    Results for a function of a difference are derived easily from the parity of
    each function (\ref{sec:trig_parity}). One might also just assume it's
    probably ok to flip signs. In either case, the full results are:

    $\sin \alpha \pm \beta \equiv
       \sin \alpha \cos \beta \pm \sin \beta \cos \alpha$
    \\$\cos \alpha \pm \beta \equiv
       \cos \alpha \cos \beta \mp \sin \alpha \sin \beta$

    $\displaystyle \tan \alpha \pm \beta \equiv
        \frac{\tan \alpha \pm \tan \beta}{1 \mp \tan \alpha \tan \beta}$

    \subsubsection{Double Angle Formulae} \label{sec:trig_double_angle}

    $\sin 2\theta \equiv
       2\sin \theta \cos \theta$
    \\$\cos 2\theta \equiv
       \cos^2 \theta - \sin^2 \theta \equiv
       2\cos^2 \theta - 1 \equiv 1 - 2\sin^2 \theta$ (by \ref{sec:trig_pythag}).

    $\displaystyle \tan 2\theta \equiv
        \frac{2\tan \theta}{1 - \tan^2 \theta}$

    From these, it follows that

    $\displaystyle \sin^2 \theta \equiv
        \frac{1 - \cos 2 \theta} 2$ and

    $\displaystyle \cos^2 \theta \equiv
        \frac{1 + \cos 2 \theta} 2$, used in \ref{calc:common}.

    $\displaystyle \tan^2 \theta \equiv \frac{\sin^2 \theta}{\cos^2 \theta}
        \equiv \frac{1 - \cos 2 \theta}{1 + \cos 2 \theta}
        \equiv \frac{1 - \cos^2 2 \theta}{(1 + \cos 2 \theta)^2}
        \equiv \left(\frac{\sin 2 \theta}{1 + \cos 2 \theta}\right)^2
        \equiv \frac{(1 - \cos 2 \theta)^2}{1 - \cos^2 2 \theta}
        \equiv \left(\frac{1 - \cos 2 \theta}{\sin 2 \theta}\right)^2$

    It follows that

    $\displaystyle \sin \theta \equiv
        \pm\sqrt{\frac{1 - \cos 2 \theta} 2}$, ($\pm$ depends on the quadrant)
    (so $\displaystyle \sin \frac \theta 2 \equiv
        \pm\sqrt{\frac{1 - \cos \theta} 2}$) and

    $\displaystyle \cos \theta \equiv
        \pm\sqrt{\frac{1 + \cos 2 \theta} 2}$, (so
    $\displaystyle \cos \frac \theta 2 \equiv
        \pm\sqrt{\frac{1 + \cos \theta} 2}$).

    $\displaystyle \tan \theta
        \equiv \frac{\sin 2 \theta}{1 + \cos 2 \theta}
        \equiv \frac{1 - \cos 2 \theta}{\sin 2 \theta}$
    (so $\displaystyle \tan \frac \theta 2
        \equiv \frac{\sin \theta}{1 + \cos \theta}
        \equiv \frac{1 - \cos \theta}{\sin \theta}$)

    \subsection{Sum-Product Identities}

    $\displaystyle \sin \alpha \cos \alpha \equiv
        \frac{\sin(\alpha + \beta) + \sin(\alpha - \beta)} 2$

    \subsection{Product-Sum Identities}

    %FIXME: annotate sections used

    The sum of sines, $\displaystyle \boxed{\sin \alpha + \sin \beta \equiv
        2 \sin \frac{\alpha + \beta}2 \cos \frac{\alpha - \beta}2}$.

    We derive by considering the RHS:

    $\displaystyle
     2 \sin \frac{\alpha + \beta} 2 \cos \frac{\alpha - \beta} 2 \equiv
     2 \left(\sin \frac \alpha 2 \cos \frac \beta 2 +
             \sin \frac \beta 2 \cos \frac \alpha 2\right)
       \left(\cos \frac \alpha 2 \cos \frac \beta 2 +
             \sin \frac \alpha 2 \sin \frac \beta 2\right) \\\\\equiv
     2 \left(\sin \frac \alpha 2 \cos \frac \alpha 2 \cos^2 \frac \beta 2 +
             \sin^2 \frac \alpha 2 \sin \frac \beta 2 \cos \frac \beta 2 +
             \sin \frac \beta 2 \cos^2 \frac \alpha 2 \cos \frac \beta 2 +
             \sin \frac \alpha 2 \sin^2 \frac \beta 2 \cos \frac \alpha 2
             \right) \\\\\equiv
     2 \left(\frac 12 \sin \alpha \cdot \frac{1 + \cos \beta} 2 +
             \frac 12 \sin \beta \cdot \frac{1 - \cos \alpha} 2 +
             \frac 12 \sin \beta \cdot \frac{1 + \cos \alpha} 2 +
             \frac 12 \sin \alpha \cdot \frac{1 - \cos \beta} 2
             \right) \\\\\equiv
     \frac 12 \left(\sin \alpha (1 + \cos \beta + 1 - \cos \beta) +
                    \sin \beta (1 + \cos \alpha + 1 - \cos \alpha)
                   \right) \\\\\equiv
     \frac 12 (2 \sin \alpha + 2 \sin \beta) \equiv
     \sin \alpha + \sin \beta$, as required.

    Now, it becomes fairly trivial to show that

    $\displaystyle \boxed{\sin \alpha - \sin \beta =
        \sin \alpha + \sin -\beta =
        2 \sin \frac{\alpha - \beta} 2 \cos\frac{\alpha + \beta} 2}$.

    We also aim to show that
    $\displaystyle \boxed{\cos \alpha + \cos \beta \equiv
        2 \cos\frac{\alpha + \beta} 2 \cos \frac{\alpha - \beta} 2}$.

    We again derive from the RHS:

    $\displaystyle 2 \cos\frac{\alpha + \beta} 2 \cos \frac{\alpha - \beta} 2
        \equiv 2 \left(\cos \frac \alpha 2 \cos \frac \beta 2 -
                       \sin \frac \alpha 2 \sin \frac \beta 2 \right)
                 \left(\cos \frac \alpha 2 \cos \frac \beta 2 +
                       \sin \frac \alpha 2 \sin \frac \beta 2 \right)
                \\\\\equiv
     2 \left(\cos^2 \frac \alpha 2 \cos^2 \frac \beta 2 -
             \sin^2 \frac \alpha 2 \sin^2 \frac \beta 2 \right) \\\\\equiv
     2 \left(\frac{1 + \cos \alpha} 2 \cdot \frac{1 + \cos \beta} 2 -
             \frac{1 - \cos \alpha} 2 \cdot \frac{1 - \cos \beta} 2 \right)
             \\\\\equiv
     \frac 12 \left((1 + \cos \alpha)(1 + \cos \beta) -
                    (1 - \cos \alpha)(1 - \cos \beta)\right) \\\\\equiv
     \frac 12 (1 + \cos \alpha + \cos \beta + \cos \alpha \cos \beta -
               1 + \cos \alpha + \cos \beta - \cos \alpha \cos \beta) \\\\\equiv
     \frac 12 (2 \cos \alpha + 2 \cos \beta) \equiv
     \cos \alpha + \cos \beta$, as required.

    We can now show that

    $\displaystyle \cos \alpha - \cos \beta \equiv
     \cos \alpha + \cos (\beta + \pi)  \equiv
     2 \cos \frac{\alpha + \beta + \pi} 2 \cos \frac {\alpha - \beta - \pi} 2
     \\\equiv
     2 \cos \left(\frac \pi 2 - \left(-\frac{\alpha + \beta} 2\right)\right)
       \cos \left(\frac \pi 2 - \frac{\alpha - \beta} 2 \right) \equiv
     2 \sin -\frac{\alpha + \beta} 2 \sin \frac{\alpha - \beta} 2 \\\\\equiv
     \boxed{-2 \sin \frac{\alpha + \beta} 2 \sin \frac{\alpha - \beta} 2}$.

    \section{Probability}

    \section{Statistics}

    \subsection{Random Variables}

    \subsection{Binomial Distribution}

    %FIXME add diagram

    The binomial distribution $\mathcal B(\nu, \pi)$ measures the probability of
    getting a certain number of successful ``events'', where there are $\nu$
    events each with probability $\pi$ of being successful.

    The PDF of $X \sim \mathcal B(\nu, \pi)$ is
    $\displaystyle f(k) = P(X = k) = {\nu \choose k} \pi^k (1 - \pi)^{\nu - k}$

    The validity of this distribution is subject to these constraints:
    \begin{enumerate}
    \item Each event must be independent.
    \end{enumerate}

    \subsection{Normal Distribution}

    %FIXME add diagram

    The PDF of $X \sim \mathcal N(\mu, \sigma^2)$ is
    $\displaystyle f(x) = \frac{1}{\sqrt{2\pi\sigma^2}}
                            e^{\frac{(x - \mu)^2}{2\sigma^2}}$

    \subsubsection{Dimensional Verification/Mnemonic}

    NB $\displaystyle\left[f(x)\right] =
        \left[\frac 1{\sqrt{\sigma^2}}\right]
        e^{\left[\frac{x^2}{\sigma^2}\right]}
      = \left[\frac 1\sigma\right]$, as
    $\left[\sigma\right] = \left[\mu\right] = \left[x\right]$. Therefore, the
    evaluation of the probability $P(a < x < b)$ by the integral
    $\displaystyle\int_a^bf(x)\dif x$ has dimension
    $\left[\frac{\dif x}{\sigma}\right] = 1$, as expected for a probability.

    \subsection{Poisson Distribution}

    \subsection{Hypothesis Testing}

    %FIXME Types of Error

    \section{Calculus}

    %FIXME: derivations from first principles

    \subsection{Definition of the derivative}

    The derivative of $f(x)$,
    $\displaystyle f'(x) = \lim_{h \to 0} \frac{f(x + h) - f(x)}{h}$.

    Other notations include, if $y = f(x)$, then
    $y' = \dod yx = f'(x)$.

    \subsection{Fundamental Theorem of Calculus} \label{sec:calc_FTC}

    \subsection{Chain Rule} \label{sec:calc_chain}

    \subsection{Product Rule (Quotient Rule)} \label{sec:calc_product}

    \subsection{Integration by Substitution} \label{sec:calc_substitution}

    \subsubsection{Trig Substitution} \label{sec:calc_trig_substitution}

    \subsection{Integration by Parts}

    \subsection{Common derivatives} \label{calc:common}

    %FIXME: add x^x

    By FTC (\ref{sec:calc_FTC}), these also give common integrations.

    \begin{center}
    \begin{longtable}{rrl}
    \toprule
    \boldmath$f(x)$ & \boldmath$f'(x)$ &
    \bfseries Comments/Derivation/Corollaries \\
    \midrule
    \endhead
    $x^n$ & $nx^{n-1}$
        & $\int x^n \dif x = \frac{1}{n + 1}x^{n + 1}$ when
          $n \neq -1$ (see $\od{}{x}\ln x$).\\
    $\ln x$ & $\frac{1}{x}$
        & $\od{}{x}\log_a x = \od{}{x} \frac{\ln x}{\ln a}$, and also
          $\int \frac{g'(x)}{g(x)} \dif x = \ln g(x)$,
            from \ref{sec:calc_chain}.\\
    $e^x$ & $e^x$ & $\od{}{x}a^x = \od{}{x}e^{x \ln a}$. \\
    $\sin x$ & $\cos x$
        & eg $\int \sin^2 x \dif x = \int \frac 12 (1 - \cos 2x) \dif x$
          by \ref{sec:comp_angle}. \\
    $\tan x$ & $\sec^2 x$ \\
    $\sec x$ & $\sec x \tan x$ \\
    $\cos x$ & $-\sin x$ \\
    $\cot x$ & $-\csc^2 x$ \\
    $\csc x$ & $-\csc x \cot x$ \\
    $\arcsin x$ & $\frac{1}{\sqrt{1 - x^2}}$
        & $\sin y = x \Rightarrow \od yx \cdot \cos y = 1 \Rightarrow
           \od yx = \frac{1}{\cos y}
         = \frac{1}{\sqrt{1 - \sin^2 y}} = \frac{1}{\sqrt{1 - x^2}}$. \\
    $\arccos x$ & $-\frac{1}{\sqrt{1 - x^2}}$
        & $\cos y = x \Rightarrow \od yx \cdot -\sin y = 1 \Rightarrow
           \od yx = -\frac{1}{\sin y}
         = -\frac{1}{\sqrt{1 - \cos^2 y}} = -\frac{1}{\sqrt{1 - x^2}}$. \\
    $\arctan x$ & $\frac{1}{x^2 + 1}$
        & $\tan y = x \Rightarrow \od yx \cdot \sec^2 y = 1 \Rightarrow
           \od yx = \frac{1}{\sec^2 y} = \frac{1}{1 + \tan^2 y}
         = \frac{1}{1 + x^2}$. \\
    \bottomrule
    \end{longtable}
    \end{center}

    \subsection{Volume of Revolution}

    \subsection{Area under Parametric Curve} \label{sec:calc_parametric_area}

    To find the area between a curve and the x-axis from $x_1$ to $x_2$ when the
    curve is given in parametric form, $x = f(t)$ and $y = g(t)$, we rewrite the
    integral

    $\displaystyle \int_{x_1}^{x_2} y \dif x =
     \int_{x_1}^{x_2} y \dod xt \dif t = \int_{x_1}^{x_2} g(t)f'(t) \dif t$.

    Similarly, the area between the curve and the y-axis from $y_1$ to $y_2$ is:

    $\displaystyle \int_{y_1}^{y_2} x \dif y =
     \int_{y_1}^{y_2} x \dod yt \dif t = \int_{y_1}^{y_2} f(t)g'(t) \dif t$.

    \subsection{Implicit differentiation}

    Where a curve is given by some implicit equation which is a sum of functions
    of $x$ and/or $y$, we can use the chain rule (\ref{sec:calc_chain}).
    Generally, where we know how to differentiate all $f_i$, $g_i$, $h_i$ then
    if we have the equation

    $\displaystyle \sum h_i(f_i(x)g_i(y)) = 0$, we can differentiate each term
    to show that

    $\displaystyle\dod{}{x} \sum h_i(f_i(x)g_i(y)) =
     \sum \left[\left(f_i'(x)g_i(y) +
                      f_i(x)g_i'(y)\dod yx\right)
                h_i'(f_i(x)g_i(y))\right] = 0$ (by \ref{sec:calc_product},
    \ref{sec:calc_chain})

    so $\displaystyle \dod yx = -\frac{\sum f_i'(x)g_i(y)h_i'(f_i(x)g_i(x))}
                                      {\sum f_i(x)g_i'(y)h_i'(f_i(x)g_i(x))}$.

    Really this is an unnecessary formulation and probably an incomplete one, as
    we could just as well take a product of any two such terms as another term,
    although that makes it laborious. Generally, I think that any differentiable
    expression will work. The idea is to differentiate with respect to $x$ on
    both sides, where we use that $\od{}{x}f(y) = f'(y)\od yx$, and then make
    $\od yx$ the subject.

    %FIXME: add x^y for fun

    An example may help to clarify.
    \begin{align*}
    0 &= y^2 + x^3 + \sin x^2 y + \ln \frac{\sqrt y} x - 5 \\
    \Rightarrow 0 &=
                2y \dod yx + 3x^2 +
                (2x \cdot y + \dod yx \cdot x^2) \cdot \cos x^2 y +
                \left(\frac{x \cdot \frac 12 y^{-\frac 12} \cdot \dod yx -
                            \sqrt y \cdot 1}
                           {x^2}\right) \cdot \frac{x}{\sqrt y} \\
    \end{align*}

    %FIXME finish this

    \subsection{Arc Length}

    \subsection{Improper Integrals}

    %FIXME add types of improper integral

    \subsection{Fake Calculus, AKA numerical approximations}

    \section{Power series} %FIXME is this analysis?

    \subsection{Taylor Series}

    The Maclaurin series is the Taylor series around $0$.

    $\displaystyle
     f(x) = f(0) + f'(0) x + \frac{f''(0)} 2 x^2 + \frac{f'''(0)}{6}x^3 \ldots
     = \sum_{k=0}^\infty \frac{f^{(k)}(0)}{k!}x^k$

    \subsection{Common series}

\end{document}
