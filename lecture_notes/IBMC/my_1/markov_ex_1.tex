\documentclass[a4paper,12pt]{article}
\title{Markov Chains}
\author{Roland Bauerschmidt}
\date{Michaelmas 2020}

\usepackage{mysty}
\usepackage{mymaths}

% Embed source files into PDF in case of loss. You can view or extract the
% source files by doing `pdfdetach -list <file.pdf>` or
% `pdfdetach -saveall <file.pdf>`, using pdfdetach from poppler, or some other
% suitable method.
\usepackage{embedall}
\embedfile{mymaths.sty}
\embedfile{mysty.sty}

\begin{document}
%TODO: fix topmargins
\thispagestyle{firststyle}
\footnotetext[1]{%
 Typesetting by Izaak van Dongen. I provide no guarantee of accurate
 transcription. Original can be found here:
 \url{http://www.statslab.cam.ac.uk/~rb812/teaching/mc2020/example1.pdf}}

\begin{enumerate}[label=\arabic*.,leftmargin=*]
 \item
  Let \(X = (X_n)_{n \ge 0}\) be a Markov chain. Show that, conditioned on
  \(X_m = i\), \mbox{\(Z = (Z_n)_{n \ge 0}\)} given by
  \(Z_n = X_{n + m}\) is a Markov chain with starting state \(i\).
 \item
  Let \(X = (X_n)_{n \ge 0}\) be a sequence of independent random variables.
  Show that \(X\) is a Markov chain. Under what condition is this chain
  homogeneous?
 \item
  Let \(X = (X_n)_{n \ge 0}\) be a sequence of fair coin tosses (with the two
  possible outcomes interpreted as \(0\) and \(1\)) and set
  \(M_n = \max_{k \le n} X_k\). Show that \((M_n)_{n \ge 0}\) is a Markov chain
  and find the transition probabilities.
 \item (Harder)
  Let \(S = (S_n)_{n \ge 0}\) be a simple (possibly asymmetric) random walk on
  \(\Z\) with \(S_0 = 0\). Show that \(X_n = \abs{S_n}\) defines a Markov chain
  and find its transition probabilities. Let \(M_n = \max_{k \le n} S_k\) and
  show that \(Y_n = M_n - S_n\) defines a Markov chain.
 \item
  Let \(X = (X_n)_{n \ge 0}\) be a Markov chain and let \((n_r)_{r \ge 0}\) be an
  unbounded increasing sequence of positive integers. Show that
  \(Y_r = X_{n_r}\) defines a (possibly inhomogeneous) Markov chain. Find the
  transition probabilities of \(Y\) when \(n_r = 2r\) and \(X\) is a simple
  random walk.
 \item
  Let \(X = (X_n)_{n \ge 0}\) and \(Y = (Y_n)_{n \ge 0}\) be Markov chains on
  the integers \(\Z\). Is \mbox{\(Z_n = X_n + Y_n\)} necessarily a Markov chain?
  Justify your answer.
 \item
  A flea hops about at random on the vertices of a triangle where each hop is
  from the currently occupied vertex to one of the other two vertices, each with
  probability \(1/2\). Find the probability that after \(n\) hops, the flea is
  back where it started.

  Now suppose that the flea is twice as likely to jump clockwise as
  anticlockwise. What is the probability that after \(n\) hops the flea is back
  where it started now?
  [Hint: \(1/2 \pm i/(2 \sqrt 3) = (1 / \sqrt 3)e^{\pm i\pi/6}\).]
 \item
  A die is `fixed' so that when it is rolled, the score cannot be the same as
  the previous score, all other scores having probability \(1/5\). If the first
  score is \(6\), what is the probability \(p\) that the \(n\)th score is \(6\)?
  What is the probability that the \(n\)th score is \(j\), where \(j \ne 6\)?

  Suppose instead that the die cannot score one greater (mod \(6\)) than the
  previous score, all other five scores having equal probability. What is the
  new value of \(p\)? [Hint: Think about the relationship between the two dice.]
 \item
  Let \(X = (X_n)_{n \ge 0}\) be a Markov chain on \(\set{1, 2, 3}\) with
  transition matrix
  \begin{equation*}
   P =
   \begin{bmatrix}
    0 & 1 & 0 \\
    0 & 2/3 & 1/3 \\
    p & 1 - p & 0
   \end{bmatrix}
  \end{equation*}
  Find \(\Prob{X_n = 1 \given X_0 = 1}\) in each of the following cases:
  \begin{multicols}{3}
   \begin{enumerate}[label=(\alph*)]
    \item
     \(p = 1/16\)
    \item
     \(p = 1/6\)
    \item[\(\text{(c)}^\ast\)]
     \(p = 1/12\)
   \end{enumerate}
  \end{multicols}
 \item
  Identify the communicating classes of the transition matrix
  \begin{equation*}
   P =
   \begin{bmatrix}
    1/2 & 0 & 0 & 0 & 1/2 \\
    0 & 1/2 & 0 & 1/2 & 0 \\
    0 & 0 & 1 & 0 & 0 \\
    0 & 1/4 & 1/4 & 1/4 & 1/4 \\
    1/2 & 0 & 0 & 0 & 1/2 \\
   \end{bmatrix}_.
  \end{equation*}
 \item
  Show that every transition matrix on a finite state space has at least one
  closed communicating class. Find an example of a transition matrix with no
  closed communicating class.
 \item
  A gambler has £2 and needs to increase it to £10 in a hurry. She can play a
  game with the following rules: a fair coin is tossed; if a player bets on the
  side which actually turns up, she wins a sum equal to her stake, and her stake
  is returned; otherwise she loses her stake. The gambler decides to use a bold
  strategy in which she stakes all her money if she has £5 or less and otherwise
  stakes just enough to increase her capital, if she wins, to £10.

  Let \(X_0 = 2\) and \(X = (X_n)_{\ge 0}\) be her capital after \(n\) throws.
  Prove that the gambler will achieve her aim with probability \(1/5\). What is
  the expected number of tosses until she either achieves her aim or loses her
  capital?
 \item (Optional)
  Let \(X = (X_n)_{n \ge 0}\) be a Markov chain on \(\set{0, 1, \dotsc}\) with
  transition probabilities given by
  \begin{align*}
   p_{0,1} = 1, &&
   p_{i,i+1} + p_{i,i-1} = 1, &&
   p_{i,i+1} = \parens[\Big]{\frac{i + 1}i}^2 p_{i,i-1},&&
   (i \ge 1)
  \end{align*}
  Show that if \(X_0\) = 0, then the probability that \(X_n \ge 1\) for all
  \(n \ge 1\) is \(6/\pi^2\).
 \item (Optional)
  Let \(Y_1, Y_2, \dotsc\) be i.i.d random variables with
  \(\Prob{Y_1 = 1} = \Prob{Y_1 = -1} = 1/2\) and set
  \(X_0 = 1\), \(X_n = X_0 + Y_1 + \dotsb + Y_n\) for \(n \ge 1\). Define
  \begin{equation*}
   H_0 = \inf \set{n \ge 0: X_n = 0}.
  \end{equation*}
  Find the probability generating function \(\phi(s) = \Exp{S^{H_0}}\).

  Suppose the common distribution of the \(Y_i\) is changed to
  \mbox{\(\Prob{Y_1 = 2} = \Prob{Y_1 = -1} = 1/2\)}. Show that the probability
  generating function \(\phi\) now satisfies
  \begin{equation*}
   s\phi^3 - 2\phi + s = 0
  \end{equation*}
\end{enumerate}
\end{document}
