\documentclass[a4paper,12pt]{article}
\author{Izaak van Dongen}
\title{Quotient Diversions}

\usepackage{mysty}
\usepackage{mymaths}

% Embed source files into PDF in case of loss. You can view or extract the
% source files by doing `pdfdetach -list <file.pdf>` or
% `pdfdetach -saveall <file.pdf>`, using pdfdetach from poppler, or some other
% suitable method.
\usepackage{embedall}
\embedfile{mymaths.sty}
\embedfile{mysty.sty}

\begin{document}
\maketitle

This document concerns the following statement regarding \emph{diversions} of
homomorphisms into quotients:
\begin{tcolorbox}
 Let \(G\) and \(H\) be groups, and let \(K \nsgp H\). Let
 \(\phi: G \to H/K\) be a homomorphism.

 Then there is some homomorphism \(\psi: G \to H\) such that
 \(\theta \compose \psi = \phi\),\footnotemark[\getrefnumber{foot_ext}]
 where \(\theta: H \to H/K\) is the quotient map.
\end{tcolorbox}
\footnotetext[1]{\label{foot_ext}%
  If you really care, \(\forall g \in G, \theta(\psi(g)) = \phi(g)\).%
}
Let \(G = C_2 = \cycsgp{-1}\), \(H = Q_8\), and \(K = \cycsgp{i}\).
Then \(K \nsgp H\) because it is of index \(2\), and \(H/K \isom C_2\).

So let \(\phi: G \to H/K\) be an isomorphism.
Note that if \(\psi: C_2 \to Q_8\) is a nontrivial homomorphism, then
\(\psi(-1)\) has order \(2\) in \(Q_8\). Therefore \(\psi(-1) = -1 \in K\). Then
\(\psi(C_2) \subseteq K\), so particularly \(\theta \compose \psi\) is trivial.

So \(\theta \compose \psi\) cannot be \(\phi\).

\begin{tcolorbox}
 Let \(V\) and \(W\) be finite-dimensional vector spaces, and let
 \(U \sspace W\). Let \(\alpha: V \to W/U\) be a linear map.

 Then there is some linear map \(\beta: V \to W\) such that
 \(\theta \compose \beta = \alpha\), where \(\theta: W \to W/U\) is the quotient
 map.
\end{tcolorbox}

Let \(\mathcal B = \set{\vec b_1, \dotsc, \vec b_n}\) be a basis of \(U\), and
extend it to a basis
\(\mathcal B'
   = \set{\vec b_1, \dotsc, \vec b_n, \vec b_{n + 1}, \dotsc, \vec b_m}\)
of \(W\). Note that
\(\mathcal B'' = \set{\vec b_{n + 1} + U, \dotsc, \vec b_m + U}\) is a basis for
\(W/U\) (exercise in notes):
\begin{itemize}
 \item
  \(\mathcal B''\) spans:

  Let \(\vec v + U \in W/U\). Since \(\vec v \in W\), we have
  \(\vec v + U
     = \sum_{i = 1}^m \lambda_i \vec b_i + U
     = \sum_{i = n + 1}^m \lambda_i (\vec b_i + U)\), for some \(\lambda_i\),
  since for \(1 \le i \le n\), \(\vec b_i + U = \vec 0 + U\).
 \item
  \(\mathcal B''\) is free:

  Suppose \(\sum_{i = n + 1}^m \lambda_i \vec b_i + U = \vec 0 + U\).
  Particularly, \(\sum_{i = n + 1}^m \lambda_i \vec b_i + \vec u = \vec 0\) for
  some \(\vec u \in U\). Since \(\vec u \in U\), we have
  \(\vec u = \sum_{i = 1}^n \mu_i \vec b_i\), for some \(\mu_i\).

  Then
  \(\sum_{i = 1}^n \mu_i \vec b_i + \sum_{i = n + 1}^m \lambda_i \vec b_i
    = \vec 0\)
  implies that \(\lambda_i \equiv 0\), because \(\mathcal B'\) is free.
\end{itemize}
All of this said, let
\(\mathcal A = \set{\vec a_1, \dotsc, \vec a_\ell}\) be a basis of \(V\). For
each \(1 \le i \le \ell\), let \(\lambda_j^{(i)}\) be such that
\(\alpha(\vec a_i) = \sum_{j = n + 1}^m \lambda_j^{(i)} \vec b_j + U\).

Then define \(\beta: \mathcal A \to W\) by
\(\beta(\vec a_i) = \sum_{j = n + 1}^m \lambda_j^{(i)} \vec b_j\), for each
\(1 \le i \le \ell\).

By the universal property of \(\mathcal A\), we can linearly extend \(\beta\) to
a linear map \(V \to W\). Then note that for all \(1 \le i \le \ell\),
\(\theta(\beta(\vec a_i))
   = \sum_{j = n + 1}^m \lambda_j^{(i)} \vec b_j + U
   = \alpha(\vec a_i)\)

Since \(\alpha\) and \(\theta \compose \beta\) agree on \(\mathcal A\), they
must be the same map (agree on all values).

\end{document}
