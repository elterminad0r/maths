\section{Miscellaneous}

\subsection{Binomial Theorem}

\begin{theorem}[Binomial theorem] Where \(n \in \Naturals\),
\begin{equation*}
(a + b)^n \equiv \sum_{r = 0}^n \binom nr a^r b^{n - r}
\end{equation*}
where the binomial coefficients are given by
\begin{equation*}
\binom nr = \nCr nr \defeq \frac{n!}{r!\cdot(n - r)!}
\end{equation*}
\end{theorem}
\begin{proof}
We can prove that this is the case by induction. If \(P(n)\) denotes the
binomial theorem for exponent \(n, \forall n \in \Integers^+\), then we
consider \(P(1)\):
\begin{equation*}
(a + b)^1 \equiv a + b
    \equiv \frac{0!}{0!\cdot 1!} a^0b^1 + \frac{1!}{1!\cdot 0!} a^1b^0
\end{equation*}
so \(P(1)\) is true. We now consider \(P(n + 1)\), supposing that \(P(n)\)
is true.
\begin{align*}
(a + b)^{n + 1} &\equiv (a + b)(a + b)^n \\
    &\equiv (a + b)\sum_{r = 0}^n \binom nr a^r b^{n - r}
        \impliedby P(n) \\
    &\equiv a\sum_{r = 0}^n \binom nr a^r b^{n - r}
          + b\sum_{r = 0}^n \binom nr a^r b^{n - r} \\
    &\equiv \sum_{r = 0}^n \binom nr a^{r + 1} b^{n - r}
          + \sum_{r = 0}^n \binom nr a^r b^{n - r + 1} \\
    &\equiv \sum_{r = 1}^{n + 1} \binom n{r - 1} a^{r} b^{n - r + 1}
          + \sum_{r = 0}^n \binom nr a^r b^{n - r + 1} \\
    &\equiv a^{n + 1} + b^{n + 1}
          + \sum_{r = 1}^n \bqty{\binom n{r - 1} + \binom nr}
            a^r b^{n - r + 1} \\
    &\equiv a^{n + 1} + b^{n + 1}
          + \sum_{r = 1}^n \binom{n + 1}r a^r b^{n - r + 1} \\
    &\equiv \sum_{r = 0}^{n + 1} \binom{n + 1}r a^r b^{(n + 1) - r} \\
\end{align*}
because
\begin{align*}
\binom n{r - 1} + \binom nr &\equiv \frac{n!}{(r - 1)! \cdot (n - r + 1)!}
                                 + \frac{n!}{r! \cdot (n - r)!}
                                    \qquad \text{by definition} \\
    &\equiv \frac{n!\cdot(r + (n - r + 1))}{r! \cdot (n - r + 1)!} \\
    &\equiv \frac{n! \cdot (n + 1)}{r! \cdot ((n + 1) - r)!} \\
    &\equiv \frac{(n + 1)!}{r! \cdot ((n + 1) - r)!} \\
    &\equiv \binom{n + 1} r \qquad \text{by definition}
\end{align*}
so \(P(n + 1)\) is true if \(P(n)\) is true, and therefore \(P(n)\) is true
for \(\forall n \in \Integers^+\) by the principle of mathematical
induction.
\end{proof}

\subsection{Numerical solutions to equations}

\subsection{Fundamental Theorem of Arithmetic}

\subsection{Prime Number Theorem}

\subsection{Fundamental Theorem of Algebra}

\subsection{Totient Function} \label{sec_totient}

\subsection{Dimensional Analysis}

\subsection{Difference of two Squares}

A difference of squares can be factorised
\begin{equation}
a^2 - b^2 \equiv (a + b)(a - b)
\end{equation}
This can be verified with fairly simple
algebra.

\subsubsection{Difference of Higher Powers}

Similar results can be found in higher powers of \(a\) and \(b\):
\begin{align*}
a^3 - b^3 &\equiv (a - b)(a^2 + ab + b^2) \\
a^4 - b^4 &\equiv (a - b)(a^3 + a^2b + b^2a + b^3) \\
\dots
\end{align*}
In general, you take out a factor of \((a - b)\) and then start with the
term \(a^{n - 1}\), and for each subsequent term decrease the power in \(a\)
and increase the power in \(b\):
\begin{equation}
a^n - b^n \equiv (a - b)(a^{n - 1} + a^{n - 2}b + a^{n - 3}b^2 + \dotsb +
                         ab^{n - 2} + b^{n - 1})
\end{equation}
This can in fact be derived from the partial sum of a geometric progression
\ref{sec_GP}. The sum \(a^n + a^{n - 1}b + \dotsb + ab^{n - 1} + b^n\) is
a geometric progression of \(n + 1\) terms with first term \(a^n\) and
common ratio \(b/a\).  Therefore,
\begin{alignat*}{2}
&&a^n + a^{n - 1}b + \dotsb + ab^{n - 1} + b^n &=
        a^n \cdot \frac{\pqty{\frac ba}^{n + 1} - 1}{\frac ba - 1} \\
&&    &= \frac{a^{n + 1} - b^{n + 1}}{a - b} \\
&\implies& (a - b)(a^n + a^{n - 1}b + \dotsb + ab^{n - 1} + b^n) &=
        a^{n + 1} - b^{n + 1}
\end{alignat*}

Note that for odd \(n\), \(a^n + b^n\) can also be factorised as
\(a^n + b^n \equiv a^n - (-b)^n\), so you get the same expansion but with
alternating positive and negative terms.

% FIXME: arguments from totient function and with different powers? number
% of common factors?

\subsubsection{Difference of Composite Powers}

In fact, in the last section, \(a^4 - b^4\) may be more fully factorised by
using:
\begin{equation*}
a^4 - b^4 \equiv (a^2)^2 - (b^2)^2 \equiv (a^2 - b^2)(a^2 + b^2) \equiv
    (a - b)(a + b)(a^2 + b^2)
\end{equation*}
This happens as \(4\) is composite. Any composite \(n\) will in fact result
in the cototient (\ref{sec_totient}) of \(n = n - \phi(n)\) factors (at
least as far as I can see, but probably can't prove). This
also holds for prime \(n\), but for prime \(n\),
\(\phi(n) = n - 2 \implies n - \phi(n) = 2\), as the only divisors of
a prime \(n\) are \(1\) and \(n\).

We may derive this from the fact that any \(n = pq: p, q \in \mathbb P\),
\(a^n - b^n\) will factorise as
\begin{equation}
a^{pq} - b^{pq} \equiv
 (a^p - b^p)(a^{p(q - 1)} + a^{p(q - 2)}b^{p(1)} + \dotsb +
             a^{p(1)}b^{p(q - 2)} + b^{p(q - 1)}
\end{equation}

\subsection{Infinitude of Primes}

\subsection[Irrationality of \(\sqrt 2\)]
           {Irrationality of \boldmath\(\sqrt 2\)}

Assume \(\sqrt 2 = \frac ab : a, b \in \Integers \land \gcd(a, b) = 1\), ie
\(a\) and \(b\) are coprime.

\subsection{Quadratic formula} \label{sec_quad_formula}

There is a formula to give the roots of a general quadratic.
\begin{theorem}[Quadratic formula]
\begin{align*}
ax^2 + bx + c &= 0\ \text{where}\ a \neq 0 \\
\iff x &= \frac{-b \pm \sqrt{b^2 - 4ac}}{2a}
\end{align*}
\end{theorem}
\begin{proof}
We can complete the square to solve a general quadratic equation.
\begin{alignat*}{2}
&&ax^2 + bx + c &= 0 \\
&\iff& x^2 + \frac{b}{a}x + \frac{c}{a} &= 0 \\
&\iff& \pqty{x + \frac{b}{2a}x}^2 &= \pqty{\frac{b}{2a}}^2 - \frac{c}{a}
    = \frac{b^2 - 4ac}{4a^2} \\
&\iff& x + \frac{b}{2a} &= \pm \sqrt{\frac{b^2 - 4ac}{4a^2}}
    = \pm \frac{\sqrt{b^2 - 4ac}}{2a} \\
&\iff& x &= \frac{-b \pm \sqrt{b^2 - 4ac}}{2a} \tag*{\qedhere}
\end{alignat*}
\end{proof}

Corollaries are that the number of real roots is determined by the
discriminant \({\Delta = b^2 - 4ac}\).
\begin{theorem}[Quadratic real roots]
Where \(P(x) \defeq ax^2 + bx + c\) and \(\Delta \defeq b^2 - 4ac\)
\begin{align*}
\Delta &= 0 \iff \text{\(P(x) = 0\) has one repeated real root} \\
\Delta &< 0 \iff \text{\(P(x) = 0\) has no real roots} \\
\Delta &> 0 \iff \text{\(P(x) = 0\) has two real roots}
\end{align*}
\end{theorem}

\subsection[Cauchy-Shwarz inequality for \(\Reals^n\)]
           {Cauchy-Shwarz inequality for \boldmath\(\Reals^n\)}

\begin{theorem}[Cauchy-Shwarz inequality]
For two equally long, finite sequences \(u_i, v_i \in \Reals\),
\begin{equation*}
\pqty{\sum u_i v_i}^2 \le \pqty{\sum u_i^2} \pqty{\sum v_i^2}
\end{equation*}
with equality iff there exists some \(k \in \Reals\) such that
\(u_i = k v_i\) for all \(i\).
\end{theorem}
\begin{proof}
We consider the polynomial
\begin{equation*}
P(x) = \sum (u_i x + v_i)^2 = 0
\end{equation*}
If there is any \(u_i\) term which is nonzero, this is a quadratic in \(x\)
(if this condition is not met, the inequality becomes obviously true with
equality).
\begin{equation*}
\pqty{\sum u_i^2} x^2 + \pqty{\sum 2 u_i v_i} x + \sum v_i^2 = 0
\end{equation*}
As it is a sum of squares of real terms, it must be nonnegative. In fact, it
can only be zero if each contributing term has precisely the same zero,
which happens only if all \(-v_i/u_i\) are equal, which leads to the
condition for equality.

As it has no zeroes or one zero (in the case of the condition for equality)
\begin{alignat*}{2}
&&\Delta = b^2 - 4ac &\le 0 \\
&\iff&
\pqty{\sum 2 u_i v_i}^2 - 4\pqty{\sum u_i^2}\pqty{\sum v_i^2} &\le 0 \\
&\iff& \pqty{\sum u_i v_i}^2 &\le \pqty{\sum u_i^2} \pqty{\sum v_i^2}
    \tag*{\qedhere}
\end{alignat*}
\end{proof}

\subsection{AM-GM inequality}

\begin{theorem}[AM-GM inequality]
For a sequence \(u_i \in \Reals\), for \(1 \le i \le n\) (ie, the
sequence is of length \(n\)) and \(u_i \ge 0\),
\begin{equation*}
\sqrt[n]{u_1 u_2 \dotsm u_n} \le \frac{u_1 + u_2 + \dotsb + u_n}{n}
\end{equation*}
with equality iff all \(u_i\) are equal.

Equivalently, using sigma and pi notation:
\begin{equation*}
\pqty{\prod u_i}^\frac{1}{n} \le \frac{1}{n}\sum u_i
\end{equation*}
\end{theorem}
\begin{proof}
We can use a kind of wonky induction. First, we verify the base
case, \(n = 2\):
\begin{alignat*}{2}
&&\sqrt{ab} &\le \frac{a + b}{2} \\
&\iff& 4ab &\le a^2 + 2ab + b^2 \\
&\iff& 0 &\le a^2 - 2ab + b^2 \\
&\iff& 0 &\le (a - b)^2\quad \text{with equality iff \(a = b\)}
\end{alignat*}
Then, supposing AM-GM holds for \(n\) and 2, we show that it holds for
\(2n\).  Taking
\begin{alignat*}{2}
&&a &= \sqrt[n]{u_1 u_2 \dotsm u_n} \\
&&b &= \sqrt[n]{u_{n+1} u_{n+2} \dotsm u_{2n}} \\
&\implies& a &\le \frac{u_1 + u_2 + \dotsb + u_n}{n}
        \quad \text{with equality iff \(u_i: 1 \le i \le n\) are equal}\\
&&b &\le \frac{u_{n + 1} + u_{n + 2} + \dotsb + u_{2n}}{n}
        \quad \text{with equality iff \(u_i: n < i \le 2n\) are equal}\\
&&\text{and}\ \sqrt{ab} &\le \frac{a + b}{2}
    \quad \text{with equality iff \(a = b\)}\\
&\implies& \sqrt[2n]{u_1 u_2 \dotsm u_{2n}} &\le
         \frac{u_1 + u_2 + \dotsb u_{2n}}{2n}
            \quad \text{with equality iff all \(u_i\) are equal}
\end{alignat*}
Now, supposing AM-GM holds for \(n\), we show that it holds for \(n - 1\).
Taking
\begin{alignat*}{2}
&&u_n &= \sqrt[n - 1]{u_1 u_2 \dotsm u_{n - 1}} \\
&\implies& \pqty{u_1 u_2 \dotsm u_{n - 1}
            \pqty{u_1 u_2 \dotsm u_{n - 1}}^{1 / (n - 1)}}^{1 / n}
         &\le \frac 1n \pqty{u_1 + u_2 + \dotsb  + u_{n - 1} +
            \pqty{u_1 u_2 \dotsm u_{n - 1}}^{1/{n - 1}}} \\
&\implies& \pqty{u_1 u_2 \dotsm u_{n - 1}}^{1 / (n - 1)} &\le
         \frac 1n \pqty{u_1 + u_2 + \dotsb u_{n - 1}} +
         \frac 1n \pqty{u_1 u_2 \dotsm u_{n - 1}}^{1 / (n - 1)} \\
&\implies& \pqty{1 - \frac 1n}
         \pqty{u_1 u_2 \dotsm u_{n - 1}}^{1 / (n - 1)} &\le
         \frac 1n \pqty{u_1 + u_2 + \dotsb + u_{n - 1}} \\
&\implies& \frac {n - 1}n
         \pqty{u_1 u_2 \dotsb u_{n - 1}}^{1 / (n - 1)} &\le
         \frac 1n \pqty{u_1 + u_2 + \dotsb + u_{n - 1}} \\
&\implies& \pqty{u_1 u_2 \dotsm u_{n - 1}}^{1 / (n - 1)} &\le
         \frac 1{n - 1} \pqty{u_1 + u_2 + \dotsb + u_{n - 1}}
\end{alignat*}
As equality for \(n\) was iff all \(u_i\) were the same, this is still true.

Now, for any \(n \in \Integers^+\), we can induct up to a power of 2 above
\(n\), and then descend from there.
\end{proof}

\subsubsection{Generalized Power Means}

\subsection{Square Triangular Numbers}

%FIXME: add link annotations, explanations, oeis
%FIXME: format table
%fixme: improve this disgusting code

The sequence of perfect squares \(\mathrm{ST}_n\) such that
\(\mathrm{ST}_n= a_n^2 = \frac 12 b_n(b_n + 1)\), where
\(a, b \in \Integers_0^+\). The first few are:

\begin{longtable}{rrrr}
\toprule
\boldmath\(n\) & \boldmath\(\text{\bfseries ST}_n\) & \boldmath\(a_n\) &
               \boldmath\(b_n\) \\
\midrule
\endhead
0 & 0 & 0 & 0 \\
1 & 1 & 1 & 1 \\
2 & 36 & 6 & 8 \\
3 & 1225 & 35 & 49 \\
4 & 41616 & 204 & 288 \\
5 & 1413721 & 1189 & 1681 \\
6 & 48024900 & 6930 & 9800 \\
7 & 1631432881 & 40391 & 57121 \\
8 & 55420693056 & 235416 & 332928 \\
9 & 1882672131025 & 1372105 & 1940449 \\
10 & 63955431761796 & 7997214 & 11309768 \\
11 & 2172602007770041 & 46611179 & 65918161 \\
\multicolumn 4c\dots \\
\bottomrule
\caption{Square triangular numbers}
\end{longtable}

Generated by Listing \ref{lst_st_gen}.

\begin{longlisting}
\begin{minted}{python}
def get_sqr_tris(n):
nums = [0, 1]
for _ in range(n):
    nums.append(34 * nums[-1] - nums[-2] + 2)
return nums

def itrirt(n):
return (isqrt(1 + 8 * n) - 1) // 2

def isqrt(n):
if n < 2:
    return n
else:
    small = isqrt(n >> 2) << 1
    big = small + 1
    if big ** 2 > n:
        return small
    else:
        return big

for n, i in enumerate(get_sqr_tris(10)):
print(" & ".join(map(str, [n, i, isqrt(i), itrirt(i)])) + " \\\\")
\end{minted}
\caption{Generating ST numbers}\label{lst_st_gen}
\end{longlisting}

%FIXME Pell's equation here

Where \(p / q\) is the \(n\)th convergent of \(\sqrt 2\),
\begin{equation}
\mathrm{ST}_n = p^2 q^2
\end{equation}

\(\mathrm{ST}_n, a_n, b_n\) also satisfy the recurrence relations where
\(\mathrm{ST}_0 = 0\) and \(\mathrm{ST}_1 = 1\)
\begin{align}
\mathrm{ST}_n &= 34\mathrm{ST}_{n - 1} - \mathrm{ST}_{n - 2} + 2\\
a_n &= 6a_{n - 1} - a_{n - 2} \\
b_n &= 6b_{n - 1} - b_{n - 2} + 2
\end{align}

See \cite{WikiSTNumbers,WolframSTNumbers} for more information.

\subsection{de Moivre's Theorem}

\begin{equation}
(\cos \theta + i \sin \theta)^n \equiv \cos n\theta + i \sin n\theta
\end{equation}

\subsection[The \(\Gamma\) function]
           {The \boldmath\(\Gamma\) function}

The ``gamma'' or \(\Gamma\) function is defined for
\(z \in \Complex, \Re(z) > 0\) as
\begin{equation}
\Gamma(z) = \int_0^{\infty} x^{z - 1}e^{-x} \dd{x}
\end{equation}
By integrating by parts, we show the following:
\begin{align*}
\Gamma(z + 1) &= \int_0^\infty x^{z}e^{-x} \dd{x} \\
              &= -x^{z} e^{-x}\eval_0^\infty
                 + \int_0^\infty zx^{z - 1}e^{-x} \dd{x} \\
              &= z\Gamma(z)
\end{align*}
Noting also that
\begin{align*}
\Gamma(1) &= \int_0^\infty e^{-x} \dd{x} \\
          &= -e^{-x}\eval_0^\infty \\
          &= 1
\end{align*}
it can be seen from this recurrence that where \(n \in \Integers^+\),
\begin{equation}
\Gamma(n) = (n - 1)!
\end{equation}
In fact, the gamma function is used as an extension of the idea of
factorials to the real and complex numbers other than the negative integers.

An interesting value that the gamma function takes is for \(z = \frac 12\).
\begin{equation*}
\Gamma(\tfrac 12) = \int_0^\infty x^{-\frac 12} e^{-x} \dd{x}
\end{equation*}
Letting \(x = u^2\)
\begin{alignat*}{2}
&\implies& \dv{x}{u} &= 2u \\
&\implies& \Gamma(\tfrac 12) &= \int_0^\infty \frac{2u}u e^{-u^2} \dd{u} \\
&&  &= 2\int_0^\infty e^{-u^2} \dd{u} \\
&&  &= \sqrt \pi
\end{alignat*}
from Theorem \ref{thm_gauss_integral}. From this we can derive the value of
any \(\Gamma(n + \frac 12)\) where \(n \in \Integers\) from the recurrence
relation on \(\Gamma\), and in some very informal sense, we can find that
the ``factorials'' of the half-integers are rational multiples of the square
root of pi. The first few are shown in Table \ref{tab_gamma_halves}.
\begin{longtable}{*{2}{>{\(\displaystyle}c<{\)}}}
\toprule
\text{\boldmath\(z\)}
    & \text{\bfseries\boldmath\(\Gamma(z)\), or ``\boldmath\((z-1)!\)''} \\
\midrule
\endhead
\frac 12 & \sqrt{\pi} \\[3ex]
\frac 32 & \frac{\sqrt{\pi}}{2} \\[3ex]
\frac 52 & \frac{3 \sqrt{\pi}}{4} \\[3ex]
\frac 72 & \frac{15 \sqrt{\pi}}{8} \\[3ex]
\frac 92 & \frac{105 \sqrt{\pi}}{16} \\[3ex]
\multicolumn 2c\dots \\
\bottomrule
\caption{Half-integer values of the gamma function}
\label{tab_gamma_halves}
\end{longtable}
