\section{Set Theory}

\subsection{Set Operations}

\subsection{Common sets}

%FIXME: sets of congruence classes
%       sets of polynomial rings: apply set mathbb A

\begin{longtable}{>{\(}c<{\)}l}
\toprule
\text{\bfseries Set} & \bfseries Description \\
\midrule
\endhead
\emptyset, \varnothing & The empty set \(\set{}\). \\
\Naturals & The set of natural numbers, \(\set{1, 2, 3, \dotsc}\).
               May or may not include 0. \\
\Integers & The set of integers
               \set{\dotsc, -2, 1, 0, 1, 2, \dotsc} \\
\Integers^+, \Integers_{> 0} & The set of strictly positive integers
               \set{1, 2, 3, \dotsc}. \\
\Integers^+_0, \Integers_{\ge 0} &
               The set of strictly nonnegative integers
               \set{0, 1, 2, \dotsc}. \\
\Integers^-, \Integers_{< 0} & The set of strictly negative integers
               \set{-1, -2, -3, \dotsc}. \\
\Integers^-_0, \Integers_{\le 0} &
               The set of strictly nonpositive integers
               \set{0, -1, -2, \dotsc}. \\
\Rationals & The set of rational numbers
               \set{\frac ab \mid a, b \in \Integers \land b \neq 0}.\\
\mathbb A & The set of algebraic numbers, ie numbers that are roots of
               polynomials in \(\Integers[x]\). \\
\Reals & The set of real numbers, which may be constructed as
               ``slices'' of \(\Rationals\). \\
\Complex & The set of complex numbers
               \(\set{a + bi \mid a, b \in \Reals}\),
               where \(i^2 = 1\).\\
\intoo{a, b} & The open interval
                 \(\set{x \in \Reals \mid a < x < b}.\)\\
\intcc{a, b} & The closed interval
                 \(\set{x \in \Reals \mid a \le x \le b}\).\\
\intco{a, b} & The half-open interval
                 \(\set{x \in \Reals \mid a \le x < b}\).\\
\intoc{a, b} & The half-open interval
                 \(\set{x \in \Reals \mid a < x \le b}\).\\
\bottomrule
\end{longtable}

\subsection{Closed, Open, Clopen}

\subsection{Axiom of Choice}

\subsection{ZFC}

\subsection{Peano Arithmetic}

\subsection{Cardinality}

% powerset of integers, diagonal argument, schroeder-bernstein, interval
% cardinality

Cardinality is a way to think about the ``size'' of sets. Cardinality is
really a kind of equivalence relation on the class of sets, where two sets
have the same cardinality iff there exists a bijection between them - ie
there is a way to produce a one-to-one mapping between the two sets.

Any two finite sets obviously have the same cardinality iff they have the
same number of elements. The cardinality of a finite set is usually just
given as the number of elements it has. For example, \(\abs{\emptyset} = 0\)
and \(\abs{\set{1, 3, 2}} = 3\), etc.

We also denote the cardinality of the natural numbers
\(\abs{\Naturals} = \aleph_0\).
