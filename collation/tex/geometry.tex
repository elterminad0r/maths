\section{Geometry}

\subsection{Pythagoras' Theorem} \label{sec_pythagoras}

%FIXME add diagram

If a triangle has sides \(a\), \(b\), \(c\) opposed by angles \(A\), \(B\),
\(C\), then
\begin{equation}
C = \frac 12 \pi \iff a^2 + b^2 = c^2
\end{equation}

\subsection{Angles in a Polygon} \label{sec_geom_polygon_angles}

\subsection{Area of a Circle}

%FIXME add diagram

\begin{theorem}[Area of a circle]
The area of a circle of radius \(r\) is given by
\begin{equation*}
A = \pi r^2
\end{equation*}
\end{theorem}
\begin{proof}
From the definition of a circle as the set of points equidistant from a
centre point, we form the equation of a circle using \ref{sec_pythagoras}:
\(x^2 + y^2 = r^2\), where the centre is \((0, 0)\) and the radius is \(r\).

We can find half the area enclosed by the curve by integrating:
\begin{equation*}
\frac 12 A = \int_{-r}^r \sqrt{r^2 - x^2} \dd{x}
\end{equation*}
We use the substitution (\ref{sec_calc_trig_substitution})
\(x = r \cos \theta \implies \dv{x}{\theta} = -r \sin \theta\) so
\begin{equation*}
\frac 12 A = \int_{\arccos -1}^{\arccos 1}
    r\sqrt{1 - \cos^2 \theta} \cdot -r \sin \theta \dd{\theta}
  = -\int_\pi^0 r^2 \sin^2 \theta \dd{\theta}
\end{equation*}
by \ref{sec_trig_pythag}. We use \ref{sec_trig_double_angle} to derive
\begin{align*}
\frac 12 A &= -\int_\pi^0
    r^2 \frac 12 (1 - \cos 2 \theta) \dd{\theta} \implies
 A = r^2 \int_0^\pi (1 - \cos 2 \theta) \dd{\theta} =
 r^2\pqty{\theta - \frac 12 \sin 2 \theta}\eval_0^\pi \\
 &= r^2 \pqty{\pi - \frac 12 \sin 2 \pi -
               \pqty{0 - \frac 12 \sin 0}} = \pi r^2 \qedhere
\end{align*}
\end{proof}
\begin{proof}[Alternative proof]
One might alternatively use the parametric form of a circle,
\(y = r \sin \theta\) and \(x = r \cos \theta\)
(note that \(x^2 + y^2 = r^2(\sin^2 \theta + \cos^2 \theta) = r^2\)
(\ref{sec_trig_pythag})), and calculate the area using
\ref{sec_calc_parametric_area}:
\begin{align*}
A &= \int_0^{2\pi}
    r \sin \theta \cdot -r \sin \theta \dd{\theta} =
 \frac 12 r^2 \int_0^{2\pi} (1 - \cos 2 \theta) \dd{\theta} =
 \frac 12 r^2 \pqty{\theta - \frac 12 \sin 2 \theta}\eval_0^{2\pi} \\
 &= \frac 12 r^2 \pqty{2\pi - \frac 12 \sin 4 \pi -
                  \pqty{0 - \sin 0}} = \pi r^2 \qedhere
\end{align*}
\end{proof}

\subsection{Volume of a sphere}

\subsection{Circle Theorems}
