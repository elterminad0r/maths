% fleqn aligns equations to the left, a4 paper size, 11pt font, article class
\documentclass[fleqn,a4paper,11pt]{article}
\title{Maths}
\author{Izaak van Dongen}

% so the title can be accessed by fancyhdr (and is automatically correctly
% spelled etc)
\makeatletter
\let\thetitle\@title
\makeatother

% better fonts, apparently
\usepackage{lmodern}
% fixes some small caps things
\usepackage{slantsc}

% for references
\usepackage[square,numbers]{natbib}

% package for drawing tex-related logos - eg \hologo(bibtex)
\usepackage{hologo}

% needed for colouring and stuff (xcolor supersedes color). provides LightSalmon
\usepackage[x11names]{xcolor}

% make the document take up more of the page
\usepackage[margin=1in,headheight=13.6pt]{geometry}

% no paragraph indent
\usepackage[parfill]{parskip}

% custom document header/footer
\usepackage{fancyhdr}
\usepackage{lastpage}

\pagestyle{fancy}
\fancyhf{}
\lhead{\thetitle}
\rhead{Izaak van Dongen}
\rfoot{Page \thepage\ of \pageref{LastPage}}

% pretty table rules and multirow entries. Also page-breaking tables
\usepackage{booktabs}
\usepackage{multirow}
\usepackage{longtable}
\usepackage{tabularx}
\usepackage{array}

\usepackage{enumitem}

% plotting mathematical functions (needs version request)
\usepackage{pgfplots}
\pgfplotsset{compat=1.15}

% \url function and clickable table of contents. no ugly red boxes though
\usepackage[hidelinks,pdfencoding=auto,psdextra]{hyperref}

% maths symbols and other stuff (supersedes the ams* packages)
\usepackage{mathtools}

% provides \set, /\int[oc][oc]/, but is mostly overwritten by physics
\usepackage{commath}

% provides typesetting of derivatives with \dv and \dd, and also \abs and \norm
% and automatic delimeter pairing with pqty, bqty, vqty, Bqty
\usepackage{physics}

% provides mathbb
\usepackage{amsfonts}
% provides uptau
\usepackage{upgreek}
% provides \varnothing
\usepackage{amssymb}
% why not
\newcommand\omicron{o}
% mathscr
\usepackage{mathrsfs}
% text(frak|swab|goth)
\usepackage{yfonts}
\def\initdefault{yinit} % fix weird font thing
% mathfrak
\usepackage{eufrak}
% witches
\usepackage{halloweenmath}
% some stackrel things
\usepackage{stackrel}

\DeclareMathOperator{\Normal}{\mathcal{N}}
\DeclareMathOperator{\Binomial}{\mathcal{B}}
\DeclareMathOperator{\Prob}{\mathrm P}
\DeclareMathOperator{\arccosh}{arccosh}
\DeclareMathOperator{\arcsinh}{arcsinh}
\DeclareMathOperator{\arctanh}{arctanh}
\DeclareMathOperator{\arcsech}{arcsech}
\DeclareMathOperator{\arccsch}{arccsch}
\DeclareMathOperator{\arccoth}{arccoth}
\DeclarePairedDelimiter{\floor}{\lfloor}{\rfloor}
\DeclarePairedDelimiter{\ceil}{\lceil}{\rceil}
\newcommand{\defeq}{\vcentcolon=}
\newcommand*\mean[1]{\overline{#1}}
\newcommand*\lseg[1]{\mathit{#1}}

\newcommand{\setstyle}{\mathbb}
\newcommand{\Naturals}{\setstyle N}
\newcommand{\Integers}{\setstyle Z}
\newcommand{\Rationals}{\setstyle Q}
\newcommand{\Reals}{\setstyle R}
\newcommand{\Complex}{\setstyle C}

% define starting paragraph letter stuff
\usepackage{lettrine}
\setcounter{DefaultLines}{4}
\setlength{\DefaultFindent}{0.5em}
\setlength{\DefaultNindent}{0em}
\renewcommand{\LettrineFontHook}{\usefont{U}{yinit}{m}{n}}

% typesetting units
\usepackage[separate-uncertainty]{siunitx}

% For framing definitions
\usepackage[framemethod=tikz]{mdframed}
\usepackage[most]{tcolorbox}

\newtcolorbox{definition}{
freelance,
before=\par\vspace{2\bigskipamount}\noindent,
after=\par\bigskip,
frame code={
  \node[
  anchor=south west,
  inner xsep=8pt,
  xshift=8pt,
  rounded corners=5pt,
  font=\bfseries\color{white},
  fill=gray] at (frame.north west) (tit) {\strut Definition:};
  \draw[
  line width=3pt,
  rounded corners=5pt,gray
  ] (tit.west) -| (frame.south west) -- ([xshift=15pt]frame.south west);
},
interior code={},
top=2pt
}

% for better table of contents stuff, providing the \listof* commands and not
% listing the tables in the table of contents
\usepackage[nottoc,notlof,notlot]{tocbibind}

% increase spacing of section numbers from headings
\usepackage{tocloft}
\setlength{\cftsubsecnumwidth}{2.9em}
\setlength{\cftsubsubsecindent}{4.4em}

% more advanced handling of utf8 and fonts or something. apparently good to have
\usepackage[utf8]{inputenc}
\usepackage[T1]{fontenc}

% bibliography management with square braces for citations
\usepackage[square,numbers]{natbib}

% graphics, like eps files and stuff (supersedes graphics)
\usepackage{graphicx}

% used to horizontally align floats
\usepackage{subfig}

% used for figures
\usepackage{float}

% lorem ipsum generator
\usepackage{lipsum}

\definecolor{codegreen}{rgb}{ 0,0.6,0}

% listings of code
\usepackage{minted}
\setminted{breaklines,
           breakbytokenanywhere,
           linenos
}
\usemintedstyle{friendly}
% bigger line numbers
\renewcommand\theFancyVerbLine{\footnotesize\arabic{FancyVerbLine}}

% that can break across pages while being captioned figures
\usepackage{caption}
\newenvironment{longlisting}
{\addvspace{\baselineskip}\captionsetup{type=listing}}
{\addvspace{\baselineskip}}

% allow maths to break across pages
\allowdisplaybreaks

\usepackage{amsthm}
\usepackage{thmtools}
\newtheorem{theorem}{Theorem}[section]
\newtheorem{corollary}{Corollary}[theorem]
\newtheorem{lemma}[theorem]{Lemma}

\begin{document}
    \maketitle\thispagestyle{empty} % no page number under title
    \tableofcontents
    \listoftables
    \listoffigures
    \listoflistings
    \listoftheorems

    %FIXME

    % mechanics: restitution etc

    % integral f(x + sqrt(1 + x^2)) dx = integral 1 / 2 (1 - 1 / t ^ 2) f(t) dt

    % frullani integral

    % x(x + 1)(x + 2)(x + 3) + 1 == (x ^ 2 + 3x + 1) ^ 2

    % can't rationalise denominator with 5 terms

    % proof of fibonacci from power series

    % STEP: area in edge of convex curve

    % substitution tan (x / 2)
    % step 2007ii4
    % proof of ceva's theorem: 2007ii8
    % proof of AM-GM: Jensen's inequality

    % https://en.wikipedia.org/wiki/Mathematical_fallacy
    % all integrals are equal to 0 - by manipulating limits
    % http://www.cut-the-knot.org/proofs/index.shtml

    % pi approximations, featuring 3b1b

    % roots of polynomials, vieta's formulas

    % sin theta tan theta = sec theta - cos thetat

    % integration reduction formulae:
    %   _0^1 cos^n x
    %   _0^1 x^n / sqrt(3 + x)

    % lambert W function - solving equations, derivative, power series

    % diophantine equations and bezout's identity

    % proof of COM of segments (using integration)

    % formal systems/proofs (with coq)

    % formula for prime numbers????
    % https://en.wikipedia.org/wiki/Formula_for_primes

    % does nonexistence of surjection imply surjection going the other way?
    % integral step I 2002 q7
    % further determinants

    % dijkstra's
    % quicksort


    % identity with (sin + cos) ^ 2 and 1 + sin(2)
    % trig harmonic form

    % olympiad:
    % http://people.bath.ac.uk/masgcs/advice.html
    % https://bmos.ukmt.org.uk/home/bmo.shtml
    % qm-am-gm-hm, generalised power mean
    % cauchy-shwarz in full , sum of squares positive,
    % rearrangement inequality,

    % factor theorem

    % circle theorems - AST, cyclic quadrilateral, centres of triangle + heron's
    % formula

    % cosine rule, "full sine rule", trig identities

    % fermat's little theorem, pell's equation

    % binomial coefficients, pigeonhole principle

    % more difficult stuff:

    % Chinese Remainder, Pell's equation and maybe Quadratic Reciprocity

    % Simson's Theorem, Ceva's theorem, excircles, Menelaus's theorem, power of
    % a point and the radical axis theorem

    % Jensen's inequality and Holder's inequality. Finally, international
    % competitors must keep a grip on Muirhead and Schur's inequalities

    % areal coordinate methods: https://bmos.ukmt.org.uk/home/areals.pdf


    % multiple angle formulas using de moivre
    % exponential representation of trig functions from euler's identity
    % (quasi-arithmetic mean)
    % limits, indeterminate forms
    % more about e as limit + exponential
    % transitive, commutative, symmetric, reflexive, etc
    % e, pi irrational
    % e^pi vs pi^e
    % nowhere continuous function (dirichlet)
    % notes on classical cryptanalysis in statistics
    % recursively enumerable sets, generally turing machines/computability
    % faulty proofs,
    %   definite integral must be 0
    %   0 == 1
    % generator for pythagorean triple, (3b1b & bprp)
    % pyth-n tuples https://www.researchgate.net/publication/250890871_A_method_of_generating_Pythagorean_n-tuples
    % - integral _a^b of f(x) / (f(x) + f(a + b - x) dx
    % - shortlist of things
    % - generally: add links to everything.
    % - modular arithmetic, congruence classes
    % - fourier, laplace transforms
    % - fractals, fractal dimension, l-systems, weierstrass functions
    % - happy number, primes, etc (autobiographical integer)
    % - inverse trig and hyperbolic trig, more on
    % - leibniz integral rule/feynmen integration/differentiation under the
    %   integral
    % - integral of interesting functions (see blackpenredpen)
    %   sec x, x^x, sqrt(tan x)
    % - mathematical fields, rings, groups
    % - frullani theorem
    % - functions: continuity, differentiability, monoticity, elementaryness
    % - central limit theorem, sum of normally distributed random variables
    % - generating functions, moment generating functions (proof)
    % - generating functions
    % - characteristic functions
    % - statistics: expectation, variance, covariance
    % - add a wee bit of physics
    % - factorisation of x^4 + 1
    % - binomial expansion, pascal's triangle, properties thereof.
    % - rational root theorem
    % - multiple angle formulae
    % - exponential representation of trig functions
    % - matrices, homeogenous coordinates, affine spaces
    % - add list of constants that are interesting.
    % - vectors, vector products, convex combination trick, decompose into new
    %   components, product spaces
    % - catenary, parabola, brachistochrone etc
    % - gamma, beta, zeta functions
    % - sets of polynomial rings
    % - gaussian integers, rationals
    % - other notation for algebraic numbers.
    % - failure of power and logarithm identities (also put these in)
    % - quaternions, octonions, sedenions, prime numbers
    % - euler, de moivre
    % - work out how to use theorems & proofs from amsthm.
    % - more math - fraktur, gothic, scr
    % - prove d/dx x^alpha holds for alpha not in Z
    % - prime number wheel erat atkin sieves, tests, li(x), pi(x), sympy prime
    % - stirling's approximation
    % - generating functions
    % - list of symbols
    % - equipotence of real interval and reals, similarly for rationals

    \section{Miscellaneous}

%\begin{thm}Here is a new theorem.\end{thm}
%
%\begin{thm}Here is a new theorem.\end{thm}

    \subsection{Numerical solutions to equations}

    \subsection{Fundamental Theorem of Arithmetic}

    \subsection{Prime Number Theorem}

    \subsection{Fundamental Theorem of Algebra}

    \subsection{Totient Function} \label{sec_totient}

    \subsection{Dimensional Analysis}

    \subsection{Difference of two Squares}

    A difference of squares can be factorised
    \begin{equation}
    a^2 - b^2 \equiv (a + b)(a - b)
    \end{equation}
    This can be verified with fairly simple
    algebra.

    \subsubsection{Difference of Higher Powers}

    Similar results can be found in higher powers of \(a\) and \(b\):
    \begin{align*}
    a^3 - b^3 &\equiv (a - b)(a^2 + ab + b^2) \\
    a^4 - b^4 &\equiv (a - b)(a^3 + a^2b + b^2a + b^3) \\
    \dots
    \end{align*}
    In general, you take out a factor of \((a - b)\) and then start with the
    term \(a^{n - 1}\), and for each subsequent term decrease the power in \(a\)
    and increase the power in \(b\):
    \begin{equation}
    a^n - b^n \equiv (a - b)(a^{n - 1} + a^{n - 2}b + a^{n - 3}b^2 + \dotsb +
                             ab^{n - 2} + b^{n - 1})
    \end{equation}
    This can in fact be derived from the partial sum of a geometric progression
    \ref{sec_GP}. The sum \(a^n + a^{n - 1}b + \dotsb + ab^{n - 1} + b^n\) is
    a geometric progression of \(n + 1\) terms with first term \(a^n\) and
    common ratio \(b/a\).  Therefore,
    \begin{alignat*}{2}
    &&a^n + a^{n - 1}b + \dotsb + ab^{n - 1} + b^n &=
            a^n \cdot \frac{\pqty{\frac ba}^{n + 1} - 1}{\frac ba - 1} \\
    &&    &= \frac{a^{n + 1} - b^{n + 1}}{a - b} \\
    &\implies& (a - b)(a^n + a^{n - 1}b + \dotsb + ab^{n - 1} + b^n) &=
            a^{n + 1} - b^{n + 1}
    \end{alignat*}

    Note that for odd \(n\), \(a^n + b^n\) can also be factorised as
    \(a^n + b^n \equiv a^n - (-b)^n\), so you get the same expansion but with
    alternating positive and negative terms.

    % FIXME: arguments from totient function and with different powers? number
    % of common factors?

    \subsubsection{Difference of Composite Powers}

    In fact, in the last section, \(a^4 - b^4\) may be more fully factorised by
    using:
    \begin{equation*}
    a^4 - b^4 \equiv (a^2)^2 - (b^2)^2 \equiv (a^2 - b^2)(a^2 + b^2) \equiv
        (a - b)(a + b)(a^2 + b^2)
    \end{equation*}
    This happens as \(4\) is composite. Any composite \(n\) will in fact result
    in the cototient (\ref{sec_totient}) of \(n = n - \phi(n)\) factors (at
    least as far as I can see, but probably can't prove). This
    also holds for prime \(n\), but for prime \(n\),
    \(\phi(n) = n - 2 \implies n - \phi(n) = 2\), as the only divisors of
    a prime \(n\) are \(1\) and \(n\).

    We may derive this from the fact that any \(n = pq: p, q \in \mathbb P\),
    \(a^n - b^n\) will factorise as
    \begin{equation}
    a^{pq} - b^{pq} \equiv
     (a^p - b^p)(a^{p(q - 1)} + a^{p(q - 2)}b^{p(1)} + \dotsb +
                 a^{p(1)}b^{p(q - 2)} + b^{p(q - 1)}
    \end{equation}

    \subsection{Infinitude of Primes}

    \subsection[Irrationality of \(\sqrt 2\)]
               {Irrationality of \boldmath\(\sqrt 2\)}

    Assume \(\sqrt 2 = \frac ab : a, b \in \Integers \land \gcd(a, b) = 1\), ie
    \(a\) and \(b\) are coprime.

    \subsection{Quadratic formula} \label{sec_quad_formula}

    There is a formula to give the roots of a general quadratic.
    \begin{theorem}[Quadratic formula]
    \begin{align*}
    ax^2 + bx + c &= 0\ \text{where}\ a \neq 0 \\
    \iff x &= \frac{-b \pm \sqrt{b^2 - 4ac}}{2a}
    \end{align*}
    \end{theorem}
    \begin{proof}
    We can complete the square to solve a general quadratic equation.
    \begin{alignat*}{2}
    &&ax^2 + bx + c &= 0 \\
    &\iff& x^2 + \frac{b}{a}x + \frac{c}{a} &= 0 \\
    &\iff& \pqty{x + \frac{b}{2a}x}^2 &= \pqty{\frac{b}{2a}}^2 - \frac{c}{a}
        = \frac{b^2 - 4ac}{4a^2} \\
    &\iff& x + \frac{b}{2a} &= \pm \sqrt{\frac{b^2 - 4ac}{4a^2}}
        = \pm \frac{\sqrt{b^2 - 4ac}}{2a} \\
    &\iff& x &= \frac{-b \pm \sqrt{b^2 - 4ac}}{2a}
    \end{alignat*}
    \end{proof}

    Corollaries are that the number of real roots is determined by the
    discriminant \({\Delta = b^2 - 4ac}\).
    \begin{theorem}[Quadratic real roots]
    Where \(P(x) \defeq ax^2 + bx + c\) and \(\Delta \defeq b^2 - 4ac\)
    \begin{align*}
    \Delta &= 0 \iff \text{\(P(x) = 0\) has one repeated real root} \\
    \Delta &< 0 \iff \text{\(P(x) = 0\) has no real roots} \\
    \Delta &> 0 \iff \text{\(P(x) = 0\) has two real roots}
    \end{align*}
    \end{theorem}

    \subsection[Cauchy-Shwarz inequality for \(\Reals^n\)]
               {Cauchy-Shwarz inequality for \boldmath\(\Reals^n\)}

    \begin{theorem}[Cauchy-Shwarz inequality]
    For two equally long, finite sequences \(u_i, v_i \in \Reals\),
    \begin{equation*}
    \pqty{\sum u_i v_i}^2 \le \pqty{\sum u_i^2} \pqty{\sum v_i^2}
    \end{equation*}
    with equality iff there exists some \(k \in \Reals\) such that
    \(u_i = k v_i\) for all \(i\).
    \end{theorem}
    \begin{proof}
    We consider the polynomial
    \begin{equation*}
    P(x) = \sum (u_i x + v_i)^2 = 0
    \end{equation*}
    If there is any \(u_i\) term which is nonzero, this is a quadratic in \(x\)
    (if this condition is not met, the inequality becomes obviously true with
    equality).
    \begin{equation*}
    \pqty{\sum u_i^2} x^2 + \pqty{\sum 2 u_i v_i} x + \sum v_i^2 = 0
    \end{equation*}
    As it is a sum of squares of real terms, it must be nonnegative. In fact, it
    can only be zero if each contributing term has precisely the same zero,
    which happens only if all \(-v_i/u_i\) are equal, which leads to the
    condition for equality.

    As it has no zeroes or one zero (in the case of the condition for equality)
    \begin{alignat*}{2}
    &&\Delta = b^2 - 4ac &\le 0 \\
    &\iff&
    \pqty{\sum 2 u_i v_i}^2 - 4\pqty{\sum u_i^2}\pqty{\sum v_i^2} &\le 0 \\
    &\iff& \pqty{\sum u_i v_i}^2 &\le \pqty{\sum u_i^2} \pqty{\sum v_i^2}
    \end{alignat*}
    \end{proof}

    \subsection{AM-GM inequality}

    \begin{theorem}[AM-GM inequality]
    For a sequence \(u_i \in \Reals\), for \(1 \le i \le n\) (ie, the
    sequence is of length \(n\)) and \(u_i \ge 0\),
    \begin{equation*}
    \sqrt[n]{u_1 u_2 \dotsm u_n} \le \frac{u_1 + u_2 + \dotsb + u_n}{n}
    \end{equation*}
    with equality iff all \(u_i\) are equal.

    Equivalently, using sigma and pi notation:
    \begin{equation*}
    \pqty{\prod u_i}^\frac{1}{n} \le \frac{1}{n}\sum u_i
    \end{equation*}
    \end{theorem}
    \begin{proof}
    We can use a kind of wonky induction. First, we verify the base
    case, \(n = 2\):
    \begin{alignat*}{2}
    &&\sqrt{ab} &\le \frac{a + b}{2} \\
    &\iff& 4ab &\le a^2 + 2ab + b^2 \\
    &\iff& 0 &\le a^2 - 2ab + b^2 \\
    &\iff& 0 &\le (a - b)^2\quad \text{with equality iff \(a = b\)}
    \end{alignat*}
    Then, supposing AM-GM holds for \(n\) and 2, we show that it holds for
    \(2n\).  Taking
    \begin{alignat*}{2}
    &&a &= \sqrt[n]{u_1 u_2 \dotsm u_n} \\
    &&b &= \sqrt[n]{u_{n+1} u_{n+2} \dotsm u_{2n}} \\
    &\implies& a &\le \frac{u_1 + u_2 + \dotsb + u_n}{n}
            \quad \text{with equality iff \(u_i: 1 \le i \le n\) are equal}\\
    &&b &\le \frac{u_{n + 1} + u_{n + 2} + \dotsb + u_{2n}}{n}
            \quad \text{with equality iff \(u_i: n < i \le 2n\) are equal}\\
    &&\text{and}\ \sqrt{ab} &\le \frac{a + b}{2}
        \quad \text{with equality iff \(a = b\)}\\
    &\implies& \sqrt[2n]{u_1 u_2 \dotsm u_{2n}} &\le
             \frac{u_1 + u_2 + \dotsb u_{2n}}{2n}
                \quad \text{with equality iff all \(u_i\) are equal}
    \end{alignat*}
    Now, supposing AM-GM holds for \(n\), we show that it holds for \(n - 1\).
    Taking
    \begin{alignat*}{2}
    &&u_n &= \sqrt[n - 1]{u_1 u_2 \dotsm u_{n - 1}} \\
    &\implies& \pqty{u_1 u_2 \dotsm u_{n - 1}
                \pqty{u_1 u_2 \dotsm u_{n - 1}}^{1 / (n - 1)}}^{1 / n}
             &\le \frac 1n \pqty{u_1 + u_2 + \dotsb  + u_{n - 1} +
                \pqty{u_1 u_2 \dotsm u_{n - 1}}^{1/{n - 1}}} \\
    &\implies& \pqty{u_1 u_2 \dotsm u_{n - 1}}^{1 / (n - 1)} &\le
             \frac 1n \pqty{u_1 + u_2 + \dotsb u_{n - 1}} +
             \frac 1n \pqty{u_1 u_2 \dotsm u_{n - 1}}^{1 / (n - 1)} \\
    &\implies& \pqty{1 - \frac 1n}
             \pqty{u_1 u_2 \dotsm u_{n - 1}}^{1 / (n - 1)} &\le
             \frac 1n \pqty{u_1 + u_2 + \dotsb + u_{n - 1}} \\
    &\implies& \frac {n - 1}n
             \pqty{u_1 u_2 \dotsb u_{n - 1}}^{1 / (n - 1)} &\le
             \frac 1n \pqty{u_1 + u_2 + \dotsb + u_{n - 1}} \\
    &\implies& \pqty{u_1 u_2 \dotsm u_{n - 1}}^{1 / (n - 1)} &\le
             \frac 1{n - 1} \pqty{u_1 + u_2 + \dotsb + u_{n - 1}}
    \end{alignat*}
    As equality for \(n\) was iff all \(u_i\) were the same, this is still true.

    Now, for any \(n \in \Integers^+\), we can induct up to a power of 2 above
    \(n\), and then descend from there.
    \end{proof}

    \subsubsection{Generalized Power Means}

    \subsection{Square Triangular Numbers}

    %FIXME: add link annotations, explanations, oeis
    %FIXME: format table
    %fixme: improve this disgusting code

    The sequence of perfect squares \(\mathrm{ST}_n\) such that
    \(\mathrm{ST}_n= a_n^2 = \frac 12 b_n(b_n + 1)\), where
    \(a, b \in \Integers_0^+\). The first few are:

    \begin{longtable}{rrrr}
    \toprule
    \boldmath\(n\) & \boldmath\(\text{\bfseries ST}_n\) & \boldmath\(a_n\) &
                   \boldmath\(b_n\) \\
    \midrule
    \endhead
    0 & 0 & 0 & 0 \\
    1 & 1 & 1 & 1 \\
    2 & 36 & 6 & 8 \\
    3 & 1225 & 35 & 49 \\
    4 & 41616 & 204 & 288 \\
    5 & 1413721 & 1189 & 1681 \\
    6 & 48024900 & 6930 & 9800 \\
    7 & 1631432881 & 40391 & 57121 \\
    8 & 55420693056 & 235416 & 332928 \\
    9 & 1882672131025 & 1372105 & 1940449 \\
    10 & 63955431761796 & 7997214 & 11309768 \\
    11 & 2172602007770041 & 46611179 & 65918161 \\
    \multicolumn{4}{c}{\dots} \\
    \bottomrule
    \caption{Square triangular numbers}
    \end{longtable}

    Generated by Listing \ref{lst_st_gen}.

    \begin{longlisting}
    \begin{minted}{python}
def get_sqr_tris(n):
    nums = [0, 1]
    for _ in range(n):
        nums.append(34 * nums[-1] - nums[-2] + 2)
    return nums

def itrirt(n):
    return (isqrt(1 + 8 * n) - 1) // 2

def isqrt(n):
    if n < 2:
        return n
    else:
        small = isqrt(n >> 2) << 1
        big = small + 1
        if big ** 2 > n:
            return small
        else:
            return big

for n, i in enumerate(get_sqr_tris(10)):
    print(" & ".join(map(str, [n, i, isqrt(i), itrirt(i)])) + " \\\\")
    \end{minted}
    \caption{Generating ST numbers}\label{lst_st_gen}
    \end{longlisting}

    %FIXME Pell's equation here

    Where \(p / q\) is the \(n\)th convergent of \(\sqrt 2\),
    \begin{equation}
    \mathrm{ST}_n = p^2 q^2
    \end{equation}

    \(\mathrm{ST}_n, a_n, b_n\) also satisfy the recurrence relations where
    \(\mathrm{ST}_0 = 0\) and \(\mathrm{ST}_1 = 1\)
    \begin{align}
    \mathrm{ST}_n &= 34\mathrm{ST}_{n - 1} - \mathrm{ST}_{n - 2} + 2\\
    a_n &= 6a_{n - 1} - a_{n - 2} \\
    b_n &= 6b_{n - 1} - b_{n - 2} + 2
    \end{align}

    See \cite{WikiSTNumbers,WolframSTNumbers} for more information.

    \subsection{de Moivre's Theorem}

    \begin{equation}
    (\cos \theta + i \sin \theta)^n \equiv \cos n\theta + i \sin n\theta
    \end{equation}

    \subsection[The \(\Gamma\) function]
               {The \boldmath\(\Gamma\) function}

    The ``gamma'' or \(\Gamma\) function is defined for
    \(z \in \Complex, \Re(z) > 0\) as
    \begin{equation}
    \Gamma(z) = \int_0^{\infty} x^{z - 1}e^{-x} \dd{x}
    \end{equation}
    By integrating by parts, we show the following:
    \begin{align*}
    \Gamma(z + 1) &= \int_0^\infty x^{z}e^{-x} \dd{x} \\
                  &= \bqty{-x^{z} e^{-x}}_0^\infty
                     + \int_0^\infty zx^{z - 1}e^{-x} \dd{x} \\
                  &= z\Gamma(z)
    \end{align*}
    Noting also that
    \begin{align*}
    \Gamma(1) &= \int_0^\infty e^{-x} \dd{x} \\
              &= \bqty{-e^{-x}}_0^\infty \\
              &= 1
    \end{align*}
    it can be seen from this recurrence that where \(n \in \Integers^+\),
    \begin{equation}
    \Gamma(n) = (n - 1)!
    \end{equation}
    In fact, the gamma function is used as an extension of the idea of
    factorials to the real and complex numbers other than the negative integers.

    \section{Discrete Maths (Computer Science)}

    \subsection{Exponentiation by Squaring} \label{sec_exp_by_squaring}

    %FIXME add fast integer square root algorithm

    For \(y \in \Integers_0^+\), \(x^y\) is given by
    \begin{equation}
    x^y =
        \begin{cases}
        1 & y = 0 \\
        x & y = 1 \\
        (x ^ 2)^{\frac 12 y} & y \equiv 0 \pmod 2\\
        x(x ^ 2)^{\frac 12 (y-1)} & y \equiv 1 \pmod 2\\
        \end{cases}
    \end{equation}

    This follows from the fact that
    \(x^{2y'} = (x^2)^{y'}\) and \(x^{2y' + 1} = x(x^2)^{y'}\).

    \section{Functions}

    % add involutions

    \subsection{Jections}

    %FIXME mnemonic or diagram

    \begin{itemize}
    \item An injection maps each element of its domain to a unique element of
          its codomain.
    \item A surjection maps an element of its domain to each element of its
          codomain.
    \item A bijection is an injection and a surjection.
    \end{itemize}

    \section{Set Theory}

    \subsection{Set Operations}

    \subsection{Common sets}

    %FIXME: sets of congruence classes
    %       sets of polynomial rings: apply set mathbb A

    \begin{longtable}{>{\(}c<{\)}l}
    \toprule
    \text{\bfseries Set} & \bfseries Description \\
    \midrule
    \endhead
    \emptyset, \varnothing & The empty set \(\set{}\). \\
    \Naturals & The set of natural numbers, \(\set{1, 2, 3, \dotsc}\).
                   May or may not include 0. \\
    \Integers & The set of integers
                   \set{\dotsc, -2, 1, 0, 1, 2, \dotsc} \\
    \Integers^+, \Integers_{> 0} & The set of strictly positive integers
                   \set{1, 2, 3, \dotsc}. \\
    \Integers^+_0, \Integers_{\ge 0} &
                   The set of strictly nonnegative integers
                   \set{0, 1, 2, \dotsc}. \\
    \Integers^-, \Integers_{< 0} & The set of strictly negative integers
                   \set{-1, -2, -3, \dotsc}. \\
    \Integers^-_0, \Integers_{\le 0} &
                   The set of strictly nonpositive integers
                   \set{0, -1, -2, \dotsc}. \\
    \Rationals & The set of rational numbers
                   \set{\frac ab \mid a, b \in \Integers \land b \neq 0}.\\
    \mathbb A & The set of algebraic numbers, ie numbers that are roots of
                   polynomials in \(\Integers[x]\). \\
    \Reals & The set of real numbers, which may be constructed as
                   ``slices'' of \(\Rationals\). \\
    \Complex & The set of complex numbers
                   \(\set{a + bi \mid a, b \in \Reals}\),
                   where \(i^2 = 1\).\\
    \intoo{a, b} & The open interval
                     \(\set{x \in \Reals \mid a < x < b}.\)\\
    \intcc{a, b} & The closed interval
                     \(\set{x \in \Reals \mid a \le x \le b}\).\\
    \intco{a, b} & The half-open interval
                     \(\set{x \in \Reals \mid a \le x < b}\).\\
    \intoc{a, b} & The half-open interval
                     \(\set{x \in \Reals \mid a < x \le b}\).\\
    \bottomrule
    \end{longtable}

    \subsection{Closed, Open, Clopen}

    \subsection{Axiom of Choice}

    \subsection{ZFC}

    \subsection{Peano Arithmetic}

    \subsection{Cardinality}

    % powerset of integers, diagonal argument, schroeder-bernstein, interval
    % cardinality

    Cardinality is a way to think about the ``size'' of sets. Cardinality is
    really a kind of equivalence relation on the class of sets, where two sets
    have the same cardinality iff there exists a bijection between them - ie
    there is a way to produce a one-to-one mapping between the two sets.

    Any two finite sets obviously have the same cardinality iff they have the
    same number of elements. The cardinality of a finite set is usually just
    given as the number of elements it has. For example, \(\abs{\emptyset} = 0\)
    and \(\abs{\set{1, 3, 2}} = 3\), etc.

    We also denote the cardinality of the natural numbers
    \(\abs{\Naturals} = \aleph_0\).

    \section{Linear Algebra}

    \section{Geometry}

    \subsection{Pythagoras' Theorem} \label{sec_pythagoras}

    %FIXME add diagram

    If a triangle has sides \(a\), \(b\), \(c\) opposed by angles \(A\), \(B\),
    \(C\), then
    \begin{equation}
    C = \frac 12 \pi \iff a^2 + b^2 = c^2
    \end{equation}

    \subsection{Angles in a Polygon} \label{sec_geom_polygon_angles}

    \subsection{Area of a Circle}

    %FIXME add diagram

    \begin{theorem}[Area of a circle]
    The area of a circle of radius \(r\) is given by
    \begin{equation*}
    A = \pi r^2
    \end{equation*}
    \end{theorem}
    \begin{proof}
    From the definition of a circle as the set of points equidistant from a
    centre point, we form the equation of a circle using \ref{sec_pythagoras}:
    \(x^2 + y^2 = r^2\), where the centre is \((0, 0)\) and the radius is \(r\).

    We can find half the area enclosed by the curve by integrating:
    \begin{equation*}
    \frac 12 A = \int_{-r}^r \sqrt{r^2 - x^2} \dd{x}
    \end{equation*}
    We use the substitution (\ref{sec_calc_trig_substitution})
    \(x = r \cos \theta \implies \dv{x}{\theta} = -r \sin \theta\) so
    \begin{equation*}
    \frac 12 A = \int_{\arccos -1}^{\arccos 1}
        r\sqrt{1 - \cos^2 \theta} \cdot -r \sin \theta \dd{\theta}
      = -\int_\pi^0 r^2 \sin^2 \theta \dd{\theta}
    \end{equation*}
    by \ref{sec_trig_pythag}. We use \ref{sec_trig_double_angle} to derive
    \begin{align*}
    \frac 12 A &= -\int_\pi^0
        r^2 \frac 12 (1 - \cos 2 \theta) \dd{\theta} \implies
     A = r^2 \int_0^\pi (1 - \cos 2 \theta) \dd{\theta} =
     r^2\bqty{\theta - \frac 12 \sin 2 \theta}_0^\pi \\
     &= r^2 \pqty{\pi - \frac 12 \sin 2 \pi -
                   \pqty{0 - \frac 12 \sin 0}} = \pi r^2
    \end{align*}
    One might alternatively use the parametric form of a circle,
    \(y = r \sin \theta\) and \(x = r \cos \theta\)
    (note that \(x^2 + y^2 = r^2(\sin^2 \theta + \cos^2 \theta) = r^2\)
    (\ref{sec_trig_pythag})), and calculate the area using
    \ref{sec_calc_parametric_area}:
    \begin{align*}
    A &= \int_0^{2\pi}
        r \sin \theta \cdot -r \sin \theta \dd{\theta} =
     \frac 12 r^2 \int_0^{2\pi} (1 - \cos 2 \theta) \dd{\theta} =
     \frac 12 r^2 \bqty{\theta - \frac 12 \sin 2 \theta}_0^{2\pi} \\
     &= \frac 12 r^2 \pqty{2\pi - \frac 12 \sin 4 \pi -
                      \pqty{0 - \sin 0}} = \pi r^2
    \end{align*}
    \end{proof}

    \subsection{Volume of a sphere}

    \subsection{Circle Theorems}

    \section{Trigonometry}

    \subsection{Definitions} \label{sec_trig_definitions}

    %FIXME add diagram

    \begin{align}
    \sin \theta &= \frac OH \\
    \cos \theta &= \frac AH \\
    \tan \theta &= \frac OA = \frac{\sin \theta}{\cos \theta}
    \end{align}

    \subsubsection{Periodicity} \label{sec_trig_periodic}

    %FIXME add more identities, diagrams (or refer to previous diagrams).

    Because they are in the same right triangle, and the sum of angles in a
    triangle is \(\pi\) (\(\ang{180}\)) (\ref{sec_geom_polygon_angles}), the
    angle other than \(\theta\) must be \(\frac 12 \pi - \theta\). Therefore,
    \begin{align}
    \sin \theta &\equiv \cos(\frac 12 \pi - \theta) \\
    \cos \theta &\equiv \sin(\frac 12 \pi - \theta)
    \end{align}
    From their parity \ref{sec_trig_parity} we may deduce
    \begin{align}
    \sin(\theta - \frac 12 \pi) &\equiv -\cos \theta \\
    \cos(\theta - \frac 12 \pi) &\equiv \sin \theta
    \end{align}
    implying
    \begin{align}
    \sin(\theta + \pi) &\equiv -\sin \theta \\
    \cos(\theta + \pi) &\equiv -\cos \theta
    \end{align}
    From this, we see that
    \begin{alignat}{2}
    \sin(\theta + 2\pi) &\equiv \sin(\theta + \pi + \pi) &&\equiv \sin \theta \\
    \cos(\theta + 2\pi) &\equiv \cos(\theta + \pi + \pi) &&\equiv \cos \theta \\
    \tan(\theta + \pi) &\equiv \frac{\sin(\theta + \pi)}{\cos(\theta + \pi)}
        &&\equiv \frac{-\sin \theta}{-\cos \theta} \equiv \tan \theta
    \end{alignat}
    and in fact,
    \begin{alignat}{2}
    \sin(\theta + 2\pi n) &\equiv \sin \theta &&\iff n \in \Integers \\
    \cos(\theta + 2\pi n) &\equiv \cos \theta &&\iff n \in \Integers \\
    \tan(\theta + \pi n) &\equiv \tan \theta &&\iff n \in \Integers
    \end{alignat}
    Therefore, the \emph{period} of \(\sin \theta\) and \(\cos \theta\) is
    \(2\pi\), whereas the period of \(\tan \theta\) is \(\pi\).

    \subsubsection{Parity} \label{sec_trig_parity}

    From \ref{sec_trig_definitions}, we see that \(\sin \theta\) and
    \(\tan \theta\) are \emph{odd} functions:
    \(\sin -x = -\sin x\) and \(\tan -x = -\tan x\),
    whereas \(\cos \theta\) is an \emph{even} function:
    \(\cos -x = \cos x\).

    \subsubsection{Reciprocal functions} \label{sec_trig_reciprocal}

    %fixme some graphs

    The reciprocals of sine, cosine and tangent are cosecant, secant and
    cotangent respectively. These are written
    \begin{align}
    \sec \theta &= \frac{1}{\cos \theta} \\
    \operatorname{cosec} \theta = \csc \theta
        &= \frac{1}{\sin \theta} \\
    \cot \theta &= \frac{1}{\tan \theta}
        = \frac{\cos \theta}{\sin \theta}
    \end{align}

    \subsection{Special Angles}

    %FIXME add diagrams, derive other angles

    From geometric construction of triangles in addition to Pythagoras' Theorem
    (\ref{sec_pythagoras}), we can determine that the following values of
    trigonometric functions at certain angles hold.

    %FIXME automatic math mode

    \begin{longtable}{*{4}{>{\(}c<{\)}}}
    \toprule
    \text{\boldmath\(\theta\)} & \text{\boldmath\(\sin \theta\)}
        & \text{\boldmath\(\cos \theta\)} & \text{\boldmath\(\tan \theta\)} \\
    \midrule
    \endhead
    0 & 0 & 1 & 0 \\[1ex]
    \dfrac \pi 6 & \dfrac 12 & \dfrac{\sqrt 3} 2 & \dfrac {\sqrt 3} 3 \\[3ex]
    \dfrac \pi 4 & \dfrac {\sqrt 2} 2 & \dfrac {\sqrt 2} 2 & 1 \\[3ex]
    \dfrac \pi 3 & \dfrac{\sqrt 3} 2 & \dfrac 12 & \sqrt 3 \\[3ex]
    \dfrac \pi 2 & 1 & 0 & \infty\\[2ex]
    \bottomrule
    \caption{The basic trigonometric constants} \\
    \end{longtable}

    Strictly, \(\tan \frac 12 \pi\) is undefined.

    The symmetry of sine and cosine going opposite ways across this table is due
    to the fact that \(\sin \theta \equiv \cos(\frac 12 \pi - \theta)\).

    We need only discuss special angles in \(\intcc{0, \dfrac \pi 2}\) because
    the corresponding angles in the other quadrants can be found from symmetries
    and periodicities. This is convenient as it means all values of trig
    functions that we discuss here are positive, so we can take positive square
    roots without remorse.

    We can always calculate half-angles in the first quadrant using the
    half-angle formulae (Theorem \ref{thm_trig_half_angle}). We can also use
    compound angle formulae to build some angles, such as
    \(\frac \pi {12} = \frac \pi 3 - \frac \pi 4\).

    A small number of more esoteric angles are presented below. Many more can be
    found \cite{WikiTrigConstants}.

    \begin{longtable}{*{4}{>{\(}c<{\)}}}
    \toprule
    \text{\boldmath\(\theta\)} & \text{\boldmath\(\sin \theta\)}
        & \text{\boldmath\(\cos \theta\)} & \text{\boldmath\(\tan \theta\)} \\
    \midrule
    \endhead
    \dfrac \pi {12} & \dfrac{\sqrt 6 - \sqrt 2} 4 & \dfrac{\sqrt 6 + \sqrt 2} 4
        & 2 - \sqrt 3 \\[3ex]
    \dfrac \pi 8 & \dfrac 12 \sqrt{2 - \sqrt 2} & \dfrac 12 \sqrt{2 + \sqrt 2}
        & \sqrt 2 - 1 \\[3ex]
    \dfrac \pi 5 & \dfrac 14 \sqrt{10 - 2\sqrt 5} & \dfrac{\sqrt 5 + 1} 4
        & \sqrt{5 - 2\sqrt 5} \\[3ex]
    \dfrac{3\pi}{10} & \dfrac{\sqrt 5 + 1} 4 & \dfrac 14 \sqrt{10 - 2\sqrt 5}
        & \frac 15 \sqrt{25 + 10\sqrt 5} \\[3ex]
    \dfrac{3\pi} 8 & \dfrac 12 \sqrt{2 + \sqrt 2} & \dfrac 12 \sqrt{2 - \sqrt 2}
        & \sqrt 2 + 1 \\[3ex]
    \dfrac{5\pi}{12} & \dfrac{\sqrt 6 + \sqrt 2} 4 & \dfrac{\sqrt 6 - \sqrt 2} 4
        & 2 + \sqrt 3  \\[2ex]
    \bottomrule
    \caption{More advanced trigonometric constants} \\
    \end{longtable}

    The second half of the table can be derived from the first half.

    The result for \(\frac 1{12} \pi\) can be derived by using
    \(\frac 1{12} \pi = \frac 13 \pi - \frac 14 \pi\).

    The result for \(\frac 18 \pi\) can be derived by using
    \(\frac 18 \pi = \frac 12 \cdot \frac 14 \pi\).

    To get the result for \(\frac 15 \pi\) we can let
    \(\varphi = \frac 15 \pi\).  Noting that
    \(\sin 2\varphi = \sin 3\varphi\), we have
    \begin{alignat*}{2}
    && 2\sin \varphi \cos \varphi
        &= \sin \varphi (2\cos^2 \varphi - 1)
         + \cos \varphi \cdot 2\sin \varphi \cos \varphi \\
    &\iff& 0 &= \sin \varphi (2\cos^2 \varphi - 1
                            + 2\cos^2 \varphi - 2\cos \varphi) \\
    &\iff& 0 &= 4\cos^2 \varphi - 2\cos \varphi - 1
        \qquad \text{as \(\sin \varphi \ne 0\)} \\
    &\iff& \cos \varphi &= \frac{1 \pm \sqrt 5} 4 \\
    &\iff& \cos \varphi &= \frac{1 + \sqrt 5} 4
        \qquad \text{as \(\cos \varphi > 0\)}
    \end{alignat*}
    We can then proceed to derive sine, and therefore also tangent, by applying
    the Pythagorean identity and simplifying.

    \subsection{Cosine Rule}

    \subsection{Sine Rule}

    \subsection{Sine Area Rule}

    \subsection{Pythagorean Identities} \label{sec_trig_pythag}

    It follows from Pythagoras' Theorem (\ref{sec_pythagoras})
    and subsequently from the definition of
    \(\tan\), \(\cot\), \(\sec\), \(\csc\) (\ref{sec_trig_reciprocal}) that
    \begin{align}
    \cos^2 \theta + \sin^2 \theta &\equiv 1 \\
    \implies \tan^2 \theta + 1 &\equiv \sec^2 \theta \\
    \cot^2 \theta + 1 &\equiv \csc^2 \theta
    \end{align}

    \subsection{Compound Angle Identities} \label{sec_comp_angle}

    %FIXME add diagram

    \begin{theorem}[Addition formula for sine]
    \begin{equation*}
    \sin(\alpha + \beta) \equiv
       \sin \alpha \cos \beta +  \sin \beta \cos \alpha
    \end{equation*}
    \end{theorem}
    \begin{proof}
    This can be shown geometrically.
    \end{proof}
    From this, it follows
    \begin{theorem}[Addition formula for cosine]
    \begin{equation*}
    \cos(\alpha + \beta) \equiv
       \cos \alpha \cos \beta - \sin \alpha \sin \beta
    \end{equation*}
    \end{theorem}
    \begin{proof}
    This could also be shown with a similar geometrical argument, but we can
    simply use section \ref{sec_trig_periodic} to show
    \begin{align*}
    \cos(\alpha + \beta) &\equiv
        \sin(\frac 12 \pi - \alpha - \beta) \equiv
        \sin(\frac 12 \pi - \alpha)\cos \beta +
            \sin \beta \cos(\frac 12 \pi - \alpha) \\
    &\equiv
        \cos \alpha \cos \beta + \sin \beta \sin \alpha
    \end{align*}
    \end{proof}
    We can now deduce
    \begin{theorem}[Addition formula for tangent]
    \begin{equation*}
    \tan(\alpha + \beta) \equiv
        \frac{\tan \alpha + \tan \beta}{1 - \tan \alpha \tan \beta}
    \end{equation*}
    \end{theorem}
    \begin{proof}
    \begin{equation*}
    \tan(\alpha + \beta) \equiv
        \frac{\sin(\alpha + \beta)}{\cos(\alpha + \beta)} \equiv
        \frac{\sin \alpha \cos \beta + \sin \beta \cos \alpha}
             {\cos \alpha \cos \beta + \sin \alpha \sin \beta} \equiv
        \frac{\frac{\sin \alpha \cos \beta}{\cos \alpha \cos \beta} +
              \frac{\sin \alpha \sin \beta}{\cos \alpha \cos \beta}}
             {\frac{\cos \alpha \cos \beta}{\cos \alpha \cos \beta} +
              \frac{\sin \alpha \sin \beta}{\cos \alpha \cos \beta}}
             \equiv
        \frac{\tan \alpha + \tan \beta}{1 - \tan \alpha \tan \beta}
    \end{equation*}
    \end{proof}
    There is a corresponding set of results also allowing for differences.
    \begin{theorem}[Compound angle formulae] \label{thm_trig_compound}
    \begin{align*}
    \sin \alpha \pm \beta &\equiv
       \sin \alpha \cos \beta \pm \sin \beta \cos \alpha \\
    \cos \alpha \pm \beta &\equiv
       \cos \alpha \cos \beta \mp \sin \alpha \sin \beta \\
    \tan \alpha \pm \beta &\equiv
        \frac{\tan \alpha \pm \tan \beta}{1 \mp \tan \alpha \tan \beta}
    \end{align*}
    \end{theorem}
    \begin{proof}
    Results for a function of a difference are derived from the parity of
    each function (\ref{sec_trig_parity}).
    \end{proof}
    One might also just assume it's probably OK to flip signs.

    \subsubsection{Double Angle Formulae} \label{sec_trig_double_angle}
    \begin{theorem}[Double angle formulae] \label{thm_trig_double_angle}
    \begin{align*}
    \sin 2\theta &\equiv
       2\sin \theta \cos \theta \\
    \cos 2\theta &\equiv
       \cos^2 \theta - \sin^2 \theta \equiv
       2\cos^2 \theta - 1 \equiv 1 - 2\sin^2 \theta
       \ \text{(by \ref{sec_trig_pythag})} \\
    \tan 2\theta &\equiv
        \frac{2\tan \theta}{1 - \tan^2 \theta}
    \end{align*}
    \end{theorem}
    \begin{proof}
    These follow from \(2\theta \equiv \theta + \theta\) in combination with
    Theorem \ref{thm_trig_compound}.
    \end{proof}
    \begin{theorem}[Squared half-angle identities]
    It follows, by rearrangement of identities in Theorem
    \ref{thm_trig_double_angle}, that
    \begin{align*}
    \sin^2 \theta &\equiv
        \frac{1 - \cos 2 \theta} 2 \\
    \cos^2 \theta &\equiv
        \frac{1 + \cos 2 \theta} 2
    \end{align*}
    and
    \begin{align*}
    \tan^2 \theta \equiv \frac{\sin^2 \theta}{\cos^2 \theta}
        \equiv \frac{1 - \cos 2 \theta}{1 + \cos 2 \theta}
        &\equiv \frac{1 - \cos^2 2 \theta}{(1 + \cos 2 \theta)^2}
        \equiv \pqty{\frac{\sin 2 \theta}{1 + \cos 2 \theta}}^2 \\
        &\equiv \frac{(1 - \cos 2 \theta)^2}{1 - \cos^2 2 \theta}
        \equiv \pqty{\frac{1 - \cos 2 \theta}{\sin 2 \theta}}^2
    \end{align*}
    \end{theorem}
    \begin{theorem}[Half angle identities] \label{thm_trig_half_angle}
    \begin{alignat*}{2}
    &&\sin \theta &\equiv
        \pm\sqrt{\frac{1 - \cos 2 \theta} 2}
     \ \pqty{\text{where \(\pm\) depends on the quadrant}} \\
    &\iff& \sin \frac \theta 2 &\equiv
        \pm\sqrt{\frac{1 - \cos \theta} 2} \\
    &&\cos \theta &\equiv
        \pm\sqrt{\frac{1 + \cos 2 \theta} 2} \\
    &\iff& \cos \frac \theta 2 &\equiv
        \pm\sqrt{\frac{1 + \cos \theta} 2} \\
    &&\tan \theta &\equiv
        \frac{\sin 2 \theta}{1 + \cos 2 \theta}
        \equiv \frac{1 - \cos 2 \theta}{\sin 2 \theta} \\
    &\iff& \tan \frac \theta 2 &\equiv
        \frac{\sin \theta}{1 + \cos \theta}
        \equiv \frac{1 - \cos \theta}{\sin \theta}
    \end{alignat*}
    \end{theorem}

    \subsubsection{Triple Angles and Beyond}

    \subsection{Sum-Product Identities} \label{sec_trig_sum_product}

    The products of trig functions can be variously rewritten as sums.

    \begin{theorem}[Product of sine and cosine]
    \begin{equation*}
    \sin \alpha \cos \beta \equiv
        \frac{\sin(\alpha + \beta) + \sin(\alpha - \beta)} 2
    \end{equation*}
    \end{theorem}
    \begin{proof}
    Derive from RHS:
    \begin{align*}
    \frac{\sin(\alpha + \beta) + \sin(\alpha - \beta)} 2 &\equiv
     \frac{\sin \alpha \cos \beta + \sin \beta \cos \alpha +
           \sin \alpha \cos \beta - \sin \beta \cos \alpha} 2 \\
    &\equiv \frac{2 \sin \alpha \cos \beta} 2 \equiv \sin \alpha \cos \beta
    \end{align*}
    \end{proof}
    \begin{theorem}[Product of cosines]
    \begin{equation*}
    \cos \alpha \cos \beta \equiv
     \frac{\cos(\alpha + \beta) + \cos(\alpha - \beta)} 2
    \end{equation*}
    \end{theorem}
    \begin{proof}
    Derive from RHS:
    \begin{align*}
    \frac{\cos(\alpha + \beta) + \cos(\alpha - \beta)} 2 &\equiv
     \frac{\cos \alpha \cos \beta - \sin \alpha \sin \beta +
           \cos \alpha \cos \beta + \sin \alpha \sin \beta} 2 \\
    &\equiv \frac{2 \cos \alpha \cos \beta} 2 \equiv \cos \alpha \cos \beta
    \end{align*}
    \end{proof}
    \begin{theorem}[Product of sines]
    \begin{equation*}
    \sin \alpha \sin \beta \equiv
     \frac{\cos(\alpha - \beta) - \cos(\alpha + \beta)} 2
    \end{equation*}
    \end{theorem}
    \begin{proof}
    Derive from RHS:
    \begin{align*}
    \frac{\cos(\alpha - \beta) - \cos(\alpha + \beta)} 2 &\equiv
     \frac{\cos \alpha \cos \beta + \sin \alpha \sin \beta -
           \cos \alpha \cos \beta + \sin \alpha \sin \beta} 2 \\
    &\equiv \frac{2 \sin \alpha \sin \beta} 2 \equiv \sin \alpha \sin \beta
    \end{align*}
    \end{proof}

    \subsection{Product-Sum Identities} \label{sec_trig_product_sum}

    %FIXME: annotate sections used

    \begin{theorem}[Relating the sum of sines to a product]
    \begin{equation*}
    \sin \alpha + \sin \beta \equiv
        2 \sin \frac{\alpha + \beta}2 \cos \frac{\alpha - \beta}2
    \end{equation*}
    \end{theorem}
    \begin{proof}
    We derive by considering \(\alpha' = \frac 12 (\alpha + \beta)\) and
    \(\beta' = \frac 12 (\alpha - \beta)\). Then,
    \begin{align*}
    \sin \alpha + \sin \beta &\equiv
     \sin(\alpha' + \beta') + \sin(\alpha' - \beta') \equiv
     2\sin \alpha' \cos \beta'\ \text{(from \ref{sec_trig_sum_product})}
    \\&\equiv 2 \sin \frac{\alpha + \beta}2 \cos \frac{\alpha - \beta} 2
    \end{align*}
    \end{proof}
    As a direct corollary of this due to the oddness of the sine
    (\ref{sec_trig_parity}), we have
    \begin{theorem}[Relating the difference of sines to a product]
    \begin{equation*}
    \sin \alpha - \sin \beta =
        \sin \alpha + \sin -\beta =
        2 \sin \frac{\alpha - \beta} 2 \cos\frac{\alpha + \beta} 2
    \end{equation*}
    \end{theorem}
    We also show
    \begin{theorem}[Relating the sum of cosines to a product]
    \begin{equation*}
    \cos \alpha + \cos \beta \equiv
        2 \cos\frac{\alpha + \beta} 2 \cos \frac{\alpha - \beta} 2
    \end{equation*}
    \end{theorem}
    \begin{proof}
    Again, let \(\alpha' = \frac 12 (\alpha + \beta)\) and
    \(\beta' = \frac 12 (\alpha - \beta)\). Then,
    \begin{align*}
     \cos \alpha + \cos \beta &\equiv
     \cos(\alpha' + \beta') + \cos(\alpha' - \beta') \equiv
     2\cos \alpha' \cos \beta'\ \text{(from \ref{sec_trig_sum_product})}
     \\&\equiv 2 \cos\frac{\alpha + \beta} 2 \cos \frac{\alpha - \beta} 2
    \end{align*}
    \end{proof}
    Finally, we show
    \begin{theorem}[Relating the difference of cosines to a product]
    \begin{equation*}
    \cos \alpha - \cos \beta \equiv
     -2 \sin \frac{\alpha + \beta} 2 \sin \frac{\alpha - \beta} 2
    \end{equation*}
    \end{theorem}
    \begin{proof}
    We can exploit the periodicity of cosine (\ref{sec_trig_periodic}).
    \begin{align*}
     \cos \alpha - \cos \beta &\equiv
     \cos \alpha + \cos (\beta + \pi)  \equiv
     2 \cos \frac{\alpha + \beta + \pi} 2 \cos \frac {\alpha - \beta - \pi} 2
     \\&\equiv
     2 \cos \pqty{\frac \pi 2 - \pqty{-\frac{\alpha + \beta} 2}}
       \cos \pqty{\frac \pi 2 - \frac{\alpha - \beta} 2} \equiv
     2 \sin -\frac{\alpha + \beta} 2 \sin \frac{\alpha - \beta} 2 \\&\equiv
     -2 \sin \frac{\alpha + \beta} 2 \sin \frac{\alpha - \beta} 2
    \end{align*}
    \end{proof}

    \subsection[The Weierstrass substitution (\(\tan(\theta / 2)\))]
       {The Weierstrass substitution \boldmath\(\pqty{\tan(\frac \theta 2)}\)}

    \begin{theorem}[The Weierstrass substitution]
    If we let \(t = \tan(\frac \theta 2)\), such that \(t\) is defined (ie
    \(\theta \ne (2n + 1)\pi, n \in \Integers\)) then we have:
    \begin{align*}
    \sin \theta &\equiv \dfrac{2t}{1 + t^2} \\
    \cos \theta &\equiv \dfrac{1 - t^2}{1 + t^2} \\
    \tan \theta &\equiv \dfrac{2t}{1 - t^2} \\
    \dv{\theta}{t} &\equiv \dfrac{2}{1 + t^2}
    \end{align*}
    \end{theorem}
    \begin{proof}
    The identity for the tangent is an immediate consequence of the double
    angle formula for the tangent (\ref{sec_trig_double_angle}), where we're
    considering \(\tan(2 \cdot \frac 12 \theta)\).

    The identity for the sine can be derived by considering the double angle
    formula for sine, and using a smattering of identities from
    (\ref{sec_trig_definitions}, \ref{sec_trig_pythag}):
    \begin{align*}
    \sin \theta &\equiv 2\sin(\tfrac \theta 2)\cos(\tfrac \theta 2)
        \equiv 2\tan(\tfrac \theta 2)\cos^2\pqty{\tfrac \theta 2} \\
        &\equiv \frac{2\tan(\frac \theta 2)}{\sec^2\pqty{\frac \theta 2}}
        \equiv \frac{2t}{1 + t^2}
    \end{align*}
    Similarly for the cosine:
    \begin{align*}
    \cos \theta
        &\equiv \cos^2\pqty{\frac \theta 2} - \sin^2 \pqty{\frac \theta 2}
        \equiv \cos^2\pqty{\frac \theta 2}
            \pqty{1 - \tan^2\pqty{\frac \theta 2}} \\
        &\equiv \frac{1 - \tan^2\pqty{\frac \theta 2}}
                     {\sec^2\pqty{\frac \theta 2}}
        \equiv \frac{1 - t^2}{1 + t^2}
    \end{align*}
    \end{proof}

    \subsubsection{Heuristic for sine and cosine}

    %FIXME add graphic.

    The formula involving the tangent can be easily derived from the double
    angle formula. Having this, we can draw a right triangle with angle
    \(\theta\), opposite side \(2t\) and adjacent side \(1 - t^2\). We can
    deduce from Pythagoras' theorem (\ref{sec_pythagoras}) that the length of
    the hypotenuse is
    \begin{equation*}
    \sqrt{(2t)^2 + (1 - t^2)^2} = \sqrt{1 + 2t^2 + t^4} = 1 + t^2
    \end{equation*}
    and then deduce the value of sine and cosine from their definitions.

    \section{Probability}

    \section{Statistics}

    \subsection{Random Variables}

    \subsection{Binomial Distribution}

    %FIXME add diagram

    The binomial distribution \(\Binomial(\nu, \pi)\) measures the probability
    of getting a certain number of successful ``events'', where there are
    \(\nu\) events each with probability \(\pi\) of being successful.

    The PMF of \(X \sim \Binomial(\nu, \pi)\) is given by
    \begin{equation*}
    f_X(k) = \Prob(X = k) = {\nu \choose k} \pi^k (1 - \pi)^{\nu - k}
        \quad \text{where \(k \in \set{0, 1, \dotsc, n}\)}
    \end{equation*}
    The validity of this distribution is subject to these constraints:
    \begin{enumerate}
    \item Each event must be independent.
    \end{enumerate}

    \subsection{Normal Distribution}

    %FIXME add diagram

    The PDF of \(X \sim \Normal(\mu, \sigma^2)\) is given by
    \begin{equation*}
    f_X(x) = \frac{1}{\sqrt{2\pi\sigma^2}} \cdot
        \exp(\frac{(x - \mu)^2}{2\sigma^2})
    \end{equation*}

    \subsubsection{Dimensional Verification/Mnemonic}

    NB \(\bqty{f(x)} =
        \bqty{\frac 1{\sqrt{\sigma^2}}}
        e^{\bqty{\frac{x^2}{\sigma^2}}}
      = \bqty{\frac 1\sigma}\), as
    \(\bqty{\sigma} = \bqty{\mu} = \bqty{x}\). Therefore, the
    evaluation of the probability \(\Prob(a < X < b)\) by the integral
    \(\int_a^bf(x) \dd{x}\) has dimension
    \(\bqty{\frac{\dd{x}}{\sigma}} = 1\), as expected for a probability.

    \subsection{Poisson Distribution}

    %FIXME derivation as limit of binomial, see generating functions

    \subsection{Hypothesis Testing}

    %FIXME Types of Error

    \section{Calculus}

    %FIXME: derivations from first principles

    \subsection{Definition of the derivative}

    The derivative of \(f(x)\),
    \begin{equation}
    f'(x) = \lim_{h \to 0} \frac{f(x + h) - f(x)}{h}
    \end{equation}
    Other notations include, if \(y = f(x)\), then
    \(\displaystyle f'(x) = \dv{x}(f(x)) = \dv{y}{x} = y'\).

    There is also various notation for repeated differentiation. If
    \(g(x) = f'(x)\), then \(g'(x) = f''(x)\).
    \(\displaystyle f''(x) = \dv[2]{y}{x} = \dv[2]{x}(f(x))\). In general, the
    \(n\)th derivative is
    \(\displaystyle f^{(n)}(x) = \dv[n]{y}{x} = \dv[n]{x}(f(x))\).

    \subsection{Basic properties of the derivative}
    \label{sec_calc_derivative_properties}

    Some basic properties of limits result in corresponding basic properties of
    the derivative.
    \begin{equation*}
    \dv{x}(a \cdot f(x)) = af'(x) \quad \text{where \(a\) is a constant}
    \end{equation*}
    \begin{equation*}
    \dv{x}(f(x) + g(x)) = f'(x) + g'(x)
    \end{equation*}

    \subsection{Fundamental Theorem of Calculus} \label{sec_calc_FTC}

    \subsection{Chain Rule} \label{sec_calc_chain}

    \subsection{Product Rule} \label{sec_calc_product}

    \subsubsection{Quotient Rule} \label{sec_calc_quotient}

    \subsection{Integration by Substitution} \label{sec_calc_substitution}

    Integration by substitution is effectively the chain rule
    (\ref{sec_calc_chain}) in reverse. If we consider the function
    \begin{equation*}
    f(x) = g(h(x)) \implies f'(x) = h'(x) \cdot g'(h(x))
    \end{equation*}
    then the integral
    \begin{equation}
    \int h'(x) \cdot g'(h(x)) \dd{x} = \int f'(x) \dd{x} = f(x) + C
    \end{equation}

    \subsubsection{Trig Substitution} \label{sec_calc_trig_substitution}

    \subsection{Integration by Parts}

    Integration by parts is derived from the product rule
    (\ref{sec_calc_product}).  Consider the function
    \begin{equation*}
    f(x) = g(x) \cdot h(x) \implies f'(x) = g'(x) \cdot h(x) + g(x) \cdot h'(x)
    \end{equation*}
    then the integral
    \begin{equation}
    \int g'(x) \cdot h(x) \dd{x} = f(x) - \int g(x) \cdot h'(x) \dd{x} + C
    \end{equation}

    \subsection{Common derivatives} \label{calc_common}

    %FIXME: add x^x. maybe more intricate proofs.

    Here follows a table of common differentiations. A number of derivative
    derivations are also given in the following subsubsections.

    Note that by FTC (\ref{sec_calc_FTC}), these also give a number of common
    integrations.

    \begin{longtable}{*{3}{>{\(}c<{\)}}c}
    \toprule
    \text{\boldmath\(f(x)\)} & \text{\boldmath\(f'(x)\)}
        & \text{\bfseries Alternatives/Notes} & \bfseries Reference\\
    \midrule
    \endhead
    g(x) + h(x) & g'(x) + h'(x) && \ref{sec_calc_derivative_properties} \\[1ex]
    a \cdot g(x) & a \cdot g'(x) && \ref{sec_calc_derivative_properties} \\[1ex]
    g(x) \cdot h(x) & g'(x) \cdot h(x) + g(x) \cdot h'(x)
        && \ref{sec_calc_product} \\[1ex]
    g(h(x)) & h'(x) \cdot g'(h(x)) && \ref{sec_calc_chain} \\[1ex]
    \dfrac{g(x)}{h(x)} & \dfrac{h(x) \cdot g'(x) - g(x) \cdot h'(x)}{(h(x))^2}
        && \ref{sec_calc_quotient} \\[3ex]
    x^n & nx^{n-1} & \text{\(n\) is constant} \\[1ex]
    e^x & e^x & \text{\(e\) is Euler's number} & \ref{sec_e} \\[1ex]
    a^x & \ln a \cdot a^x & \text{\(a\) is constant} \\[1ex]
    \ln x & \dfrac 1{x} \\[3ex]
    \sin x & \cos x \\[1ex]
    \cos x & -\sin x && \ref{sec_calc_trig_basic} \\[1ex]
    \tan x & \sec^2 x & \tan^2 x + 1 & \ref{sec_calc_trig_basic} \\[1ex]
    \sec x & \sec x \tan x && \ref{sec_calc_trig_basic} \\[1ex]
    \csc x & -\csc x \cot x && \ref{sec_calc_trig_basic} \\[1ex]
    \cot x & -\csc^2 x & -(\cot^2 x + 1) & \ref{sec_calc_trig_basic} \\[1ex]
    \arcsin x & \dfrac 1{\sqrt{1 - x^2}} \\[3ex]
    \arccos x & -\dfrac 1{\sqrt{1 - x^2}} \\[3ex]
    \arctan x & \dfrac 1{x^2 + 1} \\[3ex]
    \arcsec x & \dfrac 1{\abs{x} \sqrt{x^2 - 1}} \\[3ex]
    \arccsc x & -\dfrac 1{\abs{x} \sqrt{x^2 - 1}} \\[3ex]
    \arccot x & -\dfrac 1{x^2 + 1} \\[3ex]
    \sinh x & \cosh x \\[1ex]
    \cosh x & \sinh x \\[1ex]
    \tanh x & \sech^2 x & -\tanh^2 x + 1 \\[1ex]
    \sech x & -\tanh x \sech x \\[1ex]
    \csch x & -\coth x \csch x \\[1ex]
    \coth x & -\csch^2 x & -(\coth^2 x + 1) \\[1ex]
    \arcsinh x & \dfrac 1{\sqrt{x^2 + 1}} \\[3ex]
    \arccosh x & \dfrac 1{\sqrt{x^2 - 1}} \\[3ex]
    \arctanh x & \dfrac 1{1 - x^2} \\[3ex]
    \arcsech x & -\dfrac 1{x\sqrt{1 - x^2}} \\[3ex]
    \arccsch x & -\dfrac 1{\abs{x}\sqrt{1 + x^2}} \\[3ex]
    \arccoth x & \dfrac 1{1 - x^2} \\[3ex]
    \bottomrule
    \caption{Common derivatives} \\
    \end{longtable}

    \subsubsection{Basic trigonometric functions} \label{sec_calc_trig_basic}

    %FIXME decide how to get around this

    For now, we assume that \(\dv{x}(\sin x) = \cos x\). This can
    be shown in a number of ways. There are geometric arguments, but we could
    also appeal to the definition of sine and cosine in terms of complex
    exponentials.

    We can deduce from the chain rule that
    \begin{equation*}
    \dv{x}(\cos x) = \dv{x}(\sin(\tfrac 12 \pi - x))
        = -\cos(\tfrac 12 \pi - x) = -\sin(x)
    \end{equation*}
    Now we can use the quotient rule for the tangent.
    \begin{equation*}
    \dv{x}(\tan x) = \dv{x}(\frac{\sin x}{\cos x})
        = \frac{\cos x \cos x + \sin x \sin x}{\cos^2 x}
        = \frac 1{\cos^2 x} = \sec^2 x
    \end{equation*}
    We can use the chain rule for the reciprocal functions.
    \begin{equation*}
    \dv{x}(\sec x) = \dv{x}(\frac 1{\cos x}) = -\frac{-\sin x}{\cos^2 x}
        = \tan x \sec x
    \end{equation*}
    \begin{equation*}
    \dv{x}(\csc x) = \dv{x}(\frac 1{\sin x}) = -\frac{\cos x}{\sin^2 x}
        = -\cot x \csc x
    \end{equation*}
    The quotient rule can again be applied for the cotangent.
    \begin{equation*}
    \dv{x}(\cot x) = \dv{x}(\frac{\cos x}{\sin x})
        = \frac{-\sin x \sin x - \cos x \cos x}{\sin^2 x}
        = -\frac 1{\sin^2 x} = -\csc^2 x
    \end{equation*}

    \subsubsection{Inverse trigonometric functions} \label{sec_calc_trig_inv}

    We can use implicit differentiation and some identities to tackle
    derivatives of inverse trig functions. If we let \(y = \arcsin x\), then
    \begin{equation*}
    \sin y = x \implies \dv{y}{x} \cos y = 1
        \implies \dv{x}(\arcsin x) = \frac{1}{\cos y} = \frac{1}{\sqrt{1 - x^2}}
    \end{equation*}
    This is allowed because the domain of \(\arcsin x\) is restricted to
    \(\intcc{-1, 1}\), meaning that the range of \(y\) is
    \(\intcc{-\frac 12 \pi, \frac 12 \pi}\) and \(\cos y > 0\).

    Now, for the inverse cosine, rather than repeating this whole song and
    dance, we can use the fact that
    \(\arccos x \equiv \frac 12 \pi - \arcsin x\).
    \begin{equation*}
    \dv{x}(\arccos x) = \dv{x}(\frac \pi 2 - \arcsin x)
        = -\frac 1{\sqrt{1 - x^2}}
    \end{equation*}
    For the inverse tangent, we again let \(y = \arctan x\), so that
    \begin{equation*}
    % for some reason, \dv clobbers parenthesised things after it.
    \tan y = x \implies \dv{y}{x} {(\tan^2 y + 1)} = 1
        \implies \dv{x}(\arctan x) = \frac 1{1 + \tan^2 y}
            = \frac 1{1 + x^2}
    \end{equation*}
    For the inverse secant, we let \(y = \arcsec x\), so that
    \begin{equation*}
    \sec y = x \implies \dv{y}{x} \sec x \tan x = 1
        \implies \dv{x}(\arcsec x) = \frac 1{x \tan x}
            = \frac 1{x \sqrt{x^2 - 1}}
    \end{equation*}

    \subsection{Common integrals}

    \subsection{Volume of Revolution}

    \subsection{Area under Parametric Curve} \label{sec_calc_parametric_area}

    To find the area between a curve and the \(x\)-axis from \(x_1\) to \(x_2\)
    when the curve is given in parametric form, \((x, y) = (f(t), g(t))\),
    we rewrite the integral, effectively using a substition
    (\ref{sec_calc_substitution}) \(x = f(t)\) in order to express the integrand
    in \(t\).
    \begin{equation}
    \int_{x_1}^{x_2} y \dd{x} =
     \int_{t_1}^{t_2} y \dv{x}{t} \dd{t} =
     \int_{t_1}^{t_2} g(t)f'(t) \dd{t}
    \end{equation}
    where \(f(t_1) = x_1\) and \(f(t_2) = x_2\).

    Similarly, the area between the curve and the y-axis from \(y_1\) to \(y_2\)
    is:
    \begin{equation}
    \int_{y_1}^{y_2} x \dd{y} =
     \int_{t_1}^{t_2} x \dv{y}{t} \dd{t} =
     \int_{t_1}^{t_2} f(t)g'(t) \dd{t}
    \end{equation}
    where \(g(t_1) = y_1\) and \(g(t_2) = y_2\).

    \subsection{Implicit differentiation}

    % fix this

    %FIXME: add x^y for fun

    An example may help to clarify.
    \begin{align*}
    0 &= y^2 + x^3 + \sin x^2 y + \ln \frac{\sqrt y} x - 5 \\
    \implies 0 &=
                2y \dv{y}{x} + 3x^2 +
                \pqty{2x \cdot y + \dv{y}{x} \cdot x^2} \cdot \cos x^2 y +
                \pqty{\frac{x \cdot \frac 12 y^{-\frac 12} \cdot \dv{y}{x} -
                            \sqrt y \cdot 1}
                           {x^2}} \cdot \frac{x}{\sqrt y} \\
    \end{align*}

    %FIXME finish this

    \subsection{Arc Length}

    \subsection{Improper Integrals}

    %FIXME add types of improper integral

    \subsection{Taylor Series}

    The Maclaurin series is the Taylor series around \(0\).
    \begin{equation}
    f(x) = f(0) + f'(0) x + \frac{f''(0)} 2 x^2 + \frac{f'''(0)}{6}x^3 +\dotsb
      = \sum_{k=0}^\infty \frac{f^{(k)}(0)}{k!}x^k
    \end{equation}

    \subsection{Common series}

    \subsection{Fake Calculus, AKA numerical approximations}

    \section{Sequences and Series}

    \subsection{Fibonacci Sequence}

    \begin{theorem}[Fibonacci \(n\)th term]
    The Fibonacci numbers \(F_n: n \in \Naturals\) are such that
    \(F_1 = F_2 = 1\) and \(F_n = F_{n - 1} + F_{n - 2}\).  \(F_n\) can be given
    by the closed form
    \begin{equation*}
    F_n = \frac{\varphi^n - \psi^n}{\sqrt 5}
    \ \text{where}\ \varphi = \frac{1 + \sqrt 5} 2
    \ \text{and}\ \psi = \frac{1 - \sqrt 5} 2
    \end{equation*}
    \end{theorem}
    \begin{proof}
    Consider the power series
    \begin{equation*}
    f(x) \defeq \sum_{k=1}^\infty F_k x^k = x + x^2 + 2x^3 + 3x^4 + 5x^5
        + \dotsb
    \end{equation*}
    Note that this series has a radius of convergence, as it is strictly less
    than the series
    \begin{equation*}
    g(x) \defeq \sum_{k=1}^\infty 2^k x^k
    \end{equation*}
    This can be proven by induction. First we note that \(F_1 < 2^1\) and
    \(F_2 < 2^2\), and that \\
    \(F_{n + 2} = F_{n + 1} + F_n < 2^{n + 1} + 2^n < 2^{n + 1} + 2^{n + 1}
        = 2^{n + 2}\). So \(F_n < 2^n\) for all \(n\). Therefore, for
    \(\abs{x} < \frac 12\), \(f(x)\)  must be convergent, as \(g(x)\) is
    convergent for \(\abs{x} < \frac 12\). This means \(f(x)\) is well defined
    and we can manipulate it.

    Consider now:
    \begin{alignat*}{9}
    && f(x) &= \sum_{k = 1}^\infty F_k x^k
        &&=&\; x\, &+&\, x^2\, &+&\, 2x^3\, &+&\,
            3x^4\, &+&\, 5x^5\, &+&\, 8x^6\, + \dotsb \\
    && x f(x) &= \sum_{k = 2}^\infty F_{k - 1} x^k
        &&=& &&x^2 &+& x^3 &+& 2x^4 &+& 3x^5 &+& 5x^6 + \dotsb \\
    && x^2 f(x) &= \sum_{k = 3}^\infty F_{k - 2}x^k
        &&=& &&&& x^3 &+& x^4 &+& 2x^5 &+& 3x^6 + \dotsb \\
    &\implies& (1 - x - x^2) f(x) &= x \\
    &\implies& f(x) &= \frac x{1 - x - x^2}
    \end{alignat*}
    Now we perform a partial fraction decomposition on \(f(x)\). We do this in a
    somewhat tricky way in order to make our lives easier. Note that
    \begin{equation*}
    1 - x - x^2 = x^2 \pqty{\frac 1x^2 - \frac 1x - 1}
        = x^2\pqty{\frac 1x - \varphi}\pqty{\frac 1x - \psi}
        = (1 - \varphi x)(1 - \psi x)
    \end{equation*}
    where \(\varphi, \psi = \dfrac{1 \pm \sqrt 5}2\), ie the roots of
    \(x^2 - x - 1\), ie the roots of the golden ratio. Now,
    \begin{alignat*}{2}
    && f(x) = \frac x{(1 - \varphi x)(1 - \psi x)}
        &\equiv \frac A{1 - \varphi x} + \frac B{1 - \psi x} \\
    &\implies& A(1 - \psi x) + B(1 - \varphi x) &\equiv x \\
    &\implies& \left\{
        \begin{aligned}
            && A + B &= 0 \\
            &\implies& B &= -A  \\
            && -\psi A - \varphi B &= 1
        \end{aligned} \right. \\
    &\implies& A(\psi - \varphi) &= 1 \\
    &\implies& A(\sqrt 5) &= 1 \\
    &\implies& A &= \frac 1{\sqrt 5}, B = -\frac 1{\sqrt 5}
    \end{alignat*}
    Now we can use a binomial expansion to obtain an equivalent series.
    \begin{align*}
    f(x) &= \frac 1{\sqrt 5}\pqty{\frac 1{1 - \varphi x}
                                - \frac 1{1 - \psi x}} \\
    &= \frac 1{\sqrt 5}\pqty{\sum_{k = 0}^\infty (\varphi x)^k
                           - \sum_{k = 0}^\infty (\psi x)^k} \\
    &= \sum_{k = 0}^\infty \frac 1{\sqrt 5}(\varphi^k - \psi^k)x^k
    = \sum_{k = 1}^\infty F_k x^k
    \end{align*}
    So we have \(F_n = \dfrac 1{\sqrt 5}(\varphi^n - \psi^n)\). Note that the
    constant term is indeed zero, as \(\varphi^0 - \psi^0 = 0\).
    \end{proof}
    Combined with Exponentation by Squaring (\ref{sec_exp_by_squaring}), and
    some simple surd arithmetic, this should be a particularly fast way to
    calculate \(F_n\), as compared to utterly na\"ive recursion, or somewhat
    faster iteration or memoized recursion.

    \section{Sandbox}

    \lettrine{\color{RoyalBlue4}I}{zaak van Dongen}. \lipsum[1]

    \begin{center}
    \begin{longtable}{ccc}
    \(
    \begin{array}{cccccccc}
    \alpha&\mathrm{A}&\beta&\mathrm{B}&\gamma&\Gamma&\delta&\Delta\\
    \epsilon&\mathrm{E}&\zeta&\mathrm{Z}&\eta&\mathrm{H}&\theta&\Theta\\
    \iota&\mathrm{I}&\kappa&\mathrm{K}&\lambda&\Lambda&\mu&\mathrm{M}\\
    \nu&\mathrm{N}&\omicron&\mathrm{O}&\pi&\Pi&\rho&\mathrm{R}\\
    \sigma&\Sigma&\tau&\mathrm{T}&\upsilon&\Upsilon&\phi&\Phi\\
    \chi&\mathrm{X}&\psi&\Psi&\omega&\Omega&\xi&\Xi\\
    \vartheta&\varpi&\varphi&\varrho&\varepsilon&\varsigma&\uptau&\ell\\
    \end{array}\) &
    \(\begin{array}{cccc}
    \pm&\cap&\diamond&\oplus\\
    \mp&\cup&\bigtriangleup&\ominus\\
    \times&\uplus&\bigtriangledown&\otimes\\
    \divisionsymbol&\sqcap&\triangleleft&\oslash\\
    \ast&\sqcup&\triangleright&\odot\\
    \star&\vee \land&\lhd&\bigcirc\\
    \circ&\wedge \lor&\rhd&\dagger\\
    \bullet&\setminus&\unlhd&\ddagger\\
    \cdot&\wr&\unrhd&\amalg\\
    +&-\\
    \end{array}\) &
    \(\begin{array}{cccc}
    \leq&\geq&\equiv&\models\\
    \prec&\succ&\sim&\perp\\
    \preceq&\succeq&\simeq&\mid\\
    \ll&\gg&\asymp&\parallel\\
    \subset&\supset&\approx&\bowtie\\
    \subseteq&\supseteq&\cong&\Join\\
    \sqsubset&\sqsupset&\neq&\smile\\
    \sqsubseteq&\sqsupseteq&\doteq&\frown\\
    \in&\ni&\propto&=\\
    \vdash&\dashv&<&>\\
    \end{array}\) \\
    \(\begin{array}{ccccc}
    \arccos&\cos&\csc&\exp\\
    \ker&\limsup&\min&\sinh\\
    \arcsin&\cosh&\deg&\gcd\\
    \lg&\ln&\Pr&\sup\\
    \arctan&\cot&\det&\hom\\
    \lim&\log&\sec&\tan\\
    \arg&\coth&\dim&\inf\\
    \liminf&\max&\sin&\tanh\\
    \Re&\Im&\operatorname{Adj} A\\
    \end{array}\) &
    \(\begin{array}{ccc}
    \leftarrow&\longleftarrow&\uparrow\\
    \Leftarrow&\Longleftarrow&\Uparrow\\
    \rightarrow&\longrightarrow&\downarrow\\
    \Rightarrow&\Longrightarrow&\Downarrow\\
    \leftrightarrow&\longleftrightarrow&\updownarrow\\
    \Leftrightarrow&\Longleftrightarrow&\Updownarrow\\
    \mapsto&\longmapsto&\nearrow\\
    \hookleftarrow&\hookrightarrow&\searrow\\
    \leftharpoonup&\rightharpoonup&\swarrow\\
    \leftharpoondown&\rightharpoondown&\nwarrow\\
    \rightleftharpoons&\leadsto\\
    \end{array}\) &
    \(\begin{array}{cccc}
    \ldots&\cdots&\vdots&\ddots\\
    \aleph&\prime&\forall&\infty\\
    \hbar&\emptyset&\exists&\Box\\
    \imath&\nabla&\neg&\Diamond\\
    \jmath&\surd&\flat&\triangle\\
    \ell&\top&\natural&\clubsuit\\
    \wp&\bot&\sharp&\diamondsuit\\
    \real&\|&\backslash&\heartsuit\\
    \imaginary&\angle&\partial&\spadesuit\\
    \end{array}\) \\
    \(\begin{array}{cc}
    DIFferent          & % symbols
    \mathit{DIFferent} \\ % italic
    \mathrm{DIFferent} & % roman
    \mathsf{DIFferent} \\ % sans-serif
    \mathtt{DIFferent} & % typewriter text
    \mathbf{DIFferent} \\ % bold font
    \operatorname{DIFferen}(t) & % operator
    \mathcal{DIFFERENT} \\ % calligraphic
    \mathbb{DIFFERENT} & % blackboard bold
    \mathfrak{DIFferent} \\ % fraktur
    \textfrak{DIFferent} & % also fraktur
    \textswab{DIFferent} \\ % swabacher
    \textgoth{DIFferent} & % gothic font
    \mathscr{DIFFERENT} \\ % script font
    \overrightwitchonbroom{\mathit{DIFferent}} & % a witch on a broom
    (\vb{d} \cp \vb*{i}) \vdot \va{f} - \abs{\va*{f}} \norm{\vu{e}} + \vu*{r} \\
    \end{array}\) &
    \(\begin{array}{ccc}
    \sum&\bigcap&\bigodot\\
    \prod&\bigcup&\bigotimes\\
    \coprod&\bigsqcup&\bigoplus\\
    \int&\bigvee&\biguplus\\
    \oint&\bigwedge\\
    \floor*{} & \ceil*{} \\
    \end{array}\) &
    \(\begin{array}{ccc}
    \widetilde{abc}&\widehat{abc}&\overleftarrow{abc}\\
    \overrightarrow{abc}&\overline{abc}&\underline{abc}\\
    \overbrace{abc}&\underbrace{abc}&\sqrt{abc}\\
    \sqrt[n]{abc}&\hat{a}&\acute{a}\\
    \bar{a}&\dot{a}&\breve{a}\\
    \check{a}&\grave{a}&\vec{a}\\
    \ddot{a}&\tilde{a}\\
    \end{array}\) \\
    \multicolumn{3}{c}{
    \begin{tabular}{ccccccccccccccccccccc}
{\color{AntiqueWhite1} t}\textcolor{AntiqueWhite1}{t} &
{\color{AntiqueWhite2} t}\textcolor{AntiqueWhite2}{t} &
{\color{AntiqueWhite3} t}\textcolor{AntiqueWhite3}{t} &
{\color{AntiqueWhite4} t}\textcolor{AntiqueWhite4}{t} &
{\color{Aquamarine1} t}\textcolor{Aquamarine1}{t} &
{\color{Aquamarine2} t}\textcolor{Aquamarine2}{t} &
{\color{Aquamarine3} t}\textcolor{Aquamarine3}{t} &
{\color{Aquamarine4} t}\textcolor{Aquamarine4}{t} &
{\color{Azure1} t}\textcolor{Azure1}{t} &
{\color{Azure2} t}\textcolor{Azure2}{t} &
{\color{Azure3} t}\textcolor{Azure3}{t} &
{\color{Azure4} t}\textcolor{Azure4}{t} &
{\color{Bisque1} t}\textcolor{Bisque1}{t} &
{\color{Bisque2} t}\textcolor{Bisque2}{t} &
{\color{Bisque3} t}\textcolor{Bisque3}{t} &
{\color{Bisque4} t}\textcolor{Bisque4}{t} &
{\color{Blue1} t}\textcolor{Blue1}{t} &
{\color{Blue2} t}\textcolor{Blue2}{t} &
{\color{Blue3} t}\textcolor{Blue3}{t} &
{\color{Blue4} t}\textcolor{Blue4}{t} &
{\color{Brown1} t}\textcolor{Brown1}{t} \\
{\color{Brown2} t}\textcolor{Brown2}{t} &
{\color{Brown3} t}\textcolor{Brown3}{t} &
{\color{Brown4} t}\textcolor{Brown4}{t} &
{\color{Burlywood1} t}\textcolor{Burlywood1}{t} &
{\color{Burlywood2} t}\textcolor{Burlywood2}{t} &
{\color{Burlywood3} t}\textcolor{Burlywood3}{t} &
{\color{Burlywood4} t}\textcolor{Burlywood4}{t} &
{\color{CadetBlue1} t}\textcolor{CadetBlue1}{t} &
{\color{CadetBlue2} t}\textcolor{CadetBlue2}{t} &
{\color{CadetBlue3} t}\textcolor{CadetBlue3}{t} &
{\color{CadetBlue4} t}\textcolor{CadetBlue4}{t} &
{\color{Chartreuse1} t}\textcolor{Chartreuse1}{t} &
{\color{Chartreuse2} t}\textcolor{Chartreuse2}{t} &
{\color{Chartreuse3} t}\textcolor{Chartreuse3}{t} &
{\color{Chartreuse4} t}\textcolor{Chartreuse4}{t} &
{\color{Chocolate1} t}\textcolor{Chocolate1}{t} &
{\color{Chocolate2} t}\textcolor{Chocolate2}{t} &
{\color{Chocolate3} t}\textcolor{Chocolate3}{t} &
{\color{Chocolate4} t}\textcolor{Chocolate4}{t} &
{\color{Coral1} t}\textcolor{Coral1}{t} &
{\color{Coral2} t}\textcolor{Coral2}{t} \\
{\color{Coral3} t}\textcolor{Coral3}{t} &
{\color{Coral4} t}\textcolor{Coral4}{t} &
{\color{Cornsilk1} t}\textcolor{Cornsilk1}{t} &
{\color{Cornsilk2} t}\textcolor{Cornsilk2}{t} &
{\color{Cornsilk3} t}\textcolor{Cornsilk3}{t} &
{\color{Cornsilk4} t}\textcolor{Cornsilk4}{t} &
{\color{Cyan1} t}\textcolor{Cyan1}{t} &
{\color{Cyan2} t}\textcolor{Cyan2}{t} &
{\color{Cyan3} t}\textcolor{Cyan3}{t} &
{\color{Cyan4} t}\textcolor{Cyan4}{t} &
{\color{DarkGoldenrod1} t}\textcolor{DarkGoldenrod1}{t} &
{\color{DarkGoldenrod2} t}\textcolor{DarkGoldenrod2}{t} &
{\color{DarkGoldenrod3} t}\textcolor{DarkGoldenrod3}{t} &
{\color{DarkGoldenrod4} t}\textcolor{DarkGoldenrod4}{t} &
{\color{DarkOliveGreen1} t}\textcolor{DarkOliveGreen1}{t} &
{\color{DarkOliveGreen2} t}\textcolor{DarkOliveGreen2}{t} &
{\color{DarkOliveGreen3} t}\textcolor{DarkOliveGreen3}{t} &
{\color{DarkOliveGreen4} t}\textcolor{DarkOliveGreen4}{t} &
{\color{DarkOrange1} t}\textcolor{DarkOrange1}{t} &
{\color{DarkOrange2} t}\textcolor{DarkOrange2}{t} &
{\color{DarkOrange3} t}\textcolor{DarkOrange3}{t} \\
{\color{DarkOrange4} t}\textcolor{DarkOrange4}{t} &
{\color{DarkOrchid1} t}\textcolor{DarkOrchid1}{t} &
{\color{DarkOrchid2} t}\textcolor{DarkOrchid2}{t} &
{\color{DarkOrchid3} t}\textcolor{DarkOrchid3}{t} &
{\color{DarkOrchid4} t}\textcolor{DarkOrchid4}{t} &
{\color{DarkSeaGreen1} t}\textcolor{DarkSeaGreen1}{t} &
{\color{DarkSeaGreen2} t}\textcolor{DarkSeaGreen2}{t} &
{\color{DarkSeaGreen3} t}\textcolor{DarkSeaGreen3}{t} &
{\color{DarkSeaGreen4} t}\textcolor{DarkSeaGreen4}{t} &
{\color{DarkSlateGray1} t}\textcolor{DarkSlateGray1}{t} &
{\color{DarkSlateGray2} t}\textcolor{DarkSlateGray2}{t} &
{\color{DarkSlateGray3} t}\textcolor{DarkSlateGray3}{t} &
{\color{DarkSlateGray4} t}\textcolor{DarkSlateGray4}{t} &
{\color{DeepPink1} t}\textcolor{DeepPink1}{t} &
{\color{DeepPink2} t}\textcolor{DeepPink2}{t} &
{\color{DeepPink3} t}\textcolor{DeepPink3}{t} &
{\color{DeepPink4} t}\textcolor{DeepPink4}{t} &
{\color{DeepSkyBlue1} t}\textcolor{DeepSkyBlue1}{t} &
{\color{DeepSkyBlue2} t}\textcolor{DeepSkyBlue2}{t} &
{\color{DeepSkyBlue3} t}\textcolor{DeepSkyBlue3}{t} &
{\color{DeepSkyBlue4} t}\textcolor{DeepSkyBlue4}{t} \\
{\color{DodgerBlue1} t}\textcolor{DodgerBlue1}{t} &
{\color{DodgerBlue2} t}\textcolor{DodgerBlue2}{t} &
{\color{DodgerBlue3} t}\textcolor{DodgerBlue3}{t} &
{\color{DodgerBlue4} t}\textcolor{DodgerBlue4}{t} &
{\color{Firebrick1} t}\textcolor{Firebrick1}{t} &
{\color{Firebrick2} t}\textcolor{Firebrick2}{t} &
{\color{Firebrick3} t}\textcolor{Firebrick3}{t} &
{\color{Firebrick4} t}\textcolor{Firebrick4}{t} &
{\color{Gold1} t}\textcolor{Gold1}{t} &
{\color{Gold2} t}\textcolor{Gold2}{t} &
{\color{Gold3} t}\textcolor{Gold3}{t} &
{\color{Gold4} t}\textcolor{Gold4}{t} &
{\color{Goldenrod1} t}\textcolor{Goldenrod1}{t} &
{\color{Goldenrod2} t}\textcolor{Goldenrod2}{t} &
{\color{Goldenrod3} t}\textcolor{Goldenrod3}{t} &
{\color{Goldenrod4} t}\textcolor{Goldenrod4}{t} &
{\color{Green1} t}\textcolor{Green1}{t} &
{\color{Green2} t}\textcolor{Green2}{t} &
{\color{Green3} t}\textcolor{Green3}{t} &
{\color{Green4} t}\textcolor{Green4}{t} &
{\color{Honeydew1} t}\textcolor{Honeydew1}{t} \\
{\color{Honeydew2} t}\textcolor{Honeydew2}{t} &
{\color{Honeydew3} t}\textcolor{Honeydew3}{t} &
{\color{Honeydew4} t}\textcolor{Honeydew4}{t} &
{\color{HotPink1} t}\textcolor{HotPink1}{t} &
{\color{HotPink2} t}\textcolor{HotPink2}{t} &
{\color{HotPink3} t}\textcolor{HotPink3}{t} &
{\color{HotPink4} t}\textcolor{HotPink4}{t} &
{\color{IndianRed1} t}\textcolor{IndianRed1}{t} &
{\color{IndianRed2} t}\textcolor{IndianRed2}{t} &
{\color{IndianRed3} t}\textcolor{IndianRed3}{t} &
{\color{IndianRed4} t}\textcolor{IndianRed4}{t} &
{\color{Ivory1} t}\textcolor{Ivory1}{t} &
{\color{Ivory2} t}\textcolor{Ivory2}{t} &
{\color{Ivory3} t}\textcolor{Ivory3}{t} &
{\color{Ivory4} t}\textcolor{Ivory4}{t} &
{\color{Khaki1} t}\textcolor{Khaki1}{t} &
{\color{Khaki2} t}\textcolor{Khaki2}{t} &
{\color{Khaki3} t}\textcolor{Khaki3}{t} &
{\color{Khaki4} t}\textcolor{Khaki4}{t} &
{\color{LavenderBlush1} t}\textcolor{LavenderBlush1}{t} &
{\color{LavenderBlush2} t}\textcolor{LavenderBlush2}{t} \\
{\color{LavenderBlush3} t}\textcolor{LavenderBlush3}{t} &
{\color{LavenderBlush4} t}\textcolor{LavenderBlush4}{t} &
{\color{LemonChiffon1} t}\textcolor{LemonChiffon1}{t} &
{\color{LemonChiffon2} t}\textcolor{LemonChiffon2}{t} &
{\color{LemonChiffon3} t}\textcolor{LemonChiffon3}{t} &
{\color{LemonChiffon4} t}\textcolor{LemonChiffon4}{t} &
{\color{LightBlue1} t}\textcolor{LightBlue1}{t} &
{\color{LightBlue2} t}\textcolor{LightBlue2}{t} &
{\color{LightBlue3} t}\textcolor{LightBlue3}{t} &
{\color{LightBlue4} t}\textcolor{LightBlue4}{t} &
{\color{LightCyan1} t}\textcolor{LightCyan1}{t} &
{\color{LightCyan2} t}\textcolor{LightCyan2}{t} &
{\color{LightCyan3} t}\textcolor{LightCyan3}{t} &
{\color{LightCyan4} t}\textcolor{LightCyan4}{t} &
{\color{LightGoldenrod1} t}\textcolor{LightGoldenrod1}{t} &
{\color{LightGoldenrod2} t}\textcolor{LightGoldenrod2}{t} &
{\color{LightGoldenrod3} t}\textcolor{LightGoldenrod3}{t} &
{\color{LightGoldenrod4} t}\textcolor{LightGoldenrod4}{t} &
{\color{LightPink1} t}\textcolor{LightPink1}{t} &
{\color{LightPink2} t}\textcolor{LightPink2}{t} &
{\color{LightPink3} t}\textcolor{LightPink3}{t} \\
{\color{LightPink4} t}\textcolor{LightPink4}{t} &
{\color{LightSalmon1} t}\textcolor{LightSalmon1}{t} &
{\color{LightSalmon2} t}\textcolor{LightSalmon2}{t} &
{\color{LightSalmon3} t}\textcolor{LightSalmon3}{t} &
{\color{LightSalmon4} t}\textcolor{LightSalmon4}{t} &
{\color{LightSkyBlue1} t}\textcolor{LightSkyBlue1}{t} &
{\color{LightSkyBlue2} t}\textcolor{LightSkyBlue2}{t} &
{\color{LightSkyBlue3} t}\textcolor{LightSkyBlue3}{t} &
{\color{LightSkyBlue4} t}\textcolor{LightSkyBlue4}{t} &
{\color{LightSteelBlue1} t}\textcolor{LightSteelBlue1}{t} &
{\color{LightSteelBlue2} t}\textcolor{LightSteelBlue2}{t} &
{\color{LightSteelBlue3} t}\textcolor{LightSteelBlue3}{t} &
{\color{LightSteelBlue4} t}\textcolor{LightSteelBlue4}{t} &
{\color{LightYellow1} t}\textcolor{LightYellow1}{t} &
{\color{LightYellow2} t}\textcolor{LightYellow2}{t} &
{\color{LightYellow3} t}\textcolor{LightYellow3}{t} &
{\color{LightYellow4} t}\textcolor{LightYellow4}{t} &
{\color{Magenta1} t}\textcolor{Magenta1}{t} &
{\color{Magenta2} t}\textcolor{Magenta2}{t} &
{\color{Magenta3} t}\textcolor{Magenta3}{t} &
{\color{Magenta4} t}\textcolor{Magenta4}{t} \\
{\color{Maroon1} t}\textcolor{Maroon1}{t} &
{\color{Maroon2} t}\textcolor{Maroon2}{t} &
{\color{Maroon3} t}\textcolor{Maroon3}{t} &
{\color{Maroon4} t}\textcolor{Maroon4}{t} &
{\color{MediumOrchid1} t}\textcolor{MediumOrchid1}{t} &
{\color{MediumOrchid2} t}\textcolor{MediumOrchid2}{t} &
{\color{MediumOrchid3} t}\textcolor{MediumOrchid3}{t} &
{\color{MediumOrchid4} t}\textcolor{MediumOrchid4}{t} &
{\color{MediumPurple1} t}\textcolor{MediumPurple1}{t} &
{\color{MediumPurple2} t}\textcolor{MediumPurple2}{t} &
{\color{MediumPurple3} t}\textcolor{MediumPurple3}{t} &
{\color{MediumPurple4} t}\textcolor{MediumPurple4}{t} &
{\color{MistyRose1} t}\textcolor{MistyRose1}{t} &
{\color{MistyRose2} t}\textcolor{MistyRose2}{t} &
{\color{MistyRose3} t}\textcolor{MistyRose3}{t} &
{\color{MistyRose4} t}\textcolor{MistyRose4}{t} &
{\color{NavajoWhite1} t}\textcolor{NavajoWhite1}{t} &
{\color{NavajoWhite2} t}\textcolor{NavajoWhite2}{t} &
{\color{NavajoWhite3} t}\textcolor{NavajoWhite3}{t} &
{\color{NavajoWhite4} t}\textcolor{NavajoWhite4}{t} &
{\color{OliveDrab1} t}\textcolor{OliveDrab1}{t} \\
{\color{OliveDrab2} t}\textcolor{OliveDrab2}{t} &
{\color{OliveDrab3} t}\textcolor{OliveDrab3}{t} &
{\color{OliveDrab4} t}\textcolor{OliveDrab4}{t} &
{\color{Orange1} t}\textcolor{Orange1}{t} &
{\color{Orange2} t}\textcolor{Orange2}{t} &
{\color{Orange3} t}\textcolor{Orange3}{t} &
{\color{Orange4} t}\textcolor{Orange4}{t} &
{\color{OrangeRed1} t}\textcolor{OrangeRed1}{t} &
{\color{OrangeRed2} t}\textcolor{OrangeRed2}{t} &
{\color{OrangeRed3} t}\textcolor{OrangeRed3}{t} &
{\color{OrangeRed4} t}\textcolor{OrangeRed4}{t} &
{\color{Orchid1} t}\textcolor{Orchid1}{t} &
{\color{Orchid2} t}\textcolor{Orchid2}{t} &
{\color{Orchid3} t}\textcolor{Orchid3}{t} &
{\color{Orchid4} t}\textcolor{Orchid4}{t} &
{\color{PaleGreen1} t}\textcolor{PaleGreen1}{t} &
{\color{PaleGreen2} t}\textcolor{PaleGreen2}{t} &
{\color{PaleGreen3} t}\textcolor{PaleGreen3}{t} &
{\color{PaleGreen4} t}\textcolor{PaleGreen4}{t} &
{\color{PaleTurquoise1} t}\textcolor{PaleTurquoise1}{t} &
{\color{PaleTurquoise2} t}\textcolor{PaleTurquoise2}{t} \\
{\color{PaleTurquoise3} t}\textcolor{PaleTurquoise3}{t} &
{\color{PaleTurquoise4} t}\textcolor{PaleTurquoise4}{t} &
{\color{PaleVioletRed1} t}\textcolor{PaleVioletRed1}{t} &
{\color{PaleVioletRed2} t}\textcolor{PaleVioletRed2}{t} &
{\color{PaleVioletRed3} t}\textcolor{PaleVioletRed3}{t} &
{\color{PaleVioletRed4} t}\textcolor{PaleVioletRed4}{t} &
{\color{PeachPuff1} t}\textcolor{PeachPuff1}{t} &
{\color{PeachPuff2} t}\textcolor{PeachPuff2}{t} &
{\color{PeachPuff3} t}\textcolor{PeachPuff3}{t} &
{\color{PeachPuff4} t}\textcolor{PeachPuff4}{t} &
{\color{Pink1} t}\textcolor{Pink1}{t} &
{\color{Pink2} t}\textcolor{Pink2}{t} &
{\color{Pink3} t}\textcolor{Pink3}{t} &
{\color{Pink4} t}\textcolor{Pink4}{t} &
{\color{Plum1} t}\textcolor{Plum1}{t} &
{\color{Plum2} t}\textcolor{Plum2}{t} &
{\color{Plum3} t}\textcolor{Plum3}{t} &
{\color{Plum4} t}\textcolor{Plum4}{t} &
{\color{Purple1} t}\textcolor{Purple1}{t} &
{\color{Purple2} t}\textcolor{Purple2}{t} &
{\color{Purple3} t}\textcolor{Purple3}{t} \\
{\color{Purple4} t}\textcolor{Purple4}{t} &
{\color{Red1} t}\textcolor{Red1}{t} &
{\color{Red2} t}\textcolor{Red2}{t} &
{\color{Red3} t}\textcolor{Red3}{t} &
{\color{Red4} t}\textcolor{Red4}{t} &
{\color{RosyBrown1} t}\textcolor{RosyBrown1}{t} &
{\color{RosyBrown2} t}\textcolor{RosyBrown2}{t} &
{\color{RosyBrown3} t}\textcolor{RosyBrown3}{t} &
{\color{RosyBrown4} t}\textcolor{RosyBrown4}{t} &
{\color{RoyalBlue1} t}\textcolor{RoyalBlue1}{t} &
{\color{RoyalBlue2} t}\textcolor{RoyalBlue2}{t} &
{\color{RoyalBlue3} t}\textcolor{RoyalBlue3}{t} &
{\color{RoyalBlue4} t}\textcolor{RoyalBlue4}{t} &
{\color{Salmon1} t}\textcolor{Salmon1}{t} &
{\color{Salmon2} t}\textcolor{Salmon2}{t} &
{\color{Salmon3} t}\textcolor{Salmon3}{t} &
{\color{Salmon4} t}\textcolor{Salmon4}{t} &
{\color{SeaGreen1} t}\textcolor{SeaGreen1}{t} &
{\color{SeaGreen2} t}\textcolor{SeaGreen2}{t} &
{\color{SeaGreen3} t}\textcolor{SeaGreen3}{t} &
{\color{SeaGreen4} t}\textcolor{SeaGreen4}{t} \\
{\color{Seashell1} t}\textcolor{Seashell1}{t} &
{\color{Seashell2} t}\textcolor{Seashell2}{t} &
{\color{Seashell3} t}\textcolor{Seashell3}{t} &
{\color{Seashell4} t}\textcolor{Seashell4}{t} &
{\color{Sienna1} t}\textcolor{Sienna1}{t} &
{\color{Sienna2} t}\textcolor{Sienna2}{t} &
{\color{Sienna3} t}\textcolor{Sienna3}{t} &
{\color{Sienna4} t}\textcolor{Sienna4}{t} &
{\color{SkyBlue1} t}\textcolor{SkyBlue1}{t} &
{\color{SkyBlue2} t}\textcolor{SkyBlue2}{t} &
{\color{SkyBlue3} t}\textcolor{SkyBlue3}{t} &
{\color{SkyBlue4} t}\textcolor{SkyBlue4}{t} &
{\color{SlateBlue1} t}\textcolor{SlateBlue1}{t} &
{\color{SlateBlue2} t}\textcolor{SlateBlue2}{t} &
{\color{SlateBlue3} t}\textcolor{SlateBlue3}{t} &
{\color{SlateBlue4} t}\textcolor{SlateBlue4}{t} &
{\color{SlateGray1} t}\textcolor{SlateGray1}{t} &
{\color{SlateGray2} t}\textcolor{SlateGray2}{t} &
{\color{SlateGray3} t}\textcolor{SlateGray3}{t} &
{\color{SlateGray4} t}\textcolor{SlateGray4}{t} &
{\color{Snow1} t}\textcolor{Snow1}{t} \\
{\color{Snow2} t}\textcolor{Snow2}{t} &
{\color{Snow3} t}\textcolor{Snow3}{t} &
{\color{Snow4} t}\textcolor{Snow4}{t} &
{\color{SpringGreen1} t}\textcolor{SpringGreen1}{t} &
{\color{SpringGreen2} t}\textcolor{SpringGreen2}{t} &
{\color{SpringGreen3} t}\textcolor{SpringGreen3}{t} &
{\color{SpringGreen4} t}\textcolor{SpringGreen4}{t} &
{\color{SteelBlue1} t}\textcolor{SteelBlue1}{t} &
{\color{SteelBlue2} t}\textcolor{SteelBlue2}{t} &
{\color{SteelBlue3} t}\textcolor{SteelBlue3}{t} &
{\color{SteelBlue4} t}\textcolor{SteelBlue4}{t} &
{\color{Tan1} t}\textcolor{Tan1}{t} &
{\color{Tan2} t}\textcolor{Tan2}{t} &
{\color{Tan3} t}\textcolor{Tan3}{t} &
{\color{Tan4} t}\textcolor{Tan4}{t} &
{\color{Thistle1} t}\textcolor{Thistle1}{t} &
{\color{Thistle2} t}\textcolor{Thistle2}{t} &
{\color{Thistle3} t}\textcolor{Thistle3}{t} &
{\color{Thistle4} t}\textcolor{Thistle4}{t} &
{\color{Tomato1} t}\textcolor{Tomato1}{t} &
{\color{Tomato2} t}\textcolor{Tomato2}{t} \\
{\color{Tomato3} t}\textcolor{Tomato3}{t} &
{\color{Tomato4} t}\textcolor{Tomato4}{t} &
{\color{Turquoise1} t}\textcolor{Turquoise1}{t} &
{\color{Turquoise2} t}\textcolor{Turquoise2}{t} &
{\color{Turquoise3} t}\textcolor{Turquoise3}{t} &
{\color{Turquoise4} t}\textcolor{Turquoise4}{t} &
{\color{VioletRed1} t}\textcolor{VioletRed1}{t} &
{\color{VioletRed2} t}\textcolor{VioletRed2}{t} &
{\color{VioletRed3} t}\textcolor{VioletRed3}{t} &
{\color{VioletRed4} t}\textcolor{VioletRed4}{t} &
{\color{Wheat1} t}\textcolor{Wheat1}{t} &
{\color{Wheat2} t}\textcolor{Wheat2}{t} &
{\color{Wheat3} t}\textcolor{Wheat3}{t} &
{\color{Wheat4} t}\textcolor{Wheat4}{t} &
{\color{Yellow1} t}\textcolor{Yellow1}{t} &
{\color{Yellow2} t}\textcolor{Yellow2}{t} &
{\color{Yellow3} t}\textcolor{Yellow3}{t} &
{\color{Yellow4} t}\textcolor{Yellow4}{t} &
{\color{Gray0} t}\textcolor{Gray0}{t} &
{\color{Green0} t}\textcolor{Green0}{t} &
{\color{Grey0} t}\textcolor{Grey0}{t} \\
{\color{Maroon0} t}\textcolor{Maroon0}{t} &
{\color{Purple0} t}\textcolor{Purple0}{t} &
{\color{black} t}\textcolor{black}{t} &
{\color{blue} t}\textcolor{blue}{t} &
{\color{brown} t}\textcolor{brown}{t} &
{\color{cyan} t}\textcolor{cyan}{t} &
{\color{darkgray} t}\textcolor{darkgray}{t} &
{\color{gray} t}\textcolor{gray}{t} &
{\color{green} t}\textcolor{green}{t} &
{\color{lightgray} t}\textcolor{lightgray}{t} &
{\color{lime} t}\textcolor{lime}{t} &
{\color{magenta} t}\textcolor{magenta}{t} &
{\color{olive} t}\textcolor{olive}{t} &
{\color{orange} t}\textcolor{orange}{t} &
{\color{pink} t}\textcolor{pink}{t} &
{\color{purple} t}\textcolor{purple}{t} &
{\color{red} t}\textcolor{red}{t} &
{\color{teal} t}\textcolor{teal}{t} &
{\color{violet} t}\textcolor{violet}{t} &
{\color{white} t}\textcolor{white}{t} &
{\color{yellow} t}\textcolor{yellow}{t} \\
\end{tabular}} \\
    \begin{tabular}{cc}
    \textmd{text} {\mdseries text} & % = medium weight
    \textrm{text} {\rmfamily text} \\ % = roman font
    \textup{text} {\upshape text} & % = upright
    \textbf{text} {\bfseries text} \\ % = bold font
    \textsf{text} {\sffamily text} & % = sans-serif
    \texttt{text} {\ttfamily text} \\ % = typewriter font
    \textit{text} {\itshape text} & % = italic
    \textsl{text} {\slshape text} \\ % = slanted/oblique
    \multicolumn{2}{c}{\emph{some \emph{really} emphasised}} \\ % = emphasised
    \multicolumn{2}{c}{\em some {\em really} emphasised} \\ % = emphasised
    \textsc{I am} & {\scshape Death} \\ % = small caps
    \end{tabular} &
    \begin{tabular}{cc}
    \begin{tiny}text\end{tiny} &
    {\tiny text} \\ % smallest size
    \begin{scriptsize}text\end{scriptsize} &
    {\scriptsize text} \\ % size of (first level) super and supscripts
    \begin{footnotesize}text\end{footnotesize} &
    {\footnotesize text} \\ % size of footnotes
    \begin{small}text\end{small} &
    {\small text} \\
    \begin{normalsize}text\end{normalsize} &
    {\normalsize text} \\ % size of regular text
    \begin{large}text\end{large} &
    {\large text} \\
    \begin{Large}text\end{Large} &
    {\Large text} \\
    \begin{LARGE}text\end{LARGE} &
    {\LARGE text} \\
    \begin{huge}text\end{huge} &
    {\huge text} \\
    \begin{Huge}text\end{Huge} &
    {\Huge text} \\
    \end{tabular} \\
    \end{longtable}
    \end{center}

    re: L, \(\log_2 255 \ne 8\)

\begin{lemma}[Four by two]
Where \(4 \defeq 1 + 1 + 1 + 1\) and \(2 \defeq 1 + 1\)
\begin{equation}
\forall k \in \Reals: \frac 42 = 2
\end{equation}
\end{lemma}

\begin{proof}
    Suppose that
    \begin{equation*}
    \frac 42 = x = \frac{1 + x}{1 + x^{-1}}
    \end{equation*}
    for some \(x > 0\) (hence \(x^{-1} \neq -1\) and the expression is defined).

    Now,
    \begin{alignat*}{2}
    &&\frac 42 &= \frac{1 + x}{1 + x^{-1}} \\
    &\iff& 4(1 + x^{-1}) &= 2(1 + x) \\
    &\iff& 2x^2 + 2x - 4x - 4 &= 0 \\
    &\iff& 2(x - 2)(x + 1) &= 0 \\
    &\iff& x &\in \set{2, -1} \\
    \end{alignat*}
    But \(x > 0\) and therefore \(\frac 42 = x = 2\).
\end{proof}

%    \begin{equation}
%    \frac 1{\sqrt a + \sqrt b + \sqrt c + \sqrt d} =
%    \frac{a^{\frac{7}{2}} + 2 a^{\frac{5}{2}} \sqrt{b} \sqrt{c} + 2 a^{\frac{5}{2}} \sqrt{b} \sqrt{d} - 3 a^{\frac{5}{2}} b + 2 a^{\frac{5}{2}} \sqrt{c} \sqrt{d} - 3 a^{\frac{5}{2}} c - 3 a^{\frac{5}{2}} d - 4 a^{\frac{3}{2}} b^{\frac{3}{2}} \sqrt{c} - 4 a^{\frac{3}{2}} b^{\frac{3}{2}} \sqrt{d} - 4 a^{\frac{3}{2}} \sqrt{b} c^{\frac{3}{2}} + 4 a^{\frac{3}{2}} \sqrt{b} \sqrt{c} d + 4 a^{\frac{3}{2}} \sqrt{b} c \sqrt{d} - 4 a^{\frac{3}{2}} \sqrt{b} d^{\frac{3}{2}} + 3 a^{\frac{3}{2}} b^{2} + 4 a^{\frac{3}{2}} b \sqrt{c} \sqrt{d} + 2 a^{\frac{3}{2}} b c + 2 a^{\frac{3}{2}} b d - 4 a^{\frac{3}{2}} c^{\frac{3}{2}} \sqrt{d} - 4 a^{\frac{3}{2}} \sqrt{c} d^{\frac{3}{2}} + 3 a^{\frac{3}{2}} c^{2} + 2 a^{\frac{3}{2}} c d + 3 a^{\frac{3}{2}} d^{2} + 2 \sqrt{a} b^{\frac{5}{2}} \sqrt{c} + 2 \sqrt{a} b^{\frac{5}{2}} \sqrt{d} - 4 \sqrt{a} b^{\frac{3}{2}} c^{\frac{3}{2}} + 4 \sqrt{a} b^{\frac{3}{2}} \sqrt{c} d + 4 \sqrt{a} b^{\frac{3}{2}} c \sqrt{d} - 4 \sqrt{a} b^{\frac{3}{2}} d^{\frac{3}{2}} + 2 \sqrt{a} \sqrt{b} c^{\frac{5}{2}} + 4 \sqrt{a} \sqrt{b} c^{\frac{3}{2}} d - 6 \sqrt{a} \sqrt{b} \sqrt{c} d^{2} - 6 \sqrt{a} \sqrt{b} c^{2} \sqrt{d} + 4 \sqrt{a} \sqrt{b} c d^{\frac{3}{2}} + 2 \sqrt{a} \sqrt{b} d^{\frac{5}{2}} - \sqrt{a} b^{3} - 6 \sqrt{a} b^{2} \sqrt{c} \sqrt{d} + \sqrt{a} b^{2} c + \sqrt{a} b^{2} d + 4 \sqrt{a} b c^{\frac{3}{2}} \sqrt{d} + 4 \sqrt{a} b \sqrt{c} d^{\frac{3}{2}} + \sqrt{a} b c^{2} - 10 \sqrt{a} b c d + \sqrt{a} b d^{2} + 2 \sqrt{a} c^{\frac{5}{2}} \sqrt{d} - 4 \sqrt{a} c^{\frac{3}{2}} d^{\frac{3}{2}} + 2 \sqrt{a} \sqrt{c} d^{\frac{5}{2}} - \sqrt{a} c^{3} + \sqrt{a} c^{2} d + \sqrt{a} c d^{2} - \sqrt{a} d^{3} - a^{3} \sqrt{b} - a^{3} \sqrt{c} - a^{3} \sqrt{d} + 3 a^{2} b^{\frac{3}{2}} - 6 a^{2} \sqrt{b} \sqrt{c} \sqrt{d} + a^{2} \sqrt{b} c + a^{2} \sqrt{b} d + a^{2} b \sqrt{c} + a^{2} b \sqrt{d} + 3 a^{2} c^{\frac{3}{2}} + a^{2} \sqrt{c} d + a^{2} c \sqrt{d} + 3 a^{2} d^{\frac{3}{2}} - 3 a b^{\frac{5}{2}} + 4 a b^{\frac{3}{2}} \sqrt{c} \sqrt{d} + 2 a b^{\frac{3}{2}} c + 2 a b^{\frac{3}{2}} d + 4 a \sqrt{b} c^{\frac{3}{2}} \sqrt{d} + 4 a \sqrt{b} \sqrt{c} d^{\frac{3}{2}} + a \sqrt{b} c^{2} - 10 a \sqrt{b} c d + a \sqrt{b} d^{2} + a b^{2} \sqrt{c} + a b^{2} \sqrt{d} + 2 a b c^{\frac{3}{2}} - 10 a b \sqrt{c} d - 10 a b c \sqrt{d} + 2 a b d^{\frac{3}{2}} - 3 a c^{\frac{5}{2}} + 2 a c^{\frac{3}{2}} d + a \sqrt{c} d^{2} + a c^{2} \sqrt{d} + 2 a c d^{\frac{3}{2}} - 3 a d^{\frac{5}{2}} + b^{\frac{7}{2}} + 2 b^{\frac{5}{2}} \sqrt{c} \sqrt{d} - 3 b^{\frac{5}{2}} c - 3 b^{\frac{5}{2}} d - 4 b^{\frac{3}{2}} c^{\frac{3}{2}} \sqrt{d} - 4 b^{\frac{3}{2}} \sqrt{c} d^{\frac{3}{2}} + 3 b^{\frac{3}{2}} c^{2} + 2 b^{\frac{3}{2}} c d + 3 b^{\frac{3}{2}} d^{2} + 2 \sqrt{b} c^{\frac{5}{2}} \sqrt{d} - 4 \sqrt{b} c^{\frac{3}{2}} d^{\frac{3}{2}} + 2 \sqrt{b} \sqrt{c} d^{\frac{5}{2}} - \sqrt{b} c^{3} + \sqrt{b} c^{2} d + \sqrt{b} c d^{2} - \sqrt{b} d^{3} - b^{3} \sqrt{c} - b^{3} \sqrt{d} + 3 b^{2} c^{\frac{3}{2}} + b^{2} \sqrt{c} d + b^{2} c \sqrt{d} + 3 b^{2} d^{\frac{3}{2}} - 3 b c^{\frac{5}{2}} + 2 b c^{\frac{3}{2}} d + b \sqrt{c} d^{2} + b c^{2} \sqrt{d} + 2 b c d^{\frac{3}{2}} - 3 b d^{\frac{5}{2}} + c^{\frac{7}{2}} - 3 c^{\frac{5}{2}} d + 3 c^{\frac{3}{2}} d^{2} - \sqrt{c} d^{3} - c^{3} \sqrt{d} + 3 c^{2} d^{\frac{3}{2}} - 3 c d^{\frac{5}{2}} + d^{\frac{7}{2}}}{a^{4} - 4 a^{3} b - 4 a^{3} c - 4 a^{3} d + 6 a^{2} b^{2} + 4 a^{2} b c + 4 a^{2} b d + 6 a^{2} c^{2} + 4 a^{2} c d + 6 a^{2} d^{2} - 4 a b^{3} + 4 a b^{2} c + 4 a b^{2} d + 4 a b c^{2} - 40 a b c d + 4 a b d^{2} - 4 a c^{3} + 4 a c^{2} d + 4 a c d^{2} - 4 a d^{3} + b^{4} - 4 b^{3} c - 4 b^{3} d + 6 b^{2} c^{2} + 4 b^{2} c d + 6 b^{2} d^{2} - 4 b c^{3} + 4 b c^{2} d + 4 b c d^{2} - 4 b d^{3} + c^{4} - 4 c^{3} d + 6 c^{2} d^{2} - 4 c d^{3} + d^{4}}
%    \end{equation}

\nocite{*}

\bibliographystyle{agsm}
\bibliography{sources}
\end{document}
